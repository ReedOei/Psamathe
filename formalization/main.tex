\documentclass[10pt]{article}

\usepackage{amsmath}
\usepackage{sagetex}
\usepackage{hyperref}
\usepackage{tikz-cd}
\usepackage{amssymb}
\usepackage{amsthm}
\usepackage{bm}
\usepackage{listings}
\usepackage{bbm}
\usepackage{multicol}
\usepackage{mathtools}
\usepackage{mathpartir}
\usepackage{float}
\usepackage[inline]{enumitem}
\usepackage[margin=1.25in]{geometry}
\usepackage[T1]{fontenc}
\usepackage{kpfonts}

\usetikzlibrary{decorations.pathmorphing}

\lstset{
  frame=none,
  xleftmargin=2pt,
  stepnumber=1,
  numbers=left,
  numbersep=5pt,
  numberstyle=\ttfamily\tiny\color[gray]{0.3},
  belowcaptionskip=\bigskipamount,
  captionpos=b,
  escapeinside={*'}{'*},
  tabsize=2,
  emphstyle={\bf},
  commentstyle=\it,
  stringstyle=\mdseries\rmfamily,
  showspaces=false,
  keywordstyle=\bfseries\rmfamily,
  columns=flexible,
  basicstyle=\small\sffamily,
  showstringspaces=false,
}

\newcommand{\guard}{\bigg|\bigg|}
\newcommand{\newc}{\textbf{\texttt{new}}}
\newcommand{\everything}{\textbf{\texttt{everything}}}
\newcommand{\then}{\textbf{\texttt{then}}}
\newcommand{\this}{\textbf{\texttt{this}}}
\newcommand{\langName}{LANGUAGE-NAME\xspace}
\newcommand{\consumesarr}{\mathrel{\rotatebox[origin=c]{270}{${\looparrowright}$}}}
\newcommand{\consumes}[1]{\stackrel{#1}{\consumesarr}}
\newcommand{\sends}[1]{\stackrel{#1}{\to}}
\newcommand{\emitsarr}{\mathrel{\rotatebox[origin=c]{270}{${\looparrowleft}$}}}
\newcommand{\emits}[1]{\emitsarr #1}
\newcommand{\asset}{\textbf{\texttt{asset}}\xspace}
\newcommand{\fungible}{\textbf{\texttt{fungible}}\xspace}
\newcommand{\nonfungible}{\textbf{\texttt{nonfungible}}\xspace}
\newcommand{\addresst}{\textbf{\texttt{address}}\xspace}
\newcommand{\stringt}{\textbf{\texttt{string}}\xspace}
\newcommand{\setof}{\textbf{\texttt{set}}\xspace}
\newcommand{\optiont}{\textbf{\texttt{option}}\xspace}
\newcommand{\boolt}{\textbf{\texttt{bool}}\xspace}
\newcommand{\natt}{\textbf{\texttt{nat}}\xspace}
\newcommand{\byt}{\textbf{\texttt{by}}\xspace}
\newcommand{\stores}{\textbf{\texttt{stores}}\xspace}
\newcommand{\suchthat}{\textbf{s.t.}\xspace}
\newcommand{\returns}{\textbf{\texttt{returns}}\xspace}
\newcommand{\merge}{\rightsquigarrow}
\newcommand{\flowsto}{\rightsquigarrow}
\newcommand{\heldby}{\rightarrowtail}
\newcommand{\one}{\textbf{\texttt{one}}\xspace}
\newcommand{\some}{\textbf{\texttt{some}}\xspace}
\newcommand{\any}{\textbf{\texttt{any}}\xspace}
\newcommand{\map}{\textbf{\texttt{map}}\xspace}
\newcommand{\linking}{\textbf{\texttt{linking}}\xspace}
\newcommand{\transformer}{\textbf{\texttt{transformer}}\xspace}
\newcommand{\selects}{\textbf{\texttt{selects}}\xspace}
\newcommand{\demote}{\textbf{\texttt{demote}}\xspace}
\newcommand{\transforms}{\Rrightarrow}

\newcommand{\flowproves}{\mathbin{|\hskip -0.1em\scalebox{0.7}[1]{$\leadsto$}}}
\newcommand{\flowproveout}{\mathbin{\scalebox{0.7}[1]{$\leadsto$}\hskip -0.13em|}}


\input{../../Reed/School/LaTeX/lemmas.tex}

\lstset{
  frame=none,
  xleftmargin=2pt,
  stepnumber=1,
  numbers=left,
  numbersep=5pt,
  numberstyle=\ttfamily\tiny\color[gray]{0.3},
  belowcaptionskip=\bigskipamount,
  captionpos=b,
  escapeinside={*'}{'*},
  tabsize=2,
  emphstyle={\bf},
  commentstyle=\it,
  stringstyle=\mdseries\rmfamily,
  showspaces=false,
  keywordstyle=\bfseries\rmfamily,
  columns=flexible,
  basicstyle=\small\sffamily,
  showstringspaces=false,
}

\newcommand{\guard}{\bigg|\bigg|}
\newcommand{\newc}{\textbf{\texttt{new}}}
\newcommand{\everything}{\textbf{\texttt{everything}}}
\newcommand{\then}{\textbf{\texttt{then}}}
\newcommand{\this}{\textbf{\texttt{this}}}
\newcommand{\langName}{ResourceFlow\xspace}
\newcommand{\consumesarr}{\mathrel{\rotatebox[origin=c]{270}{${\looparrowright}$}}}
\newcommand{\consumes}[1]{\stackrel{#1}{\consumesarr}}
\newcommand{\sends}[1]{\stackrel{#1}{\to}}
\newcommand{\emitsarr}{\mathrel{\rotatebox[origin=c]{270}{${\looparrowleft}$}}}
\newcommand{\emits}[1]{\emitsarr #1}
\newcommand{\addresst}{\textbf{\texttt{address}}\xspace}
\newcommand{\stringt}{\textbf{\texttt{string}}\xspace}
\newcommand{\manyt}{\textbf{\texttt{many}}\xspace}
\newcommand{\natt}{\textbf{\texttt{nat}}\xspace}
\newcommand{\byt}{\textbf{\texttt{by}}\xspace}
\newcommand{\stores}{\textbf{\texttt{stores}}\xspace}
\newcommand{\suchthat}{\textbf{s.t.}\xspace}
\newcommand{\returns}{\textbf{\texttt{returns}}\xspace}
\newcommand{\TRN}{\textbf{\texttt{TRN}}\xspace}
\newcommand{\merge}{\rightsquigarrow}
\newcommand{\heldby}{\rightarrowtail}
\newcommand{\one}{\textbf{\texttt{one}}\xspace}
\newcommand{\some}{\textbf{\texttt{some}}\xspace}
\newcommand{\any}{\textbf{\texttt{any}}\xspace}

\begin{document}

\section{Formalization}

\subsection{Syntax}
\begin{figure}[ht]
\begin{align*}
    C &\in \textsc{ContractNames} & T &\in \textsc{TransactionNames} \\
    I &\in \textsc{InterfaceNames} & S &\in \textsc{StateNames}\\
    D &\in \textsc{ContractNames} \cup \textsc{InterfaceNames} & p &\in \Q \\
    E &\in \textsc{EventNames} & f &\in \textsc{FieldNames} \\
    x &\in \textsc{IdentifierNames} & \mathcal{R} &\in \textsc{ResourceNames} \\
\end{align*}

\begin{tabular}{l r l l}
    $\tau$ & \bnfdef & $\natt$ \bnfalt \stringt \bnfalt \addresst \bnfalt $\tau \times \tau$ \bnfalt $\manyt~\tau$ \bnfalt $\tau~\byt~\tau$ \bnfalt $\curlys{x_i~\stores~\tau_i, \alpha}_{i<n}$ \bnfalt $\mathcal{R}$ & (types) \\
    $\alpha$ & \bnfdef & & (row type variables) \\
    $Q$ & \bnfdef & \one \bnfalt \some \bnfalt \any & (flow quantities) \\
    $\delta$ & \bnfdef & $p$ \bnfalt $x$ \bnfalt $x \%$ \bnfalt $\everything$ \bnfalt $p \delta$ \bnfalt $\delta + \delta$ \bnfalt $\delta - \delta$ \bnfalt $\min(\delta, \delta)$ \bnfalt $\max(\delta, \delta)$ \bnfalt $Q~x~\suchthat~\varphi(x)$ & (prim. dist. specifier) \\
    $A$ & \bnfdef & $T(x_1, \ldots, x_n)$ \bnfalt $\emits{E(x_1, \ldots, x_n)}$ & (actions) \\
    $\mathcal{S}$ & \bnfdef & $x$ \bnfalt $\newc~x~\stores~\tau$ & (storage identifiers) \\
    $F$ & \bnfdef & $\mathcal{S} \to_\times \mathcal{S} ~\then~\curlys{A_i}_{i<n}$ & (flows) \\
    $\TRN$ & \bnfdef & $T(x_1 : \tau, \ldots, x_n : x_n) ~\returns~ \tau : \curlys{F_i}_{i < n}$ & (transactions) \\

% $T$ & \bnfdef & $T_C$@$T_{ST}$ & (types of contract references)\\[\nonterminaldefskip]
% &   \bnfalt \unit{}\\

% $T_{C}$ & \bnfdef & \generics{D}{T} & (types of contracts/interfaces)\\
% &   \bnfalt & X & (declaration variables) \\[\nonterminaldefskip]

% $T_{ST}$ & \bnfdef & \textoverline{S}  & (state disjunction)\\
% & 	\bnfalt & p & (permission/state variables)\\
% &  	\bnfalt & P & (concrete permission) \\[\nonterminaldefskip]

% $P$ &	\bnfdef & \Owned\xspace \bnfalt \Unowned\xspace \bnfalt \Shared & \\[\nonterminaldefskip]

% $T_G$ & \bnfdef & \genericParamOpt{X}{p}{\generics{I}{T}@$T_{ST}$} & (generic type parameter) \\[\nonterminaldefskip]

% $\mathit{CON}$ & \bnfdef & \contract \ \generics{C}{T_G}\ \implements\ \generics{I}{T} \{$\overline{ST}$ \textoverline{M}\} \\[\nonterminaldefskip]

% $\mathit{IFACE}$ & \bnfdef & \interface\ \generics{I}{T_G} \{ \textoverline{ST}  \textoverline{$M_{\mathit{SIG}}$} \} & \\[\nonterminaldefskip]

% $\mathit{ST}$ & \bnfdef & [\asset] $S$  $\overline{F}$ \\[\nonterminaldefskip]

% $F$ & \bnfdef & $T$ $f$ \\[\nonterminaldefskip]

% $M_{\mathit{SIG}}$ & \bnfdef & $T \: \generics{m}{T_G}$ (\textoverline{$T \trans T_{ST} \: x$}) $T_{ST} \trans T_{ST}$ & (transaction specifying types for  \\
% &			 &	& return, arguments, and receiver) \\[\nonterminaldefskip]
% &	\bnfalt  & \private\xspace \textoverline{$T_{ST} \trans T_{ST} \ f$} T \generics{m}{T_G}(\textoverline{$T \trans T_{ST} \ x$}) & (private transactions also\\
% & 			 & 	T\textsubscript{ST} \trans \ T\textsubscript{ST} & specify field types) \\[\nonterminaldefskip]

% $M$ & \bnfdef & M\textsubscript{SIG} \{ \returnExpr{e} \} \\

% $e$ & \bnfdef & s \\
% &		\bnfalt & s.f & (field access) \\
% &		\bnfalt & s.\generics{m}{T}(\textoverline{x}) \\
% &		\bnfalt & \letExpr{x}{T}{e}{e} \\
% &		\bnfalt & \new{C}{T}{S}{\textoverline{s}} &(contract fields, then state fields)\\
% &		\bnfalt & s $\transition_{\Owned \bnfalt \Shared}$ S(\textoverline{s}) & (State transition initializing fields) \\
% &		\bnfalt & s.f := s & (field update, with 1-based indexing)\\
% &		\bnfalt & \assertExpr{s}{T\textsubscript{ST}} & {(static assert)} \\
% &		\bnfalt & \ifInState{s}{P}{T\textsubscript{ST}}{e}{e} & (state test, owned or shared s) \\
% &		\bnfalt & \disown s & (drop ownership of owned ref.) \\
% &		\bnfalt & \pack \\

% $s$ & \bnfdef 	& x & (simple expressions)\\
\end{tabular}
\caption{Abstract syntax of \langName.}
\label{lang-syntax}
\end{figure}

Fig. \ref{lang-syntax} shows the syntax of \langName.

\subsection{Definitions}
\reed{I think that the distribution specifiers could be further generalized over any field $F$ instead of just $\Q$, but I'm not sure if that is actually useful; we could even do any ring if we drop the requirement of having the $/100$ thing for percents.}
\begin{definition}
    A \emph{type} is a member of the set $\mathfrak{T}$, given in Figure~\ref{lang-syntax} by the nonterminal $\tau$.
\end{definition}
\begin{definition}
    A \emph{primitive distribution specifier} is a member of the set $\mathfrak{D}_p$, given in Figure~\ref{lang-syntax} by the nonterminal $\delta$.
\end{definition}

\begin{definition}
    The \emph{interpetation of $\delta$}, $\llbracket \delta \rrbracket$, is a function $\mathcal{V} \to \Q[\mathcal{V}]$ defined below, where $\delta$ is a primitive distribution specifier.
    Let $p \in \Q$, $y \in \mathcal{V}$, and $\delta, \epsilon \in \mathfrak{D}_p$.
    Then
    \begin{align*}
        \llbracket p \rrbracket(x) &:= p \\
        \llbracket y \rrbracket(x) &:= y \\
        \llbracket y\% \rrbracket (x) &:= xy / 100 \\
        \llbracket \everything \rrbracket (x) &:= x \\
        \llbracket p \delta \rrbracket (x) &:= p \cdot \parens{ \llbracket \delta \rrbracket (x) }  \\
        \llbracket \delta + \epsilon \rrbracket(x) &:= \parens{\llbracket \delta \rrbracket (x)} + \parens{\llbracket \epsilon \rrbracket(x)} \\
        \llbracket \delta - \epsilon \rrbracket(x) &:= \parens{\llbracket \delta \rrbracket (x)} - \parens{\llbracket \epsilon \rrbracket(x)} \\
        \llbracket \min(\delta, \epsilon) \rrbracket(x) &:= \min\parens{\llbracket \delta \rrbracket (x), \llbracket \epsilon \rrbracket (x)} \\
        \llbracket \max(\delta, \epsilon) \rrbracket(x) &:= \max\parens{\llbracket \delta \rrbracket (x), \llbracket \epsilon \rrbracket (x)}
    \end{align*}
\end{definition}
\reed{This is kind of like a module-homomorphism, where the set of distribution specifiers is like a $\Q$-modules}

Note that $\llbracket \delta \rrbracket (x) \in \Q[\mathcal{V}(\delta) \cup \{x\}]$, that is, it is a polynomial with rational coefficients in the variables $\mathcal{V}(\delta) \cup \{x\}$.
\reed{I don't know if this actually matters, but it's somewhat interesting}

We consider distribution specifiers modulo their interpetation, that is, a distribution specifier is an equivalence class of primitive distribution specifiers.
\begin{definition}
    A \emph{distribution specifier} $\delta$ is a member of the set $\mathfrak{D} := \setbuild{\llbracket \delta_p \rrbracket}{\delta_p \in \mathfrak{D}_p}$.
\end{definition}

This makes $(\mathfrak{D}, +, \llbracket 0 \rrbracket)$ into an abelian group; in fact, we can make it into a $\Q$-module by defining
\[
    q \cdot \llbracket \delta \rrbracket := \llbracket q \delta \rrbracket
\]
for all $q \in \Q$.
\reed{I'm pretty sure this is true. Does it matter though? We could probably even make it into a ring if we wanted (but not a field, because there won't be an inverse for \everything), but nto sure if that's desirable.}
Note that, from now on, we use the elements of $\mathfrak{D}_p$ interchangably with their equivalence class in $\mathfrak{D}$ (e.g., $0$ instead of $\llbracket 0 \rrbracket$), to make the notation less noisy.

\begin{definition}
    Let $\delta$ be a distribution specifier.
    An \emph{arrow} $\to_\times$ is either
    \begin{enumerate*}[label=(\roman*)]
        \item a \emph{send} arrow $\sends{\delta}$;
        \item a \emph{merge} arrow $\merge$;
        \item a \emph{hold} arrow $\heldby$;
        \item or a \emph{consume} arrow $\consumes{\delta}$.
    \end{enumerate*}
\end{definition}

\framebox{$\Gamma \proves F~\mathfrak{wf}$} \textbf{Well-formed flows}

\begin{mathpar}
    \inferrule*[right=Flow-Send]{
        \Gamma \proves s~\stores~\tau
        \and
        \Gamma \proves d~\stores~\tau
        \and
        \Gamma \proves \llbracket \delta \rrbracket : \mathcal{V} \to R[\mathcal{V}]
    }{ \Gamma \proves s \sends{\delta} d~\mathfrak{wf} }

    \inferrule*[right=Flow-Merge]{
        \Gamma \proves s~\stores~\tau
        \and
        \Gamma \proves d~\stores~\tau
    }{ \Gamma \proves s \merge d~\mathfrak{wf} }

    \inferrule*[right=Flow-Hold]{
        \Gamma \proves s~\stores~\tau
        \and
        \Gamma \proves d~\stores~\tau
    }{ \Gamma \proves s \heldby d~\mathfrak{wf} }

    \inferrule*[right=Flow-Consume]{
        \Gamma \proves s~\stores~\tau
        \and
        \Gamma \proves \llbracket \delta \rrbracket_\tau
    }{ \Gamma \proves s \consumes{\delta}~\mathfrak{wf} }
\end{mathpar}

\begin{definition}
    Let $F$ be a flow and let $\Gamma$ be a type environment such that $\Gamma \proves F~\mathfrak{wf}$.
    The \emph{storage set of $F$}, $\mathcal{S}(F)$, is
    \begin{align*}
        \mathcal{S}(s \sends{\delta} d) & := \curlys{s~\stores~\Gamma(s),d} \\
        \mathcal{S}(s \merge d) & := \curlys{s,d} \\
        \mathcal{S}(s \heldby d) & := \curlys{s,d} \\
        \mathcal{S}(s \consumes{\delta}) & := \curlys{s}
    \end{align*}
\end{definition}

We write the set of flows on a set of storage identifiers $V$ as $\mathcal{F}_\ell(V)$.

\begin{definition}
    A \emph{transaction graph} is a pair $(V, E)$ where $V$ is a set of storage identifiers and $E \subseteq \mathcal{F}_\ell(V)$.
\end{definition}

\begin{definition}
    Let $T$ be a transaction and $\Gamma$ a type environment such that $\Gamma \proves T~\mathfrak{wf}$.
    The \emph{transaction graph of $T$ (in $\Gamma$)} is $G_{\Gamma}(T) := (\bigcup_{F \in T_F} \mathcal{S}_{\Gamma}(F), \mathcal{E}_{\Gamma}(T_F))$.
\end{definition}

\begin{definition}
    A \emph{resource} $\mathcal{R}$ is a tuple $(R, +, 0, \leq, -)$ where
    \begin{enumerate}[label=(\roman*)]
        \item $(R, +, 0)$ is a commutative monoid.
        \item $(R, \leq)$ is a partial order that is compatible with $(R, +, 0)$.
            That is, for any $x, y \in R$ such that $x \leq y$, and for any $z \in R$, we have $x + z \leq y + z$.
        \item $- : R \times R \to R$ is a function so that for any $x,y \in R$ such that $y \leq x$, we have $(x - y) + y = x$.
    \end{enumerate}
    The elements of a resource are called \emph{resource values}.
\end{definition}

\paragraph{Examples}

\begin{enumerate}
    \item The natural numbers with the standard operations is a resource $(\N, +, 0, \leq, -)$, where $-$ is \emph{saturating subtraction}: $n - m = 0$ if $m > n$.
    \item For any set $A$, we can build the resource $(\powerset{A}, \cup, \emptyset, \subseteq, \setminus)$, where $X \setminus Y$ is the set difference operation.
    \item Similarly, given any set $A$, we can build the resource $(\powersetfin{A}, \cup, \emptyset, \subseteq, \setminus)$; this resource is called the \emph{nonfungible resource on $A$}, written $\text{nf}(A)$.
\end{enumerate}

\begin{definition}
    We say that a resource $\mathcal{R}$ is \emph{nonfungible} if it is the nonfungible resource on $A$ for some set $A$, and otherwise we say that $\mathcal{R}$ is \emph{fungible}.
\end{definition}

\begin{definition}
    Let $\mathcal{R}$ and $\mathcal{}$ be resources.
    A function $f : \mathcal{R} \to \mathcal{S}$ is a \emph{resource homomorphism} (or simply a \emph{homomorphism}) if it is a monoid homomorphism, monotone and for all $r_1,r_2,r_3 \in \mathcal{R}$, satisfies
    \begin{enumerate}[label=(\roman*)]
        \item if $r_1 \leq r_2$, then $f(r_1) - f(r_2) = f(r_1 - r_2)$.
    \end{enumerate}
    \reed{Anything else}
\end{definition}

\reed{Maybe this is obvious, but it took me a minute to be sure of.}
\begin{proposition}
    There are fungible resources; in particular, $\N$ is a fungible resource.
\end{proposition}
\begin{proof}
    Suppose that $\N \iso \text{nf}(A)$ for some set $A$, and let $f : \N \to \text{nf}(A)$ be the isomorphism witnessing this.
    Then let $n > 0$, so $f(n) = f(1 + \cdots + 1) = f(1) + \cdots + f(1) = \bigcup_{i < n} f(1) = f(1)$, so $f$ is not a bijection, which is a contradiction.
\end{proof}

\begin{definition}
    For any two resources $\mathcal{R}$ and $\mathcal{S}$, their \emph{direct product}, $\mathcal{R} \oplus \mathcal{S}$ is $(R \times S, +_{RS}, (0_R, 0_S), \leq_{RS}, -_{RS})$, where
    \begin{align*}
        (r_1, s_1) +_{RS} (r_2, s_2) &:= (r_1 +_R r_2, s_1 +_S s_2) \\
        (r_1, s_1) -_{RS} (r_2, s_2) &:= (r_1 -_R r_2, s_1 -_S s_2) \\
        (r_1, s_1) \leq_{RS} (r_2, s_2) &:\iff r_1 \leq_R r_2 \land s_1 \leq_S s_2
    \end{align*}
\end{definition}

\begin{remark}
    The direct product of two resources is a resource.
\end{remark}

\begin{proposition}
    \reed{I think, and probably the same for nonfungible}
    The direct product of twice fungible resources is fungible.
\end{proposition}

\begin{definition}
    Let $\mathcal{R}$ be a resource.
    A \emph{selector for $\mathcal{R}$} is a function $\mathcal{R} \to \mathcal{R}$.
\end{definition}

\begin{definition}
    An \emph{environment} $\Delta$ is triple $(\mathfrak{R}, \mathcal{S}, \curlys{\#_\mathcal{R}}_{\mathcal{R} \in \mathfrak{R}})$ where $\mathfrak{R}$ is a set of resources, $\mathcal{S}$ is a set of storages and $\#_R : \mathcal{V} \to R$ is a family of functions mapping storages to resource value held by the storage.
\end{definition}

\reed{How to handle transactions which dynamically fail?}

\begin{definition}
    Let $\Delta = (\mathcal{R}, \mathcal{V}, \curlys{\#_R}_{R \in \mathcal{R}})$ be an environment.
    Then $\sum_\Delta : \mathcal{R} \to \mathcal{R}$ is a function such that
    \[
        \sum_{\Delta} R := \sum_{v \in \mathcal{V}} \#_R(v)
    \]
    That is, $\sum_\Delta$ maps every resource to the total amount of the resource currently held by variables.
\end{definition}
\reed{Not sure this is the best notation, but we can always change that}

\subsection{Safety Theorems}

\begin{theorem}[Asset Retention]
    \reed{It would be nice if we could prove this general version for any transaction graph, ignoring even where it came from.
    Could be a good start to making a more general resource flow framework}
    Let $(V, E)$ be a well-formed transaction graph.
    Let $\Delta$ be an environment containing some resource $R$.
    Let $f : \powerset{\Delta} \to \powerset{\Delta}$ be the associated resource transformation for $(V, E)$.
    Let $\mathcal{C}$ be the consumption set of $(V, E)$ and let $\mathcal{N}$ be the creation set of $(V, E)$.
    Then we have
    \[
        \sum_\mathcal{N} + \sum_\Delta = \sum_\mathcal{C} + \sum_{f(\Delta)}
    \]
    In particular, if $\mathcal{N} = \mathcal{C} = \emptyset$, then the amount of every resources stays constant.
\end{theorem}

\end{document}

