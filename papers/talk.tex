\documentclass[leqno,presentation,usenames,dvipsnames]{beamer}
\DeclareGraphicsExtensions{.eps,.jpg,.png,.tif}
\usepackage{amssymb, amsmath, pdfpages, amsfonts, calc, times, type1cm, latexsym, xcolor, colortbl, hyperref, bookmark}
\usepackage{mathtools}
\usepackage{graphicx}
\usepackage{tabularx}
\usepackage{multirow}
\usepackage{listings}
\usepackage{mathpartir}
\usepackage{xspace}
\usepackage{tipa}

\usepackage[latin1]{inputenc}
\usepackage[english]{babel}

\usetheme{Szeged}
\usecolortheme{beaver}

\definecolor{websiteGreen}{RGB}{107, 224, 134}
\definecolor{silvery}{RGB}{232, 241, 248}
\definecolor{deepOrange}{RGB}{209, 126, 0}

\definecolor{softTan}{RGB}{240, 240, 223}
\definecolor{deepGreen}{RGB}{87, 149, 115}
\definecolor{lilac}{RGB}{199, 164, 202}
\definecolor{lightOlive}{RGB}{168, 166, 96}
\definecolor{deepOlive}{RGB}{101, 109, 41}

\setbeamercolor*{palette primary}{bg=websiteGreen}
\setbeamercolor*{palette secondary}{bg=white}
\setbeamercolor*{palette tertiary}{bg=silvery}
\setbeamercolor*{palette quaternary}{bg=websiteGreen}

\setbeamercolor{section in head/foot}{fg=deepOrange}
\setbeamerfont{section in head/foot}{series=\bfseries}

\setbeamercolor{titlelike}{parent=palette primary,bg=websiteGreen,fg=white}
\setbeamercolor{frametitle}{parent=palette primary,bg=websiteGreen, fg=white}

\addtobeamertemplate{frametitle}{\vspace*{-0.65\baselineskip}}{}

\newcommand{\Int}{\textbf{Int}\xspace}
\newcommand{\overbar}[1]{\mkern1.5mu\overline{\mkern-1.5mu#1\mkern-1.5mu}\mkern1.5mu}
\newcommand{\highlight}[1]{
  \addtolength{\fboxrule}{.2ex}
  \begin{block}{}
    \begin{quote}#1
    \end{quote}
  \end{block}
}

\newtheorem*{conjecture}{Conjecture}
\newtheorem*{proposition}{Proposition}

\title{Psamathe: A DSL with Flows for Safer Blockchain Assets}
\author{\textbf{Reed Oei}}
\institute{University of Illinois, Urbana, US\\
\url{reedoei2@illinois.edu}}
\date{\today}

\definecolor{verylightgray}{rgb}{.97,.97,.97}

\lstdefinelanguage{flow}{
    keywords=[1]{var,interface,where,contract,transaction,view,transformer,type,is,returns},
    keywordstyle=[1]\color{Rhodamine}\bfseries,
    keywords=[2]{false,true,sometimes,fungible,asset,unique,immutable,consumable},
    keywordstyle=[2]\color{DarkOrchid}\bfseries,
    keywords=[3]{map,linking,set,list,address,uint256,string,bytes,nat,bool},
    keywordstyle=[3]\color{Emerald}\itshape,
    keywords=[4]{nonempty,any,every,empty,exactly,one,msg,this,in,new,or,and,not,total,consume},
    keywordstyle=[4]\color{Lavender}\itshape,
    keywords=[5]{if,else,only,when,try,catch,such,that},
    keywordstyle=[5]\color{Cerulean}\bfseries,
    keepspaces=true,
	% backgroundcolor=\color{verylightgray},
    % columns=fullflexible,
    % literate=*
    %     {(}{{{\color{red}(}}}1
    %     {)}{{{\color{red})}}}1
    %     {+}{{{\color{red}+~}}}1
    %     {-}{{{\color{red}-~}}}1
    %     {:=}{{{\color{red}:=~}}}1
    %     {\{}{{{\color{red}\{}}}1
    %     {\}}{{{\color{red}\}}}}1
    %     {[}{{{\color{red}[}}}1
    %     {]}{{{\color{red}]}}}1
    %     {*}{{{\color{red}*~}}}1
    %     {:}{{{\color{red}:~}}}1
    %     {>}{{{\color{red}>~}}}1
    %     {<}{{{\color{red}<~}}}1
    %     {<=>}{{{$\color{red}\Leftrightarrow~$}}}1
    %     {.}{{{\color{red}.~}}}1
    %     {=}{{{$\color{red}=~$}}}1
    %     {exists }{{{$\color{red}\exists$}}}1
    %     {forall }{{{$\color{red}\forall$}}}1,
    sensitive=false, % keywords are not case-sensitive
    morecomment=[l]{//}, % l is for line comment
    morecomment=[s]{/*}{*/}, % s is for start and end delimiter
    commentstyle=\color{CadetBlue}\textit,
    morestring=[b]", % defines that strings are enclosed in double quotes
    stringstyle=\color{ForestGreen}, % string literal style
    showstringspaces=false,
    xleftmargin=5.0ex,
    mathescape=true,
    escapechar=|,
    basicstyle=\small\ttfamily
}

\lstnewenvironment{flow}
{%
    \lstset{language=flow, basicstyle=\small\ttfamily, mathescape=true}%
}
{
}

\newcommand{\flowinline}[1]{\lstinline[language=flow,basicstyle=\small\ttfamily,mathescape]{#1}}


% Copyright 2017 Sergei Tikhomirov, MIT License
% https://github.com/s-tikhomirov/solidity-latex-highlighting/

\usepackage{listings, xcolor}

\definecolor{verylightgray}{rgb}{.97,.97,.97}

\lstdefinelanguage{Solidity}{
	keywords=[1]{anonymous, assembly, balance, call, callcode, class, constant, constructor, contract, debugger, delegatecall, delete, emit, event, experimental, export, external, function, gas, implements, import, in, indexed, instanceof, interface, internal, is, length, library, log0, log1, log2, log3, log4, memory, modifier, new, payable, pragma, private, protected, public, pure, push, return, returns, selfdestruct, send, solidity, storage, struct, suicide, super, then, throw, typeof, using, view, with, addmod, ecrecover, keccak256, mulmod, ripemd160, sha256, sha3}, % generic keywords including crypto operations
	keywordstyle=[1]\color{Rhodamine}\bfseries,
	keywords=[2]{address, bool, byte, bytes, bytes1, bytes2, bytes3, bytes4, bytes5, bytes6, bytes7, bytes8, bytes9, bytes10, bytes11, bytes12, bytes13, bytes14, bytes15, bytes16, bytes17, bytes18, bytes19, bytes20, bytes21, bytes22, bytes23, bytes24, bytes25, bytes26, bytes27, bytes28, bytes29, bytes30, bytes31, bytes32, enum, int, int8, int16, int24, int32, int40, int48, int56, int64, int72, int80, int88, int96, int104, int112, int120, int128, int136, int144, int152, int160, int168, int176, int184, int192, int200, int208, int216, int224, int232, int240, int248, int256, mapping, string, uint, uint8, uint16, uint24, uint32, uint40, uint48, uint56, uint64, uint72, uint80, uint88, uint96, uint104, uint112, uint120, uint128, uint136, uint144, uint152, uint160, uint168, uint176, uint184, uint192, uint200, uint208, uint216, uint224, uint232, uint240, uint248, uint256, var, void, ether, finney, szabo, wei, days, hours, minutes, seconds, weeks, years},	% types; money and time units
	keywordstyle=[2]\color{Emerald}\itshape,
	keywords=[3]{block, balance, blockhash, coinbase, difficulty, gaslimit, number, timestamp, msg, data, gas, sender, sig, now, tx, gasprice, origin, this},	% sepcial identifiers
	keywordstyle=[3]\color{Lavender}\itshape,
	keywords=[4]{assert, break, case, catch, continue, default, else, false, finally, for, if, require, revert, switch, throw, true, try, while, do}, % Control-flow
	keywordstyle=[4]\color{Cerulean}\bfseries,
	identifierstyle=\color{black},
	sensitive=false,
    keepspaces=true,
    columns=fullflexible,
    % literate=*
    %     {(}{{{\color{red}(}}}1
    %     {)}{{{\color{red})}}}1
    %     {+}{{{\color{red}+~}}}1
    %     {-}{{{\color{red}-~}}}1
    %     {:=}{{{\color{red}:=~}}}1
    %     {\{}{{{\color{red}\{}}}1
    %     {\}}{{{\color{red}\}}}}1
    %     {[}{{{\color{red}[}}}1
    %     {]}{{{\color{red}]}}}1
    %     {|}{{{\color{red}|}}}1
    %     {;}{{{\color{red};}}}1
    %     {*}{{{\color{red}*~}}}1
    %     {:}{{{\color{red}:~}}}1
    %     {>}{{{\color{red}>~}}}1
    %     {<}{{{\color{red}<~}}}1
    %     {<=>}{{{$\color{red}\Leftrightarrow~$}}}1
    %     {.}{{{\color{red}.~}}}1
    %     {!}{{{$\color{red}!~$}}}1
    %     {=}{{{$\color{red}=~$}}}1
    %     {exists }{{{$\color{red}\exists$}}}1
    %     {forall }{{{$\color{red}\forall$}}}1,
	comment=[l]{//},
	morecomment=[s]{/*}{*/},
    commentstyle=\color{CadetBlue}\textit,
	stringstyle=\color{ForestGreen},
	morestring=[b]',
	morestring=[b]",
    xleftmargin=5.0ex,
    extendedchars=true,
	basicstyle=\small\ttfamily,
	showstringspaces=false,
	showspaces=false,
	numbers=left,
	numberstyle=\footnotesize,
	numbersep=9pt,
	tabsize=2,
	breaklines=true,
	showtabs=false,
	captionpos=b
}


\lstset{
  frame=none,
  xleftmargin=2pt,
  stepnumber=1,
  numbers=left,
  numbersep=5pt,
  numberstyle=\ttfamily\tiny\color[gray]{0.3},
  belowcaptionskip=\bigskipamount,
  captionpos=b,
  escapeinside={*'}{'*},
  tabsize=2,
  emphstyle={\bf},
  commentstyle=\it,
  stringstyle=\mdseries\rmfamily,
  showspaces=false,
  keywordstyle=\bfseries\rmfamily,
  columns=flexible,
  basicstyle=\small\sffamily,
  showstringspaces=false,
}

\newcommand{\guard}{\bigg|\bigg|}
\newcommand{\newc}{\textbf{\texttt{new}}}
\newcommand{\everything}{\textbf{\texttt{everything}}}
\newcommand{\then}{\textbf{\texttt{then}}}
\newcommand{\this}{\textbf{\texttt{this}}}
\newcommand{\langName}{LANGUAGE-NAME\xspace}
\newcommand{\consumesarr}{\mathrel{\rotatebox[origin=c]{270}{${\looparrowright}$}}}
\newcommand{\consumes}[1]{\stackrel{#1}{\consumesarr}}
\newcommand{\sends}[1]{\stackrel{#1}{\to}}
\newcommand{\emitsarr}{\mathrel{\rotatebox[origin=c]{270}{${\looparrowleft}$}}}
\newcommand{\emits}[1]{\emitsarr #1}
\newcommand{\asset}{\textbf{\texttt{asset}}\xspace}
\newcommand{\fungible}{\textbf{\texttt{fungible}}\xspace}
\newcommand{\nonfungible}{\textbf{\texttt{nonfungible}}\xspace}
\newcommand{\addresst}{\textbf{\texttt{address}}\xspace}
\newcommand{\stringt}{\textbf{\texttt{string}}\xspace}
\newcommand{\setof}{\textbf{\texttt{set}}\xspace}
\newcommand{\optiont}{\textbf{\texttt{option}}\xspace}
\newcommand{\boolt}{\textbf{\texttt{bool}}\xspace}
\newcommand{\natt}{\textbf{\texttt{nat}}\xspace}
\newcommand{\byt}{\textbf{\texttt{by}}\xspace}
\newcommand{\stores}{\textbf{\texttt{stores}}\xspace}
\newcommand{\suchthat}{\textbf{s.t.}\xspace}
\newcommand{\returns}{\textbf{\texttt{returns}}\xspace}
\newcommand{\merge}{\rightsquigarrow}
\newcommand{\flowsto}{\rightsquigarrow}
\newcommand{\heldby}{\rightarrowtail}
\newcommand{\one}{\textbf{\texttt{one}}\xspace}
\newcommand{\some}{\textbf{\texttt{some}}\xspace}
\newcommand{\any}{\textbf{\texttt{any}}\xspace}
\newcommand{\map}{\textbf{\texttt{map}}\xspace}
\newcommand{\linking}{\textbf{\texttt{linking}}\xspace}
\newcommand{\transformer}{\textbf{\texttt{transformer}}\xspace}
\newcommand{\selects}{\textbf{\texttt{selects}}\xspace}
\newcommand{\demote}{\textbf{\texttt{demote}}\xspace}
\newcommand{\transforms}{\Rrightarrow}

\newcommand{\flowproves}{\mathbin{|\hskip -0.1em\scalebox{0.7}[1]{$\leadsto$}}}
\newcommand{\flowproveout}{\mathbin{\scalebox{0.7}[1]{$\leadsto$}\hskip -0.13em|}}


\lstset{
  frame=none,
  xleftmargin=2pt,
  stepnumber=1,
  numbers=left,
  numbersep=5pt,
  numberstyle=\ttfamily\tiny\color[gray]{0.3},
  belowcaptionskip=\bigskipamount,
  captionpos=b,
  escapeinside={*'}{'*},
  tabsize=2,
  emphstyle={\bf},
  commentstyle=\it,
  stringstyle=\mdseries\rmfamily,
  showspaces=false,
  keywordstyle=\bfseries\rmfamily,
  columns=flexible,
  basicstyle=\small\sffamily,
  showstringspaces=false,
}

\newcommand{\guard}{\bigg|\bigg|}
\newcommand{\newc}{\textbf{\texttt{new}}}
\newcommand{\everything}{\textbf{\texttt{everything}}}
\newcommand{\then}{\textbf{\texttt{then}}}
\newcommand{\this}{\textbf{\texttt{this}}}
\newcommand{\langName}{LANGUAGE-NAME\xspace}
\newcommand{\consumesarr}{\mathrel{\rotatebox[origin=c]{270}{${\looparrowright}$}}}
\newcommand{\consumes}[1]{\stackrel{#1}{\consumesarr}}
\newcommand{\sends}[1]{\stackrel{#1}{\to}}
\newcommand{\emitsarr}{\mathrel{\rotatebox[origin=c]{270}{${\looparrowleft}$}}}
\newcommand{\emits}[1]{\emitsarr #1}
\newcommand{\asset}{\textbf{\texttt{asset}}\xspace}
\newcommand{\fungible}{\textbf{\texttt{fungible}}\xspace}
\newcommand{\nonfungible}{\textbf{\texttt{nonfungible}}\xspace}
\newcommand{\addresst}{\textbf{\texttt{address}}\xspace}
\newcommand{\stringt}{\textbf{\texttt{string}}\xspace}
\newcommand{\setof}{\textbf{\texttt{set}}\xspace}
\newcommand{\optiont}{\textbf{\texttt{option}}\xspace}
\newcommand{\boolt}{\textbf{\texttt{bool}}\xspace}
\newcommand{\natt}{\textbf{\texttt{nat}}\xspace}
\newcommand{\byt}{\textbf{\texttt{by}}\xspace}
\newcommand{\stores}{\textbf{\texttt{stores}}\xspace}
\newcommand{\suchthat}{\textbf{s.t.}\xspace}
\newcommand{\returns}{\textbf{\texttt{returns}}\xspace}
\newcommand{\merge}{\rightsquigarrow}
\newcommand{\flowsto}{\rightsquigarrow}
\newcommand{\heldby}{\rightarrowtail}
\newcommand{\one}{\textbf{\texttt{one}}\xspace}
\newcommand{\some}{\textbf{\texttt{some}}\xspace}
\newcommand{\any}{\textbf{\texttt{any}}\xspace}
\newcommand{\map}{\textbf{\texttt{map}}\xspace}
\newcommand{\linking}{\textbf{\texttt{linking}}\xspace}
\newcommand{\transformer}{\textbf{\texttt{transformer}}\xspace}
\newcommand{\selects}{\textbf{\texttt{selects}}\xspace}
\newcommand{\demote}{\textbf{\texttt{demote}}\xspace}
\newcommand{\transforms}{\Rrightarrow}

\newcommand{\flowproves}{\mathbin{|\hskip -0.1em\scalebox{0.7}[1]{$\leadsto$}}}
\newcommand{\flowproveout}{\mathbin{\scalebox{0.7}[1]{$\leadsto$}\hskip -0.13em|}}



\begin{document}

\frame{\titlepage}

\begin{frame}{Introduction}
    \begin{itemize}
        \item Blockchains and smart contracts used/considered for:
            \begin{itemize}
                \item Supply chain management
                \item \emph{Token contracts}
                \item Voting
                \item Crowdfunding
                \item etc.
            \end{itemize}

        \item Smart contracts manage (and lose!) a lot of money: the DAO contract lost over \$40 million dollars due to a simple reentrancy bug

        \item Smart contracts cannot be patched after being deployed
    \end{itemize}
\end{frame}

\begin{frame}[fragile]{Psamathe}
    \begin{itemize}
        \item \langName (\langNamePronounce) is a new programming language we are designing for writing smart contracts
        \item It uses a new \emph{flow} abstraction, representing an atomic transfer operation
        \item It also has features to mark types with \emph{modifiers}, such as \flowinline{asset}, which combine with flows to make some classes of bugs impossible
    \end{itemize}

    \lstinputlisting[language=flow, xleftmargin=-0.5em, basicstyle=\scriptsize\ttfamily]{padl-21-examples/erc20-transfer.flow}
\end{frame}

\begin{frame}[fragile]{Advantages}
    \begin{itemize}
        \item Using the more declarative, \emph{flow-based} approach provides the following advantages over imperative state updates:
        \item \textbf{Static safety guarantees}: Each flow is guaranteed to preserve the total amount of assets (except for flows that explicitly consume or allocate assets)
        \item \textbf{Dynamic safety guarantees}: \langName automatically inserts dynamic checks of a flow's validity
            The \flowinline{unique} modifier, which restrict values to never be created more than once, is also checked dynamically.
        % \item \textbf{Data-flow tracking}: provide a clearer way of specifying how resources flow in the code itself, which may be less apparent using other approaches, especially in complicated contracts
            % Additionally, developers must explicitly mark when \assetTxt{}s are \emph{consumed}, and only assets marked as \flowinline{consumable} may be consumed.
        \item \textbf{Error messages}: When a flow fails, the \langName runtime provides automatic, descriptive error messages, such as
\begin{lstlisting}[numbers=none, basicstyle=\footnotesize\ttfamily, xleftmargin=-4.5em]
Cannot flow <amount> Token from account[<src>] to account[<dst>]:
    source only has <balance> Token.
\end{lstlisting}
    \end{itemize}
\end{frame}

\begin{frame}[fragile]{Type quantities, Modifiers}
    \begin{itemize}
        \item Type quantities: used to approximate the number of values in a variable (e.g., \flowinline{empty}, \flowinline{any}, \flowinline{one}, \flowinline{nonempty}), to track which variables can be dropped
            \begin{itemize}
                \item Can often be inferred (all type quantities int he previous example can be omitted)
            \end{itemize}

        \item Modifiers: Use to specify how variables of a type can be used (e.g., \flowinline{asset}, \flowinline{unique}, \flowinline{fungible})
    \end{itemize}

    \lstinputlisting[language=flow, xleftmargin=-0.5em, basicstyle=\scriptsize\ttfamily]{padl-21-examples/erc20-transfer-no-type-quants.flow}
\end{frame}

\begin{frame}[fragile]{Example: Type quantities, Modifiers}
    \begin{itemize}
        \item Example using modifiers and type quantities to guarantee additional correctness properties in the context of a lottery:
    \end{itemize}
\begin{lstlisting}[language=flow, xleftmargin=-0.2em, basicstyle=\scriptsize\ttfamily]
type TicketOwner is unique immutable address
type Ticket is consumable asset {
    owner : TicketOwner,
    guess : uint256
}
// ...
// End the lottery:
var winners : list Ticket <-- tickets[nonempty st isWinner(winNum, _)] |\label{line:lottery-filter}|
// Split among winners
winners --> payEach(jackpot / length(winners), _)
balance --> lotteryOwner.balance |\label{line:empty-lottery-balance}|
// Lottery is over, destroy losing tickets
tickets --> consume
\end{lstlisting}
\end{frame}

\begin{frame}[fragile]{Comparison with Solidity}
    \begin{itemize}
        \item Solidity is the most commonly-used smart contract language on the Ethereum blockchain
        \item It does not provide analogous support for managing assets
    \end{itemize}

\lstinputlisting[language=flow, xleftmargin=-0.5em, basicstyle=\scriptsize\ttfamily]{padl-21-examples/erc20-transfer-no-type-quants.flow}
\lstinputlisting[language=Solidity, xleftmargin=-0.5em, basicstyle=\scriptsize\ttfamily]{padl-21-examples/erc20-transfer.sol}
\end{frame}

\begin{frame}[fragile]{Implementation (in progress)}
    \begin{itemize}
        \item The $\mathbb{K}$ Framework is a framework for specifying programming languages
        \item An implementation in $\mathbb{K}$ of the formalization is in progress
        \item It is currently capable of running the ERC-20 example, as well as some other, more complicated programs (such as a voting contract)
    \end{itemize}

\begin{lstlisting}[xleftmargin=-0.5em, basicstyle=\scriptsize\ttfamily]
// ...
rule <k> N:Int => ListItem(selected(loc(!I), amount(N))) ... </k>
<storage> ... .Map => loc(!I) |-> resource(nat, N) ... </storage>
requires N >=Int 0
// ...
rule <k> flowVal(selected(L, K), loc(J))
     => resolveSelected(selected(L, K)) ~> subtractFrom(L) ~>
         asType(T) ~> combineWith(loc(J)) ... </k>
<storage> ... loc(J) |-> resource(T, _) ... </storage>
// ...
\end{lstlisting}
\end{frame}

% \begin{frame}{Formalization (in progress)}
%     \begin{itemize}
%         \item The $\mathbb{K}$ Framework is a framework for specifying programming languages
%     \end{itemize}
% \end{frame}

\begin{frame}{Future Work}
    \begin{itemize}
        \item Fully implement the \langName language, and prove its safety properties
        \item Study the benefits and costs of the language via case studies, performance evaluation, and the application of flows to other domains
    \end{itemize}
\end{frame}

\begin{frame}{Conclusion and Future Work}
    \begin{itemize}
        \item We have presented the \langName language for writing safer smart contracts
        \item \langName uses the new flow abstraction, \assetTxt{}s, and modifiers to provide safety guarantees for smart contracts
        \item We showed two examples of smart contracts in both Solidity and \langName, showing that \langName is capable of expressing common smart contract functionality in a concise manner, while retaining key safety properties.
    \end{itemize}
\end{frame}

\begin{frame}{Questions?}
    \begin{itemize}
        \item Questions?
    \end{itemize}
\end{frame}

\begin{frame}[fragile]{Other Random Stuff}
    \begin{itemize}
        \item Next meeting: probably type theory
        \item SPLASH conference is happening in about a month: talks will probably be streamed on YouTube
            \begin{itemize}
                \item Also, lots of other talks archived at: \url{https://www.youtube.com/channel/UCwG9512Wm7jSS6Iqshz4Dpg}
            \end{itemize}
        \item \#PLTalk stream: \url{https://www.twitch.tv/jeanqasaur}
        \item New proof of a better traveling salesman approximation algorithm: \url{https://arxiv.org/abs/2007.01409}
    \end{itemize}
\end{frame}

\end{document}

