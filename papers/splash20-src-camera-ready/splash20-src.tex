\documentclass[dvipsnames, sigplan, screen]{acmart}

\usepackage{microtype}
\usepackage{listings}
\usepackage{hyperref}
\usepackage{dblfloatfix}
\usepackage{tikz-cd}
% \usepackage{amssymb}
\usepackage{amsthm}
\usepackage{bm}
\usepackage{listings}
\usepackage{mathtools}
\usepackage{mathpartir}
\usepackage{float}
\usepackage{subcaption}
\usepackage{xcolor}
\usepackage[safe]{tipa}
\usepackage[inline]{enumitem}

%%% The following is specific to SPLASH Companion '20-SRC and the paper
%%% 'Psamathe: A DSL for Safe Blockchain Assets'
%%% by Reed Oei.
%%%
\setcopyright{acmlicensed}
\acmPrice{15.00}
\acmDOI{10.1145/3426430.3428131}
\acmYear{2020}
\copyrightyear{2020}
\acmSubmissionID{splashcomp20src-p17-p}
\acmISBN{978-1-4503-8179-6/20/11}
\acmConference[SPLASH Companion '20]{Companion Proceedings of the 2020 ACM SIGPLAN International Conference on Systems, Programming, Languages, and Applications: Software for Humanity}{November 15--20, 2020}{Virtual, USA}
\acmBooktitle{Companion Proceedings of the 2020 ACM SIGPLAN International Conference on Systems, Programming, Languages, and Applications: Software for Humanity (SPLASH Companion '20), November 15--20, 2020, Virtual, USA}


\usepackage{xcolor}
\usepackage{forloop}
\usepackage{ifthen}
\usepackage{xifthen}
\usepackage{calc}
\usepackage{xspace}
\usepackage{mathrsfs}

\newcounter{i}
\newcounter{j}
\newcounter{n1}

\newcommand{\concat}{%
  \mathbin{\raisebox{0.5ex}{\scalebox{.7}{$\frown$}}}%
}
\newcommand{\bnfdef}{\ensuremath{\Coloneqq}}
\newcommand{\bnfalt}{\ensuremath{\mid}\xspace}

\newcommand{\colvec}[1]{\ensuremath{\begin{pmatrix} #1 \end{pmatrix}}}

\newcommand{\imp}{\Rightarrow}
\renewcommand{\epsilon}{\varepsilon}

% From: https://en.wikibooks.org/wiki/LaTeX/Source_Code_Listings#Automating_file_inclusion
\newcommand{\includecode}[2][c]{\lstinputlisting[language=#1, caption=#2, basicstyle=\normalsize\ttfamily]{#2}}

\newcommand{\bigstep}[2]{#1 \Downarrow #2}

\newcommand{\eq}[1]{\begin{align*} #1 \end{align*}}

\newcommand{\Hom}{\ensuremath{\text{Hom}}}
\newcommand{\Tor}{\ensuremath{\text{Tor}}}
\newcommand{\ch}{\ensuremath{\text{ch}}}

\newcommand{\spanvec}{\ensuremath{\text{span}}}
\newcommand{\trace}[1]{\ensuremath{\text{Tr}\parens{#1}}}

\newcommand{\includeplot}[2][0.3]{%
\begin{figure}[H]
    \scalebox{#1}{\sageplot{#2}}
\end{figure}}

\newcommand{\partialdiff}[1]{\frac{\partial}{\partial #1}}
\newcommand{\diff}[1]{\frac{\text{d}}{\text{d} #1}}
\newcommand{\partialfunc}{\rightharpoonup}

\newcommand{\derive}[3][]{%
    \ifthenelse{\isempty{#1}}{%
        \frac{\text{d} #2}{\text{d} #3}%
    }{%
        \frac{\text{d}^{#1} #2}{\text{d} #3^{#1}}%
    }%
}

\newcommand{\im}[1]{\ensuremath{\text{im} {#1}}}

\newcommand{\true}{\textbf{\textsf{T}}}
\newcommand{\false}{\textbf{\textsf{F}}}

\newcommand{\join}{\bigvee}
\newcommand{\meet}{\bigwedge}

\newcommand{\piclass}[2]{\ensuremath{\bm{\Pi}^{#1}_{#2}}}
\newcommand{\sigmaclass}[2]{\ensuremath{\bm{\Sigma}^{#1}_{#2}}}
\newcommand{\deltaclass}[2]{\ensuremath{\bm{\Delta}^{#1}_{#2}}}
\newcommand{\gammaclass}{\ensuremath{\bm{\Gamma}}}
\newcommand{\borel}[1]{\ensuremath{\mathcal{B}\parens{#1}}}

\newcommand{\saturate}[2]{\squares{#1}_{#2}}

\newcommand{\abs}[1]{\ensuremath{\left| #1 \right|}}
\newcommand{\iso}{\ensuremath{\cong}}
\newcommand{\symdif}{\ensuremath{\bigtriangleup}}
\newcommand{\dual}{\ensuremath{\neg}}
\newcommand{\forces}{\ensuremath{\Vdash}}
\newcommand{\semiprod}[3][]{\ensuremath{#2 \rtimes_{#1} #3}}
\newcommand{\units}[1]{\ensuremath{#1^{\times}}}
\newcommand{\gl}[2]{\ensuremath{\text{GL}\parens{#1, #2}}}

\newcommand{\proj}[1]{\text{proj}_{#1}}

\newcommand{\lcm}{\text{lcm}}

%\newcommand{\reed}[1]{\relax}
%\newcommand{\Fix}[1]{\relax}
\newcommand{\reed}[1]{{\color{magenta}\bfseries [#1]}}
\newcommand{\Fix}[1]{{\color{red}\bfseries [#1]}}
\newcommand{\Comment}[1]{}
\newcommand{\Space}[1]{}
\newcommand{\Num}[1]{#1}

\newcommand{\e}{\ensuremath{\text{e}}}

\newcommand{\injects}{\ensuremath{\hookrightarrow}}
\newcommand{\into}{\injects}
\newcommand{\surjects}{\ensuremath{\twoheadrightarrow}}
\newcommand{\onto}{\surjects}

\newcommand{\actson}{\ensuremath{\curvearrowright}}

\newcommand{\interior}[1]{\ensuremath{\text{int}\parens{#1}}}
\newcommand{\closure}[1]{\ensuremath{\overline{#1}}}
\newcommand{\compl}[1]{\ensuremath{{#1}^c}}
\ifdefined\boundary\else\newcommand{\boundary}{\ensuremath{\partial}}\fi

\newcommand{\proves}{\vdash}
\newcommand{\satisfies}{\models}

\DeclarePairedDelimiter\ceil{\lceil}{\rceil}
\DeclarePairedDelimiter\floor{\lfloor}{\rfloor}

\ifdefined\theorem\else\newtheorem{theorem}{Theorem}\fi
\ifdefined\lemma\else\newtheorem{lemma}{Lemma}\fi
\ifdefined\proposition\else\newtheorem{proposition}{Proposition}\fi
\ifdefined\claim\else\newtheorem{claim}{Claim}\fi
\ifdefined\corollary\else\newtheorem{corollary}{Corollary}\fi
\ifdefined\definition\else\newtheorem{definition}{Definition}\fi
\ifdefined\example\else\newtheorem{example}{Example}\fi
\ifdefined\remark\else\newtheorem{remark}{Remark}\fi
\ifdefined\notation\else\newtheorem{notation}{Notation}\fi

\newcommand{\nat}{\ensuremath{\mathbb{N}}}
\newcommand{\N}{\ensuremath{\mathbb{N}}}
\newcommand{\real}{\ensuremath{\mathbb{R}}}
\newcommand{\reals}{\ensuremath{\mathbb{R}}}
\newcommand{\R}{\ensuremath{\mathbb{R}}}
\newcommand{\Z}{\ensuremath{\mathbb{Z}}}
\newcommand{\Zmod}[1]{\ensuremath{\Z/#1\Z}}
\newcommand{\integer}{\ensuremath{\mathbb{Z}}}
\newcommand{\integers}{\ensuremath{\mathbb{Z}}}
\newcommand{\rational}{\ensuremath{\mathbb{Q}}}
\newcommand{\rationals}{\ensuremath{\mathbb{Q}}}
\newcommand{\rat}{\ensuremath{\mathbb{Q}}}
\newcommand{\Q}{\ensuremath{\mathbb{Q}}}
\newcommand{\complex}{\ensuremath{\mathbb{C}}}
% \newcommand{\C}{\ensuremath{\mathbb{C}}}

\newcommand{\nonzero}[1]{\ensuremath{#1_{\neq 0}}}
\newcommand{\greaterset}[2]{\ensuremath{#1_{> #2}}}
\newcommand{\lesserset}[2]{\ensuremath{#1_{< #2}}}
\newcommand{\subgroup}{\leq}
\newcommand{\normalsubgroup}{\trianglelefteq}
\newcommand{\id}[1]{\ensuremath{\bm{1}_{#1}}}
\newcommand{\indicator}[1]{\ensuremath{\mathbbm{1}_{#1}}}

\newcommand{\notationNonZero}[1][X]{Let $\nonzero{#1}$ denote the set of $x \in #1$ so that $x \neq 0$.\xspace}
\newcommand{\notationGreaterSet}[1][X]{Let $\greaterset{#1}{y}$ denote the set of $x \in #1$ so that $x > y$.\xspace}
\newcommand{\notationLesserSet}[1][X]{Let $\lesserset{#1}{y}$ denote the set of $x \in X$ so that $x < y$.\xspace}
\newcommand{\notationSubgroup}[1][H]{Let $\subgroup{#1}{G}$ denote that $#1$ is a subgroup of $G$.\xspace}
\newcommand{\notationNormalSubgroup}[1][N]{Let $\normalsubgroup{#1}{G}$ denote that $#1$ is a normal subgroup of $G$.\xspace}
\newcommand{\notationIdentity}[1][X]{Let $\id{#1}$ denote the identity function on $#1$.\xspace}

% \Sym{X} is the set of bijections from X to X
\newcommand{\Sym}[1]{\ensuremath{\text{Sym}\parens{#1}}}
\newcommand{\Inn}[1]{\ensuremath{\text{Inn}\parens{#1}}}
\newcommand{\Out}[1]{\ensuremath{\text{Out}\parens{#1}}}
\newcommand{\Aut}[1]{\ensuremath{\text{Aut}\parens{#1}}}
\newcommand{\notationSym}[1][X]{Let $\Sym{X}$ be the group of bijections of $X$.\xspace}

\newcommand{\toplim}{\ensuremath{\text{T}\lim}}
\newcommand{\toplimup}{\ensuremath{\overline{\toplim}}}
\newcommand{\toplimlow}{\ensuremath{\underline{\toplim}}}

\newcommand{\emptysingleton}{\ensuremath{\curlys{\emptyset}}}
\newcommand{\withoutempty}{\ensuremath{\setminus \emptysingleton}}

\newcommand{\isnat}[1]{\ensuremath{#1 \in \nat}}
\newcommand{\seq}[3]{\ensuremath{{\left( #1_{#2} \right)}_{#2 \in #3}}}
\newcommand{\setint}[3]{\ensuremath{\bigcap_{#2 \in #3} #1_{#2}}}
\newcommand{\natinti}[2]{\ensuremath{\setint{#1}{#2}{\nat}}}
\newcommand{\natint}[1]{\ensuremath{\natinti{#1}{n}}}
\newcommand{\natseqi}[2]{\ensuremath{\seq{#1}{#2}{\nat}}}
\newcommand{\natseq}[1]{\ensuremath{\natseqi{#1}{n}}}

\newcommand{\diam}[1]{\ensuremath{\text{diam} \left( #1 \right)}}

\newcommand{\eventually}{\ensuremath{\forall^{\infty}}}
\newcommand{\infinitelymany}{\ensuremath{\exists^{\infty}}}

\newcommand{\power}[1]{\ensuremath{\mathcal{P} \left( #1 \right)}}
\newcommand{\card}[1]{\ensuremath{\left| #1 \right|}}

\newcommand{\brackets}[3]{\ensuremath{{\left#1 {#3} \right#2}}}
\newcommand{\parens}[1]{\brackets{(}{)}{#1}}
\newcommand{\angles}[1]{\brackets{<}{>}{#1}}
\newcommand{\curlys}[1]{\brackets{\{}{\}}{#1}}
\newcommand{\squares}[1]{\brackets{[}{]}{#1}}
\newcommand{\parensum}[3]{\ensuremath{\parens{\sum_{#1}^{#2} {#3}}}}

\newcommand{\poly}[2]{\ensuremath{{#1}\left[ #2 \right]}}

\newcommand{\tand}{\ensuremath{~\text{and}~}}
\newcommand{\tor}{\ensuremath{~\text{or}~}}
\newcommand{\twhere}{\ensuremath{~\text{where}~}}
\newcommand{\tif}{\ensuremath{~\text{if}~}}
\newcommand{\tsuchthat}{\ensuremath{~\text{s.t.}~}}
\newcommand{\owise}{\ensuremath{~\text{otherwise}~}}

\newcommand{\ff}[1]{\ensuremath{\mathbb{F}_{#1}}}

\newcommand{\beatree}[1]{\ensuremath{\subseteq {#1}^{\leq \nat}}}

\newcommand{\metricspace}[2]{\ensuremath{\parens{{#1},{#2}}}}

\newcommand{\setbuild}[2]{\ensuremath{\left\{ {#1} : {#2} \right\}}}

\newcommand{\openball}[3][]{\ensuremath{B_{#1}\parens{{#2},{#3}}}}
\newcommand{\closedball}[3][]{\ensuremath{\overline{B}_{#1}\parens{{#2},{#3}}}}

\newcommand{\powerset}[1]{\ensuremath{\mathcal{P}\parens{#1}}}
\newcommand{\powersetfin}[1]{\ensuremath{\mathcal{P}_{\text{fin}}\parens{#1}}}

\newcommand{\subgroupgen}[1]{\ensuremath{{\left< {#1} \right>}}}

\newcommand{\diagmatrix}[2]{%
    \setcounter{n1}{#1 - 1}
    \begin{pmatrix}
        \forloop{i}{0}{\value{i} < #1}{%
            \forloop{j}{0}{\value{j} < #1}{%
                \ifthenelse{\equal{\value{i}}{\value{j}}}{#2}{0}
                \ifthenelse{\value{j} < \value{n1}}{&}{}
            }
            \ifthenelse{\value{i} < \value{n1}}{\\}{}
        }
    \end{pmatrix}
}

\newcommand{\idmatrix}[1]{\diagmatrix{#1}{1}}

\newcommand{\twobytwo}[4]{\ensuremath{\begin{pmatrix} #1 & #2 \\ #3 & #4 \end{pmatrix}}}

\newcommand{\after}{\circ}


\newcommand{\finitecommabody}[1]{}
\newcommand{\countablecommabody}[1]{}

% #1 - The thing to repeat
% #2 - The operation to use to combine
% #3 - The number of times to repeat (used for comment, \cdots is used for the actual body)
\newcommand{\ntimes}[3]{\overbrace{#1 #2 \cdots #2 #1}^{#3 \text{ times}}}

\newcommand{\finiteopfromone}[3]{%
    \renewcommand{\finitecommabody}[1]{#1}%
    \ensuremath{\finitecommabody{1} #2 \finitecommabody{2} #2 \cdots #2 \finitecommabody{#3}}}
\newcommand{\finiteop}[3]{%
    \renewcommand{\finitecommabody}[1]{#1}%
    \ensuremath{\finitecommabody{0} #2 \finitecommabody{1} #2 \cdots #2 \finitecommabody{#3 - 1}}}

% #1 - The body expression (e.g., x_{#1}, where #1 is where the index will go)
% #2 - The operation to use
% #3 - The index of the element to skip
% #4 - The end index
\newcommand{\finiteopskipfromone}[4]{%
    \renewcommand{\finitecommabody}[1]{#1}%
    \ensuremath{\finitecommabody{1} #2 \cdots #2 \finitecommabody{#3 - 1} #2 \finitecommabody{#3 + 1} #2 \cdots #2 \finitecommabody{#3}}}
\newcommand{\finiteopskip}[4]{%
    \renewcommand{\finitecommabody}[1]{#1}%
    \ensuremath{\finitecommabody{0} #2 \cdots #2 \finitecommabody{#3 - 1} #2 \finitecommabody{#3 + 1} #2 \cdots #2 \finitecommabody{#3 - 1}}}

\newcommand{\finitecommafromone}[2]{%
    \renewcommand{\finitecommabody}[1]{#1}%
    \ensuremath{\finitecommabody{1}, \finitecommabody{2}, \ldots, \finitecommabody{#2}}}
\newcommand{\finitesetfromone}[2]{\ensuremath{\curlys{\finitecommafromone{#1}{#2}}}}

\newcommand{\finitecomma}[2]{%
    \renewcommand{\finitecommabody}[1]{#1}%
    \ensuremath{\finitecommabody{0}, \finitecommabody{1}, \ldots, \finitecommabody{#2 - 1}}}
\newcommand{\finiteset}[2]{\ensuremath{\curlys{\finitecomma{#1}{#2}}}}

\newcommand{\countablecommafromone}[1]{\ensuremath{%
    \renewcommand{\countablecommabody}[1]{#1}%
    \ensuremath{\countablecommabody{1}, \countablecommabody{2}, \ldots}}}
\newcommand{\ctblcommafromone}[1]{\ensuremath{\countablecommafromone{#1}}}
\newcommand{\countablesetfromone}[1]{\ensuremath{\curlys{\countablecommafromone{#1}}}}
\newcommand{\ctblsetfromone}[1]{\ensuremath{\curlys{\countablecommafromone{#1}}}}

\newcommand{\countablecomma}[1]{\ensuremath{%
    \renewcommand{\countablecommabody}[1]{#1}%
    \ensuremath{\countablecommabody{0}, \countablecommabody{1}, \ldots}}}
\newcommand{\ctblcomma}[1]{\ensuremath{\countablecomma{#1}}}
\newcommand{\countableset}[1]{\ensuremath{\curlys{\countablecomma{#1}}}}
\newcommand{\ctblset}[1]{\ensuremath{\curlys{\countablecomma{#1}}}}

\newcommand{\woutlog}{without loss of generality\xspace}
\newcommand{\Woutlog}{Without loss of generality\xspace}

\newcommand{\actOn}[2]{\ensuremath{#1 \cdot #2}}
\newcommand{\kernel}[1]{\ensuremath{\text{ker}\parens{#1}}}

\newcommand{\deffuncparam}[6]{%
    \expandafter\newcommand\csname domain#1\endcsname{\ensuremath{#3}}%
    \expandafter\newcommand\csname codomain#1\endcsname{\ensuremath{#5}}%
    \expandafter\newcommand\csname decl#1\endcsname[1]{\ensuremath{#2_{##1} : #3 #4 #5}}%
    \expandafter\newcommand\csname funcname#1\endcsname[1]{\ensuremath{#2_{##1}}}%
    \expandafter\newcommand\csname body#1\endcsname[2]{\ensuremath{#6}}%
    \expandafter\newcommand\csname call#1\endcsname[2]{\ensuremath{#2_{##1} \parens{##2}}}%
    \expandafter\newcommand\csname callbody#1\endcsname[2]{\ensuremath{\expandafter\csname call#1\endcsname{##1}{##2} = \expandafter\csname body#1\endcsname{##1}{##2}}}%
    \expandafter\newcommand\csname map#1\endcsname[2]{\ensuremath{##2 \mapsto \expandafter\csname body#1\endcsname{##1}{##2} }}%
    \expandafter\newcommand\csname inversecall#1\endcsname[2]{\ensuremath{#2_{##1} \parens{##2}^{-1}}}%
    \expandafter\newcommand\csname image#1\endcsname[1]{\ensuremath{\text{im}\parens{#2_{##1}}}}%
}

\newcommand{\defbinfunc}[6]{%
    \expandafter\newcommand\csname domain#1\endcsname{\ensuremath{#3}}%
    \expandafter\newcommand\csname codomain#1\endcsname{\ensuremath{#5}}%
    \expandafter\newcommand\csname decl#1\endcsname{\ensuremath{#2 : #3 #4 #5}}%
    \expandafter\newcommand\csname funcname#1\endcsname{\ensuremath{#2}}%
    \expandafter\newcommand\csname body#1\endcsname[2]{#6}%
    \expandafter\newcommand\csname call#1\endcsname[2]{\ensuremath{#2 \parens{##1, ##2}}}%
    \expandafter\newcommand\csname callpair#1\endcsname[1]{\ensuremath{#2 \parens{##1}}}%
    \expandafter\newcommand\csname callbody#1\endcsname[2]{\ensuremath{\expandafter\csname call#1\endcsname{##1}{##2} = \expandafter\csname body#1\endcsname{##1}{##2}}}%
    \expandafter\newcommand\csname map#1\endcsname[2]{\ensuremath{\parens{##1, ##2} \mapsto \expandafter\csname body#1\endcsname{##1}{##2} }}%
}

\newcommand{\defbinop}[6]{%
    \expandafter\newcommand\csname domain#1\endcsname{\ensuremath{#3}}%
    \expandafter\newcommand\csname codomain#1\endcsname{\ensuremath{#5}}%
    \expandafter\newcommand\csname decl#1\endcsname{\ensuremath{#2 : #3 #4 #5}}%
    \expandafter\newcommand\csname funcname#1\endcsname{\ensuremath{#2}}%
    \expandafter\newcommand\csname body#1\endcsname[2]{#6}%
    \expandafter\newcommand\csname call#1\endcsname[2]{\ensuremath{##1 #2 ##2}}%
    \expandafter\newcommand\csname callbody#1\endcsname[2]{\ensuremath{\expandafter\csname call#1\endcsname{##1}{##2} = \expandafter\csname body#1\endcsname{##1}{##2}}}%
    \expandafter\newcommand\csname map#1\endcsname[2]{\ensuremath{\parens{##1, ##2} \mapsto \expandafter\csname body#1\endcsname{##1}{##2} }}%
}

\newcommand{\deffunc}[6]{%
    \expandafter\newcommand\csname domain#1\endcsname{\ensuremath{#3}}%
    \expandafter\newcommand\csname codomain#1\endcsname{\ensuremath{#5}}%
    \expandafter\newcommand\csname decl#1\endcsname{\ensuremath{#2 : #3 #4 #5}}%
    \expandafter\newcommand\csname funcname#1\endcsname{\ensuremath{#2}}%
    \expandafter\newcommand\csname body#1\endcsname[1]{#6}%
    \expandafter\newcommand\csname call#1\endcsname[1]{\ensuremath{#2 \parens{##1}}}%
    \expandafter\newcommand\csname callbody#1\endcsname[1]{\ensuremath{\expandafter\csname call#1\endcsname{##1} = \expandafter\csname body#1\endcsname{##1}}}%
    \expandafter\newcommand\csname map#1\endcsname[1]{\ensuremath{##1 \mapsto \expandafter\csname body#1\endcsname{##1} }}%
    \expandafter\newcommand\csname invcall#1\endcsname[1]{\ensuremath{#2^{-1}\parens{##1}}}%
    \expandafter\newcommand\csname image#1\endcsname{\ensuremath{\text{im}\parens{#2}}}%
}

\newcommand{\defproj}[3]{%
    \defbinfunc{#1#2}{\pi_{#2}}{#2 \times #3}{\to}{#2}{##1}%
    \defbinfunc{#1#3}{\pi_{#3}}{#2 \times #3}{\to}{#3}{##2}%
}

% Footnoteable proofs for useful stuff
\newcommand{\notationEventually}{Define the notation $\eventually n, \varphi(n)$ to mean $\exists N \in \nat \tsuchthat \forall n \geq N, \varphi(n)$, where $\varphi$ is some formula.\xspace}
\newcommand{\eventuallyCombinesPrf}{%
    Suppose that $\eventually n, \varphi(n)$ and $\eventually n, \psi(n)$, where $\varphi$ and $\psi$ are two formulas.
    Then there is some $N_1$ so that $\varphi(n)$ holds for all $n \geq N_1$, and some $N_2$ so that $\psi(n)$ holds for all $n \geq N_2$.
    Then define $N = \max(N_1, N_2)$, and suppose $n \geq N$.
    Then $\varphi(n)$ holds because $n \geq N_1$ and $\psi(n)$ holds because $n \geq N_2$, so $\eventually n, \varphi(n) \wedge \psi(n)$.\xspace}


\newcommand{\irrationalDensePrf}{For any rational $q$, take $q + \frac{\sqrt{2}}{n}$, which is irrational. However, because we can make $\frac{1}{n}$ smaller than any $r > 0$, we can make some irrational that will be in $B(q,r)$ for any $r > 0$.\xspace}


% Copyright 2017 Sergei Tikhomirov, MIT License
% https://github.com/s-tikhomirov/solidity-latex-highlighting/

\usepackage{listings, xcolor}

\definecolor{verylightgray}{rgb}{.97,.97,.97}

\lstdefinelanguage{Solidity}{
	keywords=[1]{anonymous, assembly, balance, call, callcode, class, constant, constructor, contract, debugger, delegatecall, delete, emit, event, experimental, export, external, function, gas, implements, import, in, indexed, instanceof, interface, internal, is, length, library, log0, log1, log2, log3, log4, memory, modifier, new, payable, pragma, private, protected, public, pure, push, return, returns, selfdestruct, send, solidity, storage, struct, suicide, super, then, throw, typeof, using, view, with, addmod, ecrecover, keccak256, mulmod, ripemd160, sha256, sha3}, % generic keywords including crypto operations
	keywordstyle=[1]\color{Rhodamine}\bfseries,
	keywords=[2]{address, bool, byte, bytes, bytes1, bytes2, bytes3, bytes4, bytes5, bytes6, bytes7, bytes8, bytes9, bytes10, bytes11, bytes12, bytes13, bytes14, bytes15, bytes16, bytes17, bytes18, bytes19, bytes20, bytes21, bytes22, bytes23, bytes24, bytes25, bytes26, bytes27, bytes28, bytes29, bytes30, bytes31, bytes32, enum, int, int8, int16, int24, int32, int40, int48, int56, int64, int72, int80, int88, int96, int104, int112, int120, int128, int136, int144, int152, int160, int168, int176, int184, int192, int200, int208, int216, int224, int232, int240, int248, int256, mapping, string, uint, uint8, uint16, uint24, uint32, uint40, uint48, uint56, uint64, uint72, uint80, uint88, uint96, uint104, uint112, uint120, uint128, uint136, uint144, uint152, uint160, uint168, uint176, uint184, uint192, uint200, uint208, uint216, uint224, uint232, uint240, uint248, uint256, var, void, ether, finney, szabo, wei, days, hours, minutes, seconds, weeks, years},	% types; money and time units
	keywordstyle=[2]\color{Emerald}\itshape,
	keywords=[3]{block, balance, blockhash, coinbase, difficulty, gaslimit, number, timestamp, msg, data, gas, sender, sig, now, tx, gasprice, origin, this},	% sepcial identifiers
	keywordstyle=[3]\color{Lavender}\itshape,
	keywords=[4]{assert, break, case, catch, continue, default, else, false, finally, for, if, require, revert, switch, throw, true, try, while, do}, % Control-flow
	keywordstyle=[4]\color{Cerulean}\bfseries,
	identifierstyle=\color{black},
	sensitive=false,
    keepspaces=true,
    columns=fullflexible,
    % literate=*
    %     {(}{{{\color{red}(}}}1
    %     {)}{{{\color{red})}}}1
    %     {+}{{{\color{red}+~}}}1
    %     {-}{{{\color{red}-~}}}1
    %     {:=}{{{\color{red}:=~}}}1
    %     {\{}{{{\color{red}\{}}}1
    %     {\}}{{{\color{red}\}}}}1
    %     {[}{{{\color{red}[}}}1
    %     {]}{{{\color{red}]}}}1
    %     {|}{{{\color{red}|}}}1
    %     {;}{{{\color{red};}}}1
    %     {*}{{{\color{red}*~}}}1
    %     {:}{{{\color{red}:~}}}1
    %     {>}{{{\color{red}>~}}}1
    %     {<}{{{\color{red}<~}}}1
    %     {<=>}{{{$\color{red}\Leftrightarrow~$}}}1
    %     {.}{{{\color{red}.~}}}1
    %     {!}{{{$\color{red}!~$}}}1
    %     {=}{{{$\color{red}=~$}}}1
    %     {exists }{{{$\color{red}\exists$}}}1
    %     {forall }{{{$\color{red}\forall$}}}1,
	comment=[l]{//},
	morecomment=[s]{/*}{*/},
    commentstyle=\color{CadetBlue}\textit,
	stringstyle=\color{ForestGreen},
	morestring=[b]',
	morestring=[b]",
    xleftmargin=5.0ex,
    extendedchars=true,
	basicstyle=\small\ttfamily,
	showstringspaces=false,
	showspaces=false,
	numbers=left,
	numberstyle=\footnotesize,
	numbersep=9pt,
	tabsize=2,
	breaklines=true,
	showtabs=false,
	captionpos=b
}


\definecolor{verylightgray}{rgb}{.97,.97,.97}

\lstdefinelanguage{flow}{
    keywords=[1]{var,interface,where,contract,transaction,view,transformer,type,is,returns},
    keywordstyle=[1]\color{Rhodamine}\bfseries,
    keywords=[2]{false,true,sometimes,fungible,asset,unique,immutable,consumable},
    keywordstyle=[2]\color{DarkOrchid}\bfseries,
    keywords=[3]{map,linking,set,list,address,uint256,string,bytes,nat,bool},
    keywordstyle=[3]\color{Emerald}\itshape,
    keywords=[4]{nonempty,any,every,empty,exactly,one,msg,this,in,new,or,and,not,total,consume},
    keywordstyle=[4]\color{Lavender}\itshape,
    keywords=[5]{if,else,only,when,try,catch,such,that},
    keywordstyle=[5]\color{Cerulean}\bfseries,
    keepspaces=true,
	% backgroundcolor=\color{verylightgray},
    % columns=fullflexible,
    % literate=*
    %     {(}{{{\color{red}(}}}1
    %     {)}{{{\color{red})}}}1
    %     {+}{{{\color{red}+~}}}1
    %     {-}{{{\color{red}-~}}}1
    %     {:=}{{{\color{red}:=~}}}1
    %     {\{}{{{\color{red}\{}}}1
    %     {\}}{{{\color{red}\}}}}1
    %     {[}{{{\color{red}[}}}1
    %     {]}{{{\color{red}]}}}1
    %     {*}{{{\color{red}*~}}}1
    %     {:}{{{\color{red}:~}}}1
    %     {>}{{{\color{red}>~}}}1
    %     {<}{{{\color{red}<~}}}1
    %     {<=>}{{{$\color{red}\Leftrightarrow~$}}}1
    %     {.}{{{\color{red}.~}}}1
    %     {=}{{{$\color{red}=~$}}}1
    %     {exists }{{{$\color{red}\exists$}}}1
    %     {forall }{{{$\color{red}\forall$}}}1,
    sensitive=false, % keywords are not case-sensitive
    morecomment=[l]{//}, % l is for line comment
    morecomment=[s]{/*}{*/}, % s is for start and end delimiter
    commentstyle=\color{CadetBlue}\textit,
    morestring=[b]", % defines that strings are enclosed in double quotes
    stringstyle=\color{ForestGreen}, % string literal style
    showstringspaces=false,
    xleftmargin=5.0ex,
    mathescape=true,
    escapechar=|,
    basicstyle=\small\ttfamily
}

\lstnewenvironment{flow}
{%
    \lstset{language=flow, basicstyle=\small\ttfamily, mathescape=true}%
}
{
}

\newcommand{\flowinline}[1]{\lstinline[language=flow,basicstyle=\small\ttfamily,mathescape]{#1}}


\lstset{
  frame=none,
  xleftmargin=2pt,
  stepnumber=1,
  numbers=left,
  numbersep=5pt,
  numberstyle=\ttfamily\tiny\color[gray]{0.3},
  belowcaptionskip=\bigskipamount,
  captionpos=b,
  escapeinside={*'}{'*},
  tabsize=2,
  emphstyle={\bf},
  commentstyle=\it,
  stringstyle=\mdseries\rmfamily,
  showspaces=false,
  keywordstyle=\bfseries\rmfamily,
  columns=flexible,
  basicstyle=\small\sffamily,
  showstringspaces=false,
}

\newcommand{\guard}{\bigg|\bigg|}
\newcommand{\newc}{\textbf{\texttt{new}}}
\newcommand{\everything}{\textbf{\texttt{everything}}}
\newcommand{\then}{\textbf{\texttt{then}}}
\newcommand{\this}{\textbf{\texttt{this}}}
\newcommand{\langName}{LANGUAGE-NAME\xspace}
\newcommand{\consumesarr}{\mathrel{\rotatebox[origin=c]{270}{${\looparrowright}$}}}
\newcommand{\consumes}[1]{\stackrel{#1}{\consumesarr}}
\newcommand{\sends}[1]{\stackrel{#1}{\to}}
\newcommand{\emitsarr}{\mathrel{\rotatebox[origin=c]{270}{${\looparrowleft}$}}}
\newcommand{\emits}[1]{\emitsarr #1}
\newcommand{\asset}{\textbf{\texttt{asset}}\xspace}
\newcommand{\fungible}{\textbf{\texttt{fungible}}\xspace}
\newcommand{\nonfungible}{\textbf{\texttt{nonfungible}}\xspace}
\newcommand{\addresst}{\textbf{\texttt{address}}\xspace}
\newcommand{\stringt}{\textbf{\texttt{string}}\xspace}
\newcommand{\setof}{\textbf{\texttt{set}}\xspace}
\newcommand{\optiont}{\textbf{\texttt{option}}\xspace}
\newcommand{\boolt}{\textbf{\texttt{bool}}\xspace}
\newcommand{\natt}{\textbf{\texttt{nat}}\xspace}
\newcommand{\byt}{\textbf{\texttt{by}}\xspace}
\newcommand{\stores}{\textbf{\texttt{stores}}\xspace}
\newcommand{\suchthat}{\textbf{s.t.}\xspace}
\newcommand{\returns}{\textbf{\texttt{returns}}\xspace}
\newcommand{\merge}{\rightsquigarrow}
\newcommand{\flowsto}{\rightsquigarrow}
\newcommand{\heldby}{\rightarrowtail}
\newcommand{\one}{\textbf{\texttt{one}}\xspace}
\newcommand{\some}{\textbf{\texttt{some}}\xspace}
\newcommand{\any}{\textbf{\texttt{any}}\xspace}
\newcommand{\map}{\textbf{\texttt{map}}\xspace}
\newcommand{\linking}{\textbf{\texttt{linking}}\xspace}
\newcommand{\transformer}{\textbf{\texttt{transformer}}\xspace}
\newcommand{\selects}{\textbf{\texttt{selects}}\xspace}
\newcommand{\demote}{\textbf{\texttt{demote}}\xspace}
\newcommand{\transforms}{\Rrightarrow}

\newcommand{\flowproves}{\mathbin{|\hskip -0.1em\scalebox{0.7}[1]{$\leadsto$}}}
\newcommand{\flowproveout}{\mathbin{\scalebox{0.7}[1]{$\leadsto$}\hskip -0.13em|}}



\addtolength{\textfloatsep}{-1.5em}

\settopmatter{printfolios=false}

\begin{document}

\title{\langName: A DSL for Safe Blockchain \AssetTxt{}s}

\author{Reed Oei}
\affiliation{%
    \institution{University of Illinois}
    \city{Urbana}
    \country{USA}
}
\email{reedoei2@illinois.edu}

\begin{abstract}
Blockchains host smart contracts for voting, tokens, and other purposes.
Vulnerabilities in contracts are common, often leading to the loss of money.
Psamathe is a new language we are designing around a new flow abstraction, reducing asset bugs and making contracts more concise than in existing languages.
We present an overview of Psamathe, and discuss two example contracts in Psamathe and Solidity.
\end{abstract}

\begin{CCSXML}
<ccs2012>
<concept>
<concept_id>10011007.10011006.10011050.10011017</concept_id>
<concept_desc>Software and its engineering~Domain specific languages</concept_desc>
<concept_significance>500</concept_significance>
</concept>
</ccs2012>
\end{CCSXML}

\ccsdesc[500]{Software and its engineering~Domain specific languages}

% %% Keywords. The author(s) should pick words that accurately describe
% %% the work being presented. Separate the keywords with commas.
\keywords{domain specific languages, smart contracts}

%% This command processes the author and affiliation and title
%% information and builds the first part of the formatted document.
\maketitle

\section{Introduction}
Blockchains are increasingly used as platforms for applications called \emph{smart contracts}~\cite{Szabo97:Formalizing}, which automatically manage transactions. % in an unbiased, mutually agreed-upon way.
Commonly proposed and implemented contracts include supply chain management~\cite{SupplyChainUse}, healthcare~\cite{HealthcareUse}, voting, crowdfunding, auctions, and more~\cite{Elsden18:Making}.
% A \emph{token contract} is a contract implementing a \emph{token standard}, such as ERC-20~\cite{erc20}.
% Token contracts are common on the Ethereum blockchain~\cite{wood2014ethereum}---about 73\% of high-activity contracts are token contracts~\cite{OlivaEtAl2019}.
However, smart contracts cannot be patched after being deployed, even if a security vulnerability is discovered.
The well-known DAO attack caused the loss of over 40 million dollars~\cite{DAO}.
% Developers must carefully review contracts, and some use an independent auditing service to help with this process.
% Despite this extra care, discoveries of vulnerabilities still occur regularly, often costing large amounts of money.
% Vulnerabilities still occur regularly, often costing large amounts of money.
% Some estimates suggest that as many as 46\% of smart contracts may have some vulnerability~\cite{luuOyente}.
\begin{figure}[!b]
    \centering
    \lstinputlisting[language=flow, xleftmargin=1em, xrightmargin=-1.0em, basicstyle=\footnotesize\ttfamily]{splash20-examples/erc20-transfer.flow}
    \vspace{-1em}
    \caption{A \langName contract with a \lstinline{transfer} function, transferring \lstinline{amount} tokens from the sender's account to the destination account.
It uses a single flow, which checks all the preconditions to ensure the transfer is valid.}
    \label{fig:erc20-transfer-flow}
\end{figure}

\langName (\langNamePronounce) is a new programming language we are designing to write safer contracts, focused on a new abstraction, a \emph{flow}, representing an atomic transfer.
% which is useful in smart contracts managing digital \assetTxt{}s.
% Flows allow encoding semantic information about the flow of \assetTxt{}s into the code.
The \langName language will also provide features to mark types with \emph{modifiers}, such as \flowinline{asset}, which combine with flows to make some classes of bugs impossible.
% A formalization of \langName is in progress~\cite{psamatheRepo}, with an \emph{executable semantics} implemented in the $\mathbb{K}$-framework~\cite{rosu-serbanuta-2010-jlap}, which is already capable of running the examples shown in Figures~\ref{fig:erc20-transfer-flow} and~\ref{fig:voting-impl-flow}.

Solidity is the most commonly-used contract language on Ethereum~\cite{EthereumForDevs}, and does not provide analogous support for managing \assetTxt{}s.
Typical contracts are more \textbf{concise} in \langName than in Solidity, because \langName handles common patterns and pitfalls automatically.

Other newly-proposed blockchain languages include Flint, Move, Nomos, Obsidian, and Scilla~\cite{schrans2018flint, blackshear2019move, das2019nomos, coblenz2019obsidian, sergey2019scilla}.
Scilla and Move are intermediate languages, whereas \langName is a high-level language.
Obsidian, Move, Nomos, and Flint use linear or affine types to manage \assetTxt{}s; \langName uses \emph{type quantities}, which provide the benefits of \emph{linear types}, but allow a more precise analysis of the flow of values in a program.
None of the these languages have flows or provide support for all the modifiers that \langName does.

\section{Language}\label{sec:lang}
A \langName program is made of \emph{contracts}, each containing \emph{fields}, \emph{types}, and \emph{functions}.
Each contract instance in \langName represents a contract on the blockchain, and the fields provide persistent storage.
% Type declarations allow annotating values with modifiers, such as $\asset$.
% We distinguish between three types of function: \emph{transactions}, \emph{views}, and \emph{transformers}.
% Transactions and views are both \emph{methods}, which may access the fields of a contract; a transaction can read \textbf{and} write fields, whereas a view may only read fields.
% A transformer is a \textbf{pure} function, which is not part of any contract: it may contain flows and other statements, and it may mutate \textbf{local} state, but it cannot mutate the state of any contract.
Figure~\ref{fig:erc20-transfer-flow} shows a simple contract implementing the core of ERC-20's \lstinline{transfer} function.
ERC-20~\cite{erc20} is a standard for managing \emph{fungible} tokens, which are interchangeable, like most currencies.
\begin{figure}
    \centering
    \lstinputlisting[language=Solidity, basicstyle=\footnotesize\ttfamily]{splash20-examples/erc20-transfer.sol}
    \vspace{-1.6em}
    \caption{A Solidity implementation of ERC-20's \lstinline{transfer}, from a reference implementation~\cite{erc20Consensys}.
        Preconditions are checked manually.
        We must include the \lstinline{SafeMath} library (not shown) to use \lstinline{add} and \lstinline{sub} (to check for under/over-flow).}
    \label{fig:erc20-transfer-sol}
\end{figure}

\langName is built around \emph{flows}. %, an atomic, state-changing operation describing the transfer of values.
Using the more declarative, \emph{flow-based} approach provides the following advantages:
Flows have a \emph{source}, a \emph{destination}, and an optional \emph{selector}.
When executed, every value selected (everything, by default) in the source is removed and combined with the destination.
% On line~\ref{line:erc20-flow-ex} of Figure~\ref{fig:erc20-transfer-flow}, the source is \flowinline{balances[msg.sender]}, the destination is \flowinline{balances[dst]}, and the selector is \flowinline{amount}.
\begin{itemize}
    \item \textbf{Precondition checking}: \langName automatically inserts dynamic checks of a flow's validity; e.g., a flow of money would fail if there is not enough in the source. % , or if there is too much in the destination (e.g., due to overflow).
        % Flows can also fail for other reasons: a developer may specify that a certain flow must send all \assetTxt{}s matching a predicate, but in addition specify an expected \emph{quantity} that must be selected: any number, exactly one, or at least one.
    \item \textbf{Data-flow tracking}: We hypothesize that flows provide a clearer way of specifying how resources flow, which may be less apparent using other approaches, especially in complicated contracts. %such as those involving transfer fees.
        % Additionally, developers must explicitly mark when \assetTxt{}s are \emph{consumed}, and only assets marked as \flowinline{consumable} may be consumed. % , which we hypothesize leads to fewer accidental asset loss bugs.
    \item \textbf{Error messages}: When a flow fails, \langName provides automatic, descriptive error messages, using information in the code of the flow, such as:
\vspace{-0.5em}
\begin{lstlisting}[numbers=none, basicstyle=\small\ttfamily, xrightmargin=-5em]
Cannot flow <amount> Token from account[<src>] to
account[<dst>]: source only has <balance> Token.
\end{lstlisting}
\vspace{-0.5em}
        % Flows enable such messages by encoding all the necessary information into the program.%, instead of using low-level operations like increment and decrement.
\end{itemize}

Each variable, field, and parameter has a \emph{type quantity}, approximating the number of values in the variable, which is one of: \flowinline{empty}, \flowinline{any}, \flowinline{one}, \flowinline{nonempty}, or \flowinline{every}.
Type quantities are inferred if omitted; every type quantity in Figure~\ref{fig:erc20-transfer-flow} can be inferred.
Only \flowinline{empty} \assetTxt variables may be dropped.

\emph{Modifiers} can be used to place constraints on how values are managed: \flowinline{asset}, \flowinline{fungible}, and \flowinline{unique}. % and \flowinline{consumable}.
% \emph{Modifiers} can be used to place constraints on how values are managed: \flowinline{asset}, \flowinline{fungible}, \flowinline{unique}, \flowinline{immutable}, and \flowinline{consumable}.
An \flowinline{asset} is a value that must not be reused or accidentally lost.
A \flowinline{fungible} value represents an interchangeable value that can be \textbf{merged}. % and it is \textbf{not} \flowinline{unique}.
ERC-20 tokens are \flowinline{fungible}.
% A \flowinline{immutable} value cannot be changed; in particular, it cannot be the source or destination of a flow.
A \flowinline{unique} value only exists in at most one variable, enforced by a dynamic check when created; it must be an \flowinline{asset} to prevent duplication.
% ERC-721 tokens are \flowinline{unique}. % and \flowinline{immutable}.
% A \flowinline{consumable} value is an \flowinline{asset} that it may be appropriate to dispose of, done via the \flowinline{consume} construct, documenting that the disposal is intentional.

% \langName has transactional semantics: a sequence of flows will either all succeed, or, if a single flow fails, the rest will fail as well.
% If a sequence of flows fails, the error propagates, like an exception, until it either: a) reaches the top level, and the entire transaction fails; or b) reaches a \flowinline{catch}, and then only the changes made in the corresponding \flowinline{try} will revert, and the \flowinline{catch} block will execute.

\section{Examples}
\begin{figure}[!b]
    \centering
    \lstinputlisting[language=flow, basicstyle=\footnotesize\ttfamily]{splash20-examples/voting.flow}
    \vspace{-1em}
    \caption{A simple voting contract in \langName.}
    \label{fig:voting-impl-flow}
\end{figure}

\subsection{ERC-20}\label{sec:erc20-impl}
Figure~\ref{fig:erc20-transfer-sol} shows a Solidity implementation of the ERC-20 function \lstinline{transfer} (cf. Figure~\ref{fig:erc20-transfer-flow}).
% Each ERC-20 contract manages the ``bank accounts'' for its own tokens, keeping track of how many tokens each user has; users are represented by addresses.
This example shows the advantages of flows in precondition checking, data-flow tracking, and error messages.
In this case, the sender's balance must be at least as large as \flowinline{amount}, and the destination's balance must not overflow when it receives the tokens.
Psamathe automatically inserts code checking these two conditions, ensuring that the checks are not forgotten.

\subsection{Voting}\label{sec:voting-impl}
\begin{figure}[!b]
    \centering
    \lstinputlisting[language=Solidity, basicstyle=\footnotesize\ttfamily]{splash20-examples/voting.sol}
    \vspace{-1em}
    \caption{A simple voting contract in Solidity.}
    \label{fig:voting-impl-sol}
\end{figure}
% One proposed use for blockchains is for voting~\cite{Elsden18:Making}.
Figures~\ref{fig:voting-impl-flow} and~\ref{fig:voting-impl-sol} show the core of a voting contract in \langName and Solidity, respectively, based on the Solidity by Example tutorial~\cite{solidityByExample}.
Each contract instance has several proposals, and a chairperson, assigned in the constructor (not shown), gives users permission to vote.
Each user can vote exactly once for exactly one proposal. % and the proposal with the most votes wins.
Note that values of type \flowinline{address} are used to \emph{select} \flowinline{Voter}s on line~\ref{line:select-voter-by-addr}.
We can do this because the \emph{underlying type} of \flowinline{Voter} is \flowinline{address}---allowing us to refer to assets without the asset itself.

This example shows \langName is suited for a range of applications.
It also shows some uses of the \flowinline{unique} modifier; in this contract, \flowinline{unique} ensures that each user, represented by an \lstinline{address}, can be given permission to vote at most once, while \flowinline{asset} ensures that votes are not lost or double-counted.
% The Solidity implementation is also more verbose than the \langName implementation because it must work around the limitations of the mapping structure.
% In this example, the \lstinline{weight} and \lstinline{voted} members of the \lstinline{Voter} struct exist so that the contract can tell whether a voter has the default values, was authorized to vote, or has already voted.

\section{Conclusion and Future Work}

\langName is a language for safer contracts.
\langName uses the new flow abstraction and modifiers (e.g., \flowinline{asset}) to provide safety guarantees.
We showed examples of contracts in Solidity and \langName, showing that \langName can express common contracts concisely, retaining key safety properties.

In the future, we plan to implement the \langName language, and prove its safety properties.
We hope to study the benefits and costs of the language via case studies, performance evaluation, and the application of flows to other domains.
Finally, we would also like to conduct a user study to evaluate the usability of the flow abstraction and the design of the language. %, and to compare it to Solidity, which we hypothesize will show that developers write contracts with fewer asset management errors in \langName than in Solidity.

%% The next two lines define the bibliography style to be used, and
%% the bibliography file.
\bibliographystyle{ACM-Reference-Format}
\bibliography{biblio}

\end{document}
\endinput

