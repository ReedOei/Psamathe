\documentclass[10pt]{article}

\usepackage{amsmath}
\usepackage{sagetex}
\usepackage{hyperref}
\usepackage{tikz-cd}
\usepackage{amssymb}
\usepackage{amsthm}
\usepackage{bm}
\usepackage{listings}
\usepackage{bbm}
\usepackage{multicol}
\usepackage{mathtools}
\usepackage{mathpartir}
\usepackage{float}
\usepackage[inline]{enumitem}
\usepackage[margin=1.25in]{geometry}
\usepackage[T1]{fontenc}
\usepackage{kpfonts}

\usetikzlibrary{decorations.pathmorphing}

\lstset{
  frame=none,
  xleftmargin=2pt,
  stepnumber=1,
  numbers=left,
  numbersep=5pt,
  numberstyle=\ttfamily\tiny\color[gray]{0.3},
  belowcaptionskip=\bigskipamount,
  captionpos=b,
  escapeinside={*'}{'*},
  tabsize=2,
  emphstyle={\bf},
  commentstyle=\it,
  stringstyle=\mdseries\rmfamily,
  showspaces=false,
  keywordstyle=\bfseries\rmfamily,
  columns=flexible,
  basicstyle=\small\sffamily,
  showstringspaces=false,
}

\newcommand{\guard}{\bigg|\bigg|}
\newcommand{\newc}{\textbf{\texttt{new}}}
\newcommand{\everything}{\textbf{\texttt{everything}}}
\newcommand{\then}{\textbf{\texttt{then}}}
\newcommand{\this}{\textbf{\texttt{this}}}
\newcommand{\langName}{LANGUAGE-NAME\xspace}
\newcommand{\consumesarr}{\mathrel{\rotatebox[origin=c]{270}{${\looparrowright}$}}}
\newcommand{\consumes}[1]{\stackrel{#1}{\consumesarr}}
\newcommand{\sends}[1]{\stackrel{#1}{\to}}
\newcommand{\emitsarr}{\mathrel{\rotatebox[origin=c]{270}{${\looparrowleft}$}}}
\newcommand{\emits}[1]{\emitsarr #1}
\newcommand{\asset}{\textbf{\texttt{asset}}\xspace}
\newcommand{\fungible}{\textbf{\texttt{fungible}}\xspace}
\newcommand{\nonfungible}{\textbf{\texttt{nonfungible}}\xspace}
\newcommand{\addresst}{\textbf{\texttt{address}}\xspace}
\newcommand{\stringt}{\textbf{\texttt{string}}\xspace}
\newcommand{\setof}{\textbf{\texttt{set}}\xspace}
\newcommand{\optiont}{\textbf{\texttt{option}}\xspace}
\newcommand{\boolt}{\textbf{\texttt{bool}}\xspace}
\newcommand{\natt}{\textbf{\texttt{nat}}\xspace}
\newcommand{\byt}{\textbf{\texttt{by}}\xspace}
\newcommand{\stores}{\textbf{\texttt{stores}}\xspace}
\newcommand{\suchthat}{\textbf{s.t.}\xspace}
\newcommand{\returns}{\textbf{\texttt{returns}}\xspace}
\newcommand{\merge}{\rightsquigarrow}
\newcommand{\flowsto}{\rightsquigarrow}
\newcommand{\heldby}{\rightarrowtail}
\newcommand{\one}{\textbf{\texttt{one}}\xspace}
\newcommand{\some}{\textbf{\texttt{some}}\xspace}
\newcommand{\any}{\textbf{\texttt{any}}\xspace}
\newcommand{\map}{\textbf{\texttt{map}}\xspace}
\newcommand{\linking}{\textbf{\texttt{linking}}\xspace}
\newcommand{\transformer}{\textbf{\texttt{transformer}}\xspace}
\newcommand{\selects}{\textbf{\texttt{selects}}\xspace}
\newcommand{\demote}{\textbf{\texttt{demote}}\xspace}
\newcommand{\transforms}{\Rrightarrow}

\newcommand{\flowproves}{\mathbin{|\hskip -0.1em\scalebox{0.7}[1]{$\leadsto$}}}
\newcommand{\flowproveout}{\mathbin{\scalebox{0.7}[1]{$\leadsto$}\hskip -0.13em|}}



\lstset{
  frame=none,
  xleftmargin=2pt,
  stepnumber=1,
  numbers=left,
  numbersep=5pt,
  numberstyle=\ttfamily\tiny\color[gray]{0.3},
  belowcaptionskip=\bigskipamount,
  captionpos=b,
  escapeinside={*'}{'*},
  tabsize=2,
  emphstyle={\bf},
  commentstyle=\it,
  stringstyle=\mdseries\rmfamily,
  showspaces=false,
  keywordstyle=\bfseries\rmfamily,
  columns=flexible,
  basicstyle=\small\sffamily,
  showstringspaces=false,
}

\newcommand{\guard}{\bigg|\bigg|}
\newcommand{\newc}{\textbf{\texttt{new}}}
\newcommand{\everything}{\textbf{\texttt{everything}}}
\newcommand{\then}{\textbf{\texttt{then}}}
\newcommand{\this}{\textbf{\texttt{this}}}
\newcommand{\langName}{LANGUAGE-NAME\xspace}
\newcommand{\consumesarr}{\mathrel{\rotatebox[origin=c]{270}{${\looparrowright}$}}}
\newcommand{\consumes}[1]{\stackrel{#1}{\consumesarr}}
\newcommand{\sends}[1]{\stackrel{#1}{\to}}
\newcommand{\emitsarr}{\mathrel{\rotatebox[origin=c]{270}{${\looparrowleft}$}}}
\newcommand{\emits}[1]{\emitsarr #1}
\newcommand{\addresst}{\textbf{\texttt{address}}\xspace}
\newcommand{\stringt}{\textbf{\texttt{string}}\xspace}
\newcommand{\manyt}{\textbf{\texttt{many}}\xspace}
\newcommand{\natt}{\textbf{\texttt{nat}}\xspace}
\newcommand{\byt}{\textbf{\texttt{by}}\xspace}
\newcommand{\stores}{\textbf{\texttt{stores}}\xspace}
\newcommand{\suchthat}{\textbf{s.t.}\xspace}
\newcommand{\returns}{\textbf{\texttt{returns}}\xspace}
\newcommand{\merge}{\rightsquigarrow}
\newcommand{\heldby}{\rightarrowtail}
\newcommand{\one}{\textbf{\texttt{one}}\xspace}
\newcommand{\some}{\textbf{\texttt{some}}\xspace}
\newcommand{\any}{\textbf{\texttt{any}}\xspace}

\begin{document}

\section{Introduction}

\langName is a DSL for implementing programs which manage resources, targeted at writing smart contracts.

\subsection{Contributions}

We make the following main contributions:
\begin{itemize}
    \item \textbf{Safety guarantees}: similar to \reed{or maybe just exactly} linear types\reed{or maybe uniqueness types? need to read more about this}, preventing accidental resource loss or duplication.
        Additionally, provides some amount of reentrancy safety.

        \begin{itemize}
            \item We can evaluate these by formalizing the language and proving them; the formalization is something that would be nice to do anyway.
        \end{itemize}

    \item \textbf{Simplicity}: The language is quite simple---it makes writing typical smart contract programs easier and shorter, because many common pitfalls in Solidity are automatically handled by the language, such as overflow/underflow, checking of balances, short address attacks, etc.

        \begin{itemize}
            \item We can evaluate these by comparing LOC, cyclotomic complexity, etc.
                Not sure what the right metric would be.

            \item We can also evaluate via a user study, but that will take a long(er) time.
        \end{itemize}

    \item \reed{Optimizations?}

        \begin{itemize}
            \item We can evaluate this by profiling or a simple opcode count (which is not only a proxy for performance, but also means that deploying the contract will be cheaper).
        \end{itemize}
\end{itemize}


\subsection{Language Intro}

The basic state-changing in the language is a \emph{flow}.
This means that it is impossible to lose resources (if done properly, at least).

Note that all flows fail if they can't be performed.
For example, a flow of fungible resources fails if there aren't enough resources, and a flow of a nonfungible resource fails if it doesn't exist in the source location.

\section{More examples}

\subsection{The DAO attack}
We can prevent the DAO attack (the below is from \url{https://consensys.github.io/smart-contract-best-practices/known_attacks/}):
\begin{lstlisting}
function withdrawBalance() public {
    uint amountToWithdraw = userBalances[msg.sender];
    // At this point, the caller's code is executed, and can call withdrawBalance again
    require(msg.sender.call.value(amountToWithdraw)(""));
    userBalances[msg.sender] = 0;
}
\end{lstlisting}

In \langName, we would write this as:

\begin{lstlisting}
transaction withdrawBalance():
    flow everything from userBalances[msg.sender] to msg.sender.balance
\end{lstlisting}

Not only is this simpler, but the compiler can automatically place the actual call that does the transfer last, meaning that the mistake could simply never be made.

\subsection{approveAndCall}

Many token contracts include the concept of \lstinline{approveAndCall}, typically similarly to the following (taken from: \lstinline{0x174bfa6600bf90c885c7c01c7031389ed1461ab9}, one of the most popular contracts on the blockchain by transaction count):
\begin{lstlisting}
function approveAndCall(address _spender, uint256 _value, bytes memory _extraData) public returns (bool success) {
    tokenRecipient spender = tokenRecipient(_spender);
    if (approve(_spender, _value)) {
        spender.receiveApproval(msg.sender, _value, address(this), _extraData);
        return true;
    }
}
\end{lstlisting}

In \langName, we can do the same thing by associating a call with a flow.
\begin{lstlisting}
transaction approveAndCall(_spender : address, _value : uint256, _extraData : bytes):
    consume everything from allowance[msg.sender, _spender]
    flow _value from approvalSource to allowance[msg.sender, _spender],
        then call receiveApproval(msg.sender, _value, address(this), _extraData)
\end{lstlisting}

\section{Formalization}

\subsection{Syntax}
\begin{figure}[ht]
\begin{align*}
    C &\in \textsc{ContractNames} & T &\in \textsc{TransactionNames} \\
    I &\in \textsc{InterfaceNames} & S &\in \textsc{StateNames}\\
    D &\in \textsc{ContractNames} \cup \textsc{InterfaceNames} & p &\in \Q \\
    E &\in \textsc{EventNames} & f &\in \textsc{FieldNames} \\
    x &\in \textsc{IdentifierNames} & \mathcal{R} &\in \textsc{ResourceNames} \\
\end{align*}
\caption{Abstract syntax of \langName.}
\label{lang-syntax}
\end{figure}

\begin{definition}
    A \emph{resource} $\mathcal{R}$ is a tuple $(R, +, 0, \leq, -)$ where
    \begin{enumerate}[label=(\roman*)]
        \item $(R, +, 0)$ is a commutative monoid.
        \item $(R, \leq)$ is a partial order that is compatible with $(R, +, 0)$.
            That is, for any $x, y \in R$ such that $x \leq y$, and for any $z \in R$, we have $x + z \leq y + z$.
        \item $- : R \times R \to R$ is a function so that for any $x,y \in R$ such that $y \leq x$, we have $(x - y) + y = x$.
    \end{enumerate}
\end{definition}

\paragraph{Examples}

\begin{enumerate}
    \item The natural numbers with the standard operations is a resource $(\N, +, 0, \leq, -)$, where $-$ is \emph{saturating subtraction}: $n - m = 0$ if $m > n$.
    \item For any set $A$, we can build the resource $(\powerset{A}, \cup, \emptyset, \subseteq, \setminus)$, where $X \setminus Y$ is the set difference operation.
    \item Similarly, given any set $A$, we can build the resource $(\powersetfin{A}, \cup, \emptyset, \subseteq, \setminus)$; this resource is called the \emph{nonfungible resource on $A$}, written $\text{nf}(A)$.
\end{enumerate}

\section{Examples}
\paragraph{Basic transfer} ERC-20 transfers.
\eq{
    \texttt{transfer}(\texttt{from}, \texttt{to}, \texttt{amount}): \\
    \mathcal{A}[\texttt{from}] \xrightarrow{\texttt{amount}} \mathcal{A}[\texttt{to}]
}

\paragraph{NFT transfer}
For example, ERC-721 token transfers.

\eq{
    \texttt{transfer}(\texttt{from}, \texttt{to}, \texttt{id}): \\
    \mathcal{A}[\texttt{from}] \xrightarrow{\texttt{id}} \mathcal{A}[\texttt{to}]
}

\paragraph{Transfer with fees}
This is fairly common: for example, the contract with the most transactions does this, as do many gambling/auction contracts.
\eq{
    \texttt{transfer}(\texttt{from}, \texttt{to}, \texttt{amount}):
}
\begin{tikzcd}
    & & \mathcal{A}[\texttt{owner}] \\
    \mathcal{A}[\texttt{from}] \arrow{r}{\texttt{amount}} & \arrow{ur}{\texttt{fee} \%} \arrow[rightsquigarrow]{dr}{} & \\
    & & \mathcal{A}[\texttt{to}] \\
\end{tikzcd}
\reed{Unclear that the percent thing is actually the best idea, but we should have some way of ensuring that splitting doesn't forget anything (and doesn't create anything!)}
\reed{Not really happy with this syntax.
I think the diagram is clear in what it does, but the actual concrete syntax is not.}

\paragraph{Transfer with guards}
For example, when minting.
\eq{
    \texttt{mint}(\texttt{to}, \texttt{amount}): \\
    \mathcal{S}
    \xrightarrow{\texttt{amount}}
    \underbrace{\guard}_{\texttt{sender} = \texttt{owner}}
    \mathcal{A}[\texttt{to}]
}

\paragraph{Transfer to Timelock}
\eq{
    \texttt{timeLock}(\texttt{time}, \texttt{from}, \texttt{beneficiary}, \texttt{amount}): \\
    \mathcal{A}[\texttt{from}]
    \xrightarrow{\texttt{amount}}
    \newc ~ T(\texttt{time}, \texttt{beneficiary})
    \rightarrowtail \this
}
\begin{lstlisting}
timeLock(time, from, beneficiary, amount):
    flow amount from account[from] to new T(time, beneficiary) held by this
\end{lstlisting}

\eq{
    \texttt{claim}(): & \\
    T(\texttt{time}, \texttt{beneficiary}) &
    \rightsquigarrow
    \underbrace{\guard}_{\texttt{sender} = \texttt{beneficiary} ~\texttt{and}~ \texttt{now} > \texttt{time}}
    \mathcal{A}[\texttt{beneficiary}] \\
}
\begin{lstlisting}
claim():
    only when sender = beneficiary and now > time then
    merge T(time, beneficiary) into account[beneficiary]
\end{lstlisting}

\paragraph{Nonfungible resources} a la ERC-721, but also useful for other things, like accounts.
Saying that something \lstinline{stores account} here means that it stores a set of accounts, because accounts are a nonfungible resource.
\reed{Not sure I like this or if it will be too confusing, but will have to look into this more with more examples and such}
\begin{lstlisting}
resource account := nonfungible address
source accountSource of account

contract FreezeAccounts {
    storage {
        normal stores account
        frozen stores account
    }

    createAccount(addr: address):
        flow addr from accountSource to normal

    freezeAccount(addr: address):
        flow addr from normal to frozen
}
\end{lstlisting}
\reed{I don't think I really like this for managing states.
Will need to work on this more.}

\paragraph{RefundVault}
This is a real example from \lstinline{0xa15c7ebe1f07caf6bff097d8a589fb8ac49ae5b3.sol}.

\begin{lstlisting}
// This should probably be part of the standard library, and not allow the use of "total"
resource ether := fungible nat
resource contract := nonfungible address

source addressSource of address
source initialize of address provides 1

contract RefundVault {
    storage {
        active stores contract
        refunding stores contract
        closed stores contract

        deposited stores ether by address

        wallet stores address
        owner stores address
    }

    on create(wallet_addr) uses sender:
        merge sender into owner
        flow this from initialize to active
        flow wallet_addr from addressSource to wallet

    deposit(investor) has value:
        merge value into deposited[investor]

    close() uses sender:
        only when sender in owner then
            flow this from active to closed,
            and
            merge all in deposited into wallet // Will we be able to optimize this somehow?

    enableRefunds() uses sender:
        only when sender in owner then
            flow this from active to refunding

    refund(investor):
        only when this in refunding then
            merge deposited[investor] into investor
}
\end{lstlisting}

\end{document}

