\documentclass[10pt]{article}

\usepackage{amsmath}
\usepackage{sagetex}
\usepackage{hyperref}
\usepackage{tikz-cd}
\usepackage{amssymb}
\usepackage{amsthm}
\usepackage{bm}
\usepackage{listings}
\usepackage{bbm}
\usepackage{multicol}
\usepackage{mathtools}
\usepackage{mathpartir}
\usepackage{float}
\usepackage[inline]{enumitem}
\usepackage[margin=1.25in]{geometry}
\usepackage[T1]{fontenc}
\usepackage{kpfonts}

\usetikzlibrary{decorations.pathmorphing}

\lstset{
  frame=none,
  xleftmargin=2pt,
  stepnumber=1,
  numbers=left,
  numbersep=5pt,
  numberstyle=\ttfamily\tiny\color[gray]{0.3},
  belowcaptionskip=\bigskipamount,
  captionpos=b,
  escapeinside={*'}{'*},
  tabsize=2,
  emphstyle={\bf},
  commentstyle=\it,
  stringstyle=\mdseries\rmfamily,
  showspaces=false,
  keywordstyle=\bfseries\rmfamily,
  columns=flexible,
  basicstyle=\small\sffamily,
  showstringspaces=false,
}

\newcommand{\guard}{\bigg|\bigg|}
\newcommand{\newc}{\textbf{\texttt{new}}}
\newcommand{\everything}{\textbf{\texttt{everything}}}
\newcommand{\then}{\textbf{\texttt{then}}}
\newcommand{\this}{\textbf{\texttt{this}}}
\newcommand{\langName}{LANGUAGE-NAME\xspace}
\newcommand{\consumesarr}{\mathrel{\rotatebox[origin=c]{270}{${\looparrowright}$}}}
\newcommand{\consumes}[1]{\stackrel{#1}{\consumesarr}}
\newcommand{\sends}[1]{\stackrel{#1}{\to}}
\newcommand{\emitsarr}{\mathrel{\rotatebox[origin=c]{270}{${\looparrowleft}$}}}
\newcommand{\emits}[1]{\emitsarr #1}
\newcommand{\asset}{\textbf{\texttt{asset}}\xspace}
\newcommand{\fungible}{\textbf{\texttt{fungible}}\xspace}
\newcommand{\nonfungible}{\textbf{\texttt{nonfungible}}\xspace}
\newcommand{\addresst}{\textbf{\texttt{address}}\xspace}
\newcommand{\stringt}{\textbf{\texttt{string}}\xspace}
\newcommand{\setof}{\textbf{\texttt{set}}\xspace}
\newcommand{\optiont}{\textbf{\texttt{option}}\xspace}
\newcommand{\boolt}{\textbf{\texttt{bool}}\xspace}
\newcommand{\natt}{\textbf{\texttt{nat}}\xspace}
\newcommand{\byt}{\textbf{\texttt{by}}\xspace}
\newcommand{\stores}{\textbf{\texttt{stores}}\xspace}
\newcommand{\suchthat}{\textbf{s.t.}\xspace}
\newcommand{\returns}{\textbf{\texttt{returns}}\xspace}
\newcommand{\merge}{\rightsquigarrow}
\newcommand{\flowsto}{\rightsquigarrow}
\newcommand{\heldby}{\rightarrowtail}
\newcommand{\one}{\textbf{\texttt{one}}\xspace}
\newcommand{\some}{\textbf{\texttt{some}}\xspace}
\newcommand{\any}{\textbf{\texttt{any}}\xspace}
\newcommand{\map}{\textbf{\texttt{map}}\xspace}
\newcommand{\linking}{\textbf{\texttt{linking}}\xspace}
\newcommand{\transformer}{\textbf{\texttt{transformer}}\xspace}
\newcommand{\selects}{\textbf{\texttt{selects}}\xspace}
\newcommand{\demote}{\textbf{\texttt{demote}}\xspace}
\newcommand{\transforms}{\Rrightarrow}

\newcommand{\flowproves}{\mathbin{|\hskip -0.1em\scalebox{0.7}[1]{$\leadsto$}}}
\newcommand{\flowproveout}{\mathbin{\scalebox{0.7}[1]{$\leadsto$}\hskip -0.13em|}}


\lstset{
  frame=none,
  xleftmargin=2pt,
  stepnumber=1,
  numbers=left,
  numbersep=5pt,
  numberstyle=\ttfamily\tiny\color[gray]{0.3},
  belowcaptionskip=\bigskipamount,
  captionpos=b,
  escapeinside={*'}{'*},
  tabsize=2,
  emphstyle={\bf},
  commentstyle=\it,
  stringstyle=\mdseries\rmfamily,
  showspaces=false,
  keywordstyle=\bfseries\rmfamily,
  columns=flexible,
  basicstyle=\small\sffamily,
  showstringspaces=false,
}

\newcommand{\guard}{\bigg|\bigg|}
\newcommand{\newc}{\textbf{\texttt{new}}}
\newcommand{\everything}{\textbf{\texttt{everything}}}
\newcommand{\then}{\textbf{\texttt{then}}}
\newcommand{\this}{\textbf{\texttt{this}}}
\newcommand{\langName}{LANGUAGE-NAME\xspace}
\newcommand{\consumesarr}{\mathrel{\rotatebox[origin=c]{270}{${\looparrowright}$}}}
\newcommand{\consumes}[1]{\stackrel{#1}{\consumesarr}}
\newcommand{\sends}[1]{\stackrel{#1}{\to}}
\newcommand{\emitsarr}{\mathrel{\rotatebox[origin=c]{270}{${\looparrowleft}$}}}
\newcommand{\emits}[1]{\emitsarr #1}
\newcommand{\asset}{\textbf{\texttt{asset}}\xspace}
\newcommand{\fungible}{\textbf{\texttt{fungible}}\xspace}
\newcommand{\nonfungible}{\textbf{\texttt{nonfungible}}\xspace}
\newcommand{\addresst}{\textbf{\texttt{address}}\xspace}
\newcommand{\stringt}{\textbf{\texttt{string}}\xspace}
\newcommand{\setof}{\textbf{\texttt{set}}\xspace}
\newcommand{\optiont}{\textbf{\texttt{option}}\xspace}
\newcommand{\boolt}{\textbf{\texttt{bool}}\xspace}
\newcommand{\natt}{\textbf{\texttt{nat}}\xspace}
\newcommand{\byt}{\textbf{\texttt{by}}\xspace}
\newcommand{\stores}{\textbf{\texttt{stores}}\xspace}
\newcommand{\suchthat}{\textbf{s.t.}\xspace}
\newcommand{\returns}{\textbf{\texttt{returns}}\xspace}
\newcommand{\merge}{\rightsquigarrow}
\newcommand{\flowsto}{\rightsquigarrow}
\newcommand{\heldby}{\rightarrowtail}
\newcommand{\one}{\textbf{\texttt{one}}\xspace}
\newcommand{\some}{\textbf{\texttt{some}}\xspace}
\newcommand{\any}{\textbf{\texttt{any}}\xspace}
\newcommand{\map}{\textbf{\texttt{map}}\xspace}
\newcommand{\linking}{\textbf{\texttt{linking}}\xspace}
\newcommand{\transformer}{\textbf{\texttt{transformer}}\xspace}
\newcommand{\selects}{\textbf{\texttt{selects}}\xspace}
\newcommand{\demote}{\textbf{\texttt{demote}}\xspace}
\newcommand{\transforms}{\Rrightarrow}

\newcommand{\flowproves}{\mathbin{|\hskip -0.1em\scalebox{0.7}[1]{$\leadsto$}}}
\newcommand{\flowproveout}{\mathbin{\scalebox{0.7}[1]{$\leadsto$}\hskip -0.13em|}}



\begin{document}

\section{Specification}

\subsection{Syntax}
\begin{figure}[ht]
\begin{align*}
    C &\in \textsc{ContractNames} & m &\in \textsc{TransactionNames} \\
    t &\in \textsc{TypeNames} & x &\in \textsc{Identifiers} \\
    n &\in \Z \\
\end{align*}
\begin{tabular}{l r l l}
    $\mathfrak{q}$ & \bnfdef & $\exactlyone$ \bnfalt $\any$ \bnfalt $\nonempty$ & (selector quantifiers) \\
    $\mathcal{Q}$ & \bnfdef & $\mathfrak{q}$ \bnfalt $\emptyq$ \bnfalt $\every$ & (\typeQuantities) \\
    $\mathcal{C}$ & \bnfdef & $\optionq$ \bnfalt $\setq$ \bnfalt $\listq$ & (collection type constructors) \\
    % Removed product types for the moment because you don't really need them if you have records.
    % $T$ & \bnfdef & $\voidt$ \bnfalt \boolt \bnfalt $\natt$ \bnfalt $\mathcal{C}~\tau$ \bnfalt $\tau \times \tau$ \bnfalt $\tau \transforms \tau$ \bnfalt $\curlys{\overline{x : \tau}}$ \bnfalt $t$ & (base types) \\
    $T$ & \bnfdef & $\voidt$ \bnfalt \boolt \bnfalt $\natt$ \bnfalt $\mathcal{C}~\tau$ \bnfalt $\tau \transforms \tau$ \bnfalt $\curlys{\overline{x : \tau}}$ \bnfalt $t$ & (base types) \\
    $\tau$ & \bnfdef & $\mathcal{Q}~T$ & (types) \\
    $\mathcal{V}$ & \bnfdef & $n$ \bnfalt \true \bnfalt \false \bnfalt $\emptyq$ \bnfalt $\lambda x : \tau. E$ & (values) \\
    $\mathcal{L}$ & \bnfdef & $x$ \bnfalt $x[x]$ \bnfalt $x.x$ & (locations) \\
    $E$ & \bnfdef & $\mathcal{V}$ \bnfalt $\mathcal{L}$ \bnfalt $x.m(\overline{x})$ \bnfalt $\some(x)$ \bnfalt $s~\varin~x$ \bnfalt $\curlys{\overline{x : \tau \mapsto x}}$ \bnfalt $\letvar~x:\tau := E~\varin~E$ \bnfalt $\ifS~x~\thenS~E~\elseS~E$ & (expressions) \\
    % For the moment, removed these, but maybe put them back: $x \%$ \bnfalt $\min(f, f)$ \bnfalt $\max(f, f)$
    $s$ & \bnfdef & $\mathcal{L}$ \bnfalt $\everything$ \bnfalt $\mathfrak{q}~x : \tau~\suchthat~E$ & (selector) \\
    $\mathcal{N}$ & \bnfdef & $\mathcal{L}$ \bnfalt $\newc~C(\overline{x})$ \bnfalt $\consume$ & (nodes) \\
    $F$ & \bnfdef & $\mathcal{N} \sends{s} \mathcal{N}$ & (flows) \\
    $\Stmt$ & \bnfdef & $F$ \bnfalt $E$ \bnfalt $\revert(E)$ \bnfalt $\tryS~S~\catchS(x : \tau)~S$ \bnfalt $\ifS~x~\thenS~S~\elseS~S$ \\
            & \bnfalt & $\var~x:\tau := E$ \bnfalt $S;S$ \bnfalt \pack \bnfalt $\unpack(x)$ & (statements) \\
    $M$ & \bnfdef & \fungible \bnfalt \nonfungible \bnfalt \consumable \bnfalt \asset & (type declaration modifiers) \\
    $\mathcal{D}$ & \bnfdef & $x : \tau$ & (field) \\
                  & \bnfalt & $\type~t~\is~\overline{M}~T$ & (type declaration) \\
                  & \bnfalt & $\transaction~m(\overline{x : \tau})~\returns~x : \tau~\doC~S$ & (transactions) \\
                  & \bnfalt & $\view~m(\overline{x : \tau})~\returns~\tau := E$ & (views) \\
                  % & \bnfalt & $\oncreate(\overline{x : \tau})~\doC~S$ & (constructor) \\
    $\Con$ & \bnfdef & $\contract~C~\{~\overline{\mathcal{D}}~\}$ & (contracts) \\
    $\Prog$ & \bnfdef & $\overline{\Con}~;~S$ & (programs)

    % Maybe:  $\alpha$ & \bnfdef & & (row type variables) \\
\end{tabular}
\caption{Abstract syntax of \langName.}
\label{lang-syntax}
\end{figure}

In the surface language, ``collection types'' (i.e., $\mathcal{Q}~\mathcal{C}~\tau$ or a transformer) are by default $\any$, but all other types, like $\natt$, are $\exactlyone$.

\reed{Some simplification ideas}
\reed{Could get rid of selecting by locations and only allowed selecting with quantifiers, and just optimize things like $\exactlyone~x:\tau~\suchthat~x = y$ into a lookup.
Also, allowing \textbf{any} type quantity in a selector lets us do away with everything.
Would actually be even nicer if we allowed any type quantity to appear in the quantifier, because then we wouldn't even need a special rule for everything.}
\reed{We could also get rid of ``if'' and instead do something like $\any~x:\tau~\suchthat~\ifS~b~\thenS~x = y~\elseS~false$}

\reed{Contract types should be consumable assets by default (consuming a contract is a self-destruct?)}

\subsection{Statics}
\begin{figure}[ht]
\begin{tabular}{l r l l}
    $\Gamma,\Delta,\Xi$ & \bnfdef & $\emptyset$ \bnfalt $\Gamma, x : \tau$ & (type environments)
\end{tabular}
\label{type-env}
\end{figure}

\begin{definition}
    Define $\Quant = \curlys{\emptyq, \any, \exactlyone, \nonempty, \every}$, and call any $\mathcal{Q} \in \Quant$ a \emph{type quantity}.
    Define $\emptyq < \any < \,\, \exactlyone < \nonempty < \every$.
\end{definition}

\framebox{$\tau~\asset$} \textbf{Asset Types}

\reed{The syntax for record ``fields'' and type environments is the same...could just use it}
\begin{align*}
    (\mathcal{Q}~T)~\asset \iff \mathcal{Q} \neq \emptyq \tand (& \asset \in \modifiers(T) \tor \\
                                                                & (T = \tau \transforms \sigma \tand \sigma~\asset) \tor \\
                                                                & (T = \mathcal{C}~\tau \tand \tau~\asset) \tor \\
                                                                & (T = \curlys{\overline{y : \sigma}} \tand \exists x : \tau \in \overline{y : \sigma}. (\tau~\asset)))
\end{align*}
\reed{It should be the case that a transformer can have an output of an asset type if and only if it has an input asset type (and it the case of curried transformers, that \textbf{some} input type is an asset).}

\framebox{$\tau~\consumable$} \textbf{Consumable Types}
\begin{align*}
    (\mathcal{Q}~T)~\consumable &\iff \consumable \in \modifiers(T) \tor \lnot((\mathcal{Q}~T)~\asset)
\end{align*}

\begin{definition}
    Let $\mathcal{Q}, \mathcal{R} \in \Quant$.
    Define the commutative operator $\oplus$, called \emph{combine}, as the function:
    \begin{align*}
        \oplus : \textsc{Type} \times \Quant \to \textsc{Type} \\
        (\mathcal{Q}~T, \mathcal{R}) \mapsto \max(\mathcal{Q}, \mathcal{R})~T
    \end{align*}

    Define the operator $\ominus$, called \emph{split}, as the unique function $\Quant^2 \to \Quant$ such that
    \begin{align*}
        \mathcal{Q} \ominus \emptyq & = \mathcal{Q} \\
        \emptyq \ominus \mathcal{R} & = \emptyq \\
        \exactlyone \ominus \mathcal{R} &= \emptyq \qquad \tif \exactlyone \leq \mathcal{R} \\
        \nonempty \ominus \mathcal{R} &= \any \qquad \tif \mathcal{R} \neq \emptyq \\
        \mathcal{Q} \ominus \any &= \any \\
        \mathcal{Q} \ominus \every & = \emptyq \\
        \every \ominus \mathcal{Q} &= \every
    \end{align*}
\end{definition}

Note that we write $(\mathcal{Q}~T) \oplus \mathcal{R}$ to mean $(\mathcal{Q} \oplus \mathcal{R})~T$ and similarly $(\mathcal{Q}~T) \ominus \mathcal{R}$ to mean $(\mathcal{Q} \ominus \mathcal{R})~T$.
$\mathcal{Q} \oplus \mathcal{R}$ represents the quantity present when flowing $\mathcal{R}$ of something to a storage already containing $\mathcal{Q}$.
$\mathcal{Q} \ominus \mathcal{R}$ represents the quantity left over after flowing $\mathcal{R}$ from a storage containing $\mathcal{Q}$.

\begin{definition}
    We can consider a type environment $\Gamma$ as a function $\textsc{Identifiers}\xspace \to \textsc{Types}\xspace \cup \curlys{\bot}$ as follows:
    \[
        \Gamma(x) =
        \begin{cases}
            \tau & \tif x : \tau \in \Gamma \\
            \bot & \owise
        \end{cases}
    \]
    We write $\dom(\Gamma)$ to mean $\setbuild{x \in \textsc{Identifiers}}{\Gamma(x) \neq \bot}$, and $\Gamma|_X$ to mean the environment $\setbuild{x : \tau \in \Gamma}{x \in X}$ (restricting the domain of $\Gamma$).
\end{definition}

\begin{definition}
    Let $\mathcal{Q}$ and $\mathcal{R}$ be \typeQuantities, $T_\mathcal{Q}$ and $T_\mathcal{R}$ base types, and $\Gamma$ and $\Delta$ type environments.
    Define the following orderings, which make types and type environments into join-semilattices.
    For type quantities, define the partial order $\sqsubseteq$ by
    \[
        \mathcal{Q} \sqsubseteq \any; \quad \exactlyone \sqsubseteq \nonempty; \quad \every \sqsubseteq \nonempty
    \]
    For types, define the partial order $\leq$ by
    \[
        \mathcal{Q}~T_\mathcal{Q} \leq \mathcal{R}~T_\mathcal{R} \iff T_\mathcal{Q} = T_\mathcal{R} \tand \mathcal{Q} \sqsubseteq \mathcal{R}
    \]
    For type environments, define the partial order $\leq$ by
    \[
        \Gamma \leq \Delta \iff \forall x. \Gamma(x) \leq \Delta(x)
    \]
\end{definition}

\framebox{$\Gamma \flowproves E : \tau \flowprovesout \Delta$} \textbf{Expression Typing}
\begin{mathpar}
    \inferrule*[right=Empty-Val]{
    }{ \Gamma \flowproves \emptyq : \emptyq~\mathcal{C}~\tau \flowprovesout \Gamma }

    \inferrule*[right=Transformer]{
        \Gamma, x : \tau \flowproves E : \sigma \flowprovesout \Gamma
    }{ \Gamma \flowproves (\lambda x : \tau. E) : \emptyq~(\tau \transforms \sigma) }

    \inferrule*[right=Some]{
        \Gamma \flowproves x : \tau \flowprovesout \Delta
    }{ \Gamma \flowproves \some(x) : \,\,!~\optionq~\tau \flowprovesout \Delta }

    \inferrule*[right=Demote-Lookup]{
    }{ \Gamma, x : \tau \flowproves x : \demote(\tau) \flowprovesout \Gamma, x : \tau }

    \inferrule*[right=Lin-Lookup]{
    }{ \Gamma, x : \mathcal{Q}~T \flowproves x : \mathcal{Q}~T \flowprovesout \Gamma, x : \emptyq~T }

    \inferrule*[right=Field-Lookup]{
        \Gamma \proves x : \curlys{\overline{y : \tau}} \flowproves \Gamma
        \and
        f : \sigma \in \overline{y : \tau}
    }{ \Gamma \flowproves x.f : \sigma \flowproves \Gamma }

    \inferrule*[right=Build-Rec]{
        \Gamma \flowproves \overline{y : \tau} \flowprovesout \Delta
    }{ \Gamma \flowproves \curlys{\overline{x : \tau \mapsto y}} \flowprovesout \Delta }

    \inferrule*[right=Check-In]{
        \Gamma \proves s~\selects~\demote(\sigma)
        \and
        \Gamma \proves x :: \tau \flowsto_\mathcal{Q} \sigma
    }{ \Gamma \proves (s~\varin~x) : \boolt \flowprovesout \Gamma }

    \inferrule*[right=View-Call]{
        (\view~m(\overline{a : \tau})~\returns~\sigma := E) \in \decls(C)
        \and
        \Gamma, x : C \flowproves \overline{y : \tau} \flowprovesout \Gamma, x : C
    }{ \Gamma, x : C \flowproves x.m(\overline{y}) : \sigma \flowprovesout \Gamma, x : C }

    % \inferrule*[right=Interface-Tx-Call]{
    %     \interface~I~\{ \overline{\mathcal{D_I}} \}
    %     \and
    %     (\transaction~m(\overline{a : \tau})~\returns~\sigma) \in \overline{\mathcal{D}_I}
    %     \and
    %     \Gamma \flowproves \overline{y : \tau} \flowprovesout \Gamma
    % }{ \Gamma, x : I \flowproves x.m(\overline{y}) : \sigma \flowprovesout \Gamma, x : C }

    \inferrule*[right=Internal-Tx-Call]{
        \dom(\fields(C)) \cap \dom(\Gamma) = \emptyset
        \\
        (\transaction~m(\overline{a : \tau})~\returns~\sigma~\doC~S) \in \decls(C)
        \and
        \Gamma, \this : C \flowproves \overline{y : \tau} \flowprovesout \Delta, \this : C
    }{ \Gamma, \this : C \flowproves \this.m(\overline{y}) : \sigma \flowprovesout \Delta, \this : C }

    \inferrule*[right=If-Expr]{
        \Gamma \flowproves x : \boolt \flowprovesout \Gamma
        \and
        \Gamma \flowproves E_1 : \tau \flowprovesout \Delta
        \and
        \Gamma \flowproves E_2 : \tau \flowprovesout \Xi
    }{ \Gamma \flowproves (\ifS~x~\thenS~E_1~\elseS~E_2) : \tau \flowprovesout \Delta \vee \Xi }

    \inferrule*[right=Let]{
        \Gamma \flowproves E_1 : \tau \flowprovesout \Delta
        \and
        \Delta, x : \tau \flowproves E_2 : \pi \flowprovesout \Xi, x : \sigma
        \and
        \lnot(\sigma~\asset)
    }{ \Gamma \flowproves (\letvar~x : \tau := E_1~\varin~E_2) : \pi \flowprovesout \Xi }
\end{mathpar}

\framebox{$\Gamma \proves \mathcal{S} :: \tau \flowsto_{\mathcal{Q}} \sigma$} \textbf{Storage Typing}
The syntax $\tau \flowsto_{\mathcal{Q}} \sigma$ means that the storage accepts $\tau$ and provides $\sigma$; that is, you can flow $\tau$ into it, and when you flow out of it, you get $\sigma$; moreover, you get at most $\mathcal{Q}$ of them.
\begin{mathpar}
    \inferrule*[right=Store-One]{
    }{ \Gamma, S : \mathcal{Q}~T \proves S :: \mathcal{R}~T \flowsto_{\exactlyone} \mathcal{Q}~T }

    \inferrule*[right=Store-Col]{
    }{ \Gamma, S : \mathcal{Q}~\mathcal{C}~\tau \proves S :: \tau \flowsto_{\mathcal{Q}} \tau }

    \inferrule*[right=Store-Transformer]{
    }{ \Gamma, x : \mathcal{Q}~(\tau \transforms \sigma) \proves x :: \tau \flowsto_{\mathcal{Q}} \sigma }

    \inferrule*[right=Store-Consume]{
        \tau~\consumable
    }{ \Gamma \proves \consume :: \tau \flowsto_{\emptyq} \voidt }

    \inferrule*[right=Store-New]{
        \fields(C) = \overline{\this.f : \tau}
    }{ \Gamma \proves \newc~C :: \curlys{\overline{\this.f : \tau}} \flowsto_{\every} ~ C }

    \inferrule*[right=Store-New-Ty]{
        (\type~t~\is~\overline{M}~T) \in \decls(C)
    }{ \Gamma, \this : C \proves \newc~t :: \demote(T) \flowsto_{\every} ~ t }
\end{mathpar}

\reed{Maybe $\newc$ should just be an expression...}

\framebox{$\Gamma \proves s ~\selects_\mathcal{Q}~ \tau$} \textbf{Selectors}
\begin{mathpar}
    \inferrule*[right=Select-One]{
        \Gamma \flowproves x : \tau \flowprovesout \Gamma
    }{ \Gamma \proves x~\selects_{\exactlyone}~\tau }

    \inferrule*[right=Select-Col]{
        \Gamma \flowproves x : \mathcal{Q}~\mathcal{C}~\tau \flowprovesout \Gamma
    }{ \Gamma \proves x~\selects_{\mathcal{Q}}~\tau }

    \inferrule*[right=Select-Everything]{
    }{ \Gamma \proves \everything~\selects_\every~\tau }

    \inferrule*[right=Select-Quant]{
        \Gamma, x : \tau \flowproves p : \boolt \flowprovesout \Gamma, x : \tau
    }{ \Gamma \proves (\mathfrak{q}~x : \tau~\suchthat~p)~\selects_\mathfrak{q}~\tau }
\end{mathpar}

\framebox{$\Gamma \flowproves S~\ok \flowprovesout \Delta$} \textbf{Statement Well-formedness}
\reed{Use another word instead of well-formedness?}
\begin{mathpar}
    \inferrule*[right=Wf-Flow]{
        \Gamma \flowproves A :: \tau \flowsto_\mathcal{Q} \sigma
        \and
        \Gamma \flowproves s~\selects_\mathcal{R}~\demote(\sigma)
        \\
        \Delta = \update(\Gamma, A, \Gamma(A) \ominus \mathcal{R})
        \and
        \Delta \proves B :: \sigma \flowsto_\mathcal{S} \pi
    }{ \Gamma \flowproves (A \sends{s} B)~\ok \flowprovesout \update(\Delta, B, \Delta(B) \oplus \min(\mathcal{Q}, \mathcal{R})) }

    \inferrule*[right=Wf-Var-Def]{
        \Gamma \flowproves E : \tau \flowprovesout \Delta
        \and
        \Delta, x : \tau \flowproves S~\ok \flowprovesout \Xi, x : \sigma
        \and
        \lnot(\sigma~\asset)
    }{ \Gamma \flowproves (\var~x : \tau := E~\varin~S)~\ok \flowprovesout \Xi }

    \inferrule*[right=Wf-If]{
        \Gamma \flowproves x : \boolt \flowprovesout \Gamma
        \and
        \Gamma \flowproves S_1~\ok \flowprovesout \Delta
        \and
        \Gamma \flowproves S_2~\ok \flowprovesout \Xi
    }{ \Gamma \flowproves (\ifS~x~\thenS~S_1~\elseS~S_2)~\ok \flowprovesout \Delta \vee \Xi }

    \inferrule*[right=Wf-Try]{
        \Gamma \flowproves S_1~\ok \flowprovesout \Delta
        \and
        \Gamma, x : \tau \flowproves S_2~\ok \flowprovesout \Xi, x : \tau
    }{ \Gamma \flowproves (\tryS~S_1~\catchS~(x : \tau)~S_2)~\ok \flowprovesout \Delta \vee \Xi }

    \inferrule*[right=Wf-Revert]{
        \Gamma \flowproves E : \tau \flowprovesout \Gamma
    }{ \Gamma \flowproves \revert(E)~\ok \flowprovesout \Gamma }

    \inferrule*[right=Wf-Expr]{
        \Gamma \flowproves E : \tau \flowprovesout \Delta
        \and
        \lnot(\tau~\asset)
    }{ \Gamma \flowproves E~\ok \flowprovesout \Delta }

    \inferrule*[right=Wf-Seq]{
        \Gamma \flowproves S_1~\ok \flowprovesout \Delta
        \and
        \Delta \flowproves S_2~\ok \flowprovesout \Xi
    }{ \Gamma \flowproves (S_1 ; S_2)~\ok \flowprovesout \Xi }

    \inferrule*[right=Wf-Unpack]{
        \this.f : \tau \in \fields(C)
    }{ \Gamma, \this : C \flowproves \unpack(f)~\ok \flowprovesout \Gamma, \this : C, \this.f : \tau }

    \inferrule*[right=Wf-Pack]{
        (\Gamma|_{\dom(\fields(C))}) \leq \fields(C)
        \and
        \Delta = \setbuild{x : \tau \in \Gamma}{x \not\in \dom(\fields(C))}
    }{ \Gamma, \this : C \flowproves \pack~\ok \flowprovesout \Delta, \this : C }
\end{mathpar}

\framebox{$\proves_C \mathcal{D}~\ok$} \textbf{Declaration Well-formedness}
\begin{mathpar}
    \inferrule*[right=Wf-View]{
        \Gamma = \this : C, \fields(C), \overline{x : \tau}
        \and
        \Gamma \flowproves E : \sigma \flowprovesout \Gamma
    }{ \proves_C (\view~m(\overline{x : \tau})~\returns~\sigma := E)~\ok }

    \inferrule*[right=Wf-Tx]{
        \this : C, \fields(C), \overline{x : \tau}, y : \emptyq~T \flowproves S~\ok \flowprovesout \Delta, \this : C, y : \mathcal{Q}~T
        \\
        \dom(\fields(C)) \cap \dom(\Delta) = \emptyset
        \and
        \forall x : \tau \in \Delta. \lnot(\tau~\asset)
        % \and
        % \lnot(\mathcal{Q}~T~\asset)
    }{ \proves_C (\transaction~m(\overline{x : \tau})~\returns~y : \mathcal{Q}~T~\doC~S)~\ok }

    \inferrule*[right=Wf-Field]{
    }{ \proves_C (x : \tau)~\ok }

    \inferrule*[right=Wf-Type]{
        T~\asset \implies \asset \in \overline{M}
    }{ \proves_C (\type~t~\is~\overline{M}~T)~\ok }
\end{mathpar}

\framebox{$\Con~\ok$} \textbf{Contract Well-formedness}
\begin{mathpar}
    \inferrule*[right=Wf-Con]{
        \forall d \in \overline{\mathcal{D}}. (\proves_C d~\ok)
    }{ (\contract~C~\{ \overline{\mathcal{D}} \})~\ok }
\end{mathpar}

\framebox{$\Prog~\ok$} \textbf{Program Well-formedness}
\begin{mathpar}
    \inferrule*[right=Wf-Con]{
        \forall C \in \overline{\Con}. C~\ok
        \and
        \emptyset \flowproves S \flowprovesout \emptyset
    }{ (\overline{\Con} ; S)~\ok }
\end{mathpar}

\paragraph{Other Auxiliary Definitions}
\reed{Eliminate all the locations except for $x$ and then use flows to extract and put stuff back?}
\framebox{$\modifiers(T) = \overline{M}$} \textbf{Type Modifiers}
\begin{align*}
    \modifiers(T) =
    \begin{cases}
        \overline{M} & \tif (\type~T~\is~\overline{M}~T) \\
        \emptyset & \owise
    \end{cases}
\end{align*}

\framebox{$\demote(\tau) = \sigma$}
\framebox{$\demote_*(T_1) = T_2$} \textbf{Type Demotion}
\begin{align*}
    \demote(\mathcal{Q}~T) &= \mathcal{Q}~\demote_*(T) \\
    \demote_*(\voidt) &= \voidt \\
    \demote_*(\natt) &= \natt \\
    \demote_*(\boolt) &= \boolt \\
    \demote_*(t) &= \demote_*(T) & \twhere \type~t~\is~\overline{M}~T \\
    \demote_*(\mathcal{C}~\tau) &= \mathcal{C}~\demote(\tau) \\
    \demote_*(\curlys{\overline{x : \tau}}) &= \curlys{\overline{x : \demote(\tau)}} \\
    \demote_*(\tau \transforms \sigma) &= \demote(\tau) \transforms \demote(\sigma)
\end{align*}

\framebox{$\decls(C) = \overline{\mathcal{D}}$} \textbf{Contract Declarations}
\begin{align*}
    \decls(C) = \overline{\mathcal{D}} \twhere (\contract~C~\{ \overline{\mathcal{D}} \})
\end{align*}

\framebox{$\fields(C) = \Gamma$} \textbf{Contract Fields}
\begin{align*}
    \fields(C) = \setbuild{\this.f : \tau}{f : \tau \in \decls(C)}
\end{align*}

\framebox{$\update(\Gamma, \mathcal{S}, \tau)$} \textbf{Type environment modification}
% These auxiliary functions are used to update the statically known information about a storage after flowing in or out of it (see the \textsc{Flow} rule).
\[
    \update(\Gamma, S, \tau) =
    \begin{cases}
        \Delta, S : \tau & \tif \Gamma = \Delta, S : \sigma \\
        \Gamma & \owise
    \end{cases}
\]

\reed{Asset retention theorem?}
\reed{Resource accessiblity?}

\reed{What guarantees should we provide (no errors except for flowing a resource that doesn't exist in the source/already exists in the destination)?}

\section{Introduction}

\langName is a DSL for implementing programs which manage resources, targeted at writing smart contracts.

\subsection{Contributions}

We make the following main contributions:
\begin{itemize}
    \item \textbf{Safety guarantees}: similar to \reed{or maybe just exactly} linear types\reed{or maybe uniqueness types? need to read more about this}, preventing accidental resource loss or duplication.
        Additionally, provides some amount of reentrancy safety.

        \begin{itemize}
            \item We can evaluate these by formalizing the language and proving them; the formalization is something that would be nice to do anyway.
        \end{itemize}

    \item \textbf{Simplicity}: The language is quite simple---it makes writing typical smart contract programs easier and shorter, because many common pitfalls in Solidity are automatically handled by the language, such as overflow/underflow, checking of balances, short address attacks, etc.

        \begin{itemize}
            \item We can evaluate these by comparing LOC, cyclomatic complexity, etc.
                Not sure what the right metric would be.
                \reed{Or how cyclomatic complexity would work exactly in this language.}

            \item We can also evaluate via a user study, but that will take a long(er) time.
        \end{itemize}

    \item \reed{Optimizations?}
        Some of the Solidity contracts are actually inefficient because:
        \begin{enumerate}
            \item They use lots of modifiers which repeat checks (see reference implementation of ERC-721).
            \item They tend to use arrays to represent sets.
                Maybe this is more efficient for very small sets, but checking containment is going to be much faster with a \lstinline{mapping (X => bool)} eventually.
        \end{enumerate}

        \begin{itemize}
            \item We can evaluate this by profiling or a simple opcode count (which is not only a proxy for performance, but also means that deploying the contract will be cheaper).
        \end{itemize}
\end{itemize}


\section{Language Intro}

The basic state-changing construct in the language is a \emph{flow}.
A flow describes a transfer of a \emph{resource} from one \emph{storage} to another.
A \emph{transaction} is a sequence of statements.

Each flow has a \emph{source}, a \emph{destination}, and a \emph{selector}.
The source and destination are two storages which hold a resource, and the selector describes which part of the resource in the source should be transferred to the destination.
A flow may optionally have a \emph{name}.

Note that all flows fail if they can't be performed.
For example, a flow of fungible resources fails if there is enough of the resource, and a flow of a nonfungible resource fails if the selected value doesn't exist in the source location.

NOTE: If we wanted to be "super pure", we can implement preconditions with just flows by doing something like:
\begin{lstlisting}
{ contractCreator = msg.sender } --[ true ]-> consume
\end{lstlisting}
This works because \lstinline|{ contractCreator = msg.sender } : set bool| (specifically, a singleton), so if \lstinline{contractCreator = msg.sender} doesn't evaluate to true, then we will fail to consume true from it.
\reed{I don't think actually doing this is a good idea; at least, not in the surface language.
Maybe it would simplify the compiler and/or formalization, but it's interesting/entertaining.}

\paragraph{The DAO attack}
We can prevent the DAO attack (the below is from \url{https://consensys.github.io/smart-contract-best-practices/known_attacks/}):
\begin{lstlisting}
function withdrawBalance() public {
    uint amountToWithdraw = userBalances[msg.sender];
    // At this point, the caller's code is executed, and can call withdrawBalance again
    require(msg.sender.call.value(amountToWithdraw)(""));
    userBalances[msg.sender] = 0;
}
\end{lstlisting}

In \langName, we would write this as:

\begin{lstlisting}
transaction withdrawBalance():
    userBalances[msg.sender] --> msg.sender.balance
\end{lstlisting}

Not only is this simpler, but the compiler can automatically place the actual call that does the transfer last, meaning that the mistake could simply never be made.

\section{Random Thoughts}
\reed{Ignore this for now, just some stuff for fun that may or may not end up being useful.}
\reed{At some point, maybe for antoher language, would like to figure out the more general type quantity thing where you can freely combine $\setq$ and $\listq$ and $\nonempty$ and $\one$, etc.}
\reed{We could also add sum types and ``sum type quantities'' so you can say \lstinline{x : Left (a + b)}.
Something like $\curlys{\ell, r}$ where $\ell$ and $r$ are state specifiers.
Probably would work best with labeled sum types.}
\reed{Add at ``at most one'' quantity, which makes the \typeQuantities into a group?}
\reed{Free group of quantities list set and option makes teh type quantities we care about?
What are the inverses?}

\begin{align*}
    \nonempty &= \optionq^{-1} \\
\end{align*}
% Note: Consider $\tau \equiv \one~\tau$, so $x : \natt$ and $x : \one~\natt$ are the same (and so is $x : \one~\one~\natt$, etc.).
% Additionally, $\nonempty~\emptyq \equiv \one$, $\nonempty~\nonempty \equiv \nonempty$, \reed{$\emptyq~\nonempty \equiv \emptyq$?}

\begin{definition}
    \reed{Don't think this really has a good name}
    A \emph{resource} $\mathcal{R}$ is a tuple $(R, +, 0, \leq, -)$ where
    \begin{enumerate}[label=(\roman*)]
        \item $(R, +, 0)$ is a monoid.
        \item $(R, \leq)$ is a partial order that is compatible with $(R, +, 0)$.
            That is, for any $x, y \in R$ such that $x \leq y$, and for any $z \in R$, we have $x + z \leq y + z$.
        \item $- : R \times R \to R$ is a function so that for any $x,y \in R$ such that $y \leq x$, we have $(x - y) + y = x$.

        % NOTE: An alternate definition with splitting built-in.
        % \item $\splits$ is a function $R \times \parens{R \cup \curlys{\top}}^R \to R \times R$, where $\top$ is a value not in $R$ for which we define $r < \top$ for all $r \in R$.
        %     We write $r \stackrel{f}{\splits} (a,b)$ instead of $\splits(r, f) = (a,b)$.
        %     Let $f : R \to R \cup \curlys{\top}$ and $r \in R$.
        %     If $f(r) \leq r$, then $r \stackrel{f}{\splits} (a,b)$ such that $f(r) = b$ and $a + b = r$.
    \end{enumerate}
\end{definition}

\paragraph{Examples}

\begin{enumerate}
    \item The natural numbers with the standard operations is a resource $(\N, +, 0, \leq, -)$, where $-$ is \emph{saturating subtraction}: $n - m = 0$ if $m > n$.
    \item For any set $A$, we can build the resource $(\powerset{A}, \cup, \emptyset, \subseteq, \setminus)$, where $X \setminus Y$ is the set difference operation.
    \item Similarly, given any set $A$, we can build the resource $(\powersetfin{A}, \cup, \emptyset, \subseteq, \setminus)$; this resource is called the \emph{nonfungible resource on $A$}, written $\text{nf}(A)$.
    \item \reed{I have no idea if this is useful, but it's fun}
        The set of strings on an alphabet $\Sigma$ can be made into the \emph{prefix resource} $(\Sigma^*, \epsilon, \cdot, \leq_p, -_p)$ or the \emph{suffix resource} $(\Sigma^*, \epsilon, \cdot, \leq_s, -_s)$, where $\cdot$ is concatenation, $x \leq_p y$ if $x$ is a prefix of $y$ and similarly $x \leq_s y$ if $x$ is a suffix of $y$.
        The functions $-_p$ and $-_s$ are defined as follows:
        \[
            (x \cdot y) -_p x = y
        \]
        and
        \[
            (x \cdot y) -_s y = x
        \]
\end{enumerate}

\begin{definition}
    For any two resources $\mathcal{R}$ and $\mathcal{S}$, their \emph{direct product}, $\mathcal{R} \oplus \mathcal{S}$ is $(R \times S, +_{RS}, (0_R, 0_S), \leq_{RS}, -_{RS})$, where
    \begin{align*}
        (r_1, s_1) +_{RS} (r_2, s_2) &:= (r_1 +_R r_2, s_1 +_S s_2) \\
        (r_1, s_1) -_{RS} (r_2, s_2) &:= (r_1 -_R r_2, s_1 -_S s_2) \\
        (r_1, s_1) \leq_{RS} (r_2, s_2) &:\iff r_1 \leq_R r_2 \land s_1 \leq_S s_2
    \end{align*}
\end{definition}

\begin{definition}
    Let $\mathcal{R}$ be a resource and let $A$ be a set.
    The exponential resource $\mathcal{R}^A$ \reed{I suspect this would be an exponential, anyway} is defined so
    \[
        \mathcal{R}^A := (R^A, +, \bm{0}, \leq, -)
    \]
    where for $f,g \in R^A$, we define
    \begin{align*}
        \bm{0}(x) &:= 0 \\
        (f + g)(x) &:= f(x) + g(x) \\
        f \leq g &:\Leftrightarrow \forall x \in A. f(x) \leq g(x) \\
        (f - g)(x) &:= f(x) - g(x)
    \end{align*}

    Note that $f - g$ is only defined when $g \leq f$.
\end{definition}

\end{document}

