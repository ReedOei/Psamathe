\documentclass[leqno,presentation,usenames,dvipsnames]{beamer}
\DeclareGraphicsExtensions{.eps,.jpg,.png,.tif}
\usepackage{amssymb, amsmath, pdfpages, amsfonts, calc, times, type1cm, latexsym, xcolor, colortbl, hyperref, bookmark}
\usepackage{mathtools}
\usepackage{graphicx}
\usepackage{tabularx}
\usepackage{multirow}
\usepackage{listings}
\usepackage{mathpartir}
\usepackage{xspace}
\usepackage{tipa}

\usepackage[latin1]{inputenc}
\usepackage[english]{babel}

\usetheme{Szeged}
\usecolortheme{beaver}

\definecolor{websiteGreen}{RGB}{107, 224, 134}
\definecolor{silvery}{RGB}{232, 241, 248}
\definecolor{deepOrange}{RGB}{209, 126, 0}

\definecolor{softTan}{RGB}{240, 240, 223}
\definecolor{deepGreen}{RGB}{87, 149, 115}
\definecolor{lilac}{RGB}{199, 164, 202}
\definecolor{lightOlive}{RGB}{168, 166, 96}
\definecolor{deepOlive}{RGB}{101, 109, 41}

\definecolor{lightgray}{RGB}{245, 246, 250}
\definecolor{blue}{RGB}{64, 115, 158}
\definecolor{darkblue}{RGB}{39, 60, 117}
\definecolor{lavender}{RGB}{200,160,230}
\definecolor{darkLavender}{RGB}{160,120,200}
\definecolor{veryDarkLavender}{RGB}{130,80,160}
\definecolor{websiteGreen}{RGB}{107,224,134}
\definecolor{darkWebsiteGreen}{RGB}{53,112,67}
\definecolor{softRed}{RGB}{180,70,99}
\definecolor{vividRed}{RGB}{220,49,72}
\definecolor{beige}{RGB}{252,222,216}
\definecolor{rusticRed}{RGB}{51,9,17}

\setbeamercolor*{palette primary}{bg=softRed}
\setbeamercolor*{palette secondary}{bg=white}
\setbeamercolor*{palette tertiary}{bg=lightgray}
\setbeamercolor*{palette quaternary}{bg=softRed}

\setbeamercolor{section in head/foot}{fg=softRed}
\setbeamerfont{section in head/foot}{series=\bfseries}

\setbeamercolor{titlelike}{parent=palette primary,bg=lightgray,fg=softRed}
\setbeamercolor{frametitle}{parent=palette primary,bg=lightgray, fg=softRed}

\setbeamercolor{itemize item}{fg=lavender}
\setbeamercolor{itemize subitem}{fg=lavender}
\setbeamercolor{itemize subsubitem}{fg=lavender}

\setbeamertemplate{itemize item}[circle]
\setbeamertemplate{itemize subitem}[triangle]
\setbeamertemplate{itemize subsubitem}[square]

\addtobeamertemplate{frametitle}{\vspace*{-0.65\baselineskip}}{}

\newcommand{\Int}{\textbf{Int}\xspace}
\newcommand{\overbar}[1]{\mkern1.5mu\overline{\mkern-1.5mu#1\mkern-1.5mu}\mkern1.5mu}
\newcommand{\highlight}[1]{
  \addtolength{\fboxrule}{.2ex}
  \begin{block}{}
    \begin{quote}#1
    \end{quote}
  \end{block}
}

\newtheorem*{conjecture}{Conjecture}
\newtheorem*{proposition}{Proposition}

\title{Psamathe: A DSL with Flows for Safe Blockchain Assets}
\author{\textcolor{lavender}{\textbf{Reed Oei}}\inst{1} \and Michael Coblenz \inst{2} \and Jonathan Aldrich\inst{3}}
\institute[UIUC, UMD, CMU]{\inst{1} University of Illinois, Urbana, IL, USA\\
\url{reedoei2@illinois.edu}%
\and \inst{2} University of Maryland, College Park, MD, USA\\
\url{mcoblenz@umd.edu}%
\and \inst{3} Carnegie Mellon University, Pittsburgh, PA, USA\\
\url{jonathan.aldrich@cs.cmu.edu}}
\date{\small January 19, 2021}

\definecolor{verylightgray}{rgb}{.97,.97,.97}

\lstdefinelanguage{flow}{
    keywords=[1]{var,interface,where,contract,transaction,view,transformer,type,is,returns},
    keywordstyle=[1]\color{Rhodamine}\bfseries,
    keywords=[2]{false,true,sometimes,fungible,asset,unique,immutable,consumable},
    keywordstyle=[2]\color{DarkOrchid}\bfseries,
    keywords=[3]{map,linking,set,list,address,uint256,string,nat,bool},
    keywordstyle=[3]\color{Emerald}\itshape,
    keywords=[4]{nonempty,any,exactly,one,!,msg,this,in,new,or,and,not,total,consume},
    keywordstyle=[4]\color{Lavender}\itshape,
    keywords=[5]{if,else,only,when,try,catch,such,that},
    keywordstyle=[5]\color{Cerulean}\bfseries,
    keepspaces=true,
	backgroundcolor=\color{verylightgray},
    columns=fullflexible,
    % literate=*
    %     {(}{{{\color{red}(}}}1
    %     {)}{{{\color{red})}}}1
    %     {+}{{{\color{red}+~}}}1
    %     {-}{{{\color{red}-~}}}1
    %     {:=}{{{\color{red}:=~}}}1
    %     {\{}{{{\color{red}\{}}}1
    %     {\}}{{{\color{red}\}}}}1
    %     {[}{{{\color{red}[}}}1
    %     {]}{{{\color{red}]}}}1
    %     {*}{{{\color{red}*~}}}1
    %     {:}{{{\color{red}:~}}}1
    %     {>}{{{\color{red}>~}}}1
    %     {<}{{{\color{red}<~}}}1
    %     {<=>}{{{$\color{red}\Leftrightarrow~$}}}1
    %     {.}{{{\color{red}.~}}}1
    %     {!}{{{$\color{red}!~$}}}1
    %     {=}{{{$\color{red}=~$}}}1
    %     {exists }{{{$\color{red}\exists$}}}1
    %     {forall }{{{$\color{red}\forall$}}}1,
    sensitive=false, % keywords are not case-sensitive
    morecomment=[l]{//}, % l is for line comment
    morecomment=[s]{/*}{*/}, % s is for start and end delimiter
    commentstyle=\color{CadetBlue}\textit,
    morestring=[b]", % defines that strings are enclosed in double quotes
    stringstyle=\color{ForestGreen}, % string literal style
    showstringspaces=false,
    xleftmargin=5.0ex,
    mathescape=true,
    escapechar=|
}

\lstnewenvironment{flow}
{%
    \lstset{language=flow, basicstyle=\normalsize\ttfamily, mathescape=true}%
}
{
}

\newcommand{\flowinline}[1]{\lstinline[language=flow,basicstyle=\normalsize\ttfamily,mathescape]{#1}}


% Copyright 2017 Sergei Tikhomirov, MIT License
% https://github.com/s-tikhomirov/solidity-latex-highlighting/

\usepackage{listings, xcolor}

\definecolor{verylightgray}{rgb}{.97,.97,.97}

\lstdefinelanguage{Solidity}{
	keywords=[1]{anonymous, assembly, balance, call, callcode, class, constant, constructor, contract, debugger, delegatecall, delete, emit, event, experimental, export, external, function, gas, implements, import, in, indexed, instanceof, interface, internal, is, length, library, log0, log1, log2, log3, log4, memory, modifier, new, payable, pragma, private, protected, public, pure, push, return, returns, selfdestruct, send, solidity, storage, struct, suicide, super, then, throw, typeof, using, view, with, addmod, ecrecover, keccak256, mulmod, ripemd160, sha256, sha3}, % generic keywords including crypto operations
	keywordstyle=[1]\color{Rhodamine}\bfseries,
	keywords=[2]{address, bool, byte, bytes, bytes1, bytes2, bytes3, bytes4, bytes5, bytes6, bytes7, bytes8, bytes9, bytes10, bytes11, bytes12, bytes13, bytes14, bytes15, bytes16, bytes17, bytes18, bytes19, bytes20, bytes21, bytes22, bytes23, bytes24, bytes25, bytes26, bytes27, bytes28, bytes29, bytes30, bytes31, bytes32, enum, int, int8, int16, int24, int32, int40, int48, int56, int64, int72, int80, int88, int96, int104, int112, int120, int128, int136, int144, int152, int160, int168, int176, int184, int192, int200, int208, int216, int224, int232, int240, int248, int256, mapping, string, uint, uint8, uint16, uint24, uint32, uint40, uint48, uint56, uint64, uint72, uint80, uint88, uint96, uint104, uint112, uint120, uint128, uint136, uint144, uint152, uint160, uint168, uint176, uint184, uint192, uint200, uint208, uint216, uint224, uint232, uint240, uint248, uint256, var, void, ether, finney, szabo, wei, days, hours, minutes, seconds, weeks, years},	% types; money and time units
	keywordstyle=[2]\color{Emerald}\itshape,
	keywords=[3]{block, balance, blockhash, coinbase, difficulty, gaslimit, number, timestamp, msg, data, gas, sender, sig, now, tx, gasprice, origin, this},	% sepcial identifiers
	keywordstyle=[3]\color{Lavender}\itshape,
	keywords=[4]{assert, break, case, catch, continue, default, else, false, finally, for, if, require, revert, switch, throw, true, try, while, do}, % Control-flow
	keywordstyle=[4]\color{Cerulean}\bfseries,
	identifierstyle=\color{black},
	sensitive=false,
    keepspaces=true,
    columns=fullflexible,
    % literate=*
    %     {(}{{{\color{red}(}}}1
    %     {)}{{{\color{red})}}}1
    %     {+}{{{\color{red}+~}}}1
    %     {-}{{{\color{red}-~}}}1
    %     {:=}{{{\color{red}:=~}}}1
    %     {\{}{{{\color{red}\{}}}1
    %     {\}}{{{\color{red}\}}}}1
    %     {[}{{{\color{red}[}}}1
    %     {]}{{{\color{red}]}}}1
    %     {|}{{{\color{red}|}}}1
    %     {;}{{{\color{red};}}}1
    %     {*}{{{\color{red}*~}}}1
    %     {:}{{{\color{red}:~}}}1
    %     {>}{{{\color{red}>~}}}1
    %     {<}{{{\color{red}<~}}}1
    %     {<=>}{{{$\color{red}\Leftrightarrow~$}}}1
    %     {.}{{{\color{red}.~}}}1
    %     {!}{{{$\color{red}!~$}}}1
    %     {=}{{{$\color{red}=~$}}}1
    %     {exists }{{{$\color{red}\exists$}}}1
    %     {forall }{{{$\color{red}\forall$}}}1,
	comment=[l]{//},
	morecomment=[s]{/*}{*/},
    commentstyle=\color{CadetBlue}\textit,
	stringstyle=\color{ForestGreen},
	morestring=[b]',
	morestring=[b]",
    xleftmargin=5.0ex,
    extendedchars=true,
	basicstyle=\small\ttfamily,
	showstringspaces=false,
	showspaces=false,
	tabsize=2,
	breaklines=true,
	showtabs=false,
	captionpos=b
}


\lstset{
  frame=none,
  xleftmargin=2pt,
  stepnumber=1,
  numbers=left,
  numbersep=5pt,
  numberstyle=\ttfamily\tiny\color[gray]{0.3},
  belowcaptionskip=\bigskipamount,
  captionpos=b,
  escapeinside={*'}{'*},
  tabsize=2,
  emphstyle={\bf},
  commentstyle=\it,
  stringstyle=\mdseries\rmfamily,
  showspaces=false,
  keywordstyle=\bfseries\rmfamily,
  columns=flexible,
  basicstyle=\small\sffamily,
  showstringspaces=false,
}

\renewcommand{\setbuild}[2]{\ensuremath{\left\{ {#1} ~|~ {#2} \right\}}}
\renewcommand{\true}{\textbf{\texttt{true}}\xspace}
\renewcommand{\false}{\textbf{\texttt{false}}\xspace}
% \Leftrightarrow is shorter than \iff
\renewcommand{\iff}{\Leftrightarrow}

% \newcommand{\langName}{LANGUAGE-NAME\xspace}
\newcommand{\langName}{Psamathe\xspace}

\newcommand{\assetTxt}{asset\xspace}
\newcommand{\AssetTxt}{Asset\xspace}

\newcommand{\typeQuantity}{type quantity\xspace}
\newcommand{\typeQuantities}{type quantities\xspace}

\newcommand{\guard}{\bigg|\bigg|}
\newcommand{\newc}{\textbf{\texttt{new}}}
\newcommand{\everything}{\textbf{\texttt{everything}}}
\newcommand{\nothing}{\textbf{\texttt{nothing}}}
\newcommand{\then}{\textbf{\texttt{then}}}
\newcommand{\this}{\textbf{\texttt{this}}}
\newcommand{\fields}{\textbf{\texttt{fields}}}
\newcommand{\decls}{\textbf{\texttt{decls}}}
\newcommand{\consumes}[1]{\stackrel{#1}{\consumesarr}}
\newcommand{\sends}[1]{\stackrel{#1}{\to}}

\newcommand{\asset}{\textbf{\texttt{asset}}\xspace}
\newcommand{\fungible}{\textbf{\texttt{fungible}}\xspace}
\newcommand{\nonfungible}{\textbf{\texttt{nonfungible}}\xspace}
\newcommand{\consumable}{\textbf{\texttt{consumable}}\xspace}
\newcommand{\immutable}{\textbf{\texttt{immutable}}\xspace}
\newcommand{\unique}{\textbf{\texttt{unique}}\xspace}

\newcommand{\dom}{\textbf{\texttt{dom}}\xspace}

\newcommand{\compat}{\leftrightarrow}

\newcommand{\addresst}{\textbf{\texttt{address}}\xspace}
\newcommand{\stringt}{\textbf{\texttt{string}}\xspace}
\newcommand{\boolt}{\textbf{\texttt{bool}}\xspace}
\newcommand{\natt}{\textbf{\texttt{nat}}\xspace}
\newcommand{\voidt}{\textbf{\texttt{void}}\xspace}

\newcommand{\byt}{\textbf{\texttt{by}}\xspace}
\newcommand{\map}{\textbf{\texttt{map}}\xspace}
\newcommand{\mapitem}{\textbf{\texttt{mapitem}}\xspace}
\newcommand{\linking}{\textbf{\texttt{linking}}\xspace}
\newcommand{\link}{\textbf{\texttt{link}}\xspace}
\newcommand{\transformer}{\textbf{\texttt{transformer}}\xspace}

\newcommand{\stores}{\textbf{\texttt{stores}}\xspace}
\newcommand{\provides}{\textbf{\texttt{provides}}\xspace}
\newcommand{\accepts}{\textbf{\texttt{accepts}}\xspace}

\newcommand{\suchthat}{\textbf{s.t.}\xspace}
\newcommand{\returns}{\textbf{\texttt{returns}}\xspace}
\newcommand{\merge}{\rightsquigarrow}
\newcommand{\flowsto}{\Rrightarrow}
\newcommand{\heldby}{\rightarrowtail}

\newcommand{\ifS}{\textbf{\texttt{if}}\xspace}
\newcommand{\thenS}{\textbf{\texttt{then}}\xspace}
\newcommand{\elseS}{\textbf{\texttt{else}}\xspace}
\newcommand{\tryS}{\textbf{\texttt{try}}\xspace}
\newcommand{\catchS}{\textbf{\texttt{catch}}\xspace}

\newcommand{\contract}{\textbf{\texttt{contract}}\xspace}
\newcommand{\interface}{\textbf{\texttt{interface}}\xspace}
\newcommand{\oncreate}{\textbf{\texttt{on create}}\xspace}
\newcommand{\constructor}{\textbf{\texttt{constructor}}\xspace}

\newcommand{\Quant}{\textbf{Quant}\xspace}

\newcommand{\Con}{\textbf{\texttt{Con}}\xspace}
\newcommand{\Decl}{\textbf{\texttt{Decl}}\xspace}
\newcommand{\Trig}{\textbf{\texttt{Trig}}\xspace}
\newcommand{\Prog}{\textbf{\texttt{Prog}}\xspace}
\newcommand{\Stmt}{\textbf{\texttt{Stmt}}\xspace}

\newcommand{\type}{\textbf{\texttt{type}}\xspace}
\newcommand{\is}{\textbf{\texttt{is}}\xspace}

\newcommand{\one}{\textbf{\texttt{one}}\xspace}
\newcommand{\setq}{\textbf{\texttt{set}}\xspace}
\newcommand{\listq}{\textbf{\texttt{list}}\xspace}
\newcommand{\optionq}{\textbf{\texttt{option}}\xspace}

\newcommand{\single}{\textbf{\texttt{single}}\xspace}

\newcommand{\emptyq}{\textbf{\texttt{empty}}\xspace}
\newcommand{\atmostone}{?}
\newcommand{\nonempty}{\textbf{\texttt{nonempty}}\xspace}
\newcommand{\any}{\textbf{\texttt{any}}\xspace}
\newcommand{\every}{\textbf{\texttt{every}}\xspace}
\newcommand{\exactlyone}{!}

% Symbol versions of the type quantities:
% \newcommand{\emptyq}{\bm{0}}
% \newcommand{\atmostone}{?}
% \newcommand{\nonempty}{+}
% \newcommand{\some}{\_}
% \newcommand{\any}{*}
% \newcommand{\every}{\infty}
% \newcommand{\exactlyone}{!}

\newcommand{\elemtype}{\textbf{\texttt{elemtype}}\xspace}

\newcommand{\consume}{\textbf{\texttt{consume}}\xspace}
\newcommand{\emptyval}{\textbf{\texttt{emptyval}}\xspace}

\newcommand{\pack}{\textbf{\texttt{pack}}\xspace}
\newcommand{\unpack}{\textbf{\texttt{unpack}}\xspace}

\newcommand{\validSelect}{\textbf{\texttt{validSelect}}\xspace}

\newcommand{\total}{\textbf{\texttt{total}}\xspace}

\newcommand{\transaction}{\textbf{\texttt{transaction}}\xspace}
\newcommand{\private}{\textbf{\texttt{private}}\xspace}
\newcommand{\doC}{\textbf{\texttt{do}}\xspace}
\newcommand{\view}{\textbf{\texttt{view}}\xspace}

\newcommand{\emit}{\textbf{\texttt{emit}}\xspace}
\newcommand{\event}{\textbf{\texttt{event}}\xspace}
\newcommand{\revert}{\textbf{\texttt{revert}}\xspace}
\newcommand{\call}{\textbf{\texttt{call}}\xspace}
\newcommand{\asserting}{\textbf{\texttt{asserting}}\xspace}

\newcommand{\selects}{\textbf{\texttt{selects}}\xspace}

\newcommand{\modifiers}{\textbf{\texttt{modifiers}}\xspace}
\newcommand{\declaration}{\textbf{\texttt{declaration}}\xspace}

\newcommand{\select}{\textbf{\texttt{select}}\xspace}
\newcommand{\combine}{\textbf{\texttt{combine}}\xspace}
\newcommand{\update}{\textbf{\texttt{update}}\xspace}

\newcommand{\typeof}{\textbf{\texttt{typeof}}\xspace}

\newcommand{\ok}{\textbf{ok}\xspace}
\newcommand{\consumesarr}{\hskip -0.15em\mathrel{\rotatebox[origin=c]{270}{${\looparrowright}$}}}
\newcommand{\fillsarr}{\hskip -0.15em\mathrel{\rotatebox[origin=c]{270}{${\looparrowleft}$}}}

\newcommand{\demote}{\text{demote}\xspace}
\newcommand{\transforms}{\rightsquigarrow}
\newcommand{\orop}{\textbf{\texttt{or}}\xspace}
\newcommand{\andop}{\textbf{\texttt{and}}\xspace}
\newcommand{\var}{\textbf{\texttt{var}}\xspace}
\newcommand{\letvar}{\textbf{\texttt{let}}\xspace}
\newcommand{\varin}{\textbf{\texttt{in}}\xspace}
\newcommand{\from}{\textbf{\texttt{from}}\xspace}

% \newcommand{\flowproves}{\mathbin{|\hskip -0.1em\scalebox{0.7}[1]{$\leadsto$}}}
\newcommand{\flowproves}{\proves}
% \newcommand{\flowprovesout}{\mathbin{\scalebox{0.7}[1]{$\leadsto$}\hskip -0.13em|}}
\newcommand{\flowprovesout}{\dashv}


\lstset{
  frame=none,
  xleftmargin=2pt,
  stepnumber=1,
  numbers=left,
  numbersep=5pt,
  numberstyle=\ttfamily\tiny\color[gray]{0.3},
  belowcaptionskip=\bigskipamount,
  captionpos=b,
  escapeinside={*'}{'*},
  tabsize=2,
  emphstyle={\bf},
  commentstyle=\it,
  stringstyle=\mdseries\rmfamily,
  showspaces=false,
  keywordstyle=\bfseries\rmfamily,
  columns=flexible,
  basicstyle=\small\sffamily,
  showstringspaces=false,
}

\renewcommand{\setbuild}[2]{\ensuremath{\left\{ {#1} ~|~ {#2} \right\}}}
\renewcommand{\true}{\textbf{\texttt{true}}\xspace}
\renewcommand{\false}{\textbf{\texttt{false}}\xspace}
% \Leftrightarrow is shorter than \iff
\renewcommand{\iff}{\Leftrightarrow}

% \newcommand{\langName}{LANGUAGE-NAME\xspace}
\newcommand{\langName}{Psamathe\xspace}

\newcommand{\assetTxt}{asset\xspace}
\newcommand{\AssetTxt}{Asset\xspace}

\newcommand{\typeQuantity}{type quantity\xspace}
\newcommand{\typeQuantities}{type quantities\xspace}

\newcommand{\guard}{\bigg|\bigg|}
\newcommand{\newc}{\textbf{\texttt{new}}}
\newcommand{\everything}{\textbf{\texttt{everything}}}
\newcommand{\nothing}{\textbf{\texttt{nothing}}}
\newcommand{\then}{\textbf{\texttt{then}}}
\newcommand{\this}{\textbf{\texttt{this}}}
\newcommand{\fields}{\textbf{\texttt{fields}}}
\newcommand{\decls}{\textbf{\texttt{decls}}}
\newcommand{\consumes}[1]{\stackrel{#1}{\consumesarr}}
\newcommand{\sends}[1]{\stackrel{#1}{\to}}

\newcommand{\asset}{\textbf{\texttt{asset}}\xspace}
\newcommand{\fungible}{\textbf{\texttt{fungible}}\xspace}
\newcommand{\nonfungible}{\textbf{\texttt{nonfungible}}\xspace}
\newcommand{\consumable}{\textbf{\texttt{consumable}}\xspace}
\newcommand{\immutable}{\textbf{\texttt{immutable}}\xspace}
\newcommand{\unique}{\textbf{\texttt{unique}}\xspace}

\newcommand{\dom}{\textbf{\texttt{dom}}\xspace}

\newcommand{\compat}{\leftrightarrow}

\newcommand{\addresst}{\textbf{\texttt{address}}\xspace}
\newcommand{\stringt}{\textbf{\texttt{string}}\xspace}
\newcommand{\boolt}{\textbf{\texttt{bool}}\xspace}
\newcommand{\natt}{\textbf{\texttt{nat}}\xspace}
\newcommand{\voidt}{\textbf{\texttt{void}}\xspace}

\newcommand{\byt}{\textbf{\texttt{by}}\xspace}
\newcommand{\map}{\textbf{\texttt{map}}\xspace}
\newcommand{\mapitem}{\textbf{\texttt{mapitem}}\xspace}
\newcommand{\linking}{\textbf{\texttt{linking}}\xspace}
\newcommand{\link}{\textbf{\texttt{link}}\xspace}
\newcommand{\transformer}{\textbf{\texttt{transformer}}\xspace}

\newcommand{\stores}{\textbf{\texttt{stores}}\xspace}
\newcommand{\provides}{\textbf{\texttt{provides}}\xspace}
\newcommand{\accepts}{\textbf{\texttt{accepts}}\xspace}

\newcommand{\suchthat}{\textbf{s.t.}\xspace}
\newcommand{\returns}{\textbf{\texttt{returns}}\xspace}
\newcommand{\merge}{\rightsquigarrow}
\newcommand{\flowsto}{\Rrightarrow}
\newcommand{\heldby}{\rightarrowtail}

\newcommand{\ifS}{\textbf{\texttt{if}}\xspace}
\newcommand{\thenS}{\textbf{\texttt{then}}\xspace}
\newcommand{\elseS}{\textbf{\texttt{else}}\xspace}
\newcommand{\tryS}{\textbf{\texttt{try}}\xspace}
\newcommand{\catchS}{\textbf{\texttt{catch}}\xspace}

\newcommand{\contract}{\textbf{\texttt{contract}}\xspace}
\newcommand{\interface}{\textbf{\texttt{interface}}\xspace}
\newcommand{\oncreate}{\textbf{\texttt{on create}}\xspace}
\newcommand{\constructor}{\textbf{\texttt{constructor}}\xspace}

\newcommand{\Quant}{\textbf{Quant}\xspace}

\newcommand{\Con}{\textbf{\texttt{Con}}\xspace}
\newcommand{\Decl}{\textbf{\texttt{Decl}}\xspace}
\newcommand{\Trig}{\textbf{\texttt{Trig}}\xspace}
\newcommand{\Prog}{\textbf{\texttt{Prog}}\xspace}
\newcommand{\Stmt}{\textbf{\texttt{Stmt}}\xspace}

\newcommand{\type}{\textbf{\texttt{type}}\xspace}
\newcommand{\is}{\textbf{\texttt{is}}\xspace}

\newcommand{\one}{\textbf{\texttt{one}}\xspace}
\newcommand{\setq}{\textbf{\texttt{set}}\xspace}
\newcommand{\listq}{\textbf{\texttt{list}}\xspace}
\newcommand{\optionq}{\textbf{\texttt{option}}\xspace}

\newcommand{\single}{\textbf{\texttt{single}}\xspace}

\newcommand{\emptyq}{\textbf{\texttt{empty}}\xspace}
\newcommand{\atmostone}{?}
\newcommand{\nonempty}{\textbf{\texttt{nonempty}}\xspace}
\newcommand{\any}{\textbf{\texttt{any}}\xspace}
\newcommand{\every}{\textbf{\texttt{every}}\xspace}
\newcommand{\exactlyone}{!}

% Symbol versions of the type quantities:
% \newcommand{\emptyq}{\bm{0}}
% \newcommand{\atmostone}{?}
% \newcommand{\nonempty}{+}
% \newcommand{\some}{\_}
% \newcommand{\any}{*}
% \newcommand{\every}{\infty}
% \newcommand{\exactlyone}{!}

\newcommand{\elemtype}{\textbf{\texttt{elemtype}}\xspace}

\newcommand{\consume}{\textbf{\texttt{consume}}\xspace}
\newcommand{\emptyval}{\textbf{\texttt{emptyval}}\xspace}

\newcommand{\pack}{\textbf{\texttt{pack}}\xspace}
\newcommand{\unpack}{\textbf{\texttt{unpack}}\xspace}

\newcommand{\validSelect}{\textbf{\texttt{validSelect}}\xspace}

\newcommand{\total}{\textbf{\texttt{total}}\xspace}

\newcommand{\transaction}{\textbf{\texttt{transaction}}\xspace}
\newcommand{\private}{\textbf{\texttt{private}}\xspace}
\newcommand{\doC}{\textbf{\texttt{do}}\xspace}
\newcommand{\view}{\textbf{\texttt{view}}\xspace}

\newcommand{\emit}{\textbf{\texttt{emit}}\xspace}
\newcommand{\event}{\textbf{\texttt{event}}\xspace}
\newcommand{\revert}{\textbf{\texttt{revert}}\xspace}
\newcommand{\call}{\textbf{\texttt{call}}\xspace}
\newcommand{\asserting}{\textbf{\texttt{asserting}}\xspace}

\newcommand{\selects}{\textbf{\texttt{selects}}\xspace}

\newcommand{\modifiers}{\textbf{\texttt{modifiers}}\xspace}
\newcommand{\declaration}{\textbf{\texttt{declaration}}\xspace}

\newcommand{\select}{\textbf{\texttt{select}}\xspace}
\newcommand{\combine}{\textbf{\texttt{combine}}\xspace}
\newcommand{\update}{\textbf{\texttt{update}}\xspace}

\newcommand{\typeof}{\textbf{\texttt{typeof}}\xspace}

\newcommand{\ok}{\textbf{ok}\xspace}
\newcommand{\consumesarr}{\hskip -0.15em\mathrel{\rotatebox[origin=c]{270}{${\looparrowright}$}}}
\newcommand{\fillsarr}{\hskip -0.15em\mathrel{\rotatebox[origin=c]{270}{${\looparrowleft}$}}}

\newcommand{\demote}{\text{demote}\xspace}
\newcommand{\transforms}{\rightsquigarrow}
\newcommand{\orop}{\textbf{\texttt{or}}\xspace}
\newcommand{\andop}{\textbf{\texttt{and}}\xspace}
\newcommand{\var}{\textbf{\texttt{var}}\xspace}
\newcommand{\letvar}{\textbf{\texttt{let}}\xspace}
\newcommand{\varin}{\textbf{\texttt{in}}\xspace}
\newcommand{\from}{\textbf{\texttt{from}}\xspace}

% \newcommand{\flowproves}{\mathbin{|\hskip -0.1em\scalebox{0.7}[1]{$\leadsto$}}}
\newcommand{\flowproves}{\proves}
% \newcommand{\flowprovesout}{\mathbin{\scalebox{0.7}[1]{$\leadsto$}\hskip -0.13em|}}
\newcommand{\flowprovesout}{\dashv}



\begin{document}

\frame{\titlepage}

\begin{frame}{Introduction}
    \begin{itemize}
        \item Many uses for smart contracts on blockchains
            \begin{itemize}
                \item Cryptocurrencies, voting, crowdfunding, auctions, etc.
            \end{itemize}

        \item Smart contracts manage (and lose!) a lot of money: the DAO contract lost over \$40 million dollars in 2016 due to a simple reentrancy bug

        \item Smart contracts on Ethereum cannot be patched after being deployed
    \end{itemize}
\end{frame}

\begin{frame}[fragile]{The Psamathe Language}
    \begin{itemize}
        \item \textcolor{softRed}{\textbf{\langName}} (\langNamePronounce) is a new programming language we are designing for writing smart contracts
        \item It uses a new \emph{flow} abstraction, representing an atomic transfer operation
        \item It also has features to mark types with \emph{modifiers}, such as \flowinline{asset}, which combine with flows and \emph{type quantities} to make some classes of bugs impossible
    \end{itemize}

\begin{lstlisting}[language=flow, xleftmargin=0.4em, basicstyle=\scriptsize\ttfamily]
type Token is fungible asset uint256
transformer transfer(balances : any map one address => any Token,
                     dst : one address, amount : any uint256) {
    balances[msg.sender] --[ amount ]-> balances[dst] |\label{line:erc20-flow-ex}|
}
\end{lstlisting}

\end{frame}

\begin{frame}[fragile]{Simple Flows}
    \begin{itemize}
        \item The simplest flow \emph{selects} values from a \emph{source}, and \emph{sends} them to a \emph{destination}
    \end{itemize}
\begin{lstlisting}[language=flow, basicstyle=\normalsize\ttfamily]
Source --> Destination
\end{lstlisting}
    \begin{itemize}
        \item The flow from earlier may also be written as below:
    \end{itemize}
\begin{lstlisting}[language=flow, basicstyle=\normalsize\ttfamily]
balances[msg.sender][amount] --> balances[dst]
\end{lstlisting}
    \begin{itemize}
        \item The source is \flowinline{balances[msg.sender][amount]}, and the destination is \flowinline{balances[dst]}.
    \end{itemize}
\end{frame}

\begin{frame}[fragile]{Transformer Flows}
    \begin{itemize}
        \item The more complicated flow, with a \emph{transformer call}, selects values from the source, runs the transformer, then sends the transformed values to the destination
    \end{itemize}
\begin{lstlisting}[language=flow, basicstyle=\normalsize\ttfamily]
Source --> Transform(_)--> Destination
\end{lstlisting}
    \begin{itemize}
        \item The transformer can either be a user-defined function or a \emph{constructor}, which can create new values of named types
    \end{itemize}
\begin{lstlisting}[language=flow, basicstyle=\normalsize\ttfamily]
type Token is fungible asset uint256
10 --> new Token(_)--> var myTokens : Token
\end{lstlisting}
\end{frame}

\begin{frame}[fragile]{Type quantities}
    \begin{itemize}
        \item Type quantities: used to approximate the number of values in a variable to track \flowinline{asset} variables that can be dropped (i.e., when \flowinline{empty})
        \item They are: \flowinline{empty}, \flowinline{any}, \flowinline{one}, and \flowinline{nonempty}
        \item Type quantities are a more precise form of linear types; linear typing essentially corresponds to only having \flowinline{empty} and \flowinline{any}
        \item Can often be inferred; all type quantities in the first example can be omitted:
    \end{itemize}

\begin{lstlisting}[language=flow, xleftmargin=-0.5em, basicstyle=\footnotesize\ttfamily]
type Token is fungible asset uint256
transformer transfer(balances : map address => Token,
                     dst : address, amount : uint256) {
  balances[msg.sender] --[ amount ]-> balances[dst] |\label{line:erc20-flow-ex}|
}
\end{lstlisting}
\end{frame}

\begin{frame}[fragile]{Modifiers}
    \begin{itemize}
        \item Modifiers: Specify how variables of a type can be used
        \begin{itemize}
            \item \flowinline{asset}: must not be reused or accidentally lost, such as money (i.e., linear, or affine with \flowinline{consumable}).
            \item \flowinline{consumable}: an \flowinline{asset} that may be disposed of, via the \flowinline{consume} construct, documenting that the disposal is intentional.
                \begin{itemize}
                    \item Example: bids should not be lost \textbf{during} an auction, but it is safe to dispose of them after the auction ends.
                \end{itemize}
            \item \flowinline{unique}: each value of this type can be created only once, must be \flowinline{immutable}
            \item \flowinline{fungible}: can be \textbf{merged}, and it is \textbf{not} \flowinline{unique} (e.g., currencies).
            \item \flowinline{immutable}: cannot be \textbf{changed} (e.g., the following is not legal):
\begin{lstlisting}[language=flow, xleftmargin=0.4em, basicstyle=\footnotesize\ttfamily]
type T is immutable { x : nat, y : nat }
new T(0,0) --> var t : T
57 --> t.y // Error: `T` is immutable!
\end{lstlisting}
        \end{itemize}
    \end{itemize}
\end{frame}

\begin{frame}[fragile]{Advantages of Psamathe (Safety)}
    \begin{itemize}
        \item Using the more declarative, \emph{flow-based} approach provides the following advantages over imperative state updates:
        \item \textbf{Static safety guarantees}: The following is checked statically
            \begin{itemize}
                \item \flowinline{asset}: Each flow is guaranteed to preserve the total amount of each \flowinline{asset} type (except for flows that explicitly consume or allocate assets). %, removing the need to verify such properties.
                \item \flowinline{immutable}: prevents values from changing.
                \item Type quantities also allow us to distinguish when we are guaranteed to have an \flowinline{empty} variable vs. when we cannot know ahead of time.
                    For example, in:
\begin{lstlisting}[language=flow, xleftmargin=0.4em, basicstyle=\footnotesize\ttfamily]
x --[ one st P(_) ]--> y
\end{lstlisting}
                \item If \flowinline{x : one T}, then after the flow, \flowinline{x : empty T}, if it succeeds
            \end{itemize}
    \end{itemize}
\end{frame}

\begin{frame}[fragile]{Advantages of Psamathe (Safety, cont.)}
    \begin{itemize}
        \item \textbf{Dynamic safety guarantees}: \langName automatically inserts dynamic checks of a flow's validity. For example:
        \begin{itemize}
            \item A flow of money would fail if there is not enough money in the source, or if there is too much in the destination (e.g., due to overflow).
            \item A flow of values from a multiset could fail if there the specified values are not found in the source
            \item A flow of a value might also fail if it would overwrite an asset in the destination, e.g. \flowinline{bid --> winningBid}
            \item If \flowinline{winningBid : any Bid}, this must be checked dynamically, but in some cases, we can detect it statically using type quantities (e.g., if \flowinline{winningBid : one Bid})
        \end{itemize}
        \item The \flowinline{unique} modifier, is also checked dynamically.
    \end{itemize}
\end{frame}

\begin{frame}[fragile]{Advantages of Psamathe (Cont.)}
    \begin{itemize}
        \item \textbf{Data-flow tracking}: We hypothesize that flows provide a clearer way of specifying how resources flow in the code itself, which may be less apparent using other approaches, especially in complicated contracts.
        % Additionally, developers must explicitly mark when assets are \emph{consumed}, and only assets marked as \flowinline{consumable} may be consumed.
        \item \textbf{Error messages}: When a flow fails, the \langName runtime can provide automatic, descriptive error messages, such as
\begin{lstlisting}[numbers=none, basicstyle=\footnotesize\ttfamily, xleftmargin=-4.5em]
Cannot flow <amount> Token from account[<src>] to account[<dst>]:
    source only has <balance> Token.
\end{lstlisting}
    \end{itemize}
\end{frame}

\begin{frame}[fragile]{Example: Type quantities and Modifiers}
    \begin{itemize}
        \item Using type quantities and modifiers to guarantee correctness properties in a lottery
        \item Ensures: one ticket per user, cannot end lottery before somebody has won, jackpot is fully paid out
    \end{itemize}
\begin{lstlisting}[language=flow, xleftmargin=-0.2em, basicstyle=\scriptsize\ttfamily]
type TicketOwner is unique immutable address
type Ticket is consumable asset {
    owner : TicketOwner,
    guess : uint256
}
// ...
var winners : multiset Ticket <--
    tickets[nonempty such that isWinner(winNum, _)] |\label{line:lottery-filter}|
winners --> pay(jackpot / length(winners), _)
balance --> lotteryOwner.balance |\label{line:empty-lottery-balance}|
tickets --> consume
\end{lstlisting}
\end{frame}

\begin{frame}[fragile]{Comparison with Solidity}
    \begin{itemize}
        \item Solidity is the most commonly-used smart contract language on the Ethereum blockchain
        \item It does not provide support for managing assets
    \end{itemize}

\begin{lstlisting}[language=flow, xleftmargin=-0.5em, basicstyle=\footnotesize\ttfamily]
type Token is fungible asset uint256
transformer transfer(balances : map address => Token,
                     dst : address, amount : uint256) {
  balances[msg.sender] --[ amount ]-> balances[dst] |\label{line:erc20-flow-ex}|
}
\end{lstlisting}
\begin{lstlisting}[language=Solidity, xleftmargin=-0.5em, basicstyle=\scriptsize\ttfamily]
contract ERC20 {
  mapping (address => uint256) balances;
  function transfer(address dst, uint256 amount) public {
    require(amount <= balances[msg.sender]);
    balances[msg.sender] = balances[msg.sender].sub(amount);
    balances[dst] = balances[dst].add(amount);
  }
}
\end{lstlisting}
\end{frame}

\begin{frame}[fragile]{Example: Voting (\langName)}
    \begin{itemize}
        \item A simple voting contract that allows voters to vote on proposals; only one vote per address is allowed
    \end{itemize}
\begin{lstlisting}[language=flow, xleftmargin=-0.2em, basicstyle=\scriptsize\ttfamily]
type Voter is unique immutable asset address
type ProposalName is unique immutable asset string
type Election is asset {
  chairperson : address,
  eligibleVoters : multiset Voter,
  proposals : map ProposalName => multiset Voter
}
transformer giveRightToVote(this : Election, voter : address) {
  // msg.sender --[ this.chairperson ]-> msg.sender
  only when msg.sender = this.chairperson
  new Voter(voter) --> this.eligibleVoters
}
transformer vote(this : Election, proposal : string) {
  this.eligibleVoters --[ msg.sender ]-> this.proposals[proposal]
}
\end{lstlisting}
\end{frame}

\begin{frame}[fragile]{Example: Voting (Solidity)}
\begin{lstlisting}[language=Solidity, xleftmargin=-0.5em, basicstyle=\scriptsize\ttfamily]
contract Ballot {
  struct Voter { uint weight; bool voted; uint vote; }
  struct Proposal { bytes32 name; uint voteCount; }
  address public chairperson;
  mapping(address => Voter) public voters;
  Proposal[] public proposals;
  function giveRightToVote(address voter) public {
    require(msg.sender == chairperson,
      "Only chairperson can give right to vote.");
    require(!voters[voter].voted, "The voter already voted.");
    voters[voter].weight = 1;
  }
  function vote(uint proposal) public {
    Voter storage sender = voters[msg.sender];
    require(sender.weight != 0, "No right to vote");
    require(!sender.voted, "Already voted.");
    sender.voted = true;
    sender.vote = proposal;
    proposals[proposal].voteCount += sender.weight;
  }
}
\end{lstlisting}
\end{frame}

\begin{frame}[fragile]{Comparison with Solidity (Voting)}
    \begin{itemize}
        \item This example also shows that \langName is suited to a range of applications
        \item As in the ERC-20 example, the \langName implementation of the voting contract is more concise than the Solidity implementation
        \item It also shows more examples of using modifiers to enforce requirements of the contract
    \end{itemize}
\end{frame}

\begin{frame}{Future Work}
    \begin{itemize}
        \item Fully implement \langName by compiling to Solidity
        \item Prove the safety properties of the language: assets are not lost, immutable values are not changed, etc.
        \item Study the benefits and costs of the language via case studies, performance evaluation, and the application of flows to other domains
    \end{itemize}
\end{frame}

\begin{frame}{Conclusion}
    \begin{itemize}
        \item We have presented the \langName language for writing safer smart contracts
        \item \langName uses the new flow abstraction, \assetTxt{}s, and modifiers to provide safety guarantees for smart contracts
        \item We showed two examples of smart contracts in both Solidity and \langName, showing that \langName is capable of expressing common smart contract functionality in a concise manner, while retaining key safety properties.
    \end{itemize}
\end{frame}

\end{document}

