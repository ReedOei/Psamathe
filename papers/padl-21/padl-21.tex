\documentclass[dvipsnames,runningheads]{llncs}

\usepackage{graphicx}
\usepackage{listings}
\usepackage{hyperref}
\usepackage{dblfloatfix}
\usepackage{tikz-cd}
% \usepackage{amssymb}
% \usepackage{amsthm}
\usepackage{bm}
\usepackage{listings}
\usepackage{mathtools}
\usepackage{mathpartir}
\usepackage{float}
\usepackage{subcaption}
\usepackage{xcolor}
\usepackage{amsmath}
\usepackage{amssymb}
\usepackage[safe]{tipa}
\usepackage[inline]{enumitem}

\renewcommand\UrlFont{\color{blue}\rmfamily}

\usepackage{xcolor}
\usepackage{forloop}
\usepackage{ifthen}
\usepackage{xifthen}
\usepackage{calc}
\usepackage{xspace}
\usepackage{mathrsfs}

\newcounter{i}
\newcounter{j}
\newcounter{n1}

\newcommand{\concat}{%
  \mathbin{\raisebox{0.5ex}{\scalebox{.7}{$\frown$}}}%
}
\newcommand{\bnfdef}{\ensuremath{\Coloneqq}}
\newcommand{\bnfalt}{\ensuremath{\mid}\xspace}

\newcommand{\colvec}[1]{\ensuremath{\begin{pmatrix} #1 \end{pmatrix}}}

\newcommand{\imp}{\Rightarrow}
\renewcommand{\epsilon}{\varepsilon}

% From: https://en.wikibooks.org/wiki/LaTeX/Source_Code_Listings#Automating_file_inclusion
\newcommand{\includecode}[2][c]{\lstinputlisting[language=#1, caption=#2, basicstyle=\normalsize\ttfamily]{#2}}

\newcommand{\bigstep}[2]{#1 \Downarrow #2}

\newcommand{\eq}[1]{\begin{align*} #1 \end{align*}}

\newcommand{\Hom}{\ensuremath{\text{Hom}}}
\newcommand{\Tor}{\ensuremath{\text{Tor}}}
\newcommand{\ch}{\ensuremath{\text{ch}}}

\newcommand{\spanvec}{\ensuremath{\text{span}}}
\newcommand{\trace}[1]{\ensuremath{\text{Tr}\parens{#1}}}

\newcommand{\includeplot}[2][0.3]{%
\begin{figure}[H]
    \scalebox{#1}{\sageplot{#2}}
\end{figure}}

\newcommand{\partialdiff}[1]{\frac{\partial}{\partial #1}}
\newcommand{\diff}[1]{\frac{\text{d}}{\text{d} #1}}
\newcommand{\partialfunc}{\rightharpoonup}

\newcommand{\derive}[3][]{%
    \ifthenelse{\isempty{#1}}{%
        \frac{\text{d} #2}{\text{d} #3}%
    }{%
        \frac{\text{d}^{#1} #2}{\text{d} #3^{#1}}%
    }%
}

\newcommand{\im}[1]{\ensuremath{\text{im} {#1}}}

\newcommand{\true}{\textbf{\textsf{T}}}
\newcommand{\false}{\textbf{\textsf{F}}}

\newcommand{\join}{\bigvee}
\newcommand{\meet}{\bigwedge}

\newcommand{\piclass}[2]{\ensuremath{\bm{\Pi}^{#1}_{#2}}}
\newcommand{\sigmaclass}[2]{\ensuremath{\bm{\Sigma}^{#1}_{#2}}}
\newcommand{\deltaclass}[2]{\ensuremath{\bm{\Delta}^{#1}_{#2}}}
\newcommand{\gammaclass}{\ensuremath{\bm{\Gamma}}}
\newcommand{\borel}[1]{\ensuremath{\mathcal{B}\parens{#1}}}

\newcommand{\saturate}[2]{\squares{#1}_{#2}}

\newcommand{\abs}[1]{\ensuremath{\left| #1 \right|}}
\newcommand{\iso}{\ensuremath{\cong}}
\newcommand{\symdif}{\ensuremath{\bigtriangleup}}
\newcommand{\dual}{\ensuremath{\neg}}
\newcommand{\forces}{\ensuremath{\Vdash}}
\newcommand{\semiprod}[3][]{\ensuremath{#2 \rtimes_{#1} #3}}
\newcommand{\units}[1]{\ensuremath{#1^{\times}}}
\newcommand{\gl}[2]{\ensuremath{\text{GL}\parens{#1, #2}}}

\newcommand{\proj}[1]{\text{proj}_{#1}}

\newcommand{\lcm}{\text{lcm}}

%\newcommand{\reed}[1]{\relax}
%\newcommand{\Fix}[1]{\relax}
\newcommand{\reed}[1]{{\color{magenta}\bfseries [#1]}}
\newcommand{\Fix}[1]{{\color{red}\bfseries [#1]}}
\newcommand{\Comment}[1]{}
\newcommand{\Space}[1]{}
\newcommand{\Num}[1]{#1}

\newcommand{\e}{\ensuremath{\text{e}}}

\newcommand{\injects}{\ensuremath{\hookrightarrow}}
\newcommand{\into}{\injects}
\newcommand{\surjects}{\ensuremath{\twoheadrightarrow}}
\newcommand{\onto}{\surjects}

\newcommand{\actson}{\ensuremath{\curvearrowright}}

\newcommand{\interior}[1]{\ensuremath{\text{int}\parens{#1}}}
\newcommand{\closure}[1]{\ensuremath{\overline{#1}}}
\newcommand{\compl}[1]{\ensuremath{{#1}^c}}
\ifdefined\boundary\else\newcommand{\boundary}{\ensuremath{\partial}}\fi

\newcommand{\proves}{\vdash}
\newcommand{\satisfies}{\models}

\DeclarePairedDelimiter\ceil{\lceil}{\rceil}
\DeclarePairedDelimiter\floor{\lfloor}{\rfloor}

\ifdefined\theorem\else\newtheorem{theorem}{Theorem}\fi
\ifdefined\lemma\else\newtheorem{lemma}{Lemma}\fi
\ifdefined\proposition\else\newtheorem{proposition}{Proposition}\fi
\ifdefined\claim\else\newtheorem{claim}{Claim}\fi
\ifdefined\corollary\else\newtheorem{corollary}{Corollary}\fi
\ifdefined\definition\else\newtheorem{definition}{Definition}\fi
\ifdefined\example\else\newtheorem{example}{Example}\fi
\ifdefined\remark\else\newtheorem{remark}{Remark}\fi
\ifdefined\notation\else\newtheorem{notation}{Notation}\fi

\newcommand{\nat}{\ensuremath{\mathbb{N}}}
\newcommand{\N}{\ensuremath{\mathbb{N}}}
\newcommand{\real}{\ensuremath{\mathbb{R}}}
\newcommand{\reals}{\ensuremath{\mathbb{R}}}
\newcommand{\R}{\ensuremath{\mathbb{R}}}
\newcommand{\Z}{\ensuremath{\mathbb{Z}}}
\newcommand{\Zmod}[1]{\ensuremath{\Z/#1\Z}}
\newcommand{\integer}{\ensuremath{\mathbb{Z}}}
\newcommand{\integers}{\ensuremath{\mathbb{Z}}}
\newcommand{\rational}{\ensuremath{\mathbb{Q}}}
\newcommand{\rationals}{\ensuremath{\mathbb{Q}}}
\newcommand{\rat}{\ensuremath{\mathbb{Q}}}
\newcommand{\Q}{\ensuremath{\mathbb{Q}}}
\newcommand{\complex}{\ensuremath{\mathbb{C}}}
% \newcommand{\C}{\ensuremath{\mathbb{C}}}

\newcommand{\nonzero}[1]{\ensuremath{#1_{\neq 0}}}
\newcommand{\greaterset}[2]{\ensuremath{#1_{> #2}}}
\newcommand{\lesserset}[2]{\ensuremath{#1_{< #2}}}
\newcommand{\subgroup}{\leq}
\newcommand{\normalsubgroup}{\trianglelefteq}
\newcommand{\id}[1]{\ensuremath{\bm{1}_{#1}}}
\newcommand{\indicator}[1]{\ensuremath{\mathbbm{1}_{#1}}}

\newcommand{\notationNonZero}[1][X]{Let $\nonzero{#1}$ denote the set of $x \in #1$ so that $x \neq 0$.\xspace}
\newcommand{\notationGreaterSet}[1][X]{Let $\greaterset{#1}{y}$ denote the set of $x \in #1$ so that $x > y$.\xspace}
\newcommand{\notationLesserSet}[1][X]{Let $\lesserset{#1}{y}$ denote the set of $x \in X$ so that $x < y$.\xspace}
\newcommand{\notationSubgroup}[1][H]{Let $\subgroup{#1}{G}$ denote that $#1$ is a subgroup of $G$.\xspace}
\newcommand{\notationNormalSubgroup}[1][N]{Let $\normalsubgroup{#1}{G}$ denote that $#1$ is a normal subgroup of $G$.\xspace}
\newcommand{\notationIdentity}[1][X]{Let $\id{#1}$ denote the identity function on $#1$.\xspace}

% \Sym{X} is the set of bijections from X to X
\newcommand{\Sym}[1]{\ensuremath{\text{Sym}\parens{#1}}}
\newcommand{\Inn}[1]{\ensuremath{\text{Inn}\parens{#1}}}
\newcommand{\Out}[1]{\ensuremath{\text{Out}\parens{#1}}}
\newcommand{\Aut}[1]{\ensuremath{\text{Aut}\parens{#1}}}
\newcommand{\notationSym}[1][X]{Let $\Sym{X}$ be the group of bijections of $X$.\xspace}

\newcommand{\toplim}{\ensuremath{\text{T}\lim}}
\newcommand{\toplimup}{\ensuremath{\overline{\toplim}}}
\newcommand{\toplimlow}{\ensuremath{\underline{\toplim}}}

\newcommand{\emptysingleton}{\ensuremath{\curlys{\emptyset}}}
\newcommand{\withoutempty}{\ensuremath{\setminus \emptysingleton}}

\newcommand{\isnat}[1]{\ensuremath{#1 \in \nat}}
\newcommand{\seq}[3]{\ensuremath{{\left( #1_{#2} \right)}_{#2 \in #3}}}
\newcommand{\setint}[3]{\ensuremath{\bigcap_{#2 \in #3} #1_{#2}}}
\newcommand{\natinti}[2]{\ensuremath{\setint{#1}{#2}{\nat}}}
\newcommand{\natint}[1]{\ensuremath{\natinti{#1}{n}}}
\newcommand{\natseqi}[2]{\ensuremath{\seq{#1}{#2}{\nat}}}
\newcommand{\natseq}[1]{\ensuremath{\natseqi{#1}{n}}}

\newcommand{\diam}[1]{\ensuremath{\text{diam} \left( #1 \right)}}

\newcommand{\eventually}{\ensuremath{\forall^{\infty}}}
\newcommand{\infinitelymany}{\ensuremath{\exists^{\infty}}}

\newcommand{\power}[1]{\ensuremath{\mathcal{P} \left( #1 \right)}}
\newcommand{\card}[1]{\ensuremath{\left| #1 \right|}}

\newcommand{\brackets}[3]{\ensuremath{{\left#1 {#3} \right#2}}}
\newcommand{\parens}[1]{\brackets{(}{)}{#1}}
\newcommand{\angles}[1]{\brackets{<}{>}{#1}}
\newcommand{\curlys}[1]{\brackets{\{}{\}}{#1}}
\newcommand{\squares}[1]{\brackets{[}{]}{#1}}
\newcommand{\parensum}[3]{\ensuremath{\parens{\sum_{#1}^{#2} {#3}}}}

\newcommand{\poly}[2]{\ensuremath{{#1}\left[ #2 \right]}}

\newcommand{\tand}{\ensuremath{~\text{and}~}}
\newcommand{\tor}{\ensuremath{~\text{or}~}}
\newcommand{\twhere}{\ensuremath{~\text{where}~}}
\newcommand{\tif}{\ensuremath{~\text{if}~}}
\newcommand{\tsuchthat}{\ensuremath{~\text{s.t.}~}}
\newcommand{\owise}{\ensuremath{~\text{otherwise}~}}

\newcommand{\ff}[1]{\ensuremath{\mathbb{F}_{#1}}}

\newcommand{\beatree}[1]{\ensuremath{\subseteq {#1}^{\leq \nat}}}

\newcommand{\metricspace}[2]{\ensuremath{\parens{{#1},{#2}}}}

\newcommand{\setbuild}[2]{\ensuremath{\left\{ {#1} : {#2} \right\}}}

\newcommand{\openball}[3][]{\ensuremath{B_{#1}\parens{{#2},{#3}}}}
\newcommand{\closedball}[3][]{\ensuremath{\overline{B}_{#1}\parens{{#2},{#3}}}}

\newcommand{\powerset}[1]{\ensuremath{\mathcal{P}\parens{#1}}}
\newcommand{\powersetfin}[1]{\ensuremath{\mathcal{P}_{\text{fin}}\parens{#1}}}

\newcommand{\subgroupgen}[1]{\ensuremath{{\left< {#1} \right>}}}

\newcommand{\diagmatrix}[2]{%
    \setcounter{n1}{#1 - 1}
    \begin{pmatrix}
        \forloop{i}{0}{\value{i} < #1}{%
            \forloop{j}{0}{\value{j} < #1}{%
                \ifthenelse{\equal{\value{i}}{\value{j}}}{#2}{0}
                \ifthenelse{\value{j} < \value{n1}}{&}{}
            }
            \ifthenelse{\value{i} < \value{n1}}{\\}{}
        }
    \end{pmatrix}
}

\newcommand{\idmatrix}[1]{\diagmatrix{#1}{1}}

\newcommand{\twobytwo}[4]{\ensuremath{\begin{pmatrix} #1 & #2 \\ #3 & #4 \end{pmatrix}}}

\newcommand{\after}{\circ}


\newcommand{\finitecommabody}[1]{}
\newcommand{\countablecommabody}[1]{}

% #1 - The thing to repeat
% #2 - The operation to use to combine
% #3 - The number of times to repeat (used for comment, \cdots is used for the actual body)
\newcommand{\ntimes}[3]{\overbrace{#1 #2 \cdots #2 #1}^{#3 \text{ times}}}

\newcommand{\finiteopfromone}[3]{%
    \renewcommand{\finitecommabody}[1]{#1}%
    \ensuremath{\finitecommabody{1} #2 \finitecommabody{2} #2 \cdots #2 \finitecommabody{#3}}}
\newcommand{\finiteop}[3]{%
    \renewcommand{\finitecommabody}[1]{#1}%
    \ensuremath{\finitecommabody{0} #2 \finitecommabody{1} #2 \cdots #2 \finitecommabody{#3 - 1}}}

% #1 - The body expression (e.g., x_{#1}, where #1 is where the index will go)
% #2 - The operation to use
% #3 - The index of the element to skip
% #4 - The end index
\newcommand{\finiteopskipfromone}[4]{%
    \renewcommand{\finitecommabody}[1]{#1}%
    \ensuremath{\finitecommabody{1} #2 \cdots #2 \finitecommabody{#3 - 1} #2 \finitecommabody{#3 + 1} #2 \cdots #2 \finitecommabody{#3}}}
\newcommand{\finiteopskip}[4]{%
    \renewcommand{\finitecommabody}[1]{#1}%
    \ensuremath{\finitecommabody{0} #2 \cdots #2 \finitecommabody{#3 - 1} #2 \finitecommabody{#3 + 1} #2 \cdots #2 \finitecommabody{#3 - 1}}}

\newcommand{\finitecommafromone}[2]{%
    \renewcommand{\finitecommabody}[1]{#1}%
    \ensuremath{\finitecommabody{1}, \finitecommabody{2}, \ldots, \finitecommabody{#2}}}
\newcommand{\finitesetfromone}[2]{\ensuremath{\curlys{\finitecommafromone{#1}{#2}}}}

\newcommand{\finitecomma}[2]{%
    \renewcommand{\finitecommabody}[1]{#1}%
    \ensuremath{\finitecommabody{0}, \finitecommabody{1}, \ldots, \finitecommabody{#2 - 1}}}
\newcommand{\finiteset}[2]{\ensuremath{\curlys{\finitecomma{#1}{#2}}}}

\newcommand{\countablecommafromone}[1]{\ensuremath{%
    \renewcommand{\countablecommabody}[1]{#1}%
    \ensuremath{\countablecommabody{1}, \countablecommabody{2}, \ldots}}}
\newcommand{\ctblcommafromone}[1]{\ensuremath{\countablecommafromone{#1}}}
\newcommand{\countablesetfromone}[1]{\ensuremath{\curlys{\countablecommafromone{#1}}}}
\newcommand{\ctblsetfromone}[1]{\ensuremath{\curlys{\countablecommafromone{#1}}}}

\newcommand{\countablecomma}[1]{\ensuremath{%
    \renewcommand{\countablecommabody}[1]{#1}%
    \ensuremath{\countablecommabody{0}, \countablecommabody{1}, \ldots}}}
\newcommand{\ctblcomma}[1]{\ensuremath{\countablecomma{#1}}}
\newcommand{\countableset}[1]{\ensuremath{\curlys{\countablecomma{#1}}}}
\newcommand{\ctblset}[1]{\ensuremath{\curlys{\countablecomma{#1}}}}

\newcommand{\woutlog}{without loss of generality\xspace}
\newcommand{\Woutlog}{Without loss of generality\xspace}

\newcommand{\actOn}[2]{\ensuremath{#1 \cdot #2}}
\newcommand{\kernel}[1]{\ensuremath{\text{ker}\parens{#1}}}

\newcommand{\deffuncparam}[6]{%
    \expandafter\newcommand\csname domain#1\endcsname{\ensuremath{#3}}%
    \expandafter\newcommand\csname codomain#1\endcsname{\ensuremath{#5}}%
    \expandafter\newcommand\csname decl#1\endcsname[1]{\ensuremath{#2_{##1} : #3 #4 #5}}%
    \expandafter\newcommand\csname funcname#1\endcsname[1]{\ensuremath{#2_{##1}}}%
    \expandafter\newcommand\csname body#1\endcsname[2]{\ensuremath{#6}}%
    \expandafter\newcommand\csname call#1\endcsname[2]{\ensuremath{#2_{##1} \parens{##2}}}%
    \expandafter\newcommand\csname callbody#1\endcsname[2]{\ensuremath{\expandafter\csname call#1\endcsname{##1}{##2} = \expandafter\csname body#1\endcsname{##1}{##2}}}%
    \expandafter\newcommand\csname map#1\endcsname[2]{\ensuremath{##2 \mapsto \expandafter\csname body#1\endcsname{##1}{##2} }}%
    \expandafter\newcommand\csname inversecall#1\endcsname[2]{\ensuremath{#2_{##1} \parens{##2}^{-1}}}%
    \expandafter\newcommand\csname image#1\endcsname[1]{\ensuremath{\text{im}\parens{#2_{##1}}}}%
}

\newcommand{\defbinfunc}[6]{%
    \expandafter\newcommand\csname domain#1\endcsname{\ensuremath{#3}}%
    \expandafter\newcommand\csname codomain#1\endcsname{\ensuremath{#5}}%
    \expandafter\newcommand\csname decl#1\endcsname{\ensuremath{#2 : #3 #4 #5}}%
    \expandafter\newcommand\csname funcname#1\endcsname{\ensuremath{#2}}%
    \expandafter\newcommand\csname body#1\endcsname[2]{#6}%
    \expandafter\newcommand\csname call#1\endcsname[2]{\ensuremath{#2 \parens{##1, ##2}}}%
    \expandafter\newcommand\csname callpair#1\endcsname[1]{\ensuremath{#2 \parens{##1}}}%
    \expandafter\newcommand\csname callbody#1\endcsname[2]{\ensuremath{\expandafter\csname call#1\endcsname{##1}{##2} = \expandafter\csname body#1\endcsname{##1}{##2}}}%
    \expandafter\newcommand\csname map#1\endcsname[2]{\ensuremath{\parens{##1, ##2} \mapsto \expandafter\csname body#1\endcsname{##1}{##2} }}%
}

\newcommand{\defbinop}[6]{%
    \expandafter\newcommand\csname domain#1\endcsname{\ensuremath{#3}}%
    \expandafter\newcommand\csname codomain#1\endcsname{\ensuremath{#5}}%
    \expandafter\newcommand\csname decl#1\endcsname{\ensuremath{#2 : #3 #4 #5}}%
    \expandafter\newcommand\csname funcname#1\endcsname{\ensuremath{#2}}%
    \expandafter\newcommand\csname body#1\endcsname[2]{#6}%
    \expandafter\newcommand\csname call#1\endcsname[2]{\ensuremath{##1 #2 ##2}}%
    \expandafter\newcommand\csname callbody#1\endcsname[2]{\ensuremath{\expandafter\csname call#1\endcsname{##1}{##2} = \expandafter\csname body#1\endcsname{##1}{##2}}}%
    \expandafter\newcommand\csname map#1\endcsname[2]{\ensuremath{\parens{##1, ##2} \mapsto \expandafter\csname body#1\endcsname{##1}{##2} }}%
}

\newcommand{\deffunc}[6]{%
    \expandafter\newcommand\csname domain#1\endcsname{\ensuremath{#3}}%
    \expandafter\newcommand\csname codomain#1\endcsname{\ensuremath{#5}}%
    \expandafter\newcommand\csname decl#1\endcsname{\ensuremath{#2 : #3 #4 #5}}%
    \expandafter\newcommand\csname funcname#1\endcsname{\ensuremath{#2}}%
    \expandafter\newcommand\csname body#1\endcsname[1]{#6}%
    \expandafter\newcommand\csname call#1\endcsname[1]{\ensuremath{#2 \parens{##1}}}%
    \expandafter\newcommand\csname callbody#1\endcsname[1]{\ensuremath{\expandafter\csname call#1\endcsname{##1} = \expandafter\csname body#1\endcsname{##1}}}%
    \expandafter\newcommand\csname map#1\endcsname[1]{\ensuremath{##1 \mapsto \expandafter\csname body#1\endcsname{##1} }}%
    \expandafter\newcommand\csname invcall#1\endcsname[1]{\ensuremath{#2^{-1}\parens{##1}}}%
    \expandafter\newcommand\csname image#1\endcsname{\ensuremath{\text{im}\parens{#2}}}%
}

\newcommand{\defproj}[3]{%
    \defbinfunc{#1#2}{\pi_{#2}}{#2 \times #3}{\to}{#2}{##1}%
    \defbinfunc{#1#3}{\pi_{#3}}{#2 \times #3}{\to}{#3}{##2}%
}

% Footnoteable proofs for useful stuff
\newcommand{\notationEventually}{Define the notation $\eventually n, \varphi(n)$ to mean $\exists N \in \nat \tsuchthat \forall n \geq N, \varphi(n)$, where $\varphi$ is some formula.\xspace}
\newcommand{\eventuallyCombinesPrf}{%
    Suppose that $\eventually n, \varphi(n)$ and $\eventually n, \psi(n)$, where $\varphi$ and $\psi$ are two formulas.
    Then there is some $N_1$ so that $\varphi(n)$ holds for all $n \geq N_1$, and some $N_2$ so that $\psi(n)$ holds for all $n \geq N_2$.
    Then define $N = \max(N_1, N_2)$, and suppose $n \geq N$.
    Then $\varphi(n)$ holds because $n \geq N_1$ and $\psi(n)$ holds because $n \geq N_2$, so $\eventually n, \varphi(n) \wedge \psi(n)$.\xspace}


\newcommand{\irrationalDensePrf}{For any rational $q$, take $q + \frac{\sqrt{2}}{n}$, which is irrational. However, because we can make $\frac{1}{n}$ smaller than any $r > 0$, we can make some irrational that will be in $B(q,r)$ for any $r > 0$.\xspace}


% Copyright 2017 Sergei Tikhomirov, MIT License
% https://github.com/s-tikhomirov/solidity-latex-highlighting/

\usepackage{listings, xcolor}

\definecolor{verylightgray}{rgb}{.97,.97,.97}

\lstdefinelanguage{Solidity}{
	keywords=[1]{anonymous, assembly, balance, call, callcode, class, constant, constructor, contract, debugger, delegatecall, delete, emit, event, experimental, export, external, function, gas, implements, import, in, indexed, instanceof, interface, internal, is, length, library, log0, log1, log2, log3, log4, memory, modifier, new, payable, pragma, private, protected, public, pure, push, return, returns, selfdestruct, send, solidity, storage, struct, suicide, super, then, throw, typeof, using, view, with, addmod, ecrecover, keccak256, mulmod, ripemd160, sha256, sha3}, % generic keywords including crypto operations
	keywordstyle=[1]\color{Rhodamine}\bfseries,
	keywords=[2]{address, bool, byte, bytes, bytes1, bytes2, bytes3, bytes4, bytes5, bytes6, bytes7, bytes8, bytes9, bytes10, bytes11, bytes12, bytes13, bytes14, bytes15, bytes16, bytes17, bytes18, bytes19, bytes20, bytes21, bytes22, bytes23, bytes24, bytes25, bytes26, bytes27, bytes28, bytes29, bytes30, bytes31, bytes32, enum, int, int8, int16, int24, int32, int40, int48, int56, int64, int72, int80, int88, int96, int104, int112, int120, int128, int136, int144, int152, int160, int168, int176, int184, int192, int200, int208, int216, int224, int232, int240, int248, int256, mapping, string, uint, uint8, uint16, uint24, uint32, uint40, uint48, uint56, uint64, uint72, uint80, uint88, uint96, uint104, uint112, uint120, uint128, uint136, uint144, uint152, uint160, uint168, uint176, uint184, uint192, uint200, uint208, uint216, uint224, uint232, uint240, uint248, uint256, var, void, ether, finney, szabo, wei, days, hours, minutes, seconds, weeks, years},	% types; money and time units
	keywordstyle=[2]\color{Emerald}\itshape,
	keywords=[3]{block, balance, blockhash, coinbase, difficulty, gaslimit, number, timestamp, msg, data, gas, sender, sig, now, tx, gasprice, origin, this},	% sepcial identifiers
	keywordstyle=[3]\color{Lavender}\itshape,
	keywords=[4]{assert, break, case, catch, continue, default, else, false, finally, for, if, require, revert, switch, throw, true, try, while, do}, % Control-flow
	keywordstyle=[4]\color{Cerulean}\bfseries,
	identifierstyle=\color{black},
	sensitive=false,
    keepspaces=true,
    columns=fullflexible,
    % literate=*
    %     {(}{{{\color{red}(}}}1
    %     {)}{{{\color{red})}}}1
    %     {+}{{{\color{red}+~}}}1
    %     {-}{{{\color{red}-~}}}1
    %     {:=}{{{\color{red}:=~}}}1
    %     {\{}{{{\color{red}\{}}}1
    %     {\}}{{{\color{red}\}}}}1
    %     {[}{{{\color{red}[}}}1
    %     {]}{{{\color{red}]}}}1
    %     {|}{{{\color{red}|}}}1
    %     {;}{{{\color{red};}}}1
    %     {*}{{{\color{red}*~}}}1
    %     {:}{{{\color{red}:~}}}1
    %     {>}{{{\color{red}>~}}}1
    %     {<}{{{\color{red}<~}}}1
    %     {<=>}{{{$\color{red}\Leftrightarrow~$}}}1
    %     {.}{{{\color{red}.~}}}1
    %     {!}{{{$\color{red}!~$}}}1
    %     {=}{{{$\color{red}=~$}}}1
    %     {exists }{{{$\color{red}\exists$}}}1
    %     {forall }{{{$\color{red}\forall$}}}1,
	comment=[l]{//},
	morecomment=[s]{/*}{*/},
    commentstyle=\color{CadetBlue}\textit,
	stringstyle=\color{ForestGreen},
	morestring=[b]',
	morestring=[b]",
    xleftmargin=5.0ex,
    extendedchars=true,
	basicstyle=\small\ttfamily,
	showstringspaces=false,
	showspaces=false,
	numbers=left,
	numberstyle=\footnotesize,
	numbersep=9pt,
	tabsize=2,
	breaklines=true,
	showtabs=false,
	captionpos=b
}


\definecolor{verylightgray}{rgb}{.97,.97,.97}

\lstdefinelanguage{flow}{
    keywords=[1]{var,interface,where,contract,transaction,view,transformer,type,is,returns},
    keywordstyle=[1]\color{Rhodamine}\bfseries,
    keywords=[2]{false,true,sometimes,fungible,asset,unique,immutable,consumable},
    keywordstyle=[2]\color{DarkOrchid}\bfseries,
    keywords=[3]{map,linking,set,list,address,uint256,string,bytes,nat,bool},
    keywordstyle=[3]\color{Emerald}\itshape,
    keywords=[4]{nonempty,any,every,empty,exactly,one,msg,this,in,new,or,and,not,total,consume},
    keywordstyle=[4]\color{Lavender}\itshape,
    keywords=[5]{if,else,only,when,try,catch,such,that},
    keywordstyle=[5]\color{Cerulean}\bfseries,
    keepspaces=true,
	% backgroundcolor=\color{verylightgray},
    % columns=fullflexible,
    % literate=*
    %     {(}{{{\color{red}(}}}1
    %     {)}{{{\color{red})}}}1
    %     {+}{{{\color{red}+~}}}1
    %     {-}{{{\color{red}-~}}}1
    %     {:=}{{{\color{red}:=~}}}1
    %     {\{}{{{\color{red}\{}}}1
    %     {\}}{{{\color{red}\}}}}1
    %     {[}{{{\color{red}[}}}1
    %     {]}{{{\color{red}]}}}1
    %     {*}{{{\color{red}*~}}}1
    %     {:}{{{\color{red}:~}}}1
    %     {>}{{{\color{red}>~}}}1
    %     {<}{{{\color{red}<~}}}1
    %     {<=>}{{{$\color{red}\Leftrightarrow~$}}}1
    %     {.}{{{\color{red}.~}}}1
    %     {=}{{{$\color{red}=~$}}}1
    %     {exists }{{{$\color{red}\exists$}}}1
    %     {forall }{{{$\color{red}\forall$}}}1,
    sensitive=false, % keywords are not case-sensitive
    morecomment=[l]{//}, % l is for line comment
    morecomment=[s]{/*}{*/}, % s is for start and end delimiter
    commentstyle=\color{CadetBlue}\textit,
    morestring=[b]", % defines that strings are enclosed in double quotes
    stringstyle=\color{ForestGreen}, % string literal style
    showstringspaces=false,
    xleftmargin=5.0ex,
    mathescape=true,
    escapechar=|,
    basicstyle=\small\ttfamily
}

\lstnewenvironment{flow}
{%
    \lstset{language=flow, basicstyle=\small\ttfamily, mathescape=true}%
}
{
}

\newcommand{\flowinline}[1]{\lstinline[language=flow,basicstyle=\small\ttfamily,mathescape]{#1}}


\lstset{
  frame=none,
  xleftmargin=2pt,
  stepnumber=1,
  numbers=left,
  numbersep=5pt,
  numberstyle=\ttfamily\tiny\color[gray]{0.3},
  belowcaptionskip=\bigskipamount,
  captionpos=b,
  escapeinside={*'}{'*},
  tabsize=2,
  emphstyle={\bf},
  commentstyle=\it,
  stringstyle=\mdseries\rmfamily,
  showspaces=false,
  keywordstyle=\bfseries\rmfamily,
  columns=flexible,
  basicstyle=\small\sffamily,
  showstringspaces=false,
}

\newcommand{\guard}{\bigg|\bigg|}
\newcommand{\newc}{\textbf{\texttt{new}}}
\newcommand{\everything}{\textbf{\texttt{everything}}}
\newcommand{\then}{\textbf{\texttt{then}}}
\newcommand{\this}{\textbf{\texttt{this}}}
\newcommand{\langName}{LANGUAGE-NAME\xspace}
\newcommand{\consumesarr}{\mathrel{\rotatebox[origin=c]{270}{${\looparrowright}$}}}
\newcommand{\consumes}[1]{\stackrel{#1}{\consumesarr}}
\newcommand{\sends}[1]{\stackrel{#1}{\to}}
\newcommand{\emitsarr}{\mathrel{\rotatebox[origin=c]{270}{${\looparrowleft}$}}}
\newcommand{\emits}[1]{\emitsarr #1}
\newcommand{\asset}{\textbf{\texttt{asset}}\xspace}
\newcommand{\fungible}{\textbf{\texttt{fungible}}\xspace}
\newcommand{\nonfungible}{\textbf{\texttt{nonfungible}}\xspace}
\newcommand{\addresst}{\textbf{\texttt{address}}\xspace}
\newcommand{\stringt}{\textbf{\texttt{string}}\xspace}
\newcommand{\setof}{\textbf{\texttt{set}}\xspace}
\newcommand{\optiont}{\textbf{\texttt{option}}\xspace}
\newcommand{\boolt}{\textbf{\texttt{bool}}\xspace}
\newcommand{\natt}{\textbf{\texttt{nat}}\xspace}
\newcommand{\byt}{\textbf{\texttt{by}}\xspace}
\newcommand{\stores}{\textbf{\texttt{stores}}\xspace}
\newcommand{\suchthat}{\textbf{s.t.}\xspace}
\newcommand{\returns}{\textbf{\texttt{returns}}\xspace}
\newcommand{\merge}{\rightsquigarrow}
\newcommand{\flowsto}{\rightsquigarrow}
\newcommand{\heldby}{\rightarrowtail}
\newcommand{\one}{\textbf{\texttt{one}}\xspace}
\newcommand{\some}{\textbf{\texttt{some}}\xspace}
\newcommand{\any}{\textbf{\texttt{any}}\xspace}
\newcommand{\map}{\textbf{\texttt{map}}\xspace}
\newcommand{\linking}{\textbf{\texttt{linking}}\xspace}
\newcommand{\transformer}{\textbf{\texttt{transformer}}\xspace}
\newcommand{\selects}{\textbf{\texttt{selects}}\xspace}
\newcommand{\demote}{\textbf{\texttt{demote}}\xspace}
\newcommand{\transforms}{\Rrightarrow}

\newcommand{\flowproves}{\mathbin{|\hskip -0.1em\scalebox{0.7}[1]{$\leadsto$}}}
\newcommand{\flowproveout}{\mathbin{\scalebox{0.7}[1]{$\leadsto$}\hskip -0.13em|}}



\begin{document}

\title{\langName: A DSL with Flows for Safe Blockchain \AssetTxt{}s}

\author{Reed Oei\inst{1} \and Michael Coblenz\inst{2} \and Jonathan Aldrich\inst{3}}
\institute{University of Illinois, Urbana, IL, USA\\
\email{reedoei2@illinois.edu}%
\and University of Maryland, College Park, MD, USA\\
\email{mcoblenz@umd.edu}%
\and Carnegie Mellon University, Pittsburgh, PA, USA\\
\email{jonathan.aldrich@cs.cmu.edu}}

\maketitle

% \begin{abstract}
% Blockchains host smart contracts for crowdfunding, tokens, and many other purposes. Vulnerabilities in contracts are often discovered, leading to the loss of large quantities of money. Psamathe is a new language we are designing around a new flow abstraction, reducing asset bugs and making contracts more concise than in existing languages. We present an overview of Psamathe, and an examples contract in Psamathe, and compare it to the same program written in Solidity.
% \end{abstract}

\section{Introduction}
\vspace{-0.5em}
Blockchains are increasingly used as platforms for applications called \emph{smart contracts}~\cite{Szabo97:Formalizing}, which automatically manage transactions in an mutually agreed-upon way.
Commonly proposed and implemented applications include supply chain management, healthcare, voting, crowdfunding, auctions, and more~\cite{SupplyChainUse,HealthcareUse,Elsden18:Making}.
Smart contracts often manage \emph{digital assets}, such as cryptocurrencies, or, depending on the application, bids in an auction, votes in an election, and so on.
% One of the most common types of smart contract is a \emph{token contract}, which manage \emph{tokens}, a kind of cryptocurrency.
% Token contracts are common on the Ethereum blockchain~\cite{wood2014ethereum}---about 73\% of high-activity contracts are token contracts~\cite{OlivaEtAl2019}.
These contracts cannot be patched after deployment on Ethereum, even if security vulnerabilities are discovered.
Some estimates suggest that as many as 46\% of smart contracts may have vulnerabilities~\cite{luuOyente}.

\langName (\langNamePronounce) is a new programming language we are designing around \emph{flows}, which are a new abstraction representing an atomic transfer operation.
Together with features such as \emph{modifiers}, flows provide a \textbf{concise} way to write contracts that \textbf{safely} manage \assetTxt{}s (see Section~\ref{sec:lang}).
Solidity, the most commonly-used smart contract language on the Ethereum blockchain~\cite{EthereumForDevs}, does not provide analogous support for \assetTxt{}s.
% Typical smart contracts are more concise in \langName than in Solidity, because \langName handles common patterns and pitfalls automatically.
A formalization of \langName is in progress~\cite{psamatheRepo}, with an \emph{executable semantics} implemented in the $\mathbb{K}$-framework~\cite{rosu-serbanuta-2010-jlap}, which is already capable of running the example shown in Figure~\ref{fig:erc20-transfer-flow}.
An extended version~\cite{oei2020psamathe} of this work explains the language in more detail, and shows more examples, such as a voting contract, showing the advantages we describe below.

Other newly-proposed blockchain languages include Flint, Move, Nomos, Obsidian, and Scilla~\cite{schrans2018flint,blackshear2019move,das2019nomos,coblenz2019obsidian,sergey2019scilla}.
Scilla and Move are intermediate-level languages, whereas \langName is intended to be a high-level language.
Obsidian, Move, Nomos, and Flint use linear or affine types to manage \assetTxt{}s; \langName uses \emph{type quantities}, which extend linear types to allow a more precise analysis of the flow of values in a program.
None of the these languages have flows or provide support for all the modifiers that \langName does.

\section{Language}\label{sec:lang}
A \langName program is made of \emph{transformers} and \emph{type declarations}.
Transformers contain \emph{flows} describing the how values are transferred between variables.
Type declarations provide a way to name types and to mark values with \emph{modifiers}, such as \flowinline{asset}.
Figure~\ref{fig:erc20-transfer-flow} shows a simple contract declaring a type and a transformer, which implements the core of ERC-20's \lstinline{transfer} function.
ERC-20 is a standard, providing a bare-bones interface for token contracts managing ``bank accounts'' (identified by addresses) of \emph{fungible} tokens.
Fungible tokens are interchangeable (like most currencies), so it is only important how many tokens one owns, not \textbf{which} tokens.
% An ERC-20 contract manages ``bank accounts'' for its tokens, keeping track of how many tokens each account has; accounts are identified by addresses.

\begin{figure}[t]
    \centering
    \lstinputlisting[language=flow]{erc20-transfer.flow}
    \vspace{-1em}
    \caption{A \langName contract with a simple \flowinline{transfer} function, which transfers \flowinline{amount} tokens from the sender's account to the destination account.
It is implemented with a single flow, which checks the preconditions to ensure the transfer is valid.
The type quantities (\flowinline{any} and \flowinline{one}) can be omitted.}
    \label{fig:erc20-transfer-flow}
    \vspace{-2em}
\end{figure}

\langName is built around flows.
Using the more declarative, \emph{flow-based} approach provides the following advantages over imperative state updates:
\begin{itemize}
    \item \textbf{Static safety guarantees}: Each flow is guaranteed to preserve the total amount of assets (except for flows that explicitly consume or allocate assets). %, removing the need to verify such properties.
        % The total amount of a nonconsumable asset never decreases.
        % Each asset has exactly one reference to it, either via a variable in the current environment, or in a table/record.
        The \flowinline{immutable} modifier prevents values from changing.
    % \item \textbf{Dynamic safety guarantees}: \langName automatically inserts dynamic checks of a flow's validity; e.g., a flow of money would fail if there is not enough money in the source, or if there is too much in the destination (e.g., due to overflow).
    \item \textbf{Dynamic safety guarantees}: \langName automatically inserts dynamic checks of a flow's validity; e.g., flowing money would fail if there is not enough money, or flowing votes would fail if the selected votes are not found.
        The \flowinline{unique} modifier, which restrict values to never be created more than once, is also checked dynamically.
    \item \textbf{Data-flow tracking}: We hypothesize that flows provide a clearer way of describing how values flow in code itself, especially in complicated contracts.
    \item \textbf{Error messages}: When a flow fails, the \langName runtime provides automatic, descriptive error messages, such as
\begin{lstlisting}[numbers=none, basicstyle=\small\ttfamily]
Cannot flow <amount> Token from account[<src>] to account[<dst>]:
    source only has <balance> Token.
\end{lstlisting}
        Flows enable such messages by encoding the information (e.g., source, destination, values to transfer, etc.) into the source code.
\end{itemize}

We now give examples using modifiers and type quantities to guarantee additional correctness properties in the context of a lottery.
% We create \flowinline{TicketOwner} and \flowinline{Ticket} types.
The \flowinline{unique} and \flowinline{immutable} modifiers ensure users enter the lottery at most once, while \flowinline{asset} ensures that we do not accidentally lose tickets.
We use \flowinline{consumable} because tickets no longer have any value when the lottery is over.
\begin{lstlisting}[language=flow, xleftmargin=0.8em]
type TicketOwner is unique immutable address
type Ticket is consumable asset { owner: TicketOwner, guess: uint256 }
\end{lstlisting}
Consider the following code snippet, handling ending the lottery.
We require that lottery not end before there is a winning ticket, enforced by the \flowinline{nonempty} in the \emph{filter} on line~\ref{line:lottery-filter}; a filter is written \flowinline{[ q st P(x1,...,xn,_) ]}, selecting values \flowinline{v} \flowinline{such that} \flowinline{P(x1,...,xn,v)} returns \flowinline{true}, and \textbf{asserting} this occurs for some type quantity \flowinline{q} of the values.
Note that, as \flowinline{winners} is \flowinline{nonempty}, there cannot be a divide-by-zero error.
% The lottery cannot end before there is at least one winning ticket, enforced by the \flowinline{nonempty} in the \emph{filter} on line~\ref{line:lottery-filter}, at which point the jackpot is split evenly between the winners; note that, as \flowinline{winners} is \flowinline{nonempty}, there cannot be a divide-by-zero error.
% \flowinline{jackpot / length(winners)} is guaranteed not to divide by zero.
% Without line~\ref{line:empty-lottery-balance}, \langName would give an error indicating \flowinline{balance} has type \flowinline{any ether}, not \flowinline{empty ether}---a true error, because in the case that the jackpot cannot be evenly split between the winners, there will be some ether left over.
Without line~\ref{line:empty-lottery-balance}, \langName would give an error that \flowinline{balance} has type \flowinline{any ether}, not \flowinline{empty ether}---consider the case that the jackpot cannot be evenly split between the winners.
% Similarly, we use line~\ref{line:empty-ticket-list} to destroy the losing lottery tickets, now that lottery is over.
\begin{lstlisting}[language=flow, xleftmargin=0.8em]
var winners: list Ticket <-[nonempty st ticketWins(winNum,_)]-- tickets |\label{line:lottery-filter}|
// Split among winners using user-defined function `payEach`
winners --> payEach(jackpot / length(winners), _)
balance --> lotteryOwner.balance |\label{line:empty-lottery-balance}|
tickets --> consume // Lottery is over, destroy losing tickets
\end{lstlisting}

One could try automatically inserting such dynamic checks in a language like Solidity, but it would require extending Solidity with annotations (e.g., \flowinline{nonempty}).
Such a system would essentially reimplement flows, providing some benefits of \langName, but not the same static guarantees.

\paragraph{\textbf{Comparison with ERC-20 in Solidity}}\label{sec:erc20-impl}
Figure~\ref{fig:erc20-transfer-sol} shows a Solidity implementation of the same function as Figure~\ref{fig:erc20-transfer-flow}.
The sender's balance must be at least as large as \flowinline{amount}, and the destination's balance must not overflow when it receives the tokens.
\langName automatically inserts code checking these two conditions, ensuring the checks are not forgotten.
\begin{figure}
    \vspace{-2em}
    \centering
    \lstinputlisting[language=Solidity]{erc20-transfer.sol}
    \vspace{-1em}
    \caption{Excerpt from a Solidity reference implementation~\cite{erc20Consensys} of the \lstinline[language=Solidity]{transfer} function.
        Preconditions are checked manually.
        We must include the \lstinline[language=Solidity]{SafeMath} library (not shown) to use \lstinline[language=Solidity]{add} and \lstinline[language=Solidity]{sub}, which check for underflow/overflow.}
    \label{fig:erc20-transfer-sol}
    \vspace{-2em}
\end{figure}

\section{Conclusion and Future Work}

% We have presented the \langName language for writing safer smart contracts.
\langName uses flows, \assetTxt{}s, and type quantities to provide its safety guarantees.
We have shown an example smart contract in both \langName and Solidity, showing that \langName can express common smart contract functionality more concisely than Solidity, while retaining key safety properties.
% We have shown an example smart contract in both \langName and Solidity, showing that \langName is capable of expressing common smart contract functionality in a concise manner, while retaining key safety properties.

In the future, we plan to implement the \langName language by compiling it to Solidity, and prove its safety properties (e.g., asset retention).
We also hope to study the benefits and costs of the language via case studies, performance evaluation, and the application of flows to other domains.
Finally, we would also like to conduct a user study to evaluate the usability of the flow abstraction and the design of the language, and to compare it to Solidity, which we hypothesize will show that developers write contracts with fewer asset management errors in \langName than in Solidity.

\bibliographystyle{splncs04}
\bibliography{biblio}

\end{document}

