\documentclass[10pt]{article}

\usepackage{amsmath}
\usepackage{sagetex}
\usepackage{hyperref}
\usepackage{tikz-cd}
\usepackage{amssymb}
\usepackage{amsthm}
\usepackage{bm}
\usepackage{listings}
\usepackage{bbm}
\usepackage{multicol}
\usepackage{mathtools}
\usepackage{mathpartir}
\usepackage{float}
\usepackage[inline]{enumitem}
\usepackage[margin=1.25in]{geometry}
\usepackage[T1]{fontenc}
\usepackage{kpfonts}

\usetikzlibrary{decorations.pathmorphing}

\lstset{
  frame=none,
  xleftmargin=2pt,
  stepnumber=1,
  numbers=left,
  numbersep=5pt,
  numberstyle=\ttfamily\tiny\color[gray]{0.3},
  belowcaptionskip=\bigskipamount,
  captionpos=b,
  escapeinside={*'}{'*},
  tabsize=2,
  emphstyle={\bf},
  commentstyle=\it,
  stringstyle=\mdseries\rmfamily,
  showspaces=false,
  keywordstyle=\bfseries\rmfamily,
  columns=flexible,
  basicstyle=\small\sffamily,
  showstringspaces=false,
}

\newcommand{\guard}{\bigg|\bigg|}
\newcommand{\newc}{\textbf{\texttt{new}}}
\newcommand{\everything}{\textbf{\texttt{everything}}}
\newcommand{\then}{\textbf{\texttt{then}}}
\newcommand{\this}{\textbf{\texttt{this}}}
\newcommand{\langName}{LANGUAGE-NAME\xspace}
\newcommand{\consumesarr}{\mathrel{\rotatebox[origin=c]{270}{${\looparrowright}$}}}
\newcommand{\consumes}[1]{\stackrel{#1}{\consumesarr}}
\newcommand{\sends}[1]{\stackrel{#1}{\to}}
\newcommand{\emitsarr}{\mathrel{\rotatebox[origin=c]{270}{${\looparrowleft}$}}}
\newcommand{\emits}[1]{\emitsarr #1}
\newcommand{\asset}{\textbf{\texttt{asset}}\xspace}
\newcommand{\fungible}{\textbf{\texttt{fungible}}\xspace}
\newcommand{\nonfungible}{\textbf{\texttt{nonfungible}}\xspace}
\newcommand{\addresst}{\textbf{\texttt{address}}\xspace}
\newcommand{\stringt}{\textbf{\texttt{string}}\xspace}
\newcommand{\setof}{\textbf{\texttt{set}}\xspace}
\newcommand{\optiont}{\textbf{\texttt{option}}\xspace}
\newcommand{\boolt}{\textbf{\texttt{bool}}\xspace}
\newcommand{\natt}{\textbf{\texttt{nat}}\xspace}
\newcommand{\byt}{\textbf{\texttt{by}}\xspace}
\newcommand{\stores}{\textbf{\texttt{stores}}\xspace}
\newcommand{\suchthat}{\textbf{s.t.}\xspace}
\newcommand{\returns}{\textbf{\texttt{returns}}\xspace}
\newcommand{\merge}{\rightsquigarrow}
\newcommand{\flowsto}{\rightsquigarrow}
\newcommand{\heldby}{\rightarrowtail}
\newcommand{\one}{\textbf{\texttt{one}}\xspace}
\newcommand{\some}{\textbf{\texttt{some}}\xspace}
\newcommand{\any}{\textbf{\texttt{any}}\xspace}
\newcommand{\map}{\textbf{\texttt{map}}\xspace}
\newcommand{\linking}{\textbf{\texttt{linking}}\xspace}
\newcommand{\transformer}{\textbf{\texttt{transformer}}\xspace}
\newcommand{\selects}{\textbf{\texttt{selects}}\xspace}
\newcommand{\demote}{\textbf{\texttt{demote}}\xspace}
\newcommand{\transforms}{\Rrightarrow}

\newcommand{\flowproves}{\mathbin{|\hskip -0.1em\scalebox{0.7}[1]{$\leadsto$}}}
\newcommand{\flowproveout}{\mathbin{\scalebox{0.7}[1]{$\leadsto$}\hskip -0.13em|}}


\lstset{
  frame=none,
  xleftmargin=2pt,
  stepnumber=1,
  numbers=left,
  numbersep=5pt,
  numberstyle=\ttfamily\tiny\color[gray]{0.3},
  belowcaptionskip=\bigskipamount,
  captionpos=b,
  escapeinside={*'}{'*},
  tabsize=2,
  emphstyle={\bf},
  commentstyle=\it,
  stringstyle=\mdseries\rmfamily,
  showspaces=false,
  keywordstyle=\bfseries\rmfamily,
  columns=flexible,
  basicstyle=\small\sffamily,
  showstringspaces=false,
}

\newcommand{\guard}{\bigg|\bigg|}
\newcommand{\newc}{\textbf{\texttt{new}}}
\newcommand{\everything}{\textbf{\texttt{everything}}}
\newcommand{\then}{\textbf{\texttt{then}}}
\newcommand{\this}{\textbf{\texttt{this}}}
\newcommand{\langName}{LANGUAGE-NAME\xspace}
\newcommand{\consumesarr}{\mathrel{\rotatebox[origin=c]{270}{${\looparrowright}$}}}
\newcommand{\consumes}[1]{\stackrel{#1}{\consumesarr}}
\newcommand{\sends}[1]{\stackrel{#1}{\to}}
\newcommand{\emitsarr}{\mathrel{\rotatebox[origin=c]{270}{${\looparrowleft}$}}}
\newcommand{\emits}[1]{\emitsarr #1}
\newcommand{\asset}{\textbf{\texttt{asset}}\xspace}
\newcommand{\fungible}{\textbf{\texttt{fungible}}\xspace}
\newcommand{\nonfungible}{\textbf{\texttt{nonfungible}}\xspace}
\newcommand{\addresst}{\textbf{\texttt{address}}\xspace}
\newcommand{\stringt}{\textbf{\texttt{string}}\xspace}
\newcommand{\setof}{\textbf{\texttt{set}}\xspace}
\newcommand{\optiont}{\textbf{\texttt{option}}\xspace}
\newcommand{\boolt}{\textbf{\texttt{bool}}\xspace}
\newcommand{\natt}{\textbf{\texttt{nat}}\xspace}
\newcommand{\byt}{\textbf{\texttt{by}}\xspace}
\newcommand{\stores}{\textbf{\texttt{stores}}\xspace}
\newcommand{\suchthat}{\textbf{s.t.}\xspace}
\newcommand{\returns}{\textbf{\texttt{returns}}\xspace}
\newcommand{\merge}{\rightsquigarrow}
\newcommand{\flowsto}{\rightsquigarrow}
\newcommand{\heldby}{\rightarrowtail}
\newcommand{\one}{\textbf{\texttt{one}}\xspace}
\newcommand{\some}{\textbf{\texttt{some}}\xspace}
\newcommand{\any}{\textbf{\texttt{any}}\xspace}
\newcommand{\map}{\textbf{\texttt{map}}\xspace}
\newcommand{\linking}{\textbf{\texttt{linking}}\xspace}
\newcommand{\transformer}{\textbf{\texttt{transformer}}\xspace}
\newcommand{\selects}{\textbf{\texttt{selects}}\xspace}
\newcommand{\demote}{\textbf{\texttt{demote}}\xspace}
\newcommand{\transforms}{\Rrightarrow}

\newcommand{\flowproves}{\mathbin{|\hskip -0.1em\scalebox{0.7}[1]{$\leadsto$}}}
\newcommand{\flowproveout}{\mathbin{\scalebox{0.7}[1]{$\leadsto$}\hskip -0.13em|}}



\begin{document}

\section{Specification}

\subsection{Syntax}
\begin{figure}[ht]
\begin{align*}
    C &\in \textsc{ContractNames} & m &\in \textsc{TransactionNames} \\
    t &\in \textsc{TypeNames} & x,y,z &\in \textsc{Identifiers} \\
    n &\in \Z \\
\end{align*}
\begin{tabular}{l r l l}
    $\mathcal{Q}$, $\mathcal{R}$, $\mathcal{S}$ & \bnfdef & $\exactlyone$ \bnfalt $\any$ \bnfalt $\nonempty$ \bnfalt $\emptyq$ \bnfalt $\every$ & (\typeQuantities) \\
    % Maybe:  $\alpha$ & \bnfdef & & (row type variables) \\
    $T$ & \bnfdef & $\voidt$ \bnfalt \boolt \bnfalt $\natt$ \bnfalt $\mathcal{C}~\tau$ \bnfalt $\tau \transforms \tau$ \bnfalt $\curlys{\overline{x : \tau}}$ \bnfalt $t$ & (base types) \\
    $\tau$, $\sigma$, $\pi$ & \bnfdef & $\mathcal{Q}~T$ & (types) \\
    $\mathcal{V}$ & \bnfdef & $n$ \bnfalt \true \bnfalt \false \bnfalt $\emptyq$ \bnfalt $\lambda x : \tau. E$ & (values) \\
    $E$ & \bnfdef & $\mathcal{V}$ \bnfalt $x$ \bnfalt $x.m(\overline{x})$ \bnfalt $s~\varin~x$ \bnfalt $\curlys{\overline{x : \tau \mapsto x}}$ & \\
        & \bnfalt & $\letvar~x:\tau := E~\varin~E$ \bnfalt $\ifS~x~\thenS~E~\elseS~E$ & (expressions) \\
    $s$ & \bnfdef & $\mathcal{Q}~x : \tau~\suchthat~E$ & (selector) \\
    $\mathcal{S}$ & \bnfdef & $x$ \bnfalt $\newc~t$ & (sources) \\
    $\mathcal{D}$ & \bnfdef & $x$ \bnfalt $\consume$ \bnfalt $\var~x:\tau$ & (destinations) \\
    $F$ & \bnfdef & $\mathcal{S} \sends{s} x \to \mathcal{D}$ & (flows) \\
    $\Stmt$ & \bnfdef & $F$ \bnfalt $\revert(E)$ \bnfalt $\ifS~x~\thenS~S~\elseS~S$ \bnfalt $S;S$ & (statements) \\
    $M$ & \bnfdef & \fungible \bnfalt \nonfungible \bnfalt \consumable \bnfalt \asset & (type declaration modifiers) \\
    $\Decl$ & \bnfdef & $\type~t~\is~\overline{M}~T$ & (type declaration) \\
            & \bnfalt & $\transaction~m(\overline{x : \tau})~\returns~x : \tau~\doC~S$ & (transactions) \\
    $\Prog$ & \bnfdef & $\overline{\Decl}~;~S$ & (programs)
\end{tabular}
\caption{Abstract syntax of \langName.}
\label{lang-syntax}
\end{figure}

\subsection{Statics}
\begin{definition}
    A \emph{type environment} $\Gamma$ is a partial function $\textsc{Identifiers} \partialfunc \textsc{Types}$, where we write $x : \tau$ to mean $(x, \tau)$.
    We can consider a type environment $\Gamma$ as a (non-partial) function $\textsc{Identifiers}\xspace \to \textsc{Types}\xspace \cup \curlys{\bot}$ as follows:
    \[
        \Gamma(x) =
        \begin{cases}
            \tau & \tif x : \tau \in \Gamma \\
            \bot & \owise
        \end{cases}
    \]
    We write $\dom(\Gamma)$ to mean $\setbuild{x \in \textsc{Identifiers}}{\Gamma(x) \neq \bot}$, and $\Gamma|_X$ to mean the environment $\setbuild{x : \tau \in \Gamma}{x \in X}$ (restricting the domain of $\Gamma$).
\end{definition}

\begin{definition}
    Define $\Quant = \curlys{\emptyq, \any, \exactlyone, \nonempty, \every}$, and call any $\mathcal{Q} \in \Quant$ a \emph{type quantity}.
    Define $\emptyq < \any < \,\, \exactlyone < \nonempty < \every$.
\end{definition}

\framebox{$\tau~\asset$} \textbf{Asset Types}

\begin{align*}
    (\mathcal{Q}~T)~\asset \iff \mathcal{Q} \neq \emptyq \tand (& \asset \in \modifiers(T) \tor \\
                                                                & (T = \mathcal{C}~\tau \tand \tau~\asset) \tor \\
                                                                & (T = \curlys{\Gamma} \tand \exists x : \tau \in \Gamma. (\tau~\asset)))
\end{align*}

\framebox{$\tau~\consumable$} \textbf{Consumable Types}
\begin{align*}
    (\mathcal{Q}~T)~\consumable &\iff \consumable \in \modifiers(T) \tor \lnot((\mathcal{Q}~T)~\asset)
\end{align*}

$\mathcal{Q} \oplus \mathcal{R}$ represents the quantity present when flowing $\mathcal{R}$ of something to a storage already containing $\mathcal{Q}$.
$\mathcal{Q} \ominus \mathcal{R}$ represents the quantity left over after flowing $\mathcal{R}$ from a storage containing $\mathcal{Q}$.

\begin{definition}
    Let $\mathcal{Q}, \mathcal{R} \in \Quant$.
    Define the commutative operator $\oplus$, called \emph{combine}, as the function:
    \begin{align*}
        \oplus : \textsc{Type} \times \Quant \to \textsc{Type} \\
        (\mathcal{Q}~T, \mathcal{R}) \mapsto \max(\mathcal{Q}, \mathcal{R})~T
    \end{align*}

    Define the operator $\ominus$, called \emph{split}, as the unique function $\Quant^2 \to \Quant$ such that
    \[
        % \setlength\arraycolsep{1pt}
        \renewcommand\arraystretch{1.2}
        \begin{array}{r c l l l}
            \mathcal{Q} \ominus \emptyq & = & \mathcal{Q} & \\
            \emptyq \ominus \mathcal{R} & = & \emptyq & \\
            \mathcal{Q} \ominus \every &= & \emptyq & \\
            \every \ominus \mathcal{R} &= & \every & \tif \mathcal{R} < \every \\
            \nonempty - \mathcal{R} &= & \any & \tif \empty < \mathcal{R} < \every \\
            \exactlyone - \mathcal{R} &= & \emptyq & \tif \exactlyone \leq \mathcal{R} \\
            \exactlyone - \any &= & \any & \\
            \any - \mathcal{R} &= & \any & \tif \empty < \mathcal{R} < \every \\
        \end{array}
    \]
\end{definition}
Note that we write $(\mathcal{Q}~T) \oplus \mathcal{R}$ to mean $(\mathcal{Q} \oplus \mathcal{R})~T$ and similarly $(\mathcal{Q}~T) \ominus \mathcal{R}$ to mean $(\mathcal{Q} \ominus \mathcal{R})~T$.

\begin{definition}
    Let $\mathcal{Q}$ and $\mathcal{R}$ be \typeQuantities, $T_\mathcal{Q}$ and $T_\mathcal{R}$ base types, and $\Gamma$ and $\Delta$ type environments.
    Define the following orderings, which make types and type environments into join-semilattices.
    For type quantities, define the partial order $\sqsubseteq$ as the reflexive closure of the strict partial order $\sqsubset$ given by
    \[
        \mathcal{Q} \sqsubset \mathcal{R} \iff \mathcal{R} = \any \tor (\mathcal{Q} \in \curlys{\exactlyone, \every} \tand \mathcal{R} = \nonempty)
    \]
    For types, define the partial order $\leq$ by
    \[
        \mathcal{Q}~T_\mathcal{Q} \leq \mathcal{R}~T_\mathcal{R} \iff T_\mathcal{Q} = T_\mathcal{R} \tand \mathcal{Q} \sqsubseteq \mathcal{R}
    \]
    For type environments, define the partial order $\leq$ by
    \[
        \Gamma \leq \Delta \iff \forall x. \Gamma(x) \leq \Delta(x)
    \]
    Denote the join of $\Gamma$ and $\Delta$ by $\Gamma \sqcup \Delta$.
\end{definition}

\framebox{$\Gamma \flowproves E : \tau \flowprovesout \Delta$} \textbf{Expression Typing}

These rules are for ensuring that expressions are well-typed, and keeping track of which variables are used throughout the expression.
Most rules do \textbf{not} change the context, with the notable exceptions of internal calls and record-building operations.
We begin with the rules for typing the various literal forms of values.
\begin{mathpar}
    \inferrule*[right=Empty-Val]{
    }{ \Gamma \flowproves \emptyq : \emptyq~\mathcal{C}~\tau \flowprovesout \Gamma }

    \inferrule*[right=Transformer]{
        \Gamma, x : \tau \flowproves E : \sigma \flowprovesout \Gamma
    }{ \Gamma \flowproves (\lambda x : \tau. E) : \,\,\exactlyone~(\tau \transforms \sigma) }

    \inferrule*[right=Build-Rec]{
        \Gamma \flowproves \overline{y : \tau} \flowprovesout \Delta
    }{ \Gamma \flowproves \curlys{\overline{x : \tau \mapsto y}} \flowprovesout \Delta }
\end{mathpar}

Next, the lookup rules.
Notably, the \textsc{Demote-Lookup} rule allows the use of variables of an asset type in an expression without consuming the variable as \textsc{Lin-Lookup} does.
However, it is still safe, because it is treated as its demoted type, which is always guaranteed to be a non-asset.

\begin{mathpar}
    \inferrule*[right=Demote-Lookup]{
    }{ \Gamma, x : \tau \flowproves x : \demote(\tau) \flowprovesout \Gamma, x : \tau }

    \inferrule*[right=Lin-Lookup]{
    }{ \Gamma, x : \mathcal{Q}~T \flowproves x : \mathcal{Q}~T \flowprovesout \Gamma, x : \emptyq~T }

    \inferrule*[right=Record-Field-Lookup]{
        \Gamma \proves x : \curlys{\overline{y : \tau}} \flowproves \Gamma
        \and
        f : \sigma \in \overline{y : \tau}
    }{ \Gamma \flowproves x.f : \sigma \flowproves \Gamma }
\end{mathpar}

The expression $s~\varin~x$ allows checking whether a flow will succeed without the EAFP-style (``Easier to ask for forgiveness than permission''; e.g., Python).
A flow $A \sends{s} B$ is guaranteed to succeed when ``$s~\varin~A$'' is true and ``$s~\varin~B$'' is false.
\begin{mathpar}
    \inferrule*[right=Check-In]{
        \Gamma \proves s~\selects~\demote(\sigma)
        \and
        \Gamma \proves x :: \tau \flowsto_\mathcal{Q} \sigma
    }{ \Gamma \proves (s~\varin~x) : \boolt \flowprovesout \Gamma }
\end{mathpar}

We distinguish between three kinds of calls: view, internal, and external.
A view call is guaranteed to not change any state in the receiver, while both internal and external calls may do so.
The difference between internal and external calls is that we may transfer assets to an internal call, but \textbf{not} to an external call, because we cannot be sure any external contract will properly manage the asset of our contract.
Similarly, we cannot be sure that a view of an external contract \textbf{really} is a view, and so we can only call views of the current contract.
\begin{mathpar}
    \inferrule*[right=Tx-Call]{
        (\transaction~m(\overline{a : \tau})~\returns~\sigma~\doC~S) \in \decls
        \and
        \Gamma \flowproves \overline{y : \tau} \flowprovesout \Delta
    }{ \Gamma \flowproves m(\overline{y}) : \sigma \flowprovesout \Delta }
\end{mathpar}

Finally, the rules for If and Let expressions.
In \textsc{Let-Expr}, we ensure that the newly bound variable is either consumed or is not an asset in the body.
\begin{mathpar}
    \inferrule*[right=If-Expr]{
        \Gamma \flowproves x : \boolt \flowprovesout \Gamma
        \and
        \Gamma \flowproves E_1 : \tau \flowprovesout \Delta
        \and
        \Gamma \flowproves E_2 : \tau \flowprovesout \Xi
    }{ \Gamma \flowproves (\ifS~x~\thenS~E_1~\elseS~E_2) : \tau \flowprovesout \Delta \sqcup \Xi }

    \inferrule*[right=Let-Expr]{
        \Gamma \flowproves E_1 : \tau \flowprovesout \Delta
        \and
        \Delta, x : \tau \flowproves E_2 : \pi \flowprovesout \Xi, x : \sigma
        \and
        \lnot(\sigma~\asset)
    }{ \Gamma \flowproves (\letvar~x : \tau := E_1~\varin~E_2) : \pi \flowprovesout \Xi }
\end{mathpar}

\framebox{$\Gamma \proves \mathcal{S}~\provides_{\mathcal{Q}}~\tau$} \textbf{Source Typing}
\begin{mathpar}
    \inferrule*[right=Provide-One]{
    }{ \Gamma, S : \mathcal{Q}~T \proves S~\provides_{\mathcal{Q}}~\mathcal{R}~T }

    \inferrule*[right=Provide-Col]{
    }{ \Gamma, S : \mathcal{Q}~\mathcal{C}~\tau \proves S~\provides_{\mathcal{Q}}~\tau }

    \inferrule*[right=Provide-Source]{
        (\type~t~\is~\overline{M}~T) \in \decls(C)
    }{ \Gamma, \this : C \proves \newc~t~\provides_{\every} t }
\end{mathpar}

\framebox{$\Gamma \proves \mathcal{D}~\accepts~\tau$} \textbf{Destination Typing}
\reed{Prevent variables that are supposed to store exactly one of something from receiving another?}
\begin{mathpar}
    \inferrule*[right=Accept-One]{
    }{ \Gamma, S : \mathcal{Q}~T \proves S~\accepts~\mathcal{R}~T }

    \inferrule*[right=Accept-Col]{
    }{ \Gamma, S : \mathcal{Q}~\mathcal{C}~\tau \proves S~\accepts~\tau }

    \inferrule*[right=Accept-Consume]{
        \tau~\consumable
    }{ \Gamma \flowproves \consume~\accepts~\tau }
\end{mathpar}

\framebox{$\Gamma \proves s ~\selects_\mathcal{Q}~ \tau$} \textbf{Selectors}
\begin{mathpar}
    \inferrule*[right=Select-Quant]{
        \Gamma, x : \tau \flowproves p : \boolt \flowprovesout \Gamma, x : \tau
    }{ \Gamma \proves (\mathfrak{q}~x : \tau~\suchthat~p)~\selects_\mathfrak{q}~\tau }
\end{mathpar}

\framebox{$\Gamma \flowproves S~\ok \flowprovesout \Delta$} \textbf{Statement Well-formedness}

\begin{mathpar}
    \inferrule*[right=Wf-Flow]{
        \Gamma \proves A~\provides_\mathcal{Q}~\tau
        \and
        \Gamma \proves s~\selects_\mathcal{R}~\demote(\tau)
        \and
        \min(\mathcal{Q}, \mathcal{R}) < \every
        \\
        \Delta = \update(\Gamma, A, \Gamma(A) \ominus \mathcal{R})
        \and
        \Delta \proves f : \tau \transforms \sigma
        \and
        \Delta \proves B~\accepts~\sigma
    }{ \Gamma \flowproves (A \sends{s} f \to B)~\ok \flowprovesout \update(\Delta, B, \Delta(B) \oplus \min(\mathcal{Q}, \mathcal{R})) }
\end{mathpar}

\begin{mathpar}
    \inferrule*[right=Wf-If]{
        \Gamma \flowproves x : \boolt \flowprovesout \Gamma
        \and
        \Gamma \flowproves S_1~\ok \flowprovesout \Delta
        \and
        \Gamma \flowproves S_2~\ok \flowprovesout \Xi
    }{ \Gamma \flowproves (\ifS~x~\thenS~S_1~\elseS~S_2)~\ok \flowprovesout \Delta \sqcup \Xi }

    \inferrule*[right=Wf-Revert]{
        \Gamma \flowproves E : \tau \flowprovesout \Gamma
    }{ \Gamma \flowproves \revert(E)~\ok \flowprovesout \Gamma }

    \inferrule*[right=Wf-Seq]{
        \Gamma \flowproves S_1~\ok \flowprovesout \Delta
        \and
        \Delta \flowproves S_2~\ok \flowprovesout \Xi
    }{ \Gamma \flowproves (S_1 ; S_2)~\ok \flowprovesout \Xi }
\end{mathpar}

\framebox{$\Decl~\ok$} \textbf{Declaration Well-formedness}
\begin{mathpar}
    \inferrule*[right=Wf-Tx]{
        \overline{x : \tau}, y : \emptyq~T \flowproves S~\ok \flowprovesout \Delta, y : \mathcal{Q}~T
        \\
        \dom(\fields(C)) \cap \dom(\Delta) = \emptyset
        \and
        \forall x : \tau \in \Delta. \lnot(\tau~\asset)
    }{ (\transaction~m(\overline{x : \tau})~\returns~y : \mathcal{Q}~T~\doC~S)~\ok }

    \inferrule*[right=Wf-Type]{
        T~\asset \implies \asset \in \overline{M}
    }{ (\type~t~\is~\overline{M}~T)~\ok }
\end{mathpar}

\reed{Public transaction shouldn't be able to return an asset?
Actually, everything would be a lot nicer if you just couldn't move assets around with transactions.
Probably wouldn't be that limiting, because of the demotion system.}

\framebox{$\Prog~\ok$} \textbf{Program Well-formedness}
\begin{mathpar}
    \inferrule*[right=Wf-Prog]{
        \forall D \in \overline{\Decl}. D~\ok
        \and
        \emptyset \flowproves S \flowprovesout \emptyset
    }{ (\overline{\Decl} ; S)~\ok }
\end{mathpar}

\paragraph{Other Auxiliary Definitions}
\reed{Eliminate all the locations except for $x$ and then use flows to extract and put stuff back?}
\framebox{$\modifiers(T) = \overline{M}$} \textbf{Type Modifiers}
\begin{align*}
    \modifiers(T) =
    \begin{cases}
        \overline{M} & \tif (\type~T~\is~\overline{M}~T) \\
        \emptyset & \owise
    \end{cases}
\end{align*}

\framebox{$\demote(\tau) = \sigma$}
\framebox{$\demote_*(T_1) = T_2$} \textbf{Type Demotion}
$\demote$ and $\demote_*$ take a type and ``strip'' all the asset modifiers from it, as well as unfolding named type definitions.
This process is useful, because it allows selecting asset types without actually having a value of the desired asset type.
Note that demoting a transformer type changes nothing.
This is because a transformer is \textbf{never} an asset, regardless of the types that it operators on, because it has no storage.

\begin{align*}
    \demote(\mathcal{Q}~T) &= \mathcal{Q}~\demote_*(T) \\
    \demote_*(\voidt) &= \voidt \\
    \demote_*(\natt) &= \natt \\
    \demote_*(\boolt) &= \boolt \\
    \demote_*(t) &= \demote_*(T) & \twhere \type~t~\is~\overline{M}~T \\
    \demote_*(C) &= \demote_*(\curlys{\overline{x : \tau}}) & \twhere \fields(C) = \curlys{\overline{x : \tau}} \\
    \demote_*(\curlys{\overline{x : \tau}}) &= \curlys{\overline{x : \demote(\tau)}} \\
    % \demote_*(\tau \transforms \sigma) &= \demote(\tau) \transforms \demote(\sigma)
    \demote_*(\tau \transforms \sigma) &= \tau \transforms \sigma
\end{align*}

\framebox{$\update(\Gamma, x, \tau)$} \textbf{Type environment modification}
% These auxiliary functions are used to update the statically known information about a storage after flowing in or out of it (see the \textsc{Flow} rule).
\[
    \update(\Gamma, x, \tau) =
    \begin{cases}
        \Delta, x : \tau & \tif \Gamma = \Delta, x : \sigma \\
        \Gamma & \owise
    \end{cases}
\]

\reed{Asset retention theorem?}
\reed{Resource accessiblity?}

\reed{What guarantees should we provide (no errors except for flowing a resource that doesn't exist in the source/already exists in the destination)?}

\end{document}

