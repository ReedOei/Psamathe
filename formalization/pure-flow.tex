\documentclass[10pt]{article}

\usepackage{amsmath}
\usepackage{hyperref}
\usepackage{tikz-cd}
\usepackage{amssymb}
\usepackage{amsthm}
\usepackage{bm}
\usepackage{listings}
\usepackage{bbm}
\usepackage{multicol}
\usepackage{mathtools}
\usepackage{mathpartir}
\usepackage{float}
\usepackage[inline]{enumitem}
\usepackage[margin=1.25in]{geometry}
\usepackage[T1]{fontenc}
\usepackage{kpfonts}

\usetikzlibrary{decorations.pathmorphing}

\lstset{
  frame=none,
  xleftmargin=2pt,
  stepnumber=1,
  numbers=left,
  numbersep=5pt,
  numberstyle=\ttfamily\tiny\color[gray]{0.3},
  belowcaptionskip=\bigskipamount,
  captionpos=b,
  escapeinside={*'}{'*},
  tabsize=2,
  emphstyle={\bf},
  commentstyle=\it,
  stringstyle=\mdseries\rmfamily,
  showspaces=false,
  keywordstyle=\bfseries\rmfamily,
  columns=flexible,
  basicstyle=\small\sffamily,
  showstringspaces=false,
}

\newcommand{\guard}{\bigg|\bigg|}
\newcommand{\newc}{\textbf{\texttt{new}}}
\newcommand{\everything}{\textbf{\texttt{everything}}}
\newcommand{\then}{\textbf{\texttt{then}}}
\newcommand{\this}{\textbf{\texttt{this}}}
\newcommand{\langName}{LANGUAGE-NAME\xspace}
\newcommand{\consumesarr}{\mathrel{\rotatebox[origin=c]{270}{${\looparrowright}$}}}
\newcommand{\consumes}[1]{\stackrel{#1}{\consumesarr}}
\newcommand{\sends}[1]{\stackrel{#1}{\to}}
\newcommand{\emitsarr}{\mathrel{\rotatebox[origin=c]{270}{${\looparrowleft}$}}}
\newcommand{\emits}[1]{\emitsarr #1}
\newcommand{\asset}{\textbf{\texttt{asset}}\xspace}
\newcommand{\fungible}{\textbf{\texttt{fungible}}\xspace}
\newcommand{\nonfungible}{\textbf{\texttt{nonfungible}}\xspace}
\newcommand{\addresst}{\textbf{\texttt{address}}\xspace}
\newcommand{\stringt}{\textbf{\texttt{string}}\xspace}
\newcommand{\setof}{\textbf{\texttt{set}}\xspace}
\newcommand{\optiont}{\textbf{\texttt{option}}\xspace}
\newcommand{\boolt}{\textbf{\texttt{bool}}\xspace}
\newcommand{\natt}{\textbf{\texttt{nat}}\xspace}
\newcommand{\byt}{\textbf{\texttt{by}}\xspace}
\newcommand{\stores}{\textbf{\texttt{stores}}\xspace}
\newcommand{\suchthat}{\textbf{s.t.}\xspace}
\newcommand{\returns}{\textbf{\texttt{returns}}\xspace}
\newcommand{\merge}{\rightsquigarrow}
\newcommand{\flowsto}{\rightsquigarrow}
\newcommand{\heldby}{\rightarrowtail}
\newcommand{\one}{\textbf{\texttt{one}}\xspace}
\newcommand{\some}{\textbf{\texttt{some}}\xspace}
\newcommand{\any}{\textbf{\texttt{any}}\xspace}
\newcommand{\map}{\textbf{\texttt{map}}\xspace}
\newcommand{\linking}{\textbf{\texttt{linking}}\xspace}
\newcommand{\transformer}{\textbf{\texttt{transformer}}\xspace}
\newcommand{\selects}{\textbf{\texttt{selects}}\xspace}
\newcommand{\demote}{\textbf{\texttt{demote}}\xspace}
\newcommand{\transforms}{\Rrightarrow}

\newcommand{\flowproves}{\mathbin{|\hskip -0.1em\scalebox{0.7}[1]{$\leadsto$}}}
\newcommand{\flowproveout}{\mathbin{\scalebox{0.7}[1]{$\leadsto$}\hskip -0.13em|}}


\lstset{
  frame=none,
  xleftmargin=2pt,
  stepnumber=1,
  numbers=left,
  numbersep=5pt,
  numberstyle=\ttfamily\tiny\color[gray]{0.3},
  belowcaptionskip=\bigskipamount,
  captionpos=b,
  escapeinside={*'}{'*},
  tabsize=2,
  emphstyle={\bf},
  commentstyle=\it,
  stringstyle=\mdseries\rmfamily,
  showspaces=false,
  keywordstyle=\bfseries\rmfamily,
  columns=flexible,
  basicstyle=\small\sffamily,
  showstringspaces=false,
}

\newcommand{\guard}{\bigg|\bigg|}
\newcommand{\newc}{\textbf{\texttt{new}}}
\newcommand{\everything}{\textbf{\texttt{everything}}}
\newcommand{\then}{\textbf{\texttt{then}}}
\newcommand{\this}{\textbf{\texttt{this}}}
\newcommand{\langName}{LANGUAGE-NAME\xspace}
\newcommand{\consumesarr}{\mathrel{\rotatebox[origin=c]{270}{${\looparrowright}$}}}
\newcommand{\consumes}[1]{\stackrel{#1}{\consumesarr}}
\newcommand{\sends}[1]{\stackrel{#1}{\to}}
\newcommand{\emitsarr}{\mathrel{\rotatebox[origin=c]{270}{${\looparrowleft}$}}}
\newcommand{\emits}[1]{\emitsarr #1}
\newcommand{\asset}{\textbf{\texttt{asset}}\xspace}
\newcommand{\fungible}{\textbf{\texttt{fungible}}\xspace}
\newcommand{\nonfungible}{\textbf{\texttt{nonfungible}}\xspace}
\newcommand{\addresst}{\textbf{\texttt{address}}\xspace}
\newcommand{\stringt}{\textbf{\texttt{string}}\xspace}
\newcommand{\setof}{\textbf{\texttt{set}}\xspace}
\newcommand{\optiont}{\textbf{\texttt{option}}\xspace}
\newcommand{\boolt}{\textbf{\texttt{bool}}\xspace}
\newcommand{\natt}{\textbf{\texttt{nat}}\xspace}
\newcommand{\byt}{\textbf{\texttt{by}}\xspace}
\newcommand{\stores}{\textbf{\texttt{stores}}\xspace}
\newcommand{\suchthat}{\textbf{s.t.}\xspace}
\newcommand{\returns}{\textbf{\texttt{returns}}\xspace}
\newcommand{\merge}{\rightsquigarrow}
\newcommand{\flowsto}{\rightsquigarrow}
\newcommand{\heldby}{\rightarrowtail}
\newcommand{\one}{\textbf{\texttt{one}}\xspace}
\newcommand{\some}{\textbf{\texttt{some}}\xspace}
\newcommand{\any}{\textbf{\texttt{any}}\xspace}
\newcommand{\map}{\textbf{\texttt{map}}\xspace}
\newcommand{\linking}{\textbf{\texttt{linking}}\xspace}
\newcommand{\transformer}{\textbf{\texttt{transformer}}\xspace}
\newcommand{\selects}{\textbf{\texttt{selects}}\xspace}
\newcommand{\demote}{\textbf{\texttt{demote}}\xspace}
\newcommand{\transforms}{\Rrightarrow}

\newcommand{\flowproves}{\mathbin{|\hskip -0.1em\scalebox{0.7}[1]{$\leadsto$}}}
\newcommand{\flowproveout}{\mathbin{\scalebox{0.7}[1]{$\leadsto$}\hskip -0.13em|}}



\newcommand{\putS}{\textbf{\texttt{put}}\xspace}
\newcommand{\evaluates}{\to}
\newcommand{\compatQuan}{\textbf{\texttt{compat}}\xspace}
\newcommand{\resolve}{\textbf{\texttt{resolve}}\xspace}
\newcommand{\values}{\textbf{\texttt{values}}\xspace}
\newcommand{\skipS}{\textbf{\texttt{skip}}\xspace}

\begin{document}

\section{Formalization}

\subsection{Syntax}
\begin{align*}
    f &\in \textsc{TransformerNames} & t &\in \textsc{TypeNames} \\
    a,x,y,z &\in \textsc{Identifiers} \\
\end{align*}
\begin{tabular}{l r l l}
    $\mathcal{Q}$, $\mathcal{R}$, $\mathcal{S}$ & \bnfdef & $\exactlyone$ \bnfalt $\any$ \bnfalt $\nonempty$ \bnfalt $\emptyq$ \bnfalt $\every$ & (type quantities) \\
    $M$ & \bnfdef & \fungible \bnfalt \unique \bnfalt \immutable \bnfalt \consumable \bnfalt \asset & (type declaration modifiers) \\
    $T$ & \bnfdef & $\boolt$ \bnfalt $\natt$ \bnfalt $\type~t~\is~\overline{M}~T$ \bnfalt $\listq~\tau$ \bnfalt $\{ \overline{x : \tau} \}$ & (base types) \\
    $\tau$, $\sigma$, $\pi$ & \bnfdef & $\mathcal{Q}~T$ & (types) \\
    $\mathcal{S}$ & \bnfdef & $x$ \bnfalt $x.y$ \bnfalt $\true$ \bnfalt $\false$ \bnfalt $n$ \bnfalt $\demote(x)$ \bnfalt $[x]$ \bnfalt $\{ \overline{x : \tau \mapsto x} \}$ \bnfalt $\newc(t, \overline{M}, T)$ & (sources) \\
    $\mathcal{D}$ & \bnfdef & $x$ \bnfalt $x.y$ \bnfalt $\var~x:T$ \bnfalt $\consume$ & (destinations) \\
    $\Decl$ & \bnfdef & $\transformer~f(\overline{x : \tau}) \to x : \tau ~\{ \overline{\Stmt} \}$ & (transformers) \\
    $\Stmt$ & \bnfdef & $\skipS$ & \\
            & \bnfalt & $\mathcal{S} \to \mathcal{D}$ \bnfalt $\mathcal{S} \sends{x} \mathcal{D}$ \bnfalt $\mathcal{S} \sends{\mathcal{Q}~\suchthat~f(\overline{x})} \mathcal{D}$ \bnfalt $\mathcal{S} \to f(\overline{x}) \to \mathcal{D}$ & \\
            & \bnfalt & $\tryS~\{ \overline{\Stmt} \}~\catchS~\{ \overline{\Stmt} \}$ & \\
    $\Prog$ & \bnfdef & $\overline{\Decl}; \overline{\Stmt}$
\end{tabular}

\reed{Add rules for flow-by-variable.}
\reed{Remove bool type?
Can implement the ``filter'' selectors another way, e.g., by using a transformer returning a pair.}

\subsection{Statics}

\framebox{$\Gamma \flowproves \mathcal{S} : \tau \flowprovesout \Delta$}
\framebox{$\Gamma \flowproves \mathcal{D} : \tau \flowprovesout \Delta$} \textbf{Storage Typing}

A \emph{storage} is either a source or a destination.

\begin{mathpar}
    \inferrule*[right=Bool]{
        b \in \{ \true, \false \}
    }{ \Gamma \flowproves b : \,\,\exactlyone~\boolt \flowprovesout \Gamma }

    \inferrule*[right=Nat]{
    }{ \Gamma \flowproves n : \,\,\exactlyone~\natt \flowprovesout \Gamma }

    \inferrule*[right=Demote]{
    }{ \Gamma, x : \tau \flowproves \demote(x) : \demote(\tau) \flowprovesout \Gamma, x : \tau }

    \inferrule*[right=Var]{
        \lnot(\tau~\immutable)
    }{ \Gamma, x : \tau \flowproves x : \tau \flowprovesout \Gamma, x : \tau }

    \inferrule*[right=Field]{
        \Gamma \flowproves x : \,\,\exactlyone~T \flowprovesout \Delta
        \\
        \lnot(T~\immutable)
        \\
        \fields(T) = \overline{z : \sigma}
        \\
        y : \tau \in \overline{z : \sigma}
    }{ \Gamma \flowproves x.y : \tau \flowprovesout \Gamma }

    \inferrule*[right=Single]{
    }{ \Gamma, x : \mathcal{Q}~T \flowproves [x] : \,\,\exactlyone~\listq~\mathcal{Q}~T \flowprovesout \Gamma, x : \emptyq~T }

    \inferrule*[right=Record]{
    }{ \Gamma, \overline{y : \mathcal{Q}~T} \flowproves \{ \overline{x : \mathcal{Q}~T \mapsto y} \} : \,\,\exactlyone~\{ \overline{x : \mathcal{Q}~T} \} \flowprovesout \Gamma, \overline{y : \emptyq~T} }

    \inferrule*[right=New]{
    }{ \Gamma \flowproves \newc(t, \overline{M}, T) : \every~\listq~\exactlyone~(\type~t~\is~\overline{M}~T) \flowprovesout \Gamma }

    \inferrule*[right=VarDef]{
    }{ \Gamma \flowproves (\var~x : T) : \emptyq~T \flowprovesout \Gamma, x : \emptyq~T }

    \inferrule*[right=Consume]{
        \tau~\consumable
    }{ \Gamma \flowproves \consume : \tau \flowprovesout \Gamma }
\end{mathpar}

\framebox{$\Gamma \flowproves S~\ok \flowprovesout \Delta$} \textbf{Statement Well-formedness}
\begin{mathpar}
    \inferrule*[right=Ok-Skip]{
    }{ \Gamma \flowproves \skipS~\ok \flowprovesout \Gamma }

    \inferrule*[right=Ok-Flow-Every]{
        \Gamma \flowproves \mathcal{S} : \mathcal{Q}~T \flowprovesout \Delta
        \\
        \update(\Delta, \mathcal{S}, \Delta(\mathcal{S}) \ominus \mathcal{Q}) \flowproves \mathcal{D} : \mathcal{R}~T \flowprovesout \Xi
    }{ \Gamma \flowproves (\mathcal{S} \to \mathcal{D})~\ok \flowprovesout \update(\Xi, \mathcal{D}, \Xi(\mathcal{D}) \oplus \mathcal{Q}) }

    \inferrule*[right=Ok-Flow-Var]{
        \Gamma \flowproves \mathcal{S} : \mathcal{Q}~T \flowprovesout \Delta
        \\
        \Delta \flowproves x : \demote(\mathcal{R}~T) \flowprovesout \Delta
        \\
        \update(\Delta, \mathcal{S}, \Delta(\mathcal{S}) \ominus \mathcal{Q}) \flowproves \mathcal{D} : \mathcal{S}~T \flowprovesout \Xi
    }{ \Gamma \flowproves (\mathcal{S} \sends{x} \mathcal{D})~\ok \flowprovesout \update(\Xi, \mathcal{D}, \Xi(\mathcal{D}) \oplus \mathcal{R}) }

    \inferrule*[right=Ok-Flow-Filter]{
        \Gamma \flowproves \mathcal{S} : \mathcal{Q}~T \flowprovesout \Delta
        \\
        \transformer~f(\overline{x : \sigma}, y : \demote(\elemtype(T))) \to z : \,\,!~\boolt ~\{~\overline{\Stmt}~\}
        \\
        \forall i. \demote(\Gamma(a_i)) = \sigma_i
        \\
        \update(\Delta, \mathcal{S}, \Delta(\mathcal{S}) \ominus \mathcal{Q}) \flowproves \mathcal{D} : \mathcal{S}~T \flowprovesout \Xi
    }{ \Gamma \flowproves (\mathcal{S} \sends{\mathcal{R}~\suchthat~f(\overline{a})} \mathcal{D})~\ok \flowprovesout \update(\Xi, \mathcal{D}, \Xi(\mathcal{D}) \oplus \min(\mathcal{Q}, \mathcal{R})) }

    \inferrule*[right=Ok-Flow-Transformer]{
        \Gamma \flowproves \mathcal{S} : \mathcal{Q}~T_1 \flowprovesout \Delta
        \\
        \transformer~f(\overline{x : \sigma}, y : \demote(\elemtype(T_1))) \to z : \mathcal{R}~T_2~\{~ \overline{\Stmt} ~\}
        \\
        \forall i. \demote(\Gamma(x_i)) = \sigma_i
        \\
        \update(\Delta, \mathcal{S}, \Delta(\mathcal{S}) \ominus \mathcal{Q}) \flowproves \mathcal{D} : \mathcal{S}~T_2 \flowprovesout \Xi
    }{ \Gamma \flowproves (\mathcal{S} \to f(\overline{x}) \to \mathcal{D})~\ok \flowprovesout \update(\Xi, \mathcal{D}, \Xi(\mathcal{D}) \oplus \mathcal{Q}) }

    \inferrule*[right=Ok-Try]{
        \Gamma \flowproves \overline{S_1}~\ok \flowprovesout \Delta
        \\
        \Gamma \flowproves \overline{S_2}~\ok \flowprovesout \Xi
    }{ \Gamma \flowproves (\tryS~\{ \overline{S_1} \}~\catchS~\{ \overline{S_2} \})~\ok \flowprovesout \Delta \sqcup \Xi }
\end{mathpar}

\framebox{$\proves \Decl~\ok$} \textbf{Declaration Well-formedness}
\begin{mathpar}
    \inferrule*[right=Ok-Transformer]{
        \overline{x : \tau} \flowproves \overline{\Stmt}~\ok \flowprovesout \Gamma, y : \sigma
        \\
        \forall \pi \in \img(\Gamma). \lnot(\pi~\asset)
    }{ \proves (\transformer~f(\overline{x : \tau}) \to y : \sigma \{ \overline{\Stmt} \})~\ok }
\end{mathpar}

\framebox{$\Prog~\ok$} \textbf{Program Well-formedness}
\begin{mathpar}
    \inferrule*[right=Ok-Prog]{
        \proves \overline{\Decl}~\ok
        \\
        \emptyset \flowproves \overline{\Stmt}~\ok \flowprovesout \Gamma
        \\
        \forall \tau \in \img(\Gamma). \lnot(\tau~\asset)
    }{ (\overline{\Decl}; \overline{\Stmt})~\ok }
\end{mathpar}

\subsection{Dynamics}
\begin{tabular}{l r l l}
    $V$ & \bnfdef & $\true$ \bnfalt $\false$ \bnfalt $n$ \bnfalt $\{ x : \tau \mapsto V \}$ & \\
    $\mathcal{V}$ & \bnfdef & $\overline{V}$ & \\
    $\Stmt$ & \bnfdef & $\ldots$ \bnfalt $\putS(\mathcal{V}, \mathcal{D})$ \bnfalt $\revert$ \bnfalt $\tryS(\Sigma, \overline{S}, \overline{S})$ & \\
\end{tabular}

\begin{definition}
    An environment $\Sigma$ is a tuple $(\mu, \rho)$ where $\mu : \textsc{IdentifierNames} \partialfunc \mathbb{N}$ is the \emph{variable lookup environment}, and $\rho : \mathbb{N} \partialfunc \mathcal{V}$ is the \emph{storage environment}.
\end{definition}

\framebox{$\angles{\Sigma, \overline{\Stmt}} \to \angles{\Sigma, \overline{\Stmt}}$}

Note that we abbreviate $\angles{\Sigma, \cdot}$ as $\Sigma$, which signals the end of evaluation.

The new constructs of $\resolve(\Sigma, \mathcal{S})$ and $\putS(\mathcal{V}, \mathcal{D})$ are used to simplify the process of locating sources and updating destinations, respectively.

\begin{mathpar}
    \inferrule*[right=Seq]{
        \angles{\Sigma, S_1 } \to \angles{\Sigma', \overline{S_3}}
    }{ \angles{\Sigma, S_1 \overline{S_2}} \to \angles{\Sigma', \overline{S_3}~\overline{S_2}} }

    \inferrule*[right=Revert]{
    }{ \angles{ \Sigma, (\revert)~\overline{S} } \to \angles{ \Sigma, \revert } }

    \inferrule*[right=Skip]{
    }{ \angles{ \Sigma, \skipS } \evaluates \Sigma }
\end{mathpar}

Here we give the rules for the new $\putS(\mathcal{V}, \mathcal{D})$ statement.
\reed{TODO: Need to finalize how $\mathcal{V} + \mathcal{W}$ works; in particular, need to make sure that you can't overwrite things that shouldn't be overwritten (e.g., a nonfungible nat).
Probably need to tag types with modifiers or something.}

\begin{mathpar}
    \inferrule*[right=Put-Consume]{
    }{ \angles{\Sigma, \putS(\mathcal{V}, \consume)} \evaluates \Sigma }

    \inferrule*[right=Put-Var]{
    }{ \angles{ \Sigma, \putS(\mathcal{V}, x) } \evaluates \angles{ \Sigma, \putS(\mathcal{V}, \mu(x)) } }

    \inferrule*[right=Put-Field]{
        \rho(\mu(X)) = \{ \overline{x : \tau \mapsto \ell} \}
        \\
        (y : \sigma \mapsto k) \in \overline{x : \tau \mapsto \ell}
    }{ \angles{ \Sigma, \putS(\mathcal{V}, x.y) } \evaluates \angles{ \Sigma, \putS(\mathcal{V}, k) } }

    \inferrule*[right=Put-Loc]{
        \rho(\ell) = \mathcal{W}
        \\
        \mathcal{W} + \mathcal{V} \neq \revert
    }{ \angles{ \Sigma, \putS(\mathcal{V}, \ell) } \evaluates \Sigma[\rho \mapsto \rho[\ell \mapsto \mathcal{W} + \mathcal{V}]] }

    \inferrule*[right=Put-Loc-Fail]{
        \rho(\ell) = \mathcal{W}
        \\
        \mathcal{W} + \mathcal{V} = \revert
    }{ \angles{ \Sigma, \putS(\mathcal{V}, \ell) } \evaluates \angles{\Sigma, \revert} }

    \inferrule*[right=Put-VarDef]{
        \ell \not\in \dom(\rho)
    }{ \angles{ \Sigma, \putS(\mathcal{V}, \var~A : T) } \evaluates \Sigma[\mu \mapsto \mu[A \mapsto \ell], \rho \mapsto \rho[\ell \mapsto \mathcal{V}]] }
\end{mathpar}

\begin{mathpar}
    \inferrule*[right=Flow-Every]{
        \resolve(\Sigma, \mathcal{S}) = (\Sigma', \ell)
    }{ \angles{\Sigma, \mathcal{S} \to \mathcal{D}} \evaluates \angles{\Sigma'[\rho \mapsto \rho'[\ell \mapsto []]], \putS(\rho'(\ell), \mathcal{D}) } }

    \inferrule*[right=Flow-Var]{
        \resolve(\Sigma, \mathcal{S}) = (\Sigma', \ell)
        \\
        \rho'(\ell) = \mathcal{V}
        \\
        \rho'(\mu'(x)) = \mathcal{W}
        \\
        \mathcal{W} \leq \mathcal{V}
    }{ \angles{\Sigma, \mathcal{S} \sends{x} \mathcal{D}} \evaluates \angles{\Sigma'[\rho \mapsto \rho'[\ell \mapsto \mathcal{V} - \mathcal{W}]], \putS(\mathcal{W}, \mathcal{D}) } }

    \inferrule*[right=Flow-Var-Fail]{
        \resolve(\Sigma, \mathcal{S}) = (\Sigma', \ell)
        \\
        \rho'(\ell) = \mathcal{V}
        \\
        \rho'(\mu'(x)) = \mathcal{W}
        \\
        \mathcal{W} \not\leq \mathcal{V}
    }{ \angles{\Sigma, \mathcal{S} \sends{x} \mathcal{D}} \evaluates \angles{ \Sigma', \revert } }

    \inferrule*[right=Flow-Filter]{
        \resolve(\Sigma, \mathcal{S}) = (\Sigma', \ell)
        \\
        \rho'(\ell) = \mathcal{V}
        \\
        \mathcal{U} = [ v \in \mathcal{V} ~|~ \angles{\Sigma', f(\overline{x}, v)} \to^* \angles{\Sigma'', k} \tand \rho''(k) = \true ]
        \\
        \compatQuan(|\mathcal{U}|, |\mathcal{V}|, \mathcal{Q})
    }{ \angles{ \Sigma, (\mathcal{S} \sends{\mathcal{Q}~\suchthat~f(\overline{x})} \mathcal{D}) } \evaluates \angles{ \Sigma'[\rho' \mapsto \rho'[\ell \mapsto \rho'(\ell) - \mathcal{U}]], \putS(\mathcal{U}, \mathcal{D}) } }

    \inferrule*[right=Flow-Filter-Fail]{
        \resolve(\Sigma, \mathcal{S}) = (\Sigma', \ell)
        \\
        \rho'(\ell) = \mathcal{V}
        \\
        \mathcal{U} = [ v \in \mathcal{V} ~|~ \angles{\Sigma', f(\overline{x}, v)} \to^* \angles{\Sigma'', k} \tand \rho''(k) = \true ]
        \\
        \lnot \compatQuan(|\mathcal{U}|, |\mathcal{V}|, \mathcal{Q})
    }{ \angles{ \Sigma, (\mathcal{S} \sends{\mathcal{Q}~\suchthat~f(\overline{x})} \mathcal{D}) } \evaluates \angles{ \Sigma', \revert } }

    \inferrule*[right=Flow-Transformer]{
        \resolve(\Sigma, \mathcal{S}) = (\Sigma', \ell)
        \\
        \rho'(\ell) = v, \mathcal{V}
        \\
        \angles{ \Sigma'[\rho \mapsto \rho'[\ell \mapsto \mathcal{V}]], f(\overline{x}, v) } \to^* \angles{ \Sigma'', k }
    }{ \angles{\Sigma, \mathcal{S} \to f(\overline{x}) \to \mathcal{D}} \evaluates \angles{ (\mu', \rho''), \putS(\rho''(k), \mathcal{D}) ~ (\mathcal{S} \to f(\overline{x}) \to \mathcal{D}) } }

    \inferrule*[right=Flow-Transformer-Done]{
        \resolve(\Sigma, \mathcal{S}) = (\Sigma', \ell)
        \\
        \rho'(\ell) = []
    }{ \angles{ \Sigma, \mathcal{S} \to f(\overline{x}) \to \mathcal{D} } \evaluates \Sigma }
\end{mathpar}

\begin{mathpar}
    \inferrule*[right=Call]{
        \ell \not\in \dom(\rho)
        \\
        \transformer~f(\overline{y : \tau}) \to z : \sigma~\{~\overline{S}~\}
        \\
        \mu' = \overline{y \mapsto \mu(x)}, z \mapsto \ell
    }{ \angles{ \Sigma, f(\overline{x}) } \evaluates \angles{ (\mu', \rho[\ell \mapsto []]), \overline{S}~\ell } }
\end{mathpar}

We introduce a new statement, $\tryS(\Sigma, \overline{S_1}, \overline{S_2})$, to implement the try-catch statement, which keeps track of the environment that we begin execution in so that we can revert to the original environment in the case of a $\revert$.

\begin{mathpar}
    \inferrule*[right=Try-Start]{
    }{ \angles{ \Sigma, \tryS~\{ \overline{S_1} \}~\catchS~\{ \overline{S_2} \} } \evaluates \angles{ \Sigma, \tryS(\Sigma, \overline{S_1}, \overline{S_2}) } }

    \inferrule*[right=Try-Step]{
        \angles{ \Sigma, \overline{S_1} } \evaluates \angles{ \Sigma'', \overline{S_1'} }
    }{ \angles{ \Sigma, \tryS(\Sigma', \overline{S_1}, \overline{S_2}) } \evaluates \angles{ \Sigma'', \tryS(\Sigma', \overline{S_1'}, \overline{S_2}) } }

    \inferrule*[right=Try-Revert]{
    }{ \angles{ \Sigma, \tryS(\Sigma', \revert, \overline{S_2}) } \evaluates \angles{ \Sigma', \overline{S_2}} }

    \inferrule*[right=Try-Done]{
    }{ \angles{ \Sigma, \tryS(\Sigma', \cdot, \overline{S_2}) } \evaluates \Sigma }
\end{mathpar}

\reed{Need to handle fungible specially (or maybe only after adding nats, I'm not sure it really has any meaning without them)}

\framebox{$\resolve(\Sigma, \mathcal{S}) = (\Sigma', \ell)$} \textbf{Storage Resolution}

We use $\resolve(\Sigma, \mathcal{S})$ to get the location storing the values of $\mathcal{S}$, which returns an environment because it may need to allocate new memory (e.g., in the case of creating a new record value).

\begin{mathpar}
    \inferrule*[right=Resolve-Var]{
        \mu(\mathcal{S}) = \ell
    }{ \resolve(\Sigma, \mathcal{S}) = (\Sigma, \ell) }

    \inferrule*[right=Resolve-Field]{
        \rho(\mu(x)) = \{ \overline{z : \tau \mapsto \ell} \}
        \\
        (y : \sigma \mapsto k) \in \overline{z : \tau \mapsto \ell}
    }{ \resolve(\Sigma, x.y) = (\Sigma, k) }

    \inferrule*[right=Resolve-Single]{
        \ell \not\in \dom(\rho)
    }{ \resolve(\Sigma, [ x ]) = (\Sigma[\rho \mapsto \rho[\ell \mapsto \rho(\mu(x)), \mu(x) \mapsto []]], \ell) }

    \inferrule*[right=Resolve-Record]{
        \overline{\ell \not\in \dom(\rho)}
        \\
        k \not\in \dom(\rho) \cup \overline{\ell}
        \\
        \Sigma' = \Sigma[\rho \mapsto \rho[\overline{\mu(y) \mapsto []}, \overline{\ell \mapsto \rho(\mu(y))}, k \mapsto \{ \overline{x : \tau \mapsto \ell} \} ]]
    }{ \resolve(\Sigma, \{ \overline{x : \tau \mapsto y} \}) = (\Sigma', k) }

    \inferrule*[right=Resolve-Bool]{
        b \in \{ \true, \false \}
        \\
        \ell \not\in \dom(\rho)
    }{ \resolve(\Sigma, b) = (\Sigma[\rho \mapsto \rho[\ell \mapsto b]], \ell) }

    \inferrule*[right=Resolve-Source]{
        \mu(t) = \ell
    }{ \resolve(\Sigma, \newc(t, \overline{M}, T)) = (\Sigma, \ell) }

    \inferrule*[right=Resolve-New-Source]{
        t \not\in \dom(\mu)
        \\
        \ell \not\in \dom(\rho)
    }{ \resolve(\Sigma, \newc(t, \overline{M}, T)) = (\Sigma'[\rho \mapsto \rho[\ell \mapsto \values(T)], \mu \mapsto \mu[t \mapsto \ell]], \ell) }
\end{mathpar}

\reed{TODO: Need to be sure to handle uniqueness correctly; could do this in \textsc{Resolve-New-Source}, or in the various flow rules.}

\subsection{Auxiliaries}
\begin{definition}
    Define $\Quant = \curlys{\emptyq, \any, \exactlyone, \nonempty, \every}$, and call any $\mathcal{Q} \in \Quant$ a \emph{type quantity}.
    Define $\emptyq < \any < \,\, \exactlyone < \nonempty < \every$.
\end{definition}

\framebox{$\tau~\asset$} \textbf{Asset Types}

\begin{align*}
    (\mathcal{Q}~T)~\asset \iff \mathcal{Q} \neq \emptyq \tand (& \asset \in \modifiers(T) \tor \\
                                                                & (T = \mathcal{C}~\tau \tand \tau~\asset) \tor \\
                                                                & (T = \curlys{\overline{y : \sigma}} \tand \exists x : \tau \in \overline{y : \sigma}. (\tau~\asset)))
\end{align*}
% \reed{It should be the case that a transformer can have an output of an asset type if and only if it has an input asset type (and it the case of curried transformers, that \textbf{some} input type is an asset).}

\framebox{$\tau~\consumable$} \textbf{Consumable Types}
\begin{align*}
    (\mathcal{Q}~T)~\consumable \iff &\consumable \in \modifiers(T) \tor \\
                                     &\lnot((\mathcal{Q}~T)~\asset) \tor \\
                                     &(T = \mathcal{C}~\tau \tand \tau~\consumable) \tor \\
                                     &(T = \curlys{\overline{y : \sigma}} \tand \forall x : \tau \in \overline{y : \sigma}. (\sigma~\consumable))
\end{align*}

$\mathcal{Q} \oplus \mathcal{R}$ represents the quantity present when flowing $\mathcal{R}$ of something to a storage already containing $\mathcal{Q}$.
$\mathcal{Q} \ominus \mathcal{R}$ represents the quantity left over after flowing $\mathcal{R}$ from a storage containing $\mathcal{Q}$.

\begin{definition}
    Let $\mathcal{Q}, \mathcal{R} \in \Quant$.
    Define the commutative operator $\oplus$, called \emph{combine}, as the unique function $\Quant^2 \to \Quant$ such that
    \[
        \renewcommand\arraystretch{1.2}
        \begin{array}{r c l l l}
            \mathcal{Q} \oplus \emptyq & = & \mathcal{Q} & \\
            \mathcal{Q} \oplus \every & = & \every & \\
            \nonempty \oplus \mathcal{R} & = & \nonempty & \tif \emptyq < \mathcal{R} < \every \\
            \exactlyone \oplus \mathcal{R} & = & \nonempty & \tif \emptyq < \mathcal{R} < \every \\
            \any \oplus \any & = & \any &
        \end{array}
    \]

    Define the operator $\ominus$, called \emph{split}, as the unique function $\Quant^2 \to \Quant$ such that
    \[
        \renewcommand\arraystretch{1.2}
        \begin{array}{r c l l l}
            \mathcal{Q} \ominus \emptyq & = & \mathcal{Q} & \\
            \emptyq \ominus \mathcal{R} & = & \emptyq & \\
            \mathcal{Q} \ominus \every &= & \emptyq & \\
            \every \ominus \mathcal{R} &= & \every & \tif \mathcal{R} < \every \\
            \nonempty - \mathcal{R} &= & \any & \tif \emptyq < \mathcal{R} < \every \\
            \exactlyone - \mathcal{R} &= & \emptyq & \tif \exactlyone \leq \mathcal{R} \\
            \exactlyone - \any &= & \any & \\
            \any - \mathcal{R} &= & \any & \tif \emptyq < \mathcal{R} < \every \\
        \end{array}
    \]
\end{definition}

Note that we write $(\mathcal{Q}~T) \oplus \mathcal{R}$ to mean $(\mathcal{Q} \oplus \mathcal{R})~T$ and similarly $(\mathcal{Q}~T) \ominus \mathcal{R}$ to mean $(\mathcal{Q} \ominus \mathcal{R})~T$.

\begin{definition}
    We can consider a type environment $\Gamma$ as a function $\textsc{Identifiers}\xspace \to \textsc{Types}\xspace \cup \curlys{\bot}$ as follows:
    \[
        \Gamma(x) =
        \begin{cases}
            \tau & \tif x : \tau \in \Gamma \\
            \bot & \owise
        \end{cases}
    \]
    We write $\dom(\Gamma)$ to mean $\setbuild{x \in \textsc{Identifiers}}{\Gamma(x) \neq \bot}$, and $\Gamma|_X$ to mean the environment $\setbuild{x : \tau \in \Gamma}{x \in X}$ (restricting the domain of $\Gamma$).
\end{definition}

\begin{definition}
    Let $\mathcal{Q}$ and $\mathcal{R}$ be \typeQuantities, $T_\mathcal{Q}$ and $T_\mathcal{R}$ base types, and $\Gamma$ and $\Delta$ type environments.
    Define the following orderings, which make types and type environments into join-semilattices.
    For type quantities, define the partial order $\sqsubseteq$ as the reflexive closure of the strict partial order $\sqsubset$ given by
    \[
        \mathcal{Q} \sqsubset \mathcal{R} \iff (\mathcal{Q} \neq \any \tand \mathcal{R} = \any) \tor (\mathcal{Q} \in \curlys{\exactlyone, \every} \tand \mathcal{R} = \nonempty)
    \]
    For types, define the partial order $\leq$ by
    \[
        \mathcal{Q}~T_\mathcal{Q} \leq \mathcal{R}~T_\mathcal{R} \iff T_\mathcal{Q} = T_\mathcal{R} \tand \mathcal{Q} \sqsubseteq \mathcal{R}
    \]
    For type environments, define the partial order $\leq$ by
    \[
        \Gamma \leq \Delta \iff \forall x. \Gamma(x) \leq \Delta(x)
    \]
    Denote the join of $\Gamma$ and $\Delta$ by $\Gamma \sqcup \Delta$.
\end{definition}

\framebox{$\elemtype(T) = \tau$}
\begin{align*}
    \elemtype(T) =
    \begin{cases}
        \elemtype(T') & \tif T = \type~t~\is~\overline{M}~T' \\
        \tau & \tif T = \mathcal{C}~\tau \\
        !~T & \owise
    \end{cases}
\end{align*}

\framebox{$\modifiers(T) = \overline{M}$} \textbf{Type Modifiers}
\begin{align*}
    \modifiers(T) =
    \begin{cases}
        \overline{M} & \tif T = \type~t~\is~\overline{M}~T \\
        \emptyset & \owise
    \end{cases}
\end{align*}

\framebox{$\demote(\tau) = \sigma$}
\framebox{$\demote_*(T_1) = T_2$} \textbf{Type Demotion}
$\demote$ and $\demote_*$ take a type and ``strip'' all the asset modifiers from it, as well as unfolding named type definitions.
This process is useful, because it allows selecting asset types without actually having a value of the desired asset type.
Note that demoting a transformer type changes nothing.
This is because a transformer is \textbf{never} an asset, regardless of the types that it operators on, because it has no storage.

\begin{align*}
    \demote(\mathcal{Q}~T) &= \mathcal{Q}~\demote_*(T) \\
    \demote_*(\boolt) &= \boolt \\
    \demote_*(\natt) &= \natt \\
    \demote_*(\curlys{\overline{x : \tau}}) &= \curlys{\overline{x : \demote(\tau)}} \\
    \demote_*(\type~t~\is~\overline{M}~T) &= \demote_*(T)
\end{align*}

\framebox{$\fields(T) = \overline{x : \tau}$} \textbf{Fields}
\begin{align*}
    \fields(T) =
    \begin{cases}
        \overline{x : \tau} & \tif T = \{ \overline{x : \tau} \} \\
        \fields(T) & \tif T = \type~t~\is~\overline{M}~T \\
        \emptyset & \owise
    \end{cases}
\end{align*}

\framebox{$\update(\Gamma, x, \tau)$} \textbf{Type environment modification}
\[
    \update(\Gamma, x, \tau) =
    \begin{cases}
        \Delta, x : \tau & \tif \Gamma = \Delta, x : \sigma \\
        \Gamma & \owise
    \end{cases}
\]

\framebox{$\compatQuan(n, m, \mathcal{Q})$}
The relation $\compatQuan(n, m, \mathcal{Q})$ holds when the number of values sent, $n$, is compatible with the original number of values $m$, and the type quantity used, $\mathcal{Q}$.

\begin{align*}
    \compatQuan(n, m, \mathcal{Q}) \iff & (\mathcal{Q} = \nonempty \tand n \geq 1) \tor \\
                                     & (\mathcal{Q} = \,\,\exactlyone \tand n = 1) \tor \\
                                     & (\mathcal{Q} = \emptyq \tand n = 0) \tor \\
                                     & (\mathcal{Q} = \every \tand n = m) \tor \\
                                     & \mathcal{Q} = \any
\end{align*}

\framebox{$\values(T) = \mathcal{V}$}
The function $\values$ gives a list of all of the values of a given base type.

\begin{align*}
    \values(\boolt) & = [ \true, \false ] \\
    \values(\natt) & = [ 0, 1, 2, \ldots ] \\
    \values(\listq~T) & = [ L | L \subseteq \values(T), |L| < \infty ] \\
    \values(\type~t~\is~\overline{M}~T) &= \values(T) \\
    \values(\{ \overline{x : \mathcal{Q}~T} \}) &= [ \{ \overline{x : \tau \mapsto v} \} | \overline{v \in \values(T)} ]
\end{align*}

\end{document}

