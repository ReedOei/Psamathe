\documentclass[dvipsnames, usenames, sigconf]{acmart}

\usepackage{listings}
\usepackage{hyperref}
\usepackage{tikz-cd}
% \usepackage{amssymb}
\usepackage{amsthm}
\usepackage{bm}
\usepackage{listings}
\usepackage{mathtools}
\usepackage{mathpartir}
\usepackage{float}
\usepackage{subcaption}
\usepackage{xcolor}
% \usepackage[inline]{enumitem}

%% \BibTeX command to typeset BibTeX logo in the docs
% \AtBeginDocument{%
%   \providecommand\BibTeX{{%
%     \normalfont B\kern-0.5em{\scshape i\kern-0.25em b}\kern-0.8em\TeX}}}

%% Rights management information.  This information is sent to you
%% when you complete the rights form.  These commands have SAMPLE
%% values in them; it is your responsibility as an author to replace
%% the commands and values with those provided to you when you
%% complete the rights form.
% \setcopyright{acmcopyright}
% \copyrightyear{2018}
% \acmYear{2018}
% \acmDOI{10.1145/1122445.1122456}

%% These commands are for a PROCEEDINGS abstract or paper.
% \acmConference[Woodstock '18]{Woodstock '18: ACM Symposium on Neural
%   Gaze Detection}{June 03--05, 2018}{Woodstock, NY}
% \acmBooktitle{Woodstock '18: ACM Symposium on Neural Gaze Detection,
%   June 03--05, 2018, Woodstock, NY}
% \acmPrice{15.00}
% \acmISBN{978-1-4503-XXXX-X/18/06}

%% Submission ID.
%% Use this when submitting an article to a sponsored event. You'll
%% receive a unique submission ID from the organizers
%% of the event, and this ID should be used as the parameter to this command.
%%\acmSubmissionID{123-A56-BU3}

\lstset{
  frame=none,
  xleftmargin=2pt,
  stepnumber=1,
  numbers=left,
  numbersep=5pt,
  numberstyle=\ttfamily\tiny\color[gray]{0.3},
  belowcaptionskip=\bigskipamount,
  captionpos=b,
  escapeinside={*'}{'*},
  tabsize=2,
  emphstyle={\bf},
  commentstyle=\it,
  stringstyle=\mdseries\rmfamily,
  showspaces=false,
  keywordstyle=\bfseries\rmfamily,
  columns=flexible,
  basicstyle=\small\sffamily,
  showstringspaces=false,
}

\newcommand{\guard}{\bigg|\bigg|}
\newcommand{\newc}{\textbf{\texttt{new}}}
\newcommand{\everything}{\textbf{\texttt{everything}}}
\newcommand{\then}{\textbf{\texttt{then}}}
\newcommand{\this}{\textbf{\texttt{this}}}
\newcommand{\langName}{LANGUAGE-NAME\xspace}
\newcommand{\consumesarr}{\mathrel{\rotatebox[origin=c]{270}{${\looparrowright}$}}}
\newcommand{\consumes}[1]{\stackrel{#1}{\consumesarr}}
\newcommand{\sends}[1]{\stackrel{#1}{\to}}
\newcommand{\emitsarr}{\mathrel{\rotatebox[origin=c]{270}{${\looparrowleft}$}}}
\newcommand{\emits}[1]{\emitsarr #1}
\newcommand{\asset}{\textbf{\texttt{asset}}\xspace}
\newcommand{\fungible}{\textbf{\texttt{fungible}}\xspace}
\newcommand{\nonfungible}{\textbf{\texttt{nonfungible}}\xspace}
\newcommand{\addresst}{\textbf{\texttt{address}}\xspace}
\newcommand{\stringt}{\textbf{\texttt{string}}\xspace}
\newcommand{\setof}{\textbf{\texttt{set}}\xspace}
\newcommand{\optiont}{\textbf{\texttt{option}}\xspace}
\newcommand{\boolt}{\textbf{\texttt{bool}}\xspace}
\newcommand{\natt}{\textbf{\texttt{nat}}\xspace}
\newcommand{\byt}{\textbf{\texttt{by}}\xspace}
\newcommand{\stores}{\textbf{\texttt{stores}}\xspace}
\newcommand{\suchthat}{\textbf{s.t.}\xspace}
\newcommand{\returns}{\textbf{\texttt{returns}}\xspace}
\newcommand{\merge}{\rightsquigarrow}
\newcommand{\flowsto}{\rightsquigarrow}
\newcommand{\heldby}{\rightarrowtail}
\newcommand{\one}{\textbf{\texttt{one}}\xspace}
\newcommand{\some}{\textbf{\texttt{some}}\xspace}
\newcommand{\any}{\textbf{\texttt{any}}\xspace}
\newcommand{\map}{\textbf{\texttt{map}}\xspace}
\newcommand{\linking}{\textbf{\texttt{linking}}\xspace}
\newcommand{\transformer}{\textbf{\texttt{transformer}}\xspace}
\newcommand{\selects}{\textbf{\texttt{selects}}\xspace}
\newcommand{\demote}{\textbf{\texttt{demote}}\xspace}
\newcommand{\transforms}{\Rrightarrow}

\newcommand{\flowproves}{\mathbin{|\hskip -0.1em\scalebox{0.7}[1]{$\leadsto$}}}
\newcommand{\flowproveout}{\mathbin{\scalebox{0.7}[1]{$\leadsto$}\hskip -0.13em|}}


% Copyright 2017 Sergei Tikhomirov, MIT License
% https://github.com/s-tikhomirov/solidity-latex-highlighting/

\usepackage{listings, xcolor}

\definecolor{verylightgray}{rgb}{.97,.97,.97}

\lstdefinelanguage{Solidity}{
	keywords=[1]{anonymous, assembly, balance, call, callcode, class, constant, constructor, contract, debugger, delegatecall, delete, emit, event, experimental, export, external, function, gas, implements, import, in, indexed, instanceof, interface, internal, is, length, library, log0, log1, log2, log3, log4, memory, modifier, new, payable, pragma, private, protected, public, pure, push, return, returns, selfdestruct, send, solidity, storage, struct, suicide, super, then, throw, typeof, using, view, with, addmod, ecrecover, keccak256, mulmod, ripemd160, sha256, sha3}, % generic keywords including crypto operations
	keywordstyle=[1]\color{Rhodamine}\bfseries,
	keywords=[2]{address, bool, byte, bytes, bytes1, bytes2, bytes3, bytes4, bytes5, bytes6, bytes7, bytes8, bytes9, bytes10, bytes11, bytes12, bytes13, bytes14, bytes15, bytes16, bytes17, bytes18, bytes19, bytes20, bytes21, bytes22, bytes23, bytes24, bytes25, bytes26, bytes27, bytes28, bytes29, bytes30, bytes31, bytes32, enum, int, int8, int16, int24, int32, int40, int48, int56, int64, int72, int80, int88, int96, int104, int112, int120, int128, int136, int144, int152, int160, int168, int176, int184, int192, int200, int208, int216, int224, int232, int240, int248, int256, mapping, string, uint, uint8, uint16, uint24, uint32, uint40, uint48, uint56, uint64, uint72, uint80, uint88, uint96, uint104, uint112, uint120, uint128, uint136, uint144, uint152, uint160, uint168, uint176, uint184, uint192, uint200, uint208, uint216, uint224, uint232, uint240, uint248, uint256, var, void, ether, finney, szabo, wei, days, hours, minutes, seconds, weeks, years},	% types; money and time units
	keywordstyle=[2]\color{Emerald}\itshape,
	keywords=[3]{block, balance, blockhash, coinbase, difficulty, gaslimit, number, timestamp, msg, data, gas, sender, sig, now, tx, gasprice, origin, this},	% sepcial identifiers
	keywordstyle=[3]\color{Lavender}\itshape,
	keywords=[4]{assert, break, case, catch, continue, default, else, false, finally, for, if, require, revert, switch, throw, true, try, while, do}, % Control-flow
	keywordstyle=[4]\color{Cerulean}\bfseries,
	identifierstyle=\color{black},
	sensitive=false,
    keepspaces=true,
    columns=fullflexible,
    % literate=*
    %     {(}{{{\color{red}(}}}1
    %     {)}{{{\color{red})}}}1
    %     {+}{{{\color{red}+~}}}1
    %     {-}{{{\color{red}-~}}}1
    %     {:=}{{{\color{red}:=~}}}1
    %     {\{}{{{\color{red}\{}}}1
    %     {\}}{{{\color{red}\}}}}1
    %     {[}{{{\color{red}[}}}1
    %     {]}{{{\color{red}]}}}1
    %     {|}{{{\color{red}|}}}1
    %     {;}{{{\color{red};}}}1
    %     {*}{{{\color{red}*~}}}1
    %     {:}{{{\color{red}:~}}}1
    %     {>}{{{\color{red}>~}}}1
    %     {<}{{{\color{red}<~}}}1
    %     {<=>}{{{$\color{red}\Leftrightarrow~$}}}1
    %     {.}{{{\color{red}.~}}}1
    %     {!}{{{$\color{red}!~$}}}1
    %     {=}{{{$\color{red}=~$}}}1
    %     {exists }{{{$\color{red}\exists$}}}1
    %     {forall }{{{$\color{red}\forall$}}}1,
	comment=[l]{//},
	morecomment=[s]{/*}{*/},
    commentstyle=\color{CadetBlue}\textit,
	stringstyle=\color{ForestGreen},
	morestring=[b]',
	morestring=[b]",
    xleftmargin=5.0ex,
    extendedchars=true,
	basicstyle=\small\ttfamily,
	showstringspaces=false,
	showspaces=false,
	numbers=left,
	numberstyle=\footnotesize,
	numbersep=9pt,
	tabsize=2,
	breaklines=true,
	showtabs=false,
	captionpos=b
}


\definecolor{verylightgray}{rgb}{.97,.97,.97}

\lstdefinelanguage{flow}{
    keywords=[1]{var,interface,where,contract,transaction,view,transformer,type,is,returns},
    keywordstyle=[1]\color{Rhodamine}\bfseries,
    keywords=[2]{false,true,sometimes,fungible,asset,unique,immutable,consumable},
    keywordstyle=[2]\color{DarkOrchid}\bfseries,
    keywords=[3]{map,linking,set,list,address,uint256,string,bytes,nat,bool},
    keywordstyle=[3]\color{Emerald}\itshape,
    keywords=[4]{nonempty,any,every,empty,exactly,one,msg,this,in,new,or,and,not,total,consume},
    keywordstyle=[4]\color{Lavender}\itshape,
    keywords=[5]{if,else,only,when,try,catch,such,that},
    keywordstyle=[5]\color{Cerulean}\bfseries,
    keepspaces=true,
	% backgroundcolor=\color{verylightgray},
    % columns=fullflexible,
    % literate=*
    %     {(}{{{\color{red}(}}}1
    %     {)}{{{\color{red})}}}1
    %     {+}{{{\color{red}+~}}}1
    %     {-}{{{\color{red}-~}}}1
    %     {:=}{{{\color{red}:=~}}}1
    %     {\{}{{{\color{red}\{}}}1
    %     {\}}{{{\color{red}\}}}}1
    %     {[}{{{\color{red}[}}}1
    %     {]}{{{\color{red}]}}}1
    %     {*}{{{\color{red}*~}}}1
    %     {:}{{{\color{red}:~}}}1
    %     {>}{{{\color{red}>~}}}1
    %     {<}{{{\color{red}<~}}}1
    %     {<=>}{{{$\color{red}\Leftrightarrow~$}}}1
    %     {.}{{{\color{red}.~}}}1
    %     {=}{{{$\color{red}=~$}}}1
    %     {exists }{{{$\color{red}\exists$}}}1
    %     {forall }{{{$\color{red}\forall$}}}1,
    sensitive=false, % keywords are not case-sensitive
    morecomment=[l]{//}, % l is for line comment
    morecomment=[s]{/*}{*/}, % s is for start and end delimiter
    commentstyle=\color{CadetBlue}\textit,
    morestring=[b]", % defines that strings are enclosed in double quotes
    stringstyle=\color{ForestGreen}, % string literal style
    showstringspaces=false,
    xleftmargin=5.0ex,
    mathescape=true,
    escapechar=|,
    basicstyle=\small\ttfamily
}

\lstnewenvironment{flow}
{%
    \lstset{language=flow, basicstyle=\small\ttfamily, mathescape=true}%
}
{
}

\newcommand{\flowinline}[1]{\lstinline[language=flow,basicstyle=\small\ttfamily,mathescape]{#1}}


\lstset{
  frame=none,
  xleftmargin=2pt,
  stepnumber=1,
  numbers=left,
  numbersep=5pt,
  numberstyle=\ttfamily\tiny\color[gray]{0.3},
  belowcaptionskip=\bigskipamount,
  captionpos=b,
  escapeinside={*'}{'*},
  tabsize=2,
  emphstyle={\bf},
  commentstyle=\it,
  stringstyle=\mdseries\rmfamily,
  showspaces=false,
  keywordstyle=\bfseries\rmfamily,
  columns=flexible,
  basicstyle=\small\sffamily,
  showstringspaces=false,
}

\newcommand{\guard}{\bigg|\bigg|}
\newcommand{\newc}{\textbf{\texttt{new}}}
\newcommand{\everything}{\textbf{\texttt{everything}}}
\newcommand{\then}{\textbf{\texttt{then}}}
\newcommand{\this}{\textbf{\texttt{this}}}
\newcommand{\langName}{LANGUAGE-NAME\xspace}
\newcommand{\consumesarr}{\mathrel{\rotatebox[origin=c]{270}{${\looparrowright}$}}}
\newcommand{\consumes}[1]{\stackrel{#1}{\consumesarr}}
\newcommand{\sends}[1]{\stackrel{#1}{\to}}
\newcommand{\emitsarr}{\mathrel{\rotatebox[origin=c]{270}{${\looparrowleft}$}}}
\newcommand{\emits}[1]{\emitsarr #1}
\newcommand{\asset}{\textbf{\texttt{asset}}\xspace}
\newcommand{\fungible}{\textbf{\texttt{fungible}}\xspace}
\newcommand{\nonfungible}{\textbf{\texttt{nonfungible}}\xspace}
\newcommand{\addresst}{\textbf{\texttt{address}}\xspace}
\newcommand{\stringt}{\textbf{\texttt{string}}\xspace}
\newcommand{\setof}{\textbf{\texttt{set}}\xspace}
\newcommand{\optiont}{\textbf{\texttt{option}}\xspace}
\newcommand{\boolt}{\textbf{\texttt{bool}}\xspace}
\newcommand{\natt}{\textbf{\texttt{nat}}\xspace}
\newcommand{\byt}{\textbf{\texttt{by}}\xspace}
\newcommand{\stores}{\textbf{\texttt{stores}}\xspace}
\newcommand{\suchthat}{\textbf{s.t.}\xspace}
\newcommand{\returns}{\textbf{\texttt{returns}}\xspace}
\newcommand{\merge}{\rightsquigarrow}
\newcommand{\flowsto}{\rightsquigarrow}
\newcommand{\heldby}{\rightarrowtail}
\newcommand{\one}{\textbf{\texttt{one}}\xspace}
\newcommand{\some}{\textbf{\texttt{some}}\xspace}
\newcommand{\any}{\textbf{\texttt{any}}\xspace}
\newcommand{\map}{\textbf{\texttt{map}}\xspace}
\newcommand{\linking}{\textbf{\texttt{linking}}\xspace}
\newcommand{\transformer}{\textbf{\texttt{transformer}}\xspace}
\newcommand{\selects}{\textbf{\texttt{selects}}\xspace}
\newcommand{\demote}{\textbf{\texttt{demote}}\xspace}
\newcommand{\transforms}{\Rrightarrow}

\newcommand{\flowproves}{\mathbin{|\hskip -0.1em\scalebox{0.7}[1]{$\leadsto$}}}
\newcommand{\flowproveout}{\mathbin{\scalebox{0.7}[1]{$\leadsto$}\hskip -0.13em|}}



\settopmatter{printfolios=true}

\begin{document}

%% The "title" command has an optional parameter,
%% allowing the author to define a "short title" to be used in page headers.
\title{\langName: A DSL for Safe Blockchain \AssetTxt{}s}

%% The "author" command and its associated commands are used to define
%% the authors and their affiliations.
%% Of note is the shared affiliation of the first two authors, and the
%% "authornote" and "authornotemark" commands
%% used to denote shared contribution to the research.
% \author{Reed Oei}
% \email{reedoei2@illinois.edu}
% \affiliation{%
%   \institution{University of Illinois}
%   \city{Urbana}
%   \country{USA}
% }

%% By default, the full list of authors will be used in the page
%% headers. Often, this list is too long, and will overlap
%% other information printed in the page headers. This command allows
%% the author to define a more concise list
%% of authors' names for this purpose.
% \renewcommand{\shortauthors}{Trovato and Tobin, et al.}

%% The abstract is a short summary of the work to be presented in the
%% article.
% \begin{abstract}
%     Abstract.
% \end{abstract}

%% The code below is generated by the tool at http://dl.acm.org/ccs.cfm.
%% Please copy and paste the code instead of the example below.
% \begin{CCSXML}
% <ccs2012>
%  <concept>
%   <concept_id>10010520.10010553.10010562</concept_id>
%   <concept_desc>Computer systems organization~Embedded systems</concept_desc>
%   <concept_significance>500</concept_significance>
%  </concept>
%  <concept>
%   <concept_id>10010520.10010575.10010755</concept_id>
%   <concept_desc>Computer systems organization~Redundancy</concept_desc>
%   <concept_significance>300</concept_significance>
%  </concept>
%  <concept>
%   <concept_id>10010520.10010553.10010554</concept_id>
%   <concept_desc>Computer systems organization~Robotics</concept_desc>
%   <concept_significance>100</concept_significance>
%  </concept>
%  <concept>
%   <concept_id>10003033.10003083.10003095</concept_id>
%   <concept_desc>Networks~Network reliability</concept_desc>
%   <concept_significance>100</concept_significance>
%  </concept>
% </ccs2012>
% \end{CCSXML}

% \ccsdesc[500]{Computer systems organization~Embedded systems}
% \ccsdesc[300]{Computer systems organization~Redundancy}
% \ccsdesc{Computer systems organization~Robotics}
% \ccsdesc[100]{Networks~Network reliability}

% %% Keywords. The author(s) should pick words that accurately describe
% %% the work being presented. Separate the keywords with commas.
% \keywords{datasets, neural networks, gaze detection, text tagging}

%% This command processes the author and affiliation and title
%% information and builds the first part of the formatted document.
\maketitle

\reed{Authors/affiliations?}

\section{Introduction}

Blockchains are becoming more commonly used as platforms for applications, called \emph{smart contracts}, which automatically manage transactions in an unbiased, mutually agreed-upon way.
Commonly proposed and implemented applications for the blockchain often revolve around the management of \emph{digital assets}. \reed{cite}
One of the most forms of these are a variety of smart contracts that managed assets called \emph{tokens}.
There are many token standards, especially on the Ethereum blockchain \reed{cite}, including \reed{ERC-20, ERC-721, ERC-777, ERC-1155}, with others in various stages of the standardization process.
Other applications or proposed applications for smart contracts include voting, supply chain management, auctions, lotteries, and other applications which require careful management of their respective assets \reed{cite/find examples}.

However, smart contracts must be carefully reviewed before being deployed, as they cannot be patched if a security vulnerability is discovered.
Despite this extra care, discoveries of vulnerabilities still occur regularly, often costing large amounts of money.
\reed{Some examples}

\langName provides features to mark certain values as \emph{\assetTxt{}s} as well as a special construct called a \emph{flow}, an abstraction representing an atomic transfer operation, which is widely applicable to smart contracts managing a variety of digital \assetTxt{}s.
These features combine to prevent common issues in smart contracts.
Solidity \reed{cite} and Vyper \reed{cite}, the most common languages used to write smart contracts on the Ethereum blockchain \reed{cite}, being general purpose languages, do not provide analgous support for managing \assetTxt{}s.

\paragraph{Contributions}

We make the following contributions with \langName.
\begin{itemize}
    \item \textbf{Safety guarantees}: \langName ensures that \assetTxt{}s are properly managed, eliminating reuse, \assetTxt-loss, and duplication bugs through the use of the flow abstraction.
    \item \textbf{Flow abstraction}: \langName uses a new abstraction called a \emph{flow} to encode semantic information about the flow of \assetTxt{}s into the code.
    \item \textbf{Conciseness}: \langName makes writing typical smart contract programs more concise by handling common patterns and pitfalls automatically.
\end{itemize}

% \reed{Potential benefits of the language to discuss.
% \begin{itemize}
%     \item Maybe the language is efficient, but would need an implementation to evaluate this.
% \end{itemize}}
% \reed{Optimizations? Probably don't have time for this, unfortunately.}
%     Some of the Solidity contracts are actually inefficient because:
%     \begin{enumerate}
%         \item They use lots of modifiers which repeat checks (see reference implementation of ERC-721).
%         \item They tend to use arrays to represent sets.
%             Maybe this is more efficient for very small sets, but checking containment is going to be much faster with a \lstinline{mapping (X => bool)} eventually.
%     \end{enumerate}

%     \begin{itemize}
%         \item We can evaluate this by profiling or a simple opcode count (which is not only a proxy for performance, but also means that deploying the contract will be cheaper).
%     \end{itemize}

\section{Language Description}

A \langName program is made of many \emph{contracts}, each containing \emph{declarations}, such as \emph{fields}, \emph{types}, \emph{transactions}, and \emph{views}.
A contract is the high-level grouping of functionality, just as in Solidity; each contract instance in \langName represents a contract on the blockchain.
Fields function as the peristent storage of the contract, whose data is kept on the blockchain.
Type declarations are the primary way to use the type system of \langName, providing a way of annotating values as \assetTxt{}s (as well as other modifiers, discussed in Section~\ref{sec:lang-modifiers}).
In place of Solidity's \lstinline{function}, we have \flowinline{transaction} and \flowinline{view}.
In \langName, a transaction can change the state of the contract, and a view cannot; it can, however, read the current state of the contract, just like the \lstinline{view} modifier in Solidity.
\reed{Explain how things map a little bit more. Also, this explanation basically is mostly useful if you already know Solidity.
Probably should put more effort into explanation for people who don't know Solidity (at least, if we submit to a non-blockchain venue).}

Figure~\ref{fig:erc20-transfer-flow} shows a simple contract declaring a type, a field, and a transaction, which implements the core functionality of ERC-20's \lstinline{transfer} function (see Section~\ref{sec:case-study-erc20} for more details on ERC-20).
\begin{figure}[h]
    \centering
    \lstinputlisting[language=flow]{splash20-examples/erc20-transfer.flow}
    \caption{A contract with a simple \lstinline{transfer} function in \langName, which transfers \lstinline{amount} tokens from the first to the second account.
It is implemented with a single flow, which automatically checks all the preconditions to ensure the transfer is valid.}
    \label{fig:erc20-transfer-flow}
\end{figure}
% \begin{figure*}[h]
%     \centering
%     \begin{subfigure}[t]{0.49\textwidth}
%         \centering
%         \lstinputlisting[language=Solidity]{splash20-examples/erc20-transfer.sol}
%         \caption{Solidity implementation of ERC-20's \lstinline{transfer} function.}
%         \label{fig:erc20-transfer-sol}
%     \end{subfigure}%
%     \hfill
%     \begin{subfigure}[t]{0.49\textwidth}
%         \centering
%         \lstinputlisting[language=flow]{splash20-examples/erc20-transfer.flow}
%         \caption{\langName implementation of ERC-20's \lstinline{transfer} function.}
%         \label{fig:erc20-transfer-flow}
%     \end{subfigure}
%     % \caption{A Solidity and a \langName implementation of the core functions of the ERC-20 standard.}
%     \label{fig:erc20-transfer}
% \end{figure*}
% \reed{Not sure how to get the bounding box to be right in each column...}

\subsection{Syntax}
\begin{figure}[t]
\setlength{\tabcolsep}{1pt}
\begin{tabular}{l r l l}
    $\mathfrak{q}$ & \bnfdef & $\exactlyone$ \bnfalt $\any$ \bnfalt $\nonempty$ & (selector quantifiers) \\
    $\mathcal{Q}$ & \bnfdef & $\mathfrak{q}$ \bnfalt $\emptyq$ \bnfalt $\every$ & (\typeQuantities) \\
    $T$ & \bnfdef & \boolt \bnfalt $\natt$ \bnfalt $\map~\tau~\Rightarrow~\sigma$ \bnfalt $t$ \bnfalt $\ldots$ & (base types) \\
    $\tau$ & \bnfdef & $\mathcal{Q}~T$ & (types) \\
    $\mathcal{V}$ & \bnfdef & $n$ \bnfalt \true \bnfalt \false \bnfalt $\emptyval$ \bnfalt $\ldots$ & (values) \\
    $\mathcal{L}$ & \bnfdef & $x$ \bnfalt $x.x$ & (locations) \\
    $E$ & \bnfdef & $\mathcal{V}$ \bnfalt $\mathcal{L}$ \bnfalt $\total~t$ \bnfalt $\ldots$ & (expressions) \\
    $s$ & \bnfdef & $\mathcal{L}$ \bnfalt $\everything$ \bnfalt $\mathfrak{q}~x : \tau~\suchthat~E$ & (selector) \\
    $\mathcal{S}$ & \bnfdef & $\mathcal{L}$ \bnfalt $\newc~t$ & (sources) \\
    $\mathcal{D}$ & \bnfdef & $\mathcal{L}$ \bnfalt $\consume$ & (destinations) \\
    $F$ & \bnfdef & $\mathcal{S} \sends{s} \mathcal{D}$ & (flows) \\
    $\Stmt$ & \bnfdef & $F$ \bnfalt $\Stmt ; \Stmt$ \bnfalt $\ldots$ & (statements) \\
    $M$ & \bnfdef & \fungible \bnfalt \immutable \bnfalt \unique & \\
        & \bnfalt & \consumable \bnfalt \asset & (type modifiers) \\
    $\Decl$ & \bnfdef & $\type~t~\is~\overline{M}~T$ & (type declaration) \\
            & \bnfalt & $\transaction~m(\overline{x : \tau}) \to x : \tau~\doC~\Stmt$ & (transactions) \\
            & \bnfalt & $\ldots$ & \\
    $\Con$ & \bnfdef & $\contract~C~\{~\overline{\Decl}~\}$ & (contracts) \\
\end{tabular}
\caption{A fragment of the abstract syntax of the core calculus of \langName, a simplified form of the surface language.}
\label{fig:lang-syntax-frag}
\end{figure}

Figure~\ref{fig:lang-syntax-frag} shows a fragment of the syntax of the core calculus of \langName, which uses A-normal form and makes several other simplifications to the surface \langName language.
These simplifications are performed automatically by the compiler.
\reed{TODO: We have formalized this core calculus (in K???).}

\subsection{Modifiers}\label{sec:lang-modifiers}
\emph{Modifiers} can be used to place constraints on how values are managed: \flowinline{asset}, \flowinline{fungible}, \flowinline{unique}, \flowinline{immutable}, and \flowinline{consumable}.
A \flowinline{asset} is a value that must not be reused or accidentally lost.
A \flowinline{fungible} value represents a quantity which can be \textbf{merged}, and it is \textbf{not} \flowinline{unique}.
A \flowinline{unique} value can only exist in at most one storage; it must be \flowinline{immutable}.
A \flowinline{immutable} value cannot be changed; in particular, it cannot be the source or destination of a flow, the only state-changing construct in \langName \reed{this is the goal, anyway, which I think is true, but need to verify}.
A \flowinline{consumable} value is an \flowinline{asset} that it is sometimes appropriate to dispose of; however, this disposal must be done via the \flowinline{consume} construct, a way of documenting that the disposal is intentional.
All of these constraints, except for uniqueness, are enforced statically.

For example, ERC-20 tokens are \flowinline{fungible}, while ERC-721 tokens are best modeled as being both \flowinline{unique} and \flowinline{immutable}.
By default, neither is \flowinline{consumable}, but one of the common extensions of both standards is to add a \flowinline{burn} function, which allows tokens to be destroyed by users with the appropriate authentication.
In this case, it would be appropriate to add the \flowinline{consumable} modifier.

It supports data structures that make working with \assetTxt{}s easier, such as \emph{linkings}, a bidirectional map between keys and collections of values, with special operations to support modeling of an \emph{account} (e.g., an address which has a balance consisting of a set of tokens).
A linking is essentially a \flowinline{map K => C V} with extra operations, where \flowinline{K} is the key type, \flowinline{V} is the value type, and \flowinline{C} is some collection type constructor, such as \flowinline{list} or \flowinline{set}.
These extra operations include querying which key owns which value, removing the need to maintain parallel data structures which separately keep track of various information about an \assetTxt; this operation is often useful (see Figure~\ref{fig:erc721-impl-flow} for some examples).
Though one can partially implement linkings as a library in Solidity, using the $\unique$ feature of \langName, we can guarantee that looking up the owner of a value which has a unique type is a \textbf{well-defined} operation.

\subsection{Flows}
\langName is built around the concept of a \emph{flow}, an atomic, state-changing operation describing the transfer of a \assetTxt.
Each flow has a \emph{source} and a \emph{destination}; they may optionally have a \emph{selector} or a \emph{transformer}, which default to $\everything$ and the identity transformer, respectively.
The source of a flow \emph{provides} values, the destination of the flow \emph{accepts} these values, and the selector describes which subpart of the value(s) in the source should be transferred to the destination.
Selectors act like filters from functional programming, or like the \lstinline{WHERE} clause of a \lstinline{SELECT} in a SQL database.
Using a transformer in a flow is like mapping a function over the collection of values being flowed.

All flows fail if the selected \assetTxt{}s are not present in the source, or if the selected \assetTxt{}s cannot be added to the destination.
For example, a flow of fungible \assetTxt{}s fails if there is not enough of the \assetTxt in the source, or if there is too much in the destination; for example, the latter may occur in the case of overflow.
Flows can also fail for other reasons: a developer may specify that a certain flow must send all \assetTxt{}s matching a predicate, but in addition specify an expected \emph{quantity} that must be selected: any number, exactly one, or at least one.
The former will always succeed, but the latter two may cause the flow to fail.

\reed{Discuss transformers}

\subsection{Error Handling}
Computation on blockchains like the Ethereum blockchain is grouped into units called \emph{transactions}.
Transactions either succeed or they fail and revert all changes.
\langName also has transactional semantics: a sequence of flows will either all succeed, or, if a single flow fails, the rest will fail as well.
If a sequence of flows fails, it ``bubbles up'', like an exception, until it either: a) reaches the top level, at which point the entire transaction fails; or b) reaches a \flowinline{catch}, in which case only the changes made inside the corresponding \flowinline{try} block will be reverted, and the code inside the \flowinline{catch} block will be executed.

\section{Examples}
\reed{section intro}
\subsection{ERC-20}
ERC-20 \reed{cite} is a standard for smart contracts that manage \textbf{fungible} tokens, and provides a bare-bones interface for this purpose.
Each ERC-20 contract manages the ``bank accounts'' for its own tokens, keeping track of which users, identified by addresses, have some number of tokens.
We focus on one core function from ERC-20, the \lstinline{transfer} function.
ERC-20 is one of the commonly implemented standards on the Ethereum blockchain \reed{cite}.
\reed{Cite all Solidity code}
\begin{figure}[h]
    \centering
    \lstinputlisting[language=Solidity]{splash20-examples/erc20-transfer.sol}
    \caption{An implementation of ERC-20's \lstinline{transfer} function in Solidity.
        All preconditions are checked manually.
        Note that we must include the \lstinline{SafeMath} library (not shown), which checks for underflow/overflow, to use the \lstinline{add} and \lstinline{sub} functions.}
    \label{fig:erc20-transfer-sol}
\end{figure}
Figure~\ref{fig:erc20-transfer-sol} shows a Solidity implementation of the ERC-20 function \lstinline{transfer} (cf. Figure~\ref{fig:erc20-transfer-flow}).
Note that event code has been omitted, because \langName handles events in the same way as Solidity.
This example shows several advantages of the flow abstraction:
\begin{itemize}
    \item \textbf{Precondition checking}: For a flow to succeed, the source must have enough \assetTxt{}s and the destination must be capable of receiving the \assetTxt{}s flowed.
        In this case, the balance of the sender must be greater than the amount sent, and the balance of the destination must not overflow when it receives the tokens.
        Code checking these two conditions is automatically inserted, ensuring that the checks cannot be forgotten.
    \item \textbf{Data-flow tracking}: It is clear where the resources are flowing from the code itself, which may not be apparent in more complicated implementations, such as those involving transfer fees.
        Furthermore, developers must explicitly mark all times that \assetTxt{}s are \emph{consumed}, and only assets marked as consumable may be consumed.
        This restriction prevents, in this example, tokens from being consumed, and can also be used to ensure that other \assetTxt{}s, like ether, are not consumed.
    \item \textbf{Error messages}: When a flow fails, \langName provides automatic, descriptive error messages, such as \lstinline{"Cannot flow '<amount>' Token from account[<src>] to account[<dst>]: source only has <balance> Token."}.
        The reference Solidity implementation provides no error message, forcing developers to write their own.
        Flows enable the generation of the messages by encoding the semantic information of a transfer into the program, instead of using low-level operations like increment and decrement or insert and delete.
\end{itemize}

\subsection{ERC-721}\label{sec:erc721-impl}
ERC-721 \reed{cite}, like ERC-20, is a token standard for the Ethereum blockchain.
ERC-721 is a standard for tokens managing \emph{nonfungible} tokens; that is, tokens that are unique.
The ERC-721 standard requires many invariants hold, including:
    the tokens must be unique,
    the tokens must be owned by at most one account,
    at most one non-owning account can have \emph{approval} for a token, and only if that token has been minted,
    we must be able to support \emph{operators} who can manage all of the tokens of a user,
    among others.
Because \langName is designed to manage \assetTxt{}s, it has features to help developers ensure that these correctness properties hold.

\begin{figure*}[h]
    \centering
    \begin{subfigure}[t]{0.49\textwidth}
        \centering
        \lstinputlisting[language=Solidity]{splash20-examples/erc721.sol}
        \caption{A Solidity implementation of ERC-721's \lstinline{transferFrom}.}
        \label{fig:erc721-impl-sol}
    \end{subfigure}%
    \hfill
    \begin{subfigure}[t]{0.49\textwidth}
        \centering
        \lstinputlisting[language=flow]{splash20-examples/erc721.flow}
        \caption{\langName implementation of ERC-721's \lstinline{transferFrom}.}
        \label{fig:erc721-impl-flow}
    \end{subfigure}
    % \caption{A Solidity and a \langName implementation of the \lstinline{transferFrom} function of the ERC-721 standard.}
    \label{fig:erc721-impl}
\end{figure*}

Figures~\ref{fig:erc721-impl-sol} and~\ref{fig:erc721-impl-flow} show implementations in Solidity and \langName, respectively, of the ERC-721 function \lstinline{transferFrom}.
The Solidity implementation is extracted from a reference implementation of ERC-721 given on the official Ethereum EIP page.

A \langName implementation has several benefits: because of the \assetTxt abstraction, we can be sure that token references will not be duplicated or lost; because \lstinline{Token} has been declared as $\unique$, we can be sure that we will not mint two of the same token.
In addition to the invariants required by the specification, there are also internal invariants which the contract must maintain, such as the connection between \lstinline{idToOwner} and \lstinline{ownerToNFTokenCount}, which are handled automatically in the \langName version.
This example demonstrates the benefits of having $\unique$ \assetTxt{}s and the linking data structure built into the language itself.

\subsection{Voting}
One proposed use for the blockchain is smart contracts for managing voting \reed{cite}.
Figures~\ref{fig:voting-impl-sol} and~\ref{fig:voting-impl-flow} show the core of an implementation of a simple voting contract in Solidity and \langName, respectively.
This example shows that \langName is suited for a range of applications, as we can use the $\unique$ modifier to remove certain incorrect behaviors, shown in Figure~\ref{fig:voting-impl}.
In this application, the use of $\unique$ ensures that each user, represented by an \lstinline{address}, can be given permission to vote at most once, while the use of $\asset$ ensures that votes are not lost or double-counted.

\reed{Solidity impl. comes from ``Solidity by Example'' page}
\begin{figure*}[h]
    \centering
    \begin{subfigure}[t]{0.49\textwidth}
        \centering
        \lstinputlisting[language=Solidity]{splash20-examples/voting.sol}
        \caption{Solidity}
        \label{fig:voting-impl-sol}
    \end{subfigure}%
    \hfill
    \begin{subfigure}[t]{0.49\textwidth}
        \centering
        \lstinputlisting[language=flow]{splash20-examples/voting.flow}
        \caption{\langName}
        \label{fig:voting-impl-flow}
    \end{subfigure}
    \caption{A voting contract with a set of proposals, for which each user must first be given persmission to vote by the chairperson, assigned in the constructor of the contract (not shown)>
    Each user can vote exactly once for exactly one proposal.
    The proposal with the most votes wins.}
    \label{fig:voting-impl}
\end{figure*}

\subsection{The DAO attack}
One of the most financially impactful bugs in a smart contract on the Ethereum blockchain was the bug in the DAO contract which allowed a large quantity of ether, worth about \$50 million dollars at the time, to be stolen \reed{cite, verifying dollar amount}.
The bug was caused by a reentrancy-unsafe function in the contract, illustrated below.

The Solidity function shown below allows a user to withdraw their balance of ether whenever they wish.
It does this by first transferring the ether from the contract to the user's address, then decreasing the user's balance.
However, in Ethereum, a transfer of ether also calls a special function on the receiving address, when it is a contract address, called a \emph{fallback} function.
In this function, arbitrary code may be executed, and because \lstinline{call.value()} was used, this call has access to all of the remaining gas of the transaction.
A malicious contract can use this to perform a \emph{reentrant} call back into the \lstinline{withdrawBalance} function, and withdraw their balance again---the user's balance has not been decreased yet.
The fix in this case is a simple swap of the last two lines of the function, but reentrancy issues can be more complicated than this.
\reed{The below is from \url{https://consensys.github.io/smart-contract-best-practices/known_attacks/}}
\begin{lstlisting}[language=Solidity]
function withdrawBalance() public {
    uint amountToWithdraw = userBalances[msg.sender];
    // At this point, the caller's code is executed, and
    // can call withdrawBalance again
    require(msg.sender.call.value(amountToWithdraw)(""));
    userBalances[msg.sender] = 0;
}
\end{lstlisting}

In \langName, this attack could not have occurred for several reasons.
Consider the following implementation of the same function in \langName given below.
\begin{lstlisting}[language=flow]
transaction withdrawBalance():
    userBalances[msg.sender] --> msg.sender.balance
\end{lstlisting}
Because of the additional information encoded in the flow construct, the compiler can output the safe version of the above code---reducing the balance before performing the external call---without any developer intervention.
Additionally, \langName forbids any reentrant call from an external source, a similar approach to the Obsidian language \reed{cite}, which would also prevent more complicated reentrancy attacks.

\section{Discussion}

\section{Related Work}
\reed{Obsidian, Scilla, Move, etc.?}
\reed{TODO: The safety guarantees provided by \langName differ from those provided by other languages because...Obsidian has similar concept of assets, but doesn't allow expressing immutability or uniqueness (could do fungibility via an interface, I suppose); this is probably going to be similar to other languages built around linear type systems. Obsidian's reentrancy scheme is slightly different, Nomos has a similar global lock, it seems.}

\section{Future Work}

\section{Conclusion}

%% The next two lines define the bibliography style to be used, and
%% the bibliography file.
\bibliographystyle{ACM-Reference-Format}
\bibliography{biblio}

\appendix
\section{Formalization}

\subsection{Syntax}
\begin{figure*}[h]
    \centering
\begin{align*}
    C &\in \textsc{ContractNames} & m &\in \textsc{TransactionNames} \\
    t &\in \textsc{TypeNames} & x,y,z &\in \textsc{Identifiers} \\
    n &\in \Z \\
\end{align*}
\begin{tabular}{l r l l}
    $\mathfrak{q}$ & \bnfdef & $\exactlyone$ \bnfalt $\any$ \bnfalt $\nonempty$ & (selector quantifiers) \\
    $\mathcal{Q}$, $\mathcal{R}$, $\mathcal{S}$ & \bnfdef & $\mathfrak{q}$ \bnfalt $\emptyq$ \bnfalt $\every$ & (\typeQuantities) \\
    $\mathcal{C}$ & \bnfdef & $\optionq$ \bnfalt $\setq$ \bnfalt $\listq$ & (collection type constructors) \\
    $T$ & \bnfdef & \boolt \bnfalt $\natt$ \bnfalt $\mathcal{C}~\tau$ & \\
        & \bnfalt & $\map~\tau \Rightarrow \tau$ \bnfalt $\mapitem~\tau \Rightarrow \tau$ & \\
        & \bnfalt & $\linking~\tau \Leftrightarrow \tau$ \bnfalt $\link~\tau \Leftrightarrow \tau$ & \\
        & \bnfalt & $\tau \transforms \tau$ \bnfalt $\curlys{\overline{x : \tau}}$ \bnfalt $t$ & (base types) \\
    $\tau$, $\sigma$, $\pi$ & \bnfdef & $\mathcal{Q}~T$ & (types) \\
    $\mathcal{V}$ & \bnfdef & $n$ \bnfalt \true \bnfalt \false \bnfalt $\emptyval$ \bnfalt $\lambda x : \tau. E$ & (values) \\
    $\mathcal{L}$, $\mathcal{M}$ & \bnfdef & $x$ \bnfalt $\mathcal{L}.x$ \bnfalt $\mathcal{L}[\mathcal{L}]$ & (locations) \\
    $E$ & \bnfdef & $\mathcal{V}$ \bnfalt $\mathcal{L}$ \bnfalt $x.m(\overline{x})$ \bnfalt $\single(x)$ \bnfalt $s~\varin~x$ \bnfalt $\curlys{\overline{x : \tau \mapsto x}}$ & \\
        & \bnfalt & $\letvar~x:\tau := E~\varin~E$ \bnfalt $\ifS~x~\thenS~E~\elseS~E$ & \\
        & \bnfalt & $x = x$ \bnfalt $x \neq x$ \bnfalt $\total~x$ \bnfalt $\total~t$ & (expressions) \\
    % For the moment, removed these, but maybe put them back: $x \%$ \bnfalt $\min(f, f)$ \bnfalt $\max(f, f)$
    $s$ & \bnfdef & $\mathcal{L}$ \bnfalt $\everything$ \bnfalt $\mathfrak{q}~x : \tau~\suchthat~E$ & (selector) \\
    % $\mathcal{N}$ & \bnfdef & $\mathcal{L}$ \bnfalt $\newc~C(\overline{x})$ \bnfalt $\newc~t$ \bnfalt $\consume$ & (nodes) \\
    % $F$ & \bnfdef & $\mathcal{N} \sends{s} \mathcal{N}$ & (flows) \\
    $\mathcal{S}$ & \bnfdef & $\mathcal{L}$ \bnfalt $\newc~t$ & (sources) \\
    $\mathcal{D}$ & \bnfdef & $\mathcal{L}$ \bnfalt $\consume$ & (destinations) \\
    $F$ & \bnfdef & $\mathcal{S} \sends{s} x \to \mathcal{D}$ & (flows) \\
    $\Stmt$ & \bnfdef & $F$ \bnfalt $E$ \bnfalt $\revert(E)$ \bnfalt \pack \bnfalt $\unpack(x)$ \bnfalt $\emit~E(\overline{x})$ & \\
            & \bnfalt & $\tryS~\Stmt~\catchS(x : \tau)~\Stmt$ \bnfalt $\ifS~x~\thenS~\Stmt~\elseS~\Stmt$ & \\
            & \bnfalt & $\var~x:\tau := E~\varin~\Stmt$ \bnfalt $\Stmt;\Stmt$ & (statements) \\
    $M$ & \bnfdef & \fungible \bnfalt \unique \bnfalt \immutable \bnfalt \consumable \bnfalt \asset & (type declaration modifiers) \\
    $\Decl$ & \bnfdef & $x : \tau$ & (field) \\
            & \bnfalt & $\event~E(\overline{x : \tau})$ & (event declaration) \\
            & \bnfalt & $\type~t~\is~\overline{M}~T$ & (type declaration) \\
            & \bnfalt & $[\private]~\transaction~m(\overline{x : \tau}) \to x : \tau~\doC~\Stmt$ & (transactions) \\
            & \bnfalt & $\view~m(\overline{x : \tau}) \to \tau := E$ & (views) \\
            & \bnfalt & $\oncreate(\overline{x : \tau})~\doC~\Stmt$ & (constructor) \\
    $\Con$ & \bnfdef & $\contract~C~\{~\overline{\Decl}~\}$ & (contracts) \\
    $\Prog$ & \bnfdef & $\overline{\Con}~;~S$ & (programs)

    % Maybe:  $\alpha$ & \bnfdef & & (row type variables) \\
\end{tabular}
\caption{Abstract syntax of the core calculus of \langName.}
\label{fig:lang-syntax}
\end{figure*}

\reed{We have public and private transactions...we could also have a public/private type?
The difference being that public types can be transferred between contracts.}
\reed{Random note: We can optimize merge operations on unique types by dropping some checks.}

In the surface language, ``collection types'' (i.e., $\mathcal{Q}~\mathcal{C}~\tau$ or a transformer) are by default $\any$, but all other types, like $\natt$, are $\exactlyone$.

\reed{Some simplification ideas}
\reed{Could get rid of selecting by locations and only allowed selecting with quantifiers, and just optimize things like $\exactlyone~x:\tau~\suchthat~x = y$ into a lookup.
Also, allowing \textbf{any} type quantity in a selector lets us do away with everything.
Would actually be even nicer if we allowed any type quantity to appear in the quantifier, because then we wouldn't even need a special rule for everything.}
\reed{We could also get rid of ``if'' and instead do something like $\any~x:\tau~\suchthat~\ifS~b~\thenS~x = y~\elseS~false$}

\reed{Contract types should be consumable assets by default (consuming a contract is a self-destruct?)}

\subsection{Statics}
\begin{figure}[ht]
\begin{tabular}{l r l l}
    $\Gamma,\Delta,\Xi$ & \bnfdef & $\emptyset$ \bnfalt $\Gamma, x : \tau$ & (type environments)
\end{tabular}
\label{type-env}
\end{figure}

\begin{definition}
    Define $\Quant = \curlys{\emptyq, \any, \exactlyone, \nonempty, \every}$, and call any $\mathcal{Q} \in \Quant$ a \emph{type quantity}.
    Define $\emptyq < \any < \,\, \exactlyone < \nonempty < \every$.
\end{definition}

\framebox{$\tau~\asset$} \textbf{Asset Types}

\reed{The syntax for record ``fields'' and type environments is the same...could just use it}
\begin{align*}
    (\mathcal{Q}~T)~\asset \iff \mathcal{Q} \neq \emptyq \tand (& \asset \in \modifiers(T) \tor \\
                                                                % & (T = \tau \transforms \sigma \tand \sigma~\asset) \tor \\
                                                                & (T = \mathcal{C}~\tau \tand \tau~\asset) \tor \\
                                                                & (T = \curlys{\overline{y : \sigma}} \tand \exists x : \tau \in \overline{y : \sigma}. (\tau~\asset)))
\end{align*}
% \reed{It should be the case that a transformer can have an output of an asset type if and only if it has an input asset type (and it the case of curried transformers, that \textbf{some} input type is an asset).}

\framebox{$\tau~\consumable$} \textbf{Consumable Types}
\begin{align*}
    (\mathcal{Q}~T)~\consumable &\iff \consumable \in \modifiers(T) \tor \lnot((\mathcal{Q}~T)~\asset)
\end{align*}

$\mathcal{Q} \oplus \mathcal{R}$ represents the quantity present when flowing $\mathcal{R}$ of something to a storage already containing $\mathcal{Q}$.
$\mathcal{Q} \ominus \mathcal{R}$ represents the quantity left over after flowing $\mathcal{R}$ from a storage containing $\mathcal{Q}$.

\begin{definition}
    Let $\mathcal{Q}, \mathcal{R} \in \Quant$.
    Define the commutative operator $\oplus$, called \emph{combine}, as the unique function $\Quant^2 \to \Quant$ such that
    \[
        \renewcommand\arraystretch{1.2}
        \begin{array}{r c l l l}
            \mathcal{Q} \oplus \emptyq & = & \mathcal{Q} & \\
            \mathcal{Q} \oplus \every & = & \every & \\
            \nonempty \oplus \mathcal{R} & = & \nonempty & \tif \emptyq < \mathcal{R} < \every \\
            \exactlyone \oplus \mathcal{R} & = & \nonempty & \tif \emptyq < \mathcal{R} < \every \\
            \any \oplus \any & = & \any &
        \end{array}
    \]

    Define the operator $\ominus$, called \emph{split}, as the unique function $\Quant^2 \to \Quant$ such that
    \[
        \renewcommand\arraystretch{1.2}
        \begin{array}{r c l l l}
            \mathcal{Q} \ominus \emptyq & = & \mathcal{Q} & \\
            \emptyq \ominus \mathcal{R} & = & \emptyq & \\
            \mathcal{Q} \ominus \every &= & \emptyq & \\
            \every \ominus \mathcal{R} &= & \every & \tif \mathcal{R} < \every \\
            \nonempty - \mathcal{R} &= & \any & \tif \emptyq < \mathcal{R} < \every \\
            \exactlyone - \mathcal{R} &= & \emptyq & \tif \exactlyone \leq \mathcal{R} \\
            \exactlyone - \any &= & \any & \\
            \any - \mathcal{R} &= & \any & \tif \emptyq < \mathcal{R} < \every \\
        \end{array}
    \]
\end{definition}

Note that we write $(\mathcal{Q}~T) \oplus \mathcal{R}$ to mean $(\mathcal{Q} \oplus \mathcal{R})~T$ and similarly $(\mathcal{Q}~T) \ominus \mathcal{R}$ to mean $(\mathcal{Q} \ominus \mathcal{R})~T$.

\begin{definition}
    We can consider a type environment $\Gamma$ as a function $\textsc{Identifiers}\xspace \to \textsc{Types}\xspace \cup \curlys{\bot}$ as follows:
    \[
        \Gamma(x) =
        \begin{cases}
            \tau & \tif x : \tau \in \Gamma \\
            \bot & \owise
        \end{cases}
    \]
    We write $\dom(\Gamma)$ to mean $\setbuild{x \in \textsc{Identifiers}}{\Gamma(x) \neq \bot}$, and $\Gamma|_X$ to mean the environment $\setbuild{x : \tau \in \Gamma}{x \in X}$ (restricting the domain of $\Gamma$).
\end{definition}

\begin{definition}
    Let $\mathcal{Q}$ and $\mathcal{R}$ be \typeQuantities, $T_\mathcal{Q}$ and $T_\mathcal{R}$ base types, and $\Gamma$ and $\Delta$ type environments.
    Define the following orderings, which make types and type environments into join-semilattices.
    For type quantities, define the partial order $\sqsubseteq$ as the reflexive closure of the strict partial order $\sqsubset$ given by
    \[
        \mathcal{Q} \sqsubset \mathcal{R} \iff (\mathcal{Q} \neq \any \tand \mathcal{R} = \any) \tor (\mathcal{Q} \in \curlys{\exactlyone, \every} \tand \mathcal{R} = \nonempty)
    \]
    For types, define the partial order $\leq$ by
    \[
        \mathcal{Q}~T_\mathcal{Q} \leq \mathcal{R}~T_\mathcal{R} \iff T_\mathcal{Q} = T_\mathcal{R} \tand \mathcal{Q} \sqsubseteq \mathcal{R}
    \]
    For type environments, define the partial order $\leq$ by
    \[
        \Gamma \leq \Delta \iff \forall x. \Gamma(x) \leq \Delta(x)
    \]
    Denote the join of $\Gamma$ and $\Delta$ by $\Gamma \sqcup \Delta$.
\end{definition}

\framebox{$\Gamma \flowproves E : \tau \flowprovesout \Delta$} \textbf{Expression Typing}

These rules are for ensuring that expressions are well-typed, and keeping track of which variables are used throughout the expression.
Most rules do \textbf{not} change the context, with the notable exceptions of internal calls and record-building operations.
We begin with the rules for typing the various literal forms of values.
\reed{TODO: Single rule should work for any fungible (or mergeable??) type}
\begin{mathpar}
    \inferrule*[right=Empty-Val]{
    }{ \Gamma \flowproves \emptyval : \emptyq~\mathcal{C}~\tau \flowprovesout \Gamma }

    \inferrule*[right=Single]{
        \Gamma \flowproves x : \tau \flowprovesout \Delta
    }{ \Gamma \flowproves \single(x) : \,\,\exactlyone~\mathcal{C}~\tau \flowprovesout \Delta }

    \inferrule*[right=Transformer]{
        \Gamma, x : \tau \flowproves E : \sigma \flowprovesout \Gamma, x : \pi
        \and
        \lnot(\pi~\asset)
    }{ \Gamma \flowproves (\lambda x : \tau. E) : \,\,\exactlyone~(\tau \transforms \sigma) \flowprovesout \Gamma }

    \inferrule*[right=Build-Rec]{
        \Gamma \flowproves \overline{y : \tau} \flowprovesout \Delta
    }{ \Gamma \flowproves \curlys{\overline{x : \tau \mapsto y}} \flowprovesout \Delta }
\end{mathpar}

Next, the lookup rules.
Notably, the \textsc{Demote-Lookup} rule allows the use of variables of an asset type in an expression without consuming the variable as \textsc{Lin-Lookup} does.
However, it is still safe, because it is treated as its demoted type, which is always guaranteed to be a non-asset \reed{probably should prove this to be sure}.

\begin{mathpar}
    \inferrule*[right=Demote-Lookup]{
    }{ \Gamma, x : \tau \flowproves x : \demote(\tau) \flowprovesout \Gamma, x : \tau }

    \inferrule*[right=Lin-Lookup]{
        \demote(\tau) \neq \tau
    }{ \Gamma, x : \mathcal{Q}~T \flowproves x : \mathcal{Q}~T \flowprovesout \Gamma, x : \emptyq~T }

    \inferrule*[right=Record-Demote-Lookup]{
        f : \sigma \in \overline{y : \tau}
    }{ \Gamma, x : \curlys{\overline{y : \tau}} \flowproves x.f : \demote(\sigma) \flowproves \Gamma, x : \curlys{\overline{y : \tau}} }

    \inferrule*[right=Record-Lin-Lookup]{
        f : \mathcal{Q}~T \in \overline{y : \tau}
        \and
        \demote(\sigma) \neq \sigma
        \\
        \overline{z : \sigma} = (\overline{y : \tau} \setminus \curlys{f : \mathcal{Q}~T}) \cup \curlys{f : \emptyq~T}
    }{ \Gamma, x : \curlys{\overline{y : \tau}} \flowproves x.f : \mathcal{Q}~T \flowproves \Gamma, x : \curlys{\overline{z : \sigma}} }
\end{mathpar}

The expression $s~\varin~x$ allows checking whether a flow will succeed without the EAFP-style (``Easier to ask for forgiveness than permission''; e.g., Python).
A flow $A \sends{s} B$ is guaranteed to succeed when ``$s~\varin~A$'' is true and ``$s~\varin~B$'' is false.
\begin{mathpar}
    \inferrule*[right=Check-In]{
        \Gamma \proves x~\provides_{\mathcal{Q}}~\tau
        \and
        \Gamma \proves s~\selects~\demote(\tau)
    }{ \Gamma \proves (s~\varin~x) : \boolt \flowprovesout \Gamma }
\end{mathpar}

We distinguish between three kinds of calls: view, internal, and external.
A view call is guaranteed to not change any state in the receiver, while both internal and external calls may do so.
The difference between internal and external calls is that we may transfer assets to an internal call, but \textbf{not} to an external call, because we cannot be sure any external contract will properly manage the asset of our contract.
\begin{mathpar}
    \inferrule*[right=View-Call]{
        \dom(\fields(C)) \cap \dom(\Gamma) = \emptyset
        \\
        \typeof(C, m) = \curlys{\overline{a : \tau}} \transforms \sigma
        \and
        \Gamma, x : C \flowproves \overline{y : \tau} \flowprovesout \Gamma, x : C
    }{ \Gamma, x : C \flowproves x.m(\overline{y}) : \sigma \flowprovesout \Gamma, x : C }

    \inferrule*[right=Internal-Tx-Call]{
        \dom(\fields(C)) \cap \dom(\Gamma) = \emptyset
        \\
        \typeof(C, m) = \curlys{\overline{a : \tau}} \transforms \sigma
        \\
        \Gamma, \this : C \flowproves \overline{y : \tau} \flowprovesout \Delta, \this : C
    }{ \Gamma, \this : C \flowproves \this.m(\overline{y}) : \sigma \flowprovesout \Delta, \this : C }

    \inferrule*[right=External-Tx-Call]{
        \dom(\fields(C)) \cap \dom(\Gamma) = \emptyset
        \\
        (\transaction~m(\overline{a : \tau}) \to \sigma~\doC~S) \in \decls(D)
        \\
        \Gamma, \this : C, x : D \flowproves \overline{y : \tau} \flowprovesout \Gamma, \this : C, x : D
    }{ \Gamma, \this : C, x : D \flowproves x.m(\overline{y}) : \sigma \flowprovesout \Gamma, \this : C, x : D }

    \inferrule*[right=New-Con]{
        (\oncreate(\overline{x : \tau})~\doC~S) \in \decls(C)
        \and
        \Gamma \flowproves \overline{y : \tau} \flowprovesout \Gamma
    }{ \Gamma \flowproves \newc~C(\overline{y}) : C }
\end{mathpar}

Finally, the rules for If and Let expressions.
In \textsc{Let-Expr}, we must ensure that the newly bound variable is either consumed or is not an asset after the body runs.
\begin{mathpar}
    \inferrule*[right=If-Expr]{
        \Gamma \flowproves x : \boolt \flowprovesout \Gamma
        \and
        \Gamma \flowproves E_1 : \tau \flowprovesout \Delta
        \and
        \Gamma \flowproves E_2 : \tau \flowprovesout \Xi
    }{ \Gamma \flowproves (\ifS~x~\thenS~E_1~\elseS~E_2) : \tau \flowprovesout \Delta \sqcup \Xi }

    \inferrule*[right=Let-Expr]{
        \Gamma \flowproves E_1 : \tau \flowprovesout \Delta
        \and
        \Delta, x : \tau \flowproves E_2 : \pi \flowprovesout \Xi, x : \sigma
        \and
        \lnot(\sigma~\asset)
    }{ \Gamma \flowproves (\letvar~x : \tau := E_1~\varin~E_2) : \pi \flowprovesout \Xi }
\end{mathpar}

\framebox{$\elemtype(T) = \sigma$}
\begin{align*}
    \elemtype(T) =
    \begin{cases}
        \elemtype(S) & \tif \type~T~\is~\overline{M}~S \\
        \sigma & \tif T = \mathcal{C}~\sigma \\
        !~T & \owise
    \end{cases}
\end{align*}

\framebox{$\Gamma \proves \mathcal{L} : \tau$} \textbf{Non-consuming location typing}
This judgement is exclusively used for checking the types of the storages in a flow.
It is safe to use because the flow rule handles updating the storages after the flow.
\reed{Would be nice to combine the linking and map rules, and especially nice to avoid special casing them altogether.
Could be done with a ``$\textbf{\texttt{keytype}}(T)$'' function.
Maybe we could make it work on sets of $\mapitem$ or $\link$, and then avoid even defining maps and linkings.}
\begin{mathpar}
    \inferrule*[right=Loc-Var]{
    }{ \Gamma, x : \tau \proves x : \tau }

    \inferrule*[right=Loc-Member]{
        \Gamma \proves \mathcal{L} : \mathcal{Q}~T
        \and
        (f : \sigma) \in \fields(T)
        \and
        \lnot(T~\immutable)
    }{ \Gamma \proves \mathcal{L}.f : \sigma }

    \inferrule*[right=Loc-Map]{
        \Gamma \proves \mathcal{L} : \mathcal{Q}~(\map~\tau~\Rightarrow~\sigma)
        \and
        \Gamma \proves \mathcal{M} : \tau
    }{ \Gamma \proves \mathcal{L}[\mathcal{M}] : \sigma }

    \inferrule*[right=Loc-Linking]{
        \Gamma \proves \mathcal{L} : \mathcal{Q}~(\linking~\tau~\Leftrightarrow~\sigma)
        \and
        \Gamma \proves \mathcal{M} : \tau
    }{ \Gamma \proves \mathcal{L}[\mathcal{M}] : \sigma }
\end{mathpar}

\reed{How to handle immutable properly?}
\framebox{$\Gamma \proves \mathcal{S}~\provides_{\mathcal{Q}}~\tau$} \textbf{Source Typing}
\begin{mathpar}
    \inferrule*[right=Provide-Var]{
        \Gamma \proves S : \mathcal{Q}~T
    }{ \Gamma \proves S~\provides_{\mathcal{Q}}~\elemtype(T) }

    \inferrule*[right=Provide-Source]{
        (\type~t~\is~\overline{M}~T) \in \decls(C)
    }{ \Gamma, \this : C \proves (\newc~t)~\provides_{\every}~\exactlyone~t }
\end{mathpar}

\framebox{$\Gamma \proves \mathcal{D}~\accepts~\tau$} \textbf{Destination Typing}
Note that the \typeQuantities in \textsc{Accept-One} are different on the left and right of the turnstile.
This is because, for example, when I have $D : \nonempty~\setq~\natt$, it is reasonable to flow into it some $S : \any~\setq~\natt$.
\begin{mathpar}
    \inferrule*[right=Accept-Var]{
        \Gamma \proves D : \mathcal{Q}~T
    }{ \Gamma \proves D~\accepts~\elemtype(T) }

    \inferrule*[right=Accept-Consume]{
        \tau~\consumable
    }{ \Gamma \flowproves \consume~\accepts~\tau }
\end{mathpar}

\framebox{$\Gamma \proves s ~\selects_\mathcal{Q}~ \tau$} \textbf{Selectors}
\begin{mathpar}
    \inferrule*[right=Select-Var]{
        \Gamma \flowproves \mathcal{L} : \mathcal{Q}~T \flowprovesout \Gamma
    }{ \Gamma \proves \mathcal{L}~\selects_{\mathcal{Q}}~\elemtype(T) }

    \inferrule*[right=Select-Everything]{
    }{ \Gamma \proves \everything~\selects_{\every}~\tau }

    \inferrule*[right=Select-Quant]{
        \Gamma, x : \tau \flowproves p : \boolt \flowprovesout \Gamma, x : \tau
    }{ \Gamma \proves (\mathfrak{q}~x : \tau~\suchthat~p)~\selects_\mathfrak{q}~\tau }
\end{mathpar}

\framebox{$\validSelect(s, \mathcal{R}, \mathcal{Q})$}
We need to ensure that the resources to be selected are easily computable.
In particular, we wish to enforce that we never select $\everything$ from a source containing $\every$ of something, nor do we use a selector like $\mathfrak{q} x : \tau \tsuchthat E$ on a source contianing $\every$ of something.
The following definition captures these restrictions.
\begin{align*}
    \validSelect(s, \mathcal{R}, \mathcal{Q}) \iff \min(\mathcal{Q}, \mathcal{R}) < \every \tand (\mathcal{Q} = \every \implies \exists \mathcal{L}. s = \mathcal{L})
\end{align*}

\framebox{$\Gamma \flowproves S~\ok \flowprovesout \Delta$} \textbf{Statement Well-formedness}

Flows are the main construct for transferring resources.
A flow has four parts: a source, a selector, a transformer, and a destination.
The selector acts as a function that ``chooses'' part of the source's resources to flow.
These resources then get applied to the transformer, which is an applicative functor applied to a function type.
\reed{Bringing back $\one$ would let us do all the collections the same way in all of these flow-related rules, which would be nice.}

\begin{mathpar}
    \inferrule*[right=Ok-Flow]{
        \Gamma \proves A~\provides_\mathcal{Q}~\tau
        \\
        \Gamma \proves s~\selects_\mathcal{R}~\tau
        \and
        \validSelect(s, \mathcal{R}, \mathcal{Q})
        \\
        \Delta = \update(\Gamma, A, \Gamma(A) \ominus \mathcal{R})
        \and
        \Delta \flowproves f : \tau \transforms \sigma \flowprovesout \Delta
        \\
        \Delta \proves B~\accepts~\sigma
    }{ \Gamma \flowproves (A \sends{s} f \to B)~\ok \flowprovesout \update(\Delta, B, \Delta(B) \oplus \min(\mathcal{Q}, \mathcal{R})) }
\end{mathpar}

\reed{TODO: Finish handling currying transformers.}

\begin{mathpar}
    \inferrule*[right=Ok-Var-Def]{
        \Gamma \flowproves E : \tau \flowprovesout \Delta
        \and
        \Delta, x : \tau \flowproves S~\ok \flowprovesout \Xi, x : \sigma
        \and
        \lnot(\sigma~\asset)
    }{ \Gamma \flowproves (\var~x : \tau := E~\varin~S)~\ok \flowprovesout \Xi }

    \inferrule*[right=Ok-If]{
        \Gamma \flowproves x : \boolt \flowprovesout \Gamma
        \and
        \Gamma \flowproves S_1~\ok \flowprovesout \Delta
        \and
        \Gamma \flowproves S_2~\ok \flowprovesout \Xi
    }{ \Gamma \flowproves (\ifS~x~\thenS~S_1~\elseS~S_2)~\ok \flowprovesout \Delta \sqcup \Xi }

    \inferrule*[right=Ok-Try]{
        \Gamma \flowproves S_1~\ok \flowprovesout \Delta
        \and
        \Gamma, x : \tau \flowproves S_2~\ok \flowprovesout \Xi, x : \sigma
        \and
        \lnot(\sigma~\asset)
    }{ \Gamma \flowproves (\tryS~S_1~\catchS~(x : \tau)~S_2)~\ok \flowprovesout \Delta \sqcup \Xi }

    \inferrule*[right=Ok-Revert]{
        \Gamma \flowproves E : \tau \flowprovesout \Gamma
        \and
        \lnot(\tau~\asset)
    }{ \Gamma \flowproves \revert(E)~\ok \flowprovesout \Gamma }

    \inferrule*[right=Ok-Expr]{
        \Gamma \flowproves E : \tau \flowprovesout \Delta
        \and
        \lnot(\tau~\asset)
    }{ \Gamma \flowproves E~\ok \flowprovesout \Delta }

    \inferrule*[right=Ok-Seq]{
        \Gamma \flowproves S_1~\ok \flowprovesout \Delta
        \and
        \Delta \flowproves S_2~\ok \flowprovesout \Xi
    }{ \Gamma \flowproves (S_1 ; S_2)~\ok \flowprovesout \Xi }

    \inferrule*[right=Ok-Unpack]{
        \this.f : \tau \in \fields(C)
    }{ \Gamma, \this : C \flowproves \unpack(f)~\ok \flowprovesout \Gamma, \this : C, \this.f : \tau }

    \inferrule*[right=Ok-Pack]{
        (\Gamma|_{\dom(\fields(C))}) \leq \fields(C)
        \\
        \Delta = \setbuild{x : \tau \in \Gamma}{x \not\in \dom(\fields(C))}
    }{ \Gamma, \this : C \flowproves \pack~\ok \flowprovesout \Delta, \this : C }
\end{mathpar}

\framebox{$\proves_C \Decl~\ok$} \textbf{Declaration Well-formedness}
\begin{mathpar}
    \inferrule*[right=Ok-View]{
        \Gamma = \this : C, \fields(C), \overline{x : \tau}
        \and
        \Gamma \flowproves E : \sigma \flowprovesout \Gamma
    }{ \proves_C (\view~m(\overline{x : \tau}) \to \sigma := E)~\ok }

    \inferrule*[right=Ok-Tx-Public]{
        \this : C, \overline{x : \tau}, y : \emptyq~T \flowproves S~\ok \flowprovesout \Delta, \this : C, y : \mathcal{Q}~T
        \\
        \dom(\fields(C)) \cap \dom(\Delta) = \emptyset
        \\
        \forall x : \tau \in \Delta. \lnot(\tau~\asset)
        \and
        \lnot(\mathcal{Q}~T~\asset)
    }{ \proves_C (\transaction~m(\overline{x : \tau}) \to y : \mathcal{Q}~T~\doC~S)~\ok }

    \inferrule*[right=Ok-Tx-Private]{
        \this : C, \overline{x : \tau}, y : \emptyq~T \flowproves S~\ok \flowprovesout \Delta, \this : C, y : \mathcal{Q}~T
        \\
        \dom(\fields(C)) \cap \dom(\Delta) = \emptyset
        \and
        \forall x : \tau \in \Delta. \lnot(\tau~\asset)
    }{ \proves_C (\private~\transaction~m(\overline{x : \tau}) \to y : \mathcal{Q}~T~\doC~S)~\ok }
\end{mathpar}

A field definition is always okay, as long as the type doesn't have the $\every$ modifier.
\reed{Add this restriction to the rest of the places where we write types.}
\reed{Maybe we should always restrict variable definitions so that you can only write named types that appear in the current contract.
This isn't strictly necessary, because everything will still work, but you'll simply never be able to get a value of an asset type not created in the current contract.}
\begin{mathpar}
    \inferrule*[right=Ok-Field]{
        \mathcal{Q} \neq \every
    }{ \proves_C (x : \mathcal{Q}~T)~\ok }
\end{mathpar}

A type declaration is okay as long as it has the $\asset$ modifier if its base type is an asset.
Note that this restriction isn't necessary, but is intended to help users realize which types are assets without unfolding the entire type definition.
\reed{TODO: What if the base type is a non-consumable asset, and we write consumable. Is that okay? Probably not.}
\begin{mathpar}
    \inferrule*[right=Ok-Type]{
        T~\asset \implies \asset \in \overline{M}
    }{ \proves_C (\type~t~\is~\overline{M}~T)~\ok }
\end{mathpar}

Note that we need to have constructors, because only the contract that defines a named type is allowed to create values of that type, and so it is not always possible to externally initialize all contract fields.
\begin{mathpar}
    \inferrule*[right=Ok-Constructor]{
        \fields(C) = \overline{\this.f : \mathcal{Q}~T}
        \and
        \this : C, \overline{x : \tau}, \overline{\this.f : \emptyq~T} \flowproves S~\ok \flowprovesout \Delta
        \and
        \forall y : \sigma \in \Delta. \lnot(\sigma~\asset)
    }{ \proves_C (\oncreate(\overline{x : \tau})~\doC~S)~\ok }
\end{mathpar}

\framebox{$\Con~\ok$} \textbf{Contract Well-formedness}
\begin{mathpar}
    \inferrule*[right=Ok-Con]{
        \forall d \in \overline{\Decl}. (\proves_C d~\ok)
        \and
        \exists! d \in \overline{\Decl}. \exists \overline{x : \tau}, S. d = \oncreate(\overline{x : \tau})~\doC~S
    }{ (\contract~C~\{ \overline{\Decl} \})~\ok }
\end{mathpar}

\framebox{$\Prog~\ok$} \textbf{Program Well-formedness}
\begin{mathpar}
    \inferrule*[right=Ok-Prog]{
        \forall C \in \overline{\Con}. C~\ok
        \and
        \emptyset \flowproves S \flowprovesout \emptyset
    }{ (\overline{\Con} ; S)~\ok }
\end{mathpar}

\paragraph{Other Auxiliary Definitions}

\framebox{$\modifiers(T) = \overline{M}$} \textbf{Type Modifiers}
\begin{align*}
    \modifiers(T) =
    \begin{cases}
        \overline{M} & \tif (\type~T~\is~\overline{M}~T) \\
        \emptyset & \owise
    \end{cases}
\end{align*}

\framebox{$\demote(\tau) = \sigma$}
\framebox{$\demote_*(T_1) = T_2$} \textbf{Type Demotion}
$\demote$ and $\demote_*$ take a type and ``strip'' all the asset modifiers from it, as well as unfolding named type definitions.
This process is useful, because it allows selecting asset types without actually having a value of the desired asset type.
Note that demoting a transformer type changes nothing.
This is because a transformer is \textbf{never} an asset, regardless of the types that it operators on, because it has no storage.

\begin{align*}
    \demote(\mathcal{Q}~T) &= \mathcal{Q}~\demote_*(T) \\
    % \demote_*(\voidt) &= \voidt \\
    \demote_*(\natt) &= \natt \\
    \demote_*(\boolt) &= \boolt \\
    \demote_*(t) &= \demote_*(T) & \twhere \type~t~\is~\overline{M}~T \\
    \demote_*(C) &= \demote_*(\curlys{\overline{x : \tau}}) & \twhere \fields(C) = \curlys{\overline{x : \tau}} \\
    \demote_*(\mathcal{C}~\tau) &= \mathcal{C}~\demote(\tau) \\
    \demote_*(\curlys{\overline{x : \tau}}) &= \curlys{\overline{x : \demote(\tau)}} \\
    \demote_*(\tau \transforms \sigma) &= \tau \transforms \sigma
\end{align*}

\framebox{$\decls(C) = \overline{\Decl}$} \textbf{Contract Declarations}
\begin{align*}
    \decls(C) = \overline{\Decl} \twhere (\contract~C~\{ \overline{\Decl} \})
\end{align*}

\framebox{$\fields(C) = \Gamma$} \textbf{Contract Fields}
\begin{align*}
    \fields(C) = \setbuild{\this.f : \tau}{f : \tau \in \decls(C)}
\end{align*}

\framebox{$\typeof(C, m) = \tau \transforms \sigma$} \textbf{Method Type Lookup}
\begin{align*}
    \typeof(C, m) =
    \begin{cases}
        \curlys{\overline{x : \tau}} \transforms \sigma & \tif (\private~\transaction~m(\overline{x : \tau})~\returns~y:\sigma~\doC~S) \in \decls(C) \\
        \curlys{\overline{x : \tau}} \transforms \sigma & \tif (\transaction~m(\overline{x : \tau})~\returns~y:\sigma~\doC~S) \in \decls(C) \\
        \curlys{\overline{x : \tau}} \transforms \sigma & \tif (\view~m(\overline{x : \tau})~\returns~\sigma := E) \in \decls(C)
    \end{cases}
\end{align*}

\framebox{$\update(\Gamma, x, \tau)$} \textbf{Type environment modification}
\[
    \update(\Gamma, x, \tau) =
    \begin{cases}
        \Delta, x : \tau & \tif \Gamma = \Delta, x : \sigma \\
        \Gamma & \owise
    \end{cases}
\]

\reed{Asset retention theorem?}
\reed{Resource accessiblity?}

\reed{What guarantees should we provide (no errors except for flowing a resource that doesn't exist in the source/already exists in the destination)?}

NOTE: We can implement preconditions (i.e., ``only when'') with just flows by doing something like:
\begin{lstlisting}
{ contractCreator = msg.sender } --[ true ]-> consume
\end{lstlisting}
This works because \lstinline|{ contractCreator = msg.sender } : set bool| (specifically, a singleton), so if \lstinline{contractCreator = msg.sender} doesn't evaluate to true, then we will fail to consume true from it.

\end{document}
\endinput

