\documentclass[10pt]{article}

\usepackage{amsmath}
\usepackage{hyperref}
\usepackage{tikz-cd}
\usepackage{amssymb}
\usepackage{amsthm}
\usepackage{bm}
\usepackage{listings}
\usepackage{bbm}
\usepackage{multicol}
\usepackage{mathtools}
\usepackage{mathpartir}
\usepackage{float}
\usepackage[inline]{enumitem}
\usepackage[margin=1.25in]{geometry}
\usepackage[T1]{fontenc}
\usepackage{kpfonts}

\usetikzlibrary{decorations.pathmorphing}

\lstset{
  frame=none,
  xleftmargin=2pt,
  stepnumber=1,
  numbers=left,
  numbersep=5pt,
  numberstyle=\ttfamily\tiny\color[gray]{0.3},
  belowcaptionskip=\bigskipamount,
  captionpos=b,
  escapeinside={*'}{'*},
  tabsize=2,
  emphstyle={\bf},
  commentstyle=\it,
  stringstyle=\mdseries\rmfamily,
  showspaces=false,
  keywordstyle=\bfseries\rmfamily,
  columns=flexible,
  basicstyle=\small\sffamily,
  showstringspaces=false,
}

\newcommand{\guard}{\bigg|\bigg|}
\newcommand{\newc}{\textbf{\texttt{new}}}
\newcommand{\everything}{\textbf{\texttt{everything}}}
\newcommand{\then}{\textbf{\texttt{then}}}
\newcommand{\this}{\textbf{\texttt{this}}}
\newcommand{\langName}{LANGUAGE-NAME\xspace}
\newcommand{\consumesarr}{\mathrel{\rotatebox[origin=c]{270}{${\looparrowright}$}}}
\newcommand{\consumes}[1]{\stackrel{#1}{\consumesarr}}
\newcommand{\sends}[1]{\stackrel{#1}{\to}}
\newcommand{\emitsarr}{\mathrel{\rotatebox[origin=c]{270}{${\looparrowleft}$}}}
\newcommand{\emits}[1]{\emitsarr #1}
\newcommand{\asset}{\textbf{\texttt{asset}}\xspace}
\newcommand{\fungible}{\textbf{\texttt{fungible}}\xspace}
\newcommand{\nonfungible}{\textbf{\texttt{nonfungible}}\xspace}
\newcommand{\addresst}{\textbf{\texttt{address}}\xspace}
\newcommand{\stringt}{\textbf{\texttt{string}}\xspace}
\newcommand{\setof}{\textbf{\texttt{set}}\xspace}
\newcommand{\optiont}{\textbf{\texttt{option}}\xspace}
\newcommand{\boolt}{\textbf{\texttt{bool}}\xspace}
\newcommand{\natt}{\textbf{\texttt{nat}}\xspace}
\newcommand{\byt}{\textbf{\texttt{by}}\xspace}
\newcommand{\stores}{\textbf{\texttt{stores}}\xspace}
\newcommand{\suchthat}{\textbf{s.t.}\xspace}
\newcommand{\returns}{\textbf{\texttt{returns}}\xspace}
\newcommand{\merge}{\rightsquigarrow}
\newcommand{\flowsto}{\rightsquigarrow}
\newcommand{\heldby}{\rightarrowtail}
\newcommand{\one}{\textbf{\texttt{one}}\xspace}
\newcommand{\some}{\textbf{\texttt{some}}\xspace}
\newcommand{\any}{\textbf{\texttt{any}}\xspace}
\newcommand{\map}{\textbf{\texttt{map}}\xspace}
\newcommand{\linking}{\textbf{\texttt{linking}}\xspace}
\newcommand{\transformer}{\textbf{\texttt{transformer}}\xspace}
\newcommand{\selects}{\textbf{\texttt{selects}}\xspace}
\newcommand{\demote}{\textbf{\texttt{demote}}\xspace}
\newcommand{\transforms}{\Rrightarrow}

\newcommand{\flowproves}{\mathbin{|\hskip -0.1em\scalebox{0.7}[1]{$\leadsto$}}}
\newcommand{\flowproveout}{\mathbin{\scalebox{0.7}[1]{$\leadsto$}\hskip -0.13em|}}


\lstset{
  frame=none,
  xleftmargin=2pt,
  stepnumber=1,
  numbers=left,
  numbersep=5pt,
  numberstyle=\ttfamily\tiny\color[gray]{0.3},
  belowcaptionskip=\bigskipamount,
  captionpos=b,
  escapeinside={*'}{'*},
  tabsize=2,
  emphstyle={\bf},
  commentstyle=\it,
  stringstyle=\mdseries\rmfamily,
  showspaces=false,
  keywordstyle=\bfseries\rmfamily,
  columns=flexible,
  basicstyle=\small\sffamily,
  showstringspaces=false,
}

\newcommand{\guard}{\bigg|\bigg|}
\newcommand{\newc}{\textbf{\texttt{new}}}
\newcommand{\everything}{\textbf{\texttt{everything}}}
\newcommand{\then}{\textbf{\texttt{then}}}
\newcommand{\this}{\textbf{\texttt{this}}}
\newcommand{\langName}{LANGUAGE-NAME\xspace}
\newcommand{\consumesarr}{\mathrel{\rotatebox[origin=c]{270}{${\looparrowright}$}}}
\newcommand{\consumes}[1]{\stackrel{#1}{\consumesarr}}
\newcommand{\sends}[1]{\stackrel{#1}{\to}}
\newcommand{\emitsarr}{\mathrel{\rotatebox[origin=c]{270}{${\looparrowleft}$}}}
\newcommand{\emits}[1]{\emitsarr #1}
\newcommand{\asset}{\textbf{\texttt{asset}}\xspace}
\newcommand{\fungible}{\textbf{\texttt{fungible}}\xspace}
\newcommand{\nonfungible}{\textbf{\texttt{nonfungible}}\xspace}
\newcommand{\addresst}{\textbf{\texttt{address}}\xspace}
\newcommand{\stringt}{\textbf{\texttt{string}}\xspace}
\newcommand{\setof}{\textbf{\texttt{set}}\xspace}
\newcommand{\optiont}{\textbf{\texttt{option}}\xspace}
\newcommand{\boolt}{\textbf{\texttt{bool}}\xspace}
\newcommand{\natt}{\textbf{\texttt{nat}}\xspace}
\newcommand{\byt}{\textbf{\texttt{by}}\xspace}
\newcommand{\stores}{\textbf{\texttt{stores}}\xspace}
\newcommand{\suchthat}{\textbf{s.t.}\xspace}
\newcommand{\returns}{\textbf{\texttt{returns}}\xspace}
\newcommand{\merge}{\rightsquigarrow}
\newcommand{\flowsto}{\rightsquigarrow}
\newcommand{\heldby}{\rightarrowtail}
\newcommand{\one}{\textbf{\texttt{one}}\xspace}
\newcommand{\some}{\textbf{\texttt{some}}\xspace}
\newcommand{\any}{\textbf{\texttt{any}}\xspace}
\newcommand{\map}{\textbf{\texttt{map}}\xspace}
\newcommand{\linking}{\textbf{\texttt{linking}}\xspace}
\newcommand{\transformer}{\textbf{\texttt{transformer}}\xspace}
\newcommand{\selects}{\textbf{\texttt{selects}}\xspace}
\newcommand{\demote}{\textbf{\texttt{demote}}\xspace}
\newcommand{\transforms}{\Rrightarrow}

\newcommand{\flowproves}{\mathbin{|\hskip -0.1em\scalebox{0.7}[1]{$\leadsto$}}}
\newcommand{\flowproveout}{\mathbin{\scalebox{0.7}[1]{$\leadsto$}\hskip -0.13em|}}



\newcommand{\putS}{\textbf{\texttt{put}}\xspace}
\newcommand{\evaluates}{\to}
\newcommand{\compat}{\text{compat}\xspace}
\newcommand{\resolve}{\textbf{\texttt{resolve}}\xspace}
\newcommand{\values}{\text{values}\xspace}
\newcommand{\storage}{\text{storage}\xspace}
\newcommand{\loc}{\text{loc}\xspace}
\newcommand{\Loc}{\mathcal{L}}
\newcommand{\LocK}{\mathcal{K}}
\newcommand{\copyC}{\textbf{\texttt{copy}}\xspace}
\newcommand{\Trfm}{\textbf{\texttt{Trfm}}\xspace}
\newcommand{\tableT}{\textbf{\texttt{table}}\xspace}
\newcommand{\Env}{\textbf{Env}\xspace}
\newcommand{\useField}{\text{useField}\xspace}

\begin{document}

\section{Formalization}

\subsection{Syntax}
\begin{align*}
    f &\in \textsc{TransformerNames} & t &\in \textsc{TypeNames} \\
    a,x,y,z &\in \textsc{Identifiers} & \alpha, \beta &\in \textsc{TypeVariables} \\
\end{align*}
\begin{tabular}{l r l l}
    $\mathcal{Q}$, $\mathcal{R}$, $\mathcal{S}$ & \bnfdef & $\exactlyone$ \bnfalt $\any$ \bnfalt $\nonempty$ \bnfalt $\emptyq$ \bnfalt $\every$ & (type quantities) \\
    $M$ & \bnfdef & \fungible \bnfalt \unique \bnfalt \immutable \bnfalt \consumable \bnfalt \asset & (type declaration modifiers) \\
    $T$ & \bnfdef & $\boolt$ \bnfalt $\natt$ \bnfalt $\alpha$ \bnfalt $t[\overline{T}]$ \bnfalt $\tableT(\overline{x})~\{ \overline{x : \tau} \}$ & (base types) \\
    $\tau$, $\sigma$, $\pi$ & \bnfdef & $\mathcal{Q}~T$ & (types) \\
    $T_V$ & \bnfdef & $\alpha~\is~\overline{M}$ & (type variable declaration) \\
    $\Loc$, $\LocK$ & \bnfdef & $\true$ \bnfalt $\false$ \bnfalt $n$ & \\
           & \bnfalt & $x$ \bnfalt $\Loc.x$ \bnfalt $\var~x : T$ \bnfalt $[ \overline{\Loc} ]$ \bnfalt $\{ \overline{x : \tau \mapsto \Loc} \}$ & \\
           & \bnfalt & $\demote(\Loc)$ \bnfalt $\copyC(\Loc)$ & \\
           & \bnfalt & $\Loc[\Loc]$ \bnfalt $\Loc[\mathcal{Q} \tsuchthat f[\overline{T}](\overline{L})]$ \bnfalt $\consume$ & \\
    $\Decl$ & \bnfdef & $\transformer~f[\overline{T_V}](\overline{x : \tau}) \to x : \tau ~\{ \overline{\Stmt} \}$ & (transformers) \\
            & \bnfalt & $\type~t[\overline{T_V}]~\is~\overline{M}~T$ & (type decl.) \\
    $\Trfm$ & \bnfdef & $\newc~t[\overline{T}](\overline{\Loc})$ \bnfalt $f[\overline{T}](\overline{\Loc})$ & (transformer calls) \\
    $\Stmt$ & \bnfdef & $\Loc \to \Loc$ \bnfalt $\Loc \to \Trfm \to \Loc$ & (flows) \\
            & \bnfalt & $\tryS~\{ \overline{\Stmt} \}~\catchS~\{ \overline{\Stmt} \}$ & (try-catch) \\
    $\Decl$ & \bnfdef & $\transformer~f[\overline{T_V}](\overline{x : \tau}) \to x : \tau ~\{ \overline{\Stmt} \}$ & (transformers) \\
    $\Prog$ & \bnfdef & $\overline{\Decl}; \overline{\Stmt}$ & (programs)
\end{tabular}

We use the following abbreviations
\begin{itemize}
    \item $(\map~\tau~\Rightarrow~\sigma) := \tableT(\texttt{key})~\{ \texttt{key} : \tau, \texttt{value} : \sigma \}$
    \item $(\listq~\tau) := \tableT(\texttt{idx})~\{ \texttt{idx} : \natt, \texttt{value} : \tau \}$
    \item $\{ \overline{x : \tau} \} := \tableT(\cdot)~\{ \overline{x : \tau} \}$
\end{itemize}

\reed{TODO: Need type formation rules to make sure things like types that are used are declared and that the key variables are a subset of the fields of a table}

\subsection{Statics}

Define $\# : \N \cup \curlys{\infty} \to \mathcal{Q}$ so that $\#(n)$ is the best approximation by type quantity of $n$, i.e.,
\[
    \#(n) =
    \begin{cases}
        \emptyq & \tif n = 0 \\
        \exactlyone & \tif n = 1 \\
        \nonempty & \tif n > 1 \\
        \every & \tif n = \infty \\
    \end{cases}
\]

\framebox{$\Gamma \flowproves (\Loc : \tau) ; u$} \textbf{Locator Typing}

Here $u$ is an \emph{updater}, that is, $u \in \curlys{\bot} \cup ((\Gamma \times (\tau \to \tau)) \to \Gamma)$.
The updater describes how the types of the values specified by the locator will be modified given some function $f : \tau \to \tau$.
If $u = \bot$, that means that the updater cannot be applied---this typically \reed{atm, always} means that the locator value(s) are immutable.

Define
\[
    u ||_T =
    \begin{cases}
        \bot & \tif T~\immutable \\
        u & \owise
    \end{cases}
\]

Let $\id{u}$ be the identity updater, that is, $\id{u}(\Delta, f) = \Delta$ for all $\Delta$ and $f$.

\begin{mathpar}
    \inferrule*[right=Bool]{
        b \in \{ \true, \false \}
    }{ \Gamma \flowproves b : \,\,\exactlyone~\boolt; \id{u} }

    \inferrule*[right=Nat]{
    }{ \Gamma \flowproves n : \#(n)~\natt; \id{u} }
\end{mathpar}

\reed{The idea is that both $\demote(\Loc)$ and $\copyC(\Loc)$ give a demoted value, but $\demote(\Loc)$ gives a read-only value (so no copy needs to happen), whereas copy will actually copy all the data.
The results below intentionally throw out the environment $\Delta$, because we don't want to actually consume whatever references we used to get $\Loc : \tau$.}

\begin{mathpar}
    \inferrule*[right=Demote]{
        \Gamma \flowproves (\Loc : \tau) ; u
    }{ \Gamma \flowproves (\demote(\Loc) : \demoteT(\tau)) ; \bot }

    \inferrule*[right=Copy]{
        \Gamma \flowproves (\Loc : \tau) ; u
    }{ \Gamma \flowproves (\copyC(\Loc) : \demoteT(\tau)) ; \id{u} }
\end{mathpar}

\reed{TODO: Finish adding the various updater functions as done in the Haskell file}

\begin{mathpar}
    \inferrule*[right=Var]{
    }{ \Gamma, x : \mathcal{Q}~T \flowproves (x : \mathcal{Q}~T) ; ((\Delta, f) \mapsto \Delta[x \mapsto f(\Delta(x))])||_T }

    \inferrule*[right=Field]{
        \Gamma \flowproves (\Loc : \mathcal{Q}~\exactlyone~T) ; u
        \\
        (x : \tau) \in \fields(T)
    }{ \Gamma \flowproves (\Loc.x : \tau); ((\Delta, f) \mapsto u(\Delta, \useField_{T,x}(f)))||_{T} }
\end{mathpar}

where $\useField_{T,x}(f)$ is defined by:
\reed{Can we simplify this a little?}
\reed{Can/should define keys and fields functions so we can do more easily}
\reed{How ot keep track that a value used to be of some named typed????}
\[
    \useField_{T,x}(f) =
    \begin{cases}
        \tableT(\overline{k})~\{ \overline{y : \tau}, x : f(\sigma), \overline{z : \pi} \} & \tif T = \tableT(\overline{k})~\{ \overline{y : \tau}, x : \sigma, \overline{z : \pi} \} \\
        \tableT(\overline{k})~\{ \overline{y : \tau}, x : f(\sigma), \overline{z : \pi} \} & \tif \type~T~\is~\overline{M}~\tableT(\overline{k})~\{ \overline{y : \tau}, x : \sigma, \overline{z : \pi} \} \\
    \end{cases}
\]

\begin{mathpar}
    \inferrule*[right=List]{
        \Gamma \flowproves \overline{\Loc : \tau ; u}
    }{ \Gamma \flowproves ([ \overline{\Loc} ] : \#(|\overline{\Loc}|)~\listq~\tau) ; (\Delta, f) \mapsto u_n(u_{n-1}(\cdots u_1(\Delta, f) \cdots, f), f)}

    \inferrule*[right=Record]{
        \Gamma \flowproves \overline{\Loc : \tau} \flowprovesout \Delta
    }{ \Gamma \flowproves \{ \overline{x : \tau \mapsto \Loc} \} : \,\,\exactlyone~\{ \overline{x : \tau} \} \flowprovesout \Delta }
\end{mathpar}

\reed{VarDef needs some way to get out of the definition so it can be used later? We could use the updater for this.}

\begin{mathpar}
    \inferrule*[right=VarDef]{
    }{ \Gamma \flowproves ((\var~x : T) : \emptyq~T) ; (\Delta, f) \mapsto \Delta[x \mapsto f(\emptyq~T)] }

    \inferrule*[right=Consume]{
        \tau~\consumable
    }{ \Gamma \flowproves \consume : \tau }
\end{mathpar}

\begin{mathpar}
    \inferrule*[right=Select]{
        \Gamma \flowproves (\Loc : \mathcal{Q}~T) ; u
        \\
        \Gamma \flowproves (\demote(\LocK) : \demoteT(\mathcal{R}~T)) ; v
    }{ \Gamma \flowproves (\Loc[\LocK] : \mathcal{R}~T) ; \select(\mathcal{R}, u) }

    % \inferrule*[right=Filter]{
    %     \Gamma \flowproves \Loc : \mathcal{Q}~T \flowprovesout \Delta
    %     \\
    %     \typeof(f, \overline{T}) = (\overline{x : \sigma}, y : \demoteT(\elemtype(T))) \to z : \,\,!~\boolt
    %     \\
    %     \forall i. \demoteT(\Gamma(a_i)) = \sigma_i
    %     \\
    %     \update(\Delta, \mathcal{S}, \Delta(\mathcal{S}) \ominus \mathcal{Q}) \flowproves \mathcal{D} : \mathcal{S}~T \flowprovesout \Xi
    % }{ \Gamma \flowproves \Loc[\mathcal{R}~\suchthat~f[\overline{T}](\overline{a})] : \mathcal{R}~T \flowprovesout \update(\Xi, \mathcal{D}, \Xi(\mathcal{D}) \oplus \min(\mathcal{Q}, \mathcal{R})) }
\end{mathpar}

where
\[
    \select(\mathcal{Q}, u)(\Delta, f) =
    \begin{cases}
        \Delta & \tif \mathcal{Q} = \emptyq \\
        u(\Delta, f) & \tif \mathcal{Q} = \every \\
        u(\Delta, \tau \mapsto \tau \sqcup f(\tau)) & \owise
    \end{cases}
\]

\framebox{$\Gamma \flowproves S~\ok \flowprovesout \Delta$} \textbf{Statement Well-formedness}
\begin{mathpar}
    \inferrule*[right=Ok-Flow-Every]{
        \Gamma \flowproves \Loc : \mathcal{Q}~T \flowprovesout \Delta
        \\
        \update(\Delta, \Loc, \Delta(\Loc) \ominus \mathcal{Q}) \flowproves \LocK : \mathcal{R}~T \flowprovesout \Xi
    }{ \Gamma \flowproves (\Loc \to \LocK)~\ok \flowprovesout \update(\Xi, \LocK, \Xi(\LocK) \oplus \mathcal{Q}) }

    \inferrule*[right=Ok-Flow-Transformer]{
        \Gamma \flowproves \Loc : \mathcal{Q}~T_1 \flowprovesout \Delta
        \\
        % \transformer~f(\overline{x : \sigma}, y : \elemtype(T_1)) \to z : \mathcal{R}~T_2~\{~ \overline{\Stmt} ~\}
        \typeof(f, \overline{T}) = (\overline{x : \sigma}, y : \elemtype(T_1)) \to z : \mathcal{R}~T_2~\{~ \overline{\Stmt} ~\}
        \\
        \forall i. \demoteT(\Gamma(x_i)) = \sigma_i
        \\
        \update(\Delta, \Loc, \Delta(\Loc) \ominus \mathcal{Q}) \flowproves \LocK : \mathcal{S}~T_2 \flowprovesout \Xi
    }{ \Gamma \flowproves (\Loc \to f[\overline{T}](\overline{x}) \to \LocK)~\ok \flowprovesout \update(\Xi, \LocK, \Xi(\LocK) \oplus \mathcal{Q}) }

    \inferrule*[right=Ok-Try]{
        \Gamma \flowproves \overline{S_1}~\ok \flowprovesout \Delta
        \\
        \Gamma \flowproves \overline{S_2}~\ok \flowprovesout \Xi
    }{ \Gamma \flowproves (\tryS~\{ \overline{S_1} \}~\catchS~\{ \overline{S_2} \})~\ok \flowprovesout \Delta \sqcup \Xi }
\end{mathpar}

\reed{Rule \textsc{Ok-Flow-Transformer} actually doesn't necessary add $\mathcal{Q}$ things to the destination because it's like a concat operation, need to either change that or change how much gets added.}

\framebox{$\proves \Decl~\ok$} \textbf{Declaration Well-formedness}
\begin{mathpar}
    \inferrule*[right=Ok-Transformer]{
        \overline{T_V}, \overline{x : \tau} \flowproves \overline{\Stmt}~\ok \flowprovesout \Gamma, y : \sigma
        \\
        \forall \pi \in \img(\Gamma). \lnot\isAsset(\overline{T_V}, \pi)
    }{ \proves (\transformer~f[\overline{T_V}](\overline{x : \tau}) \to y : \sigma \{ \overline{\Stmt} \})~\ok }
\end{mathpar}

\framebox{$\Prog~\ok$} \textbf{Program Well-formedness}
\begin{mathpar}
    \inferrule*[right=Ok-Prog]{
        \proves \overline{\Decl}~\ok
        \\
        \emptyset \flowproves \overline{\Stmt}~\ok \flowprovesout \Gamma
        \\
        \forall \tau \in \img(\Gamma). \lnot \isAsset(\emptyset, \tau)
    }{ (\overline{\Decl}; \overline{\Stmt})~\ok }
\end{mathpar}

\subsection{Dynamics}
\begin{tabular}{l r l l}
    $V$ & \bnfdef & $\true$ \bnfalt $\false$ \bnfalt $n$ \bnfalt $\{ \overline{x : \tau \mapsto n} \}$ & \\
    $\mathcal{V}$ & \bnfdef & $\overline{V}$ & \\
    $\Stmt$ & \bnfdef & $\ldots$ \bnfalt $\revert$ \bnfalt $\tryS(\Sigma, \overline{\Stmt}, \overline{\Stmt})$ & \\
\end{tabular}

\begin{definition}
    An environment $\Sigma$ is a tuple $(\mu, \rho)$ where $\mu : \textsc{IdentifierNames} \partialfunc \mathbb{N}$ is the \emph{variable lookup environment}, and $\rho : \mathbb{N} \partialfunc \mathcal{V}$ is the \emph{storage environment}.
\end{definition}

\framebox{$\angles{\Sigma, \overline{\Stmt}} \to \angles{\Sigma, \overline{\Stmt}}$}

Note that we abbreviate $\angles{\Sigma, \cdot}$ as $\Sigma$, which signals the end of evaluation.

\begin{mathpar}
    \inferrule*[right=Seq]{
        % S_1 \neq \tryS~\{ \overline{S_4} \}~\catchS~\{ \overline{S_5} \}
        % \\
        \angles{\Sigma, S_1 } \to \angles{\Sigma', \overline{S_3}}
    }{ \angles{\Sigma, S_1 \overline{S_2}} \to \angles{\Sigma', \overline{S_3}~\overline{S_2}} }

    \inferrule*[right=Revert]{
    }{ \angles{ \Sigma, (\revert)~\overline{S} } \to \angles{ \Sigma, \revert } }
\end{mathpar}

Locators.

\begin{mathpar}
    \inferrule*[right=Loc-Id]{
    }{ \angles{ \Sigma, x } \to \angles{ \Sigma, \storage(\mu(x), \rho(\mu(x))) } }

    \inferrule*[right=Loc-VarDef]{
        \ell \not\in \dom(\rho)
    }{ \angles{ \Sigma, \var~x : T } \to \angles{ \Sigma[\mu \mapsto \mu[x \mapsto \ell], \rho \mapsto \rho[\ell \mapsto []]], \ell } }

    \inferrule*[right=Loc-Field-Congr]{
        \angles{ \Sigma, \Loc } \to \angles{ \Sigma', \Loc' }
    }{ \angles{ \Sigma, \Loc.x } \to \angles{ \Sigma', \Loc'.x } }

    \inferrule*[right=Loc-Field]{
        \rho(\ell) = \overline{k}
        \\
        \overline{\rho(k).x = j}
    }{ \angles{ \Sigma, \ell.x } \to \angles{ \Sigma', \overline{j} } }

    \inferrule*[right=Loc-List-Congr]{
        \angles{ \Sigma, \Loc } \to \angles{ \Sigma', \Loc' }
    }{ \angles{ \Sigma, [ \overline{\ell}, \Loc, \overline{\Loc'} ] } \to \angles{ \Sigma', [ \overline{\ell}, \Loc', \overline{\Loc'} ] } }

    \inferrule*[right=Loc-List]{
        k \not\in \dom(\rho)
    }{ \angles{ \Sigma, [ \overline{\ell} ] } \to \angles{ \Sigma[\rho \mapsto \rho[k \mapsto \overline{\ell}]] , k } }

    \inferrule*[right=Loc-Val-Src-Congr]{
        \angles{ \Sigma, \Loc } \to \angles{ \Sigma'', \Loc'' }
    }{ \angles{ \Sigma, \Loc[\Loc'] } \to \angles{ \Sigma'', \Loc''[\Loc'] } }

    \inferrule*[right=Loc-Val-Sel-Congr]{
        \angles{ \Sigma, \Loc' } \to \angles{ \Sigma'', \Loc'' }
    }{ \angles{ \Sigma, \ell[\Loc'] } \to \angles{ \Sigma'', \ell[\Loc'] } }

    \inferrule*[right=Loc-Val]{
    }{ \angles{ \Sigma, \overline{\ell}[\overline{k}] } \to \angles{ \Sigma, \select(\rho, \overline{\ell}, \overline{k}) } }

    % \inferrule*[right=Loc-Filter]{
    %     y \not\in \dom(\mu)
    %     \\
    %     \overline{j} = [ k \in \overline{\ell} : \angles{ \Sigma, f(\overline{x}, y) } \to^* \angles{ \Sigma
    % }{ \angles{ \Sigma, \overline{\ell} \sends{\mathcal{Q}~\tsuchthat~f(\overline{x},  @x } \to \angles{ \Sigma', \Loc' \to @x } }
\end{mathpar}

\[
    \select(\rho, \overline{\ell}, \overline{k}) =
    \begin{cases}
        [] & \tif \overline{k} = [] \\
        \revert & \tif \overline{\ell} = [] \tand \overline{k} \neq [] \\
        j, \select(\rho, \overline{\ell} \setminus j, \overline{k'}) & \tif \overline{k} = (i, \overline{k'}) \tand j \in \overline{\ell} \tand \rho(i) = \rho(j) \\
        \select(\rho, \overline{\ell}, \overline{k'}) & \tif \overline{k} = (i, \overline{k'}) \tand \forall j \in \overline{\ell}. \rho(i) \neq \rho(j) \\
        % \select(\overline{\ell'}, \overline{k'})
    \end{cases}
\]

\reed{TODO Finish this rule/figuring out exactly how all this reference stuff works...}
\begin{mathpar}
    \inferrule*[right=Flow]{
    }{ \angles{ \Sigma, \ell \to k } \to \angles{ \Sigma[\rho \mapsto \rho[\ell \mapsto \rho(\ell) \setminus, k \mapsto \rho(k) + \rho(\ell)]] } }
\end{mathpar}

\begin{mathpar}
    \inferrule*[right=Flow-Every]{
        \resolve(\Sigma, \mathcal{S}) = (\Sigma', \ell)
    }{ \angles{\Sigma, \mathcal{S} \to \mathcal{D}} \evaluates \angles{\Sigma'[\rho \mapsto \rho'[\ell \mapsto []]], \putS(\rho'(\ell), \mathcal{D}) } }

    \inferrule*[right=Flow-Var]{
        \resolve(\Sigma, \mathcal{S}) = (\Sigma', \ell)
        \\
        \rho'(\ell) = \mathcal{V}
        \\
        \rho'(\mu'(x)) = \mathcal{W}
        \\
        \mathcal{W} \leq \mathcal{V}
    }{ \angles{\Sigma, \mathcal{S} \sends{x} \mathcal{D}} \evaluates \angles{\Sigma'[\rho \mapsto \rho'[\ell \mapsto \mathcal{V} - \mathcal{W}]], \putS(\mathcal{W}, \mathcal{D}) } }

    \inferrule*[right=Flow-Var-Fail]{
        \resolve(\Sigma, \mathcal{S}) = (\Sigma', \ell)
        \\
        \rho'(\ell) = \mathcal{V}
        \\
        \rho'(\mu'(x)) = \mathcal{W}
        \\
        \mathcal{W} \not\leq \mathcal{V}
    }{ \angles{\Sigma, \mathcal{S} \sends{x} \mathcal{D}} \evaluates \angles{ \Sigma', \revert } }

    \inferrule*[right=Flow-Filter]{
        \resolve(\Sigma, \mathcal{S}) = (\Sigma', \ell)
        \\
        \rho'(\ell) = \mathcal{V}
        \\
        \mathcal{U} = [ v \in \mathcal{V} ~|~ \angles{\Sigma', f(\overline{x}, v)} \to^* \angles{\Sigma'', k} \tand \rho''(k) = \true ]
        \\
        \compat(|\mathcal{U}|, |\mathcal{V}|, \mathcal{Q})
    }{ \angles{ \Sigma, \mathcal{S} \sends{\mathcal{Q}~\suchthat~f(\overline{x})} \mathcal{D} } \evaluates \angles{ \Sigma'[\rho' \mapsto \rho'[\ell \mapsto \rho'(\ell) - \mathcal{U}]], \putS(\mathcal{U}, \mathcal{D}) } }

    \inferrule*[right=Flow-Filter-Fail]{
        \resolve(\Sigma, \mathcal{S}) = (\Sigma', \ell)
        \\
        \rho'(\ell) = \mathcal{V}
        \\
        \mathcal{U} = [ v \in \mathcal{V} ~|~ \angles{\Sigma', f(\overline{x}, v)} \to^* \angles{\Sigma'', k} \tand \rho''(k) = \true ]
        \\
        \lnot \compat(|\mathcal{U}|, |\mathcal{V}|, \mathcal{Q})
    }{ \angles{ \Sigma, \mathcal{S} \sends{\mathcal{Q}~\suchthat~f(\overline{x})} \mathcal{D} } \evaluates \angles{ \Sigma', \revert } }

    \inferrule*[right=Flow-Transformer]{
        \resolve(\Sigma, \mathcal{S}) = (\Sigma', \ell)
        \\
        \rho'(\ell) = v, \mathcal{V}
        \\
        \angles{ \Sigma'[\rho \mapsto \rho'[\ell \mapsto \mathcal{V}]], f(\overline{x}, v) } \to^* \angles{ \Sigma'', k }
    }{ \angles{\Sigma, \mathcal{S} \to f(\overline{x}) \to \mathcal{D}} \evaluates \angles{ (\mu', \rho''), \putS([ \rho''(k) ], \mathcal{D}) ~ (\mathcal{S} \to f(\overline{x}) \to \mathcal{D}) } }

    \inferrule*[right=Flow-Transformer-Done]{
        \resolve(\Sigma, \mathcal{S}) = (\Sigma', \ell)
        \\
        \rho'(\ell) = []
    }{ \angles{ \Sigma, \mathcal{S} \to f(\overline{x}) \to \mathcal{D} } \evaluates \angles{ \Sigma, \putS([], \mathcal{D}) } }
\end{mathpar}

\reed{NOTE: It is important that we flow an empty list in the \textsc{Flow-Transformer-Done} rule, otherwise we may fail to allocate a variable as expected.}

\begin{mathpar}
    \inferrule*[right=Call]{
        \ell \not\in \dom(\rho)
        \\
        \transformer~f(\overline{y : \tau}) \to z : \sigma~\{~\overline{S}~\}
        \\
        \mu' = \overline{y \mapsto \mu(x)}, z \mapsto \ell
    }{ \angles{ \Sigma, f(\overline{x}) } \evaluates \angles{ (\mu', \rho[\ell \mapsto []]), \overline{S}~\ell } }
\end{mathpar}

We introduce a new statement, $\tryS(\Sigma, \overline{S_1}, \overline{S_2})$, to implement the try-catch statement, which keeps track of the environment that we begin execution in so that we can revert to the original environment in the case of a $\revert$.

\begin{mathpar}
    \inferrule*[right=Try-Start]{
    }{ \angles{ \Sigma, \tryS~\{ \overline{S_1} \}~\catchS~\{ \overline{S_2} \} } \evaluates \angles{ \Sigma, \tryS(\Sigma, \overline{S_1}, \overline{S_2}) } }

    \inferrule*[right=Try-Step]{
        \angles{ \Sigma, \overline{S_1} } \evaluates \angles{ \Sigma'', \overline{S_1'} }
    }{ \angles{ \Sigma, \tryS(\Sigma', \overline{S_1}, \overline{S_2}) } \evaluates \angles{ \Sigma'', \tryS(\Sigma', \overline{S_1'}, \overline{S_2}) } }

    \inferrule*[right=Try-Revert]{
    }{ \angles{ \Sigma, \tryS(\Sigma', \revert, \overline{S_2}) } \evaluates \angles{ \Sigma', \overline{S_2}} }

    \inferrule*[right=Try-Done]{
    }{ \angles{ \Sigma, \tryS(\Sigma', \cdot, \overline{S_2}) } \evaluates \Sigma }
\end{mathpar}

\reed{TODO: Need to be sure to handle uniqueness correctly}

\subsection{Auxiliaries}
\begin{definition}
    Define $\Quant = \curlys{\emptyq, \any, \exactlyone, \nonempty, \every}$, and call any $\mathcal{Q} \in \Quant$ a \emph{type quantity}.
    Define $\emptyq < \any < \,\, \exactlyone < \nonempty < \every$.
\end{definition}

\framebox{$\isAsset(\overline{T_V}, \tau)$} \textbf{Asset Types}

\begin{align*}
    \isAsset(\overline{T_V}, \mathcal{Q}~T) \iff \mathcal{Q} \neq \emptyq \tand (& \asset \in \modifiers(\overline{T_V}, T) \tor \\
                                                                                 & (T = \mathcal{C}~\tau \tand \isAsset(\overline{T_V}, \tau)) \tor \\
                                                                                 & (T = \curlys{\overline{y : \sigma}} \tand \exists x : \tau \in \overline{y : \sigma}. \isAsset(\overline{T_V}, \tau)) \tor \\
\end{align*}
% \reed{It should be the case that a transformer can have an output of an asset type if and only if it has an input asset type (and it the case of curried transformers, that \textbf{some} input type is an asset).}

\framebox{$\tau~\consumable$} \textbf{Consumable Types}
\begin{align*}
    (\mathcal{Q}~T)~\consumable \iff &\consumable \in \modifiers(T) \tor \\
                                     &\lnot((\mathcal{Q}~T)~\asset) \tor \\
                                     &(T = \mathcal{C}~\tau \tand \tau~\consumable) \tor \\
                                     &(T = \curlys{\overline{y : \sigma}} \tand \forall x : \tau \in \overline{y : \sigma}. (\sigma~\consumable))
\end{align*}

$\mathcal{Q} \oplus \mathcal{R}$ represents the quantity present when flowing $\mathcal{R}$ of something to a storage already containing $\mathcal{Q}$.
$\mathcal{Q} \ominus \mathcal{R}$ represents the quantity left over after flowing $\mathcal{R}$ from a storage containing $\mathcal{Q}$.

\begin{definition}
    Let $\mathcal{Q}, \mathcal{R} \in \Quant$.
    Define the commutative operator $\oplus$, called \emph{combine}, as the unique function $\Quant^2 \to \Quant$ such that
    \[
        \renewcommand\arraystretch{1.2}
        \begin{array}{r c l l l}
            \mathcal{Q} \oplus \emptyq & = & \mathcal{Q} & \\
            \mathcal{Q} \oplus \every & = & \every & \\
            \nonempty \oplus \mathcal{R} & = & \nonempty & \tif \emptyq < \mathcal{R} < \every \\
            \exactlyone \oplus \mathcal{R} & = & \nonempty & \tif \emptyq < \mathcal{R} < \every \\
            \any \oplus \any & = & \any &
        \end{array}
    \]

    Define the operator $\ominus$, called \emph{split}, as the unique function $\Quant^2 \to \Quant$ such that
    \[
        \renewcommand\arraystretch{1.2}
        \begin{array}{r c l l l}
            \mathcal{Q} \ominus \emptyq & = & \mathcal{Q} & \\
            \emptyq \ominus \mathcal{R} & = & \emptyq & \\
            \mathcal{Q} \ominus \every &= & \emptyq & \\
            \every \ominus \mathcal{R} &= & \every & \tif \mathcal{R} < \every \\
            \nonempty - \mathcal{R} &= & \any & \tif \emptyq < \mathcal{R} < \every \\
            \exactlyone - \mathcal{R} &= & \emptyq & \tif \exactlyone \leq \mathcal{R} \\
            \exactlyone - \any &= & \any & \\
            \any - \mathcal{R} &= & \any & \tif \emptyq < \mathcal{R} < \every \\
        \end{array}
    \]
\end{definition}

Note that we write $(\mathcal{Q}~T) \oplus \mathcal{R}$ to mean $(\mathcal{Q} \oplus \mathcal{R})~T$ and similarly $(\mathcal{Q}~T) \ominus \mathcal{R}$ to mean $(\mathcal{Q} \ominus \mathcal{R})~T$.

\begin{definition}
    We can consider a type environment $\Gamma$ as a function $\textsc{Identifiers}\xspace \to \textsc{Types}\xspace \cup \curlys{\bot}$ as follows:
    \[
        \Gamma(x) =
        \begin{cases}
            \tau & \tif x : \tau \in \Gamma \\
            \bot & \owise
        \end{cases}
    \]
    We write $\dom(\Gamma)$ to mean $\setbuild{x \in \textsc{Identifiers}}{\Gamma(x) \neq \bot}$, and $\Gamma|_X$ to mean the environment $\setbuild{x : \tau \in \Gamma}{x \in X}$ (restricting the domain of $\Gamma$).
\end{definition}

\begin{definition}
    Let $\mathcal{Q}$ and $\mathcal{R}$ be \typeQuantities, $T_\mathcal{Q}$ and $T_\mathcal{R}$ base types, and $\Gamma$ and $\Delta$ type environments.
    Define the following orderings, which make types and type environments into join-semilattices.
    For type quantities, define the partial order $\sqsubseteq$ as the reflexive closure of the strict partial order $\sqsubset$ given by
    \[
        \mathcal{Q} \sqsubset \mathcal{R} \iff (\mathcal{Q} \neq \any \tand \mathcal{R} = \any) \tor (\mathcal{Q} \in \curlys{\exactlyone, \every} \tand \mathcal{R} = \nonempty)
    \]
    For types, define the partial order $\leq$ by
    \[
        \mathcal{Q}~T_\mathcal{Q} \leq \mathcal{R}~T_\mathcal{R} \iff T_\mathcal{Q} = T_\mathcal{R} \tand \mathcal{Q} \sqsubseteq \mathcal{R}
    \]
    For type environments, define the partial order $\leq$ by
    \[
        \Gamma \leq \Delta \iff \forall x. \Gamma(x) \leq \Delta(x)
    \]
    Denote the join of $\Gamma$ and $\Delta$ by $\Gamma \sqcup \Delta$.
\end{definition}

\framebox{$\elemtype(T) = \tau$}
\begin{align*}
    \elemtype(T) =
    \begin{cases}
        \elemtype(T') & \tif T = \type~t~\is~\overline{M}~T' \\
        \tau & \tif T = \mathcal{C}~\tau \\
        !~T & \owise
    \end{cases}
\end{align*}

\framebox{$\modifiers(\overline{T_V}, T) = \overline{M}$} \textbf{Type Modifiers}
\begin{align*}
    \modifiers(\overline{T_V}, T) =
    \begin{cases}
        \overline{M} & \tif T = \type~t~\is~\overline{M}~T' \\
        \overline{M} & \tif (T~\is~\overline{M}) \in \overline{T_V} \\
        \emptyset & \owise
    \end{cases}
\end{align*}

\framebox{$\demoteT(\tau) = \sigma$}
\framebox{$\demoteT_*(T_1) = T_2$} \textbf{Type Demotion}
$\demoteT$ and $\demoteT_*$ take a type and ``strip'' all the asset modifiers from it, as well as unfolding named type definitions.
This process is useful, because it allows selecting asset types without actually having a value of the desired asset type.
Note that demoting a transformer type changes nothing.
This is because a transformer is \textbf{never} an asset, regardless of the types that it operators on, because it has no storage.

\begin{align*}
    \demoteT(\mathcal{Q}~T) &= \mathcal{Q}~\demoteT_*(T) \\
    \demoteT_*(\boolt) &= \boolt \\
    \demoteT_*(\natt) &= \natt \\
    \demoteT_*(\curlys{\overline{x : \tau}}) &= \curlys{\overline{x : \demoteT(\tau)}} \\
    \demoteT_*(\type~t~\is~\overline{M}~T) &= \demoteT_*(T)
\end{align*}

\framebox{$\fields(T) = \overline{x : \tau}$} \textbf{Fields}
\begin{align*}
    \fields(T) =
    \begin{cases}
        \overline{x : \tau} & \tif T = \{ \overline{x : \tau} \} \\
        \fields(T) & \tif T = \type~t~\is~\overline{M}~T \\
        \emptyset & \owise
    \end{cases}
\end{align*}

\framebox{$\update(\Gamma, x, \tau)$} \textbf{Type environment modification}
\[
    \update(\Gamma, x, \tau) =
    \begin{cases}
        \Delta, x : \tau & \tif \Gamma = \Delta, x : \sigma \\
        \Gamma & \owise
    \end{cases}
\]

\framebox{$\compat(n, m, \mathcal{Q})$}
The relation $\compat(n, m, \mathcal{Q})$ holds when the number of values sent, $n$, is compatible with the original number of values $m$, and the type quantity used, $\mathcal{Q}$.

\begin{align*}
    \compat(n, m, \mathcal{Q}) \iff & (\mathcal{Q} = \nonempty \tand n \geq 1) \tor \\
                                    & (\mathcal{Q} = \,\,\exactlyone \tand n = 1) \tor \\
                                    & (\mathcal{Q} = \emptyq \tand n = 0) \tor \\
                                    & (\mathcal{Q} = \every \tand n = m) \tor \\
                                    & \mathcal{Q} = \any
\end{align*}

\framebox{$\values(T) = \mathcal{V}$}
The function $\values$ gives a list of all of the values of a given base type.

\begin{align*}
    \values(\boolt) & = [ \true, \false ] \\
    \values(\natt) & = [ 0, 1, 2, \ldots ] \\
    \values(\listq~T) & = [ L | L \subseteq \values(T), |L| < \infty ] \\
    \values(\type~t~\is~\overline{M}~T) &= \values(T) \\
    \values(\{ \overline{x : \mathcal{Q}~T} \}) &= [ \{ \overline{x : \tau \mapsto v} \} | \overline{v \in \values(T)} ]
\end{align*}

\end{document}

